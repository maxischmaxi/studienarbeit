\newpage
\thispagestyle{empty}
\begin{minipage}[\textheight]{145mm}
%\textbf{{\textsf{\large Bibliografischer Nachweis}}}
\minisec{\large Bibliografischer Nachweis}

\vspace{10mm}

Name, Vorname: Bohdanova, Henriieta\\
Studienarbeit\\

Titel der Studienarbeit: Stakeholder-Kostenanalyse an Flughäfen bei Einführung neuer Luftfahrzeugantriebe\\

Technische Universität Dresden\\
Fakultät Verkehrswissenschaften \glqq Friedrich List\grqq{}\\
Institut für Luftfahrt und Logistik\\

Studiengang Verkehrsingenieurwesen\\
XX Seiten, XX Bilder, XX Tabellen, XX Formeln, XX Quellenangaben 
		
\vspace{15mm}

\begin{minipage}[\textheight]{145mm}
%\textbf{{\textsf{\large Autorenreferat}}}
\minisec{\large Autorenreferat}
 
\vspace{10mm}

 
Innovative Luftfahrzeugantriebe können dazu beitragen, das Netto-Null-Ziel der EU zu erreichen und die \ce{CO_2}-Neutralität im Luftverkehr voranzutreiben.
In Rahmen dieser Arbeit wird untersucht, welche Änderung in den Betrieb- und Infrastrukturkosten entstehen, 
als auch welche zusätzliche Ausbildung benötigt wird. Durch eine umfassende Recherche wurde ein Überblick über mögliche alternative Antriebe geschaffen, aus dem die Grundlagen für Kostenberechnungen abgeleitet wurden.
Es zeigt sich, dass innovative Antriebe zu höheren Betriebskosten führen, während sich die Infrastrukturkosten je nach Antriebstechnologie unterschiedlich entwickeln.
Obwohl neue Antriebskonzepte mit höheren Kosten verbunden sind, könnten sie sich aufgrund ihrer Umweltvorteile 
langfristig dennoch als eine sinnvolle Investition darstellen.


 
\end{minipage}\hspace{4cm}



\end{minipage}
\cleardoublepage