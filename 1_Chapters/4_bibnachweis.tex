\newpage
\thispagestyle{empty}
\begin{minipage}[\textheight]{145mm}
%\textbf{{\textsf{\large Bibliografischer Nachweis}}}
\minisec{\large Bibliografischer Nachweis}

\vspace{10mm}

Name, Vorname: Bohdanova, Henriieta\\
Studienarbeit\\

Titel der Studienarbeit: Stakeholder-Kostenanalyse an Flughäfen bei Einführung neuer Luftfahrzeugantriebe\\

Technische Universität Dresden\\
Fakultät Verkehrswissenschaften \glqq Friedrich List\grqq{}\\
Institut für Luftfahrt und Logistik\\

Studiengang Verkehrsingenieurwesen\\
XX Seiten, XX Bilder, XX Tabellen, XX Formeln, XX Quellenangaben 
		
\vspace{15mm}

\begin{minipage}[\textheight]{145mm}
%\textbf{{\textsf{\large Autorenreferat}}}
\minisec{\large Autorenreferat}
 
\vspace{10mm}

Die Mitteldeutsche Flughafen AG hat sich zum Ziel gesetzt, bis zum Jahr 2030 einen \acs{CO2}-neutralen Betrieb zu erreichen. In diesem Kontext gilt es, den betriebenen Fuhrpark zu einem möglichst großen Anteil auf alternative Antriebe umzustellen. Als Ergänzung zu bereits betriebenen Elektro- und Hybridfahrzeugen
wird im Rahmen dieser Arbeit die Einsatzmöglichkeit von Wasserstofffahrzeugen untersucht.\\
Auf einen Überblick über die Technologie und ihren aktuellen Einsatz folgt die Untersuchung des Fuhrparks des Praxispartners auf Grundlage eines zur Verfügung gestellten Datensatzes. Dabei werden verschiedene Aspekte von Fahrzeugbetrieb und die damit verbundenen Kosten mittels entwickelter Modelle beleuchtet.\\
Haupterkenntnis ist, dass ein hinsichtlich Antrieben homogener Fuhrpark nicht zielführend ist. Durch individuelle Abwägung operativer Anforderungen sollte ein hybrider Fuhrpark mit Elektro- und Wasserstofffahrzeugen gebildet werden. Dieser gewährleistet operative Zuverlässigkeit bei Nullemission. 
\end{minipage}\hspace{4cm}



\end{minipage}
\cleardoublepage