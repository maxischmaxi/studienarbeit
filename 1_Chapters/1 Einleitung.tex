\chapter{Motivation}
\label{ch:Einleitung}

Laut Verordnung (EU) 2021 soll die EU zum Jahr 2050 klimaneutral sein und bereits zum Jahr 2030 sollen die Treibhauseffekte um mindestens 55\%
im Vergleich zum Jahr 1990 reduziert werden.
Im Jahr 2023 ermittelte Umweltbundesamt, dass die Treibhausgase im Vergleich zum Vorjahr um mehr als zehn Prozent gesunken sind.
Nichtsdestotrotz wurden allein in Deutschland im Jahr 2023 673 Mio. Tonnen Treibhausgase freigesetzt \cite{bundesregierung}.
\ce{CO2}-Emissionen haben die Auswirkungen auf die Umwelt und sind eine Ursache für den Klimawandel. Luftverkehr hat auch seine Rolle in diesen Einfluss. 
Etwa 2,5 \% von ganzen anthropogenen \ce{CO2}-Emissionen weltweit werden vom Luftverkehr
durch die Treibstoffverbrennung verursacht \cite{conrady2019luftverkehr}.

Ob nachhaltige Alternativen die Emissionswerte mindern oder vermeiden können, ist derzeit ein begehrtes Thema. 

%Wenn die Emissionen weiter mit der Wachstumsrate (2002-2022) von ca.1,8\% (Daten aus: Worldatabank https://data.worldbank.org/indicator/EN.GHG.ALL.MT.CE.AR5) 
%jährlich erhöhen werden, erreichen die Treibhausgase im Jahr 2050 den Wert von 88 Gt ohne Berücksichtigung von Landnutzung, 
%Landnutzungsänderungen und Forstwirtschaft. (64\% Steigung?)

Neben dem Kohlendioxid \ce{CO2} und Wasserstoff \ce{H2O} entstehen bei der Verbrennung des Treibstoffs andere Nebenprodukte, die 
das Klima beeinflussen, wie Stickoxide \ce{NOx}, die für Ozonbildung in der Stratosphäre verantwortlich sind \cite{conrady2019luftverkehr}.

Diese Arbeit widmet sich dem Thema nachhaltige Antriebe, nämlich im Fokus stehen nachhaltige Kraftstoffe (SAF), Wassertreibstoff und 
batterieelektrische Antriebe.
%Mittlerweile sind viele Betriebe verpflichtet auf die Umweltneutralität zu achten. Die Flughäfen stellen ihre Bodendienste auf elektrische Antriebe um.
Durch die neuartigen Konfigurationen und alternative Kraftstoffe und Antriebe existiert die Möglichkeit die unnötigen Emissionen zu vermeiden.
Innovative Antriebe, wie Batterie, Wassertreibstoff oder Sustainable Aviation Fuel (SAF), versprechen unter nachhaltiger Produktion und
Logistik geringere Einfluss auf die Umwelt und somit die Reduktion der Emissionen. 

Eine zusammenfassende Berechnung der Kosten für die alle drei Alternativen durch die Einführung dieser Antriebe und den Betrieb entstehen, 
wurde bislang noch nicht systematisch
untersucht. In der wissenschaftlichen Arbeiten sind die bereits getrennte Kostenberechnungen für die einzelne Alternative, meistens für 
elektrische oder wasserstoffbetriebene Flugzeuge, zu finden. Die Interesse für das Thema im Vergleich zu ... deutlich gestiegen.
Neue Erkenntnisse und Vorschläge.

Dafür interessant anzuschauen, wie sich konventionelle Kraftstoff-Flugzeugen und innovativen Antriebe unterscheiden 
und welche Veränderungen der Betrieb-, Infrastruktur- und Ausbildungskosten dadurch entstehen.

Dieses Thema ist für die Praxis relevant, weil alle Fluggesellschaften bestimmten Kriterien unterliegen. 
Geprägt von strengeren und wachsenden Maßnahmen in Bezug auf die Treibhausgase, brauchen die Betriebsunternehmen neue Technologien, um die  
höhere \ce{CO2}-Emissionen und damit verbundene höheren Kosten zu vermeiden.
Es soll untersucht werden, ob die nachhaltigen Antriebe eine Möglichkeit haben kostengünstig im Markt zu gelangen und Wettbewerb durchstehen oder 
sogar als Ersatz für die konventionellen Kraftstoffe, wie Kerosin, dienen können.

Im Rahmen dieser Studienarbeit werden folgende Themen berührt:
Kapitel \ref{ch:Relevante Grundlagen und Überblick über alternative Antriebe} stellt die relevanten Grundlagen zur weiteren Forschung dar, wie Stakeholder am Flughafen und deren Teil an der Abfertigung eines Flugzeugs,
die Betriebskosten, gesetzliche Einflüsse auf der Luftverkehr sowie die zukünftigen Flugzeugkonfigurationen mit neuer Antriebstechnologien.
Darauf aufbauend werden im Kapitel \ref{ch:Änderungen durch neue Antriebe, Annahmen und Methodik} die Methodik für die Kostenberechnung und betriebliche Szenarien für einen Flughafen definiert, 
als auch die getroffenen Annahmen erörtert.
Kapitel \ref{ch:Auswertung den Ergebnissen} begebt sich um die Auswertung der Kostenanalyse für den Betrieb und aufgestellten Betriebsszenarien, dazugehörige Sensitivitätsanalyse und
kritische Auseinandersetzung mit den Ergebnissen. Zusätzlich werden in diesem Kapitel die aufgestellten Hypothesen diskutiert.
%
Kapitel \ref{ch:Fazit} enthält eine abschließende Zusammenfassung und erdenkliche Richtung für die weiterführende Forschungsarbeiten.
Im Anhang sind die zusätzliche Information zu den einzelnen Kapiteln aufgeführt.
%einen Ausblick auf die Unsicherheiten in der Arbeit 