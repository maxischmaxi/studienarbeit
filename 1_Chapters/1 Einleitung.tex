\chapter{Motivation}
\label{ch:Einleitung}

Laut Verordnung (EU) 2021/1119 soll die EU zum Jahr 2050 klimaneutral sein. Bereits im Jahr 2030 sollen 
die Treibhauseffekte um mindestens 55 \%
verglichen mit dem Jahr 1990 reduziert und die Klimaerwärmung auf 1,5 °C gegenüber 
des vorindustriellen Niveaus begrenzt werden.
Treibhausgase haben Auswirkungen auf das Wetter und führen zu globaler Erderwärmung, 
was letztlich die menschliche Gesundheit bedroht. % vielleicht Existenz?
Obwohl das Umweltbundesamt im Jahr 2023 ermittelte, dass die Treibhausgase im Vergleich zum Vorjahr um mehr als zehn Prozent gesunken sind, 
wurden in dem Jahr allein in Deutschland 673 Mio. Tonnen Treibhausgase freigesetzt \cite{bundesregierung}.
Vor allem durch ihre Langlebigkeit \cite{filonchyk2024greenhouse} spielen \ce{CO2}-Emissionen eine zentrale Rolle unter den Treibhausgasen. 
Auch der Luftverkehr trägt Verantwortung für diesen Einfluss. 
Etwa 2,5 \% der gesamten anthropogenen \ce{CO2}-Emissionen weltweit werden vom Luftverkehr
durch die Treibstoffverbrennung verursacht \cite{conrady2019luftverkehr}. 

Neben dem Kohlendioxid \ce{CO2} und Wasserstoff \ce{H2O} entstehen bei der Verbrennung des Treibstoffs andere Nebenprodukte und Rußpartikel, wie Stickoxide \ce{NO_x},
die das Klima beeinflussen und für die Ozonbildung in der Stratosphäre verantwortlich sind \cite{conrady2019luftverkehr}.

Die internationale Gesellschaft sucht nach Lösungen, um die Klimakrise zu bewältigen. 
Ob nachhaltige Alternativen durch den Luftverkehr verursachte Emissionswerte mindern können, 
ist derzeit ein weit verbreitetes Thema.
Durch neuartige Konfigurationen und alternative Kraftstoffe und Antriebe besteht die Möglichkeit Emissionen zu vermeiden.
Innovative Energieträger, wie Batterie, Wasserstoff oder Sustainable Aviation Fuel (SAF), 
versprechen unter nachhaltiger Produktion und
Logistik eine Reduktion der Emissionen zu erwirken und somit geringen Einfluss auf die Umwelt zu haben.
Allein durch den Einsatz von Elektro- und Wasserstoffantrieben könnte 17 \% des Netto-Null-Ziels erreicht werden \cite{gao2022hydrogen}.
%Außerdem sind fossile Energie \cite{gao}
%Wenn die Emissionen weiter mit der Wachstumsrate (2002-2022) von ca.1,8\% (Daten aus: Worldatabank https://data.worldbank.org/indicator/EN.GHG.ALL.MT.CE.AR5) 
%jährlich erhöhen werden, erreichen die Treibhausgase im Jahr 2050 den Wert von 88 Gt ohne Berücksichtigung von Landnutzung, 
%Landnutzungsänderungen und Forstwirtschaft. (64\% Steigung?)
%
%Diese Arbeit widmet sich dem Thema nachhaltige Antriebe, nämlich im Fokus stehen nachhaltige Kraftstoffe (SAF), Wassertreibstoff und 
%batterieelektrische Antriebe.
%Mittlerweile sind viele Betriebe verpflichtet auf die Umweltneutralität zu achten. Die Flughäfen stellen ihre Bodendienste auf elektrische Antriebe um.

Ein zusammenfassender Unterschied der Kosten für alle drei Alternativen durch die Einführung dieser Antriebe 
wurde bislang noch nicht systematisch untersucht. 
Forscher haben bereits getrennte Kostenberechnungen für einzelne neuartige Antriebe durchgeführt, mit dem Fokus auf
elektrische oder wasserstoffbetriebene Flugzeuge. 
%Dafür interessant anzuschauen, wie sich konventionelle Kraftstoff-Flugzeugen und innovativen Antriebe unterscheiden 
%und welche Veränderungen der Betrieb-, Infrastruktur- und Ausbildungskosten dadurch entstehen.

Geprägt von wachsend strengeren Maßnahmen in Bezug auf die Treibhausgase, brauchen Betreiber neue Technologien um
höhere \ce{CO2}-Emissionen und damit verbundene höhere Kosten zu vermeiden.
Im Rahmen dieser Arbeit soll untersucht werden, ob nachhaltige Antriebe eine Möglichkeit haben, 
kostengünstig in den Markt zu gelangen und den Wettbewerb zu überstehen oder 
sogar als Ersatz für konventionelle Kraftstoffe, wie Kerosin, dienen zu können. 
Daraus ergeben sich die Fragestellungen, welchen Einfluss hat die Einführung neuartiger Antriebe 
auf Stakeholder und auf den Markt, und welche Betriebs-, Infrastruktur- und Ausbildungsdifferenzen entstehen dadurch.

Im Rahmen dieser Studienarbeit werden folgende Themen berührt:
Kapitel \ref{ch:Relevante Grundlagen und Überblick über alternative Antriebe} stellt 
die relevanten Grundlagen zur weiteren Forschung dar, wie Stakeholder am Flughafen und 
deren Einfluss auf die Abfertigung eines Flugzeugs, die Betriebskosten, 
gesetzliche Einflüsse auf den Luftverkehr sowie die zukünftigen Flugzeugkonfigurationen 
mit neuen Antriebstechnologien.
Darauf aufbauend wird im Kapitel \ref{ch:Änderungen durch neue Antriebe, Annahmen und Methodik} 
anhand Recherche die Methodik für die Kostenberechnung und betriebliche Szenarien für einen Flughafen definiert, 
sowie getroffene Annahmen erörtert.
Kapitel \ref{ch:Auswertung der Ergebnisse} bearbeitet die Auswertung der Kostenanalyse 
für den Betrieb und aufgestellter Betriebsszenarien, die dazugehörige Sensitivitätsanalyse und
eine kritische Auseinandersetzung mit den Ergebnissen. 
Zusätzlich werden in diesem Kapitel die aufgestellten Hypothesen diskutiert.
Kapitel \ref{ch:Fazit} enthält eine abschließende Zusammenfassung 
und zeigt mögliche Richtungen für weiterführende Forschungsarbeiten.
Im Anhang sind zusätzliche Informationen zu den einzelnen Kapiteln aufgeführt, 
auf die in der Arbeit Bezug genommen wird.
%einen Ausblick auf die Unsicherheiten in der Arbeit 
