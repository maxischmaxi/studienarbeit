\chapter{Einleitung}
\label{ch:Einleitung}



Laut Verordnung (EU) 2021 soll die EU zum Jahr 2050 klimaneutral sein, aber auch zum Jahr 2030 sollen die Treibhauseffekte um mindestens 55\%
im Vergleich zum Jahr 1990 reduziert werden. 

Ob nachhaltige Alternativen die Emissionswerte mindern oder vermeiden können, ist derzeit ein begehrtes Thema. 

Diese Arbeit widmet sich dem Thema nachhaltige Antriebe, nämlich im Fokus stehen nachhaltige Kraftstoffe (SAF), Wassertreibstoff und 
batterieelektrische Antriebe.
 
Im Jahr 2023 ermittelte Umweltbundesamt, dass die Treibhausgase im Vergleich zum Vorjahr um mehr als zehn Prozent gesunken sind.
Nichtsdestotrotz wurden allein in Deutschland im Jahr 2023 673 Mio. Tonnen Treibhausgase freigesetzt \cite{bundesregierung}.

\ce{CO2} Emissionen sind als eine Ursache für den Klimawandel gesehen. Luftverkehr hat auch seine Rolle in diesen Auswirkungen. 
Etwa 2,5\% von ganzen anthropogenen \ce{CO2} Emissionen weltweit werden vom Luftverkehr
durch die Treibstoffverbrennung verursacht \cite{conrady2019luftverkehr}.


Wenn die Emissionen weiter mit der Wachstumsrate (2002-2022) von ca.1,8\% (Daten aus: Worldatabank https://data.worldbank.org/indicator/EN.GHG.ALL.MT.CE.AR5) 
jährlich erhöhen werden, erreichen die Treibhausgase im Jahr 2050 den Wert von 88 Gt ohne Berücksichtigung von Landnutzung, 
Landnutzungsänderungen und Forstwirtschaft. (64\% Steigung?)

Neben dem Kohlendioxid \ce{CO2} und Wasserstoff \ce{H2O} entstehen bei der Verbrennung des Treibstoffs andere Nebenprodukte, die 
das Klima beeinflussen, wie Stickoxide \ce{NOx}, die für Ozonbildung in der Stratosphäre verantwortlich sind \cite{conrady2019luftverkehr}.

Mittlerweile sind viele Betriebe verpflichtet auf die Umweltneutralität zu achten. Die Flughäfen stellen ihre Bodendienste auf elektrische Antriebe um.

Durch die neuartige Konfiguration und alternative Kraftstoffe und Antriebe existiert die Möglichkeit die unnötige Emissionen zu vermeiden.
Alternative Antriebe, wie Batterie, Wassertreibstoff oder Sustainable Aviation Fuel (SAF) weisen unter nachhaltigen Produktion und Logistik kein Einfluss 
auf der Umwelt auf und somit helfen die Emissionen zu reduzieren. 


Die Frage, welche Kosten für die alle drei Alternativen durch diese Einführung und den Betrieb entstehen, wurde bislang noch nicht systematisch
untersucht. In der wissenschaftlichen Arbeiten sind die bereits getrennte Kostenberechnungen für die einzelne Alternative, meistens für 
elektrische oder wasserstoffbetriebene Flugzeuge, zu finden.

Jedoch eine zusammenfassende Berechnung der Kosten für alle drei Alternativen wurde bis jetzt nicht erforscht.
Dafür interessant anzuschauen, wie sich konventionelle Kraftstoff-Flugzeugen und neuartigen Antriebe unterscheiden 
und welche Veränderungen der Betrieb-, Infrastruktur- und Ausbildungskosten durch alternativen Antriebe entstehen.

Dieses Thema ist für die Praxis relevant, weil alle Fluggesellschaften bestimmten Kriterien unterliegen. 
Geprägt von strengeren und wachsenden Maßnahmen in Bezug auf die Treibhausgase, brauchen die Betriebe neue Technologien, um die  
höhere \ce{CO2} Ausstoß und damit verbunden höheren Kosten zu vermeiden.
Es soll untersucht werden, ob die nachhaltigen Antriebe eine Möglichkeit haben kostengünstig im Markt zu gelangen und Wettbewerb durchstehen oder 
sogar als Ersatz für die konventionellen Kraftstoffe, wie Kerosin, dienen können.

Aufbau der Arbeit: Im Rahmen dieser Studienarbeit...
Kapitel 2 stellt die relevanten Grundlagen zur weiteren Forschung dar, wie Stakeholder am Flughafen und deren Teil an der Abfertigung eines Flugzeugs,
die Betriebskosten, gesetzliche Einflüsse auf der Luftverkehr sowie die zukünftigen Flugzeugkonfigurationen mit neuer Antriebstechnologien.
Darauf aufbauend werden im Kapitel 3 die Methodik für die Kostenberechnung und betriebliche Szenarien für einen Flughafen definiert, 
als auch die getroffenen Annahmen erörtert.
Kapitel 4 begebt sich um die Auswertung der Kostenanalyse für die ... und dazugehörige Sensitivitätsanalyse.

Kapitel 5 enthält eine abschließende Zusammenfassung und einen Ausblick auf die Unsicherheiten in der Arbeit 
und mögliche Richtung für die weiterführende Forschungsarbeiten.

