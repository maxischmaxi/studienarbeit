\chapter*{Thesen zur Arbeit}
\label{ch:Thesen}
\thispagestyle{empty}

\begin{enumerate}
    \item These 1: Betriebskosten
    \item These 2: Die Verteilung alternativen Antrieben einer Flotte kann einen großen Einfluss auf die Infrastrukturkosten eines Flughafens haben.
    \item These 3: Die Infrastrukturkosten eines Flughafenbetreibers bleiben über die Jahre konstant, 
    da die einmaligen Investitionen über die Abschreibungsdauer verteilt werden, jedoch keine signifikanten Schwankungen in 
    den Betriebskosten auftreten.
    
    \item These 4: Die Infrastrukturkosten sind schwer vorherzusagen, da unerwartete Reparaturen 
    \item Am Ende der Abschreibungsperiode müssen erhebliche Investitionen getätigt werden, um die Infrastruktur zu erneuern, 
    was zu einem sprunghaften Anstieg der Infrastrukturkosten nach der Abschreibung führt.
    \item These 5: Obwohl bestimmte Szenarien höhere Anschaffungskosten aufweisen, 
    wird der finanzielle Effekt dieser höheren Investitionen durch die längere Abschreibungsdauer relativiert, 
    sodass die jährlichen Belastungen vergleichbar oder sogar geringer bleiben.
    \item These 6: Die Abschreibungsmethode bewirkt, dass trotz hoher Investitionskosten die jährlichen Kosten belastbarer 
    und niedriger sind, während bei geringeren Anschaffungskosten eine ineffiziente Kostenverteilung zu 
    höheren jährlichen Belastungen führen kann.
\end{enumerate}