\chapter*{Thesen zur Arbeit}
\label{ch:Thesen}
\thispagestyle{empty}

\begin{enumerate}
    \item Die Einführung innovativer Antriebe führt zu höheren Betriebskosten einer Fluggesellschaft im Vergleich zu herkömmlichen
    Jettriebwerken, da andere technische Anforderungen und zusätzlich neue Abfertigungsprozesse notwendig sind.
    %\item These 2: Die Verteilung alternativen Antrieben einer Flotte kann einen großen Einfluss auf die Infrastrukturkosten eines Flughafens haben.
    %\item These 3: Die Infrastrukturkosten eines Flughafenbetreibers bleiben über die Jahre konstant, 
    %da die einmaligen Investitionen über die Abschreibungsdauer verteilt werden...
    %
    %\item Die Wahl der Flottenzusammensetzung mit innovativen Antriebstechnologien in verschiedenen 
    %Betriebsstrategien beeinflusst signifikant die Betriebskosten von Fluggesellschaften. 
    %Interessanterweise könnte in Szenarien mit den größten Betriebskosten (z. B. bei langen Strecken) die Infrastrukturkosten geringer ausfallen, 
    %da manche innovative Antriebe weniger spezifische oder kostspielige Infrastruktur erfordern.
    \item Die Wahl der Flottenzusammensetzung mit innovativen Antriebstechnologien in verschiedenen 
    Betriebsstrategien beeinflusst die Betriebskosten von Fluggesellschaften, wobei Szenarien mit höheren 
    Betriebskosten geringere Infrastrukturkosten erfordern.
    \item Die Abschreibungsmethode bewirkt, dass trotz hoher Investitionskosten die jährlichen Kosten belastbarer 
    und niedriger sind, während bei geringeren Anschaffungskosten eine ineffiziente Kostenverteilung zu 
    höheren jährlichen Belastungen führen kann.
\end{enumerate}

%These 4: Die Infrastrukturkosten sind schwer vorherzusagen, da unerwartete Reparaturen 
%Am Ende der Abschreibungsperiode müssen erhebliche Investitionen getätigt werden, um die Infrastruktur zu erneuern, 
%was zu einem sprunghaften Anstieg der Infrastrukturkosten nach der Abschreibung führt.

%These 5: Obwohl bestimmte Szenarien höhere Anschaffungskosten aufweisen, 
%    wird der finanzielle Effekt dieser höheren Investitionen durch die längere Abschreibungsdauer relativiert, 
%    sodass die jährlichen Belastungen vergleichbar oder sogar geringer bleiben.
