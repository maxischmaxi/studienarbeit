\chapter{Fazit}
\label{ch:Fazit}
Innovative Antriebe und Kraftstoffe, wie SAF, Wasserstoff und der Batterieantrieb tragen dazu bei, die Auswirkungen des Luftverkehrs auf die Umwelt zu verringern.
Deren Wirtschaftlichkeit ist jedoch diskussionswürdig.
Diese Arbeit hat sich mit dem Thema Implementierung neuartiger Antriebe am Flughafen auseinandergesetzt. 
Dabei wurden anhand von Recherche sowohl die Antriebe als auch die Infrastruktur und Versorgungswege untersucht.
Schließlich wurden sowohl die erwarteten Betriebskosten für Fluggesellschaften, als auch Kapitalkosten für Flughäfen berechnet.

Zwar werden die Betriebskosten bei alternativen Antrieben im Vergleich zu herkömmlichen Jets steigen, 
jedoch sind die Kosten gegenüber den Auswirkungen auf die Umwelt gerechtfertigt, sodass sie in Kauf genommen werden müssen.
Zudem wurden die Betriebskosten und Infrastrukturkosten anhand drei Szenarien berechnet. 
Die Infrastrukturkosten am Flughafen drehten sich um die Swap-Methode für Batterieantriebe 
und die Nutzung von kryogenem Wasserstoff.
Die Ergebnisse weisen darauf hin, dass eine vermehrte Nutzung von Wasserstoffantrieben zu 
höheren Infrastrukturkosten führen kann.
Jedoch macht die Nutzung der Abschreibungsmethode ein Unterschied zwischen den Szenarien aus. 
Dies beantwortet die eingangs gestellte Forschungsfrage, welcher Einfluss von der Einführung alternativer Antriebe ausgeht.

Diese Arbeit bietet eine Übersicht zu wirtschaftlichen Zusammenhängen mit Bezug auf neue Antriebsverfahren.
Um das Potenzial von neuen Technologien jedoch vollständig auszuschöpfen, müssen weitere Untersuchungen 
mit tiefen Einblicken in die Infrastruktur, Ausbildung und Technologien in den Luftfahrzeugen selbst erfolgen.
% Dabei lieferte die Verteilung, wo die Kurzstrecken durch die Batterieantrieb komplett ersetzt wurden.
%Die Mittelstrecken durch WA und die Langstrecken durch die SAF bedient die beste Ergebnisse sowohl für Betrieb- als auch für einmalige Anschaffungskosten.
%Durch die
