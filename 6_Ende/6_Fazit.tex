\chapter{Fazit}
\label{ch:Fazit}
Die innovative Antriebe und Kraftstoffe, wie SAF, Wasserstoff und Batterieantrieb, schlagen Verkleinerung der Auswirkungen von Luftverkehr auf die Umwelt. 
Die Wirtschaftlichkeit von denen ist jedoch diskussionswürdig.
Diese Arbeit hat sich mit dem Thema Implementierung den neuartigen Antrieben am Flughafen auseinandergesetzt. Dabei wurde anhand Recherche 
Schließlich wurden die erwarteten Betriebskosten für Fluggesellschaften, als auch Kapitalkosten für die Flughäfen. 

Die Betriebskosten werden zwar bei alternativen Antrieben im Vergleich zu traditionellen Jets steigen, 
jedoch mit den Vorteilen, die sie im Umweltauswirkungen bringen, 
müssen die preislichen Anstiege in Kauf genommen werden. Zudem wurden die Betriebskosten und Infrastrukturkosten anhand drei Szenarien
berechnet. Die Infrastrukturkosten am Flughafen drehte sich um Swap-Methode für Batterieantriebe und Nutzung von kryogenen flüssigem Wasserstoff.
Die Ergebnisse zum Schluss gekommen, dass größere Nutzung von Wasserstoff-Antrieben zu größere Infrastrukturkosten führen kann.
Jedoch die Nutzung der Abschreibungsmethode ein Unterschied zwischen Szenarien ausmachen kann. Dies beantwortet die 
eingangs gestellte Forschungsfrage, welcher Einfluss wird von Einführung geprägt

Diese Arbeit bietet eine Übersicht zu wirtschaftlichen Zusammenhängen mit Bezug auf neuen Antriebsverfahren.
Um das Potenzial von neuen Technologien jedoch vollständig auszuschöpfen, müssen weitere Untersuchungen mit den tiefen Ausblicken
in Infrastruktur, Ausbildung und Technologien am Luftfahrzeugen selbst erfolgen.
% Dabei lieferte die Verteilung, wo die Kurzstrecken durch die Batterieantrieb komplett ersetzt wurden.
%Die Mittelstrecken durch WA und die Langstrecken durch die SAF bedient die beste Ergebnisse sowohl für Betrieb- als auch für einmalige Anschaffungskosten.
%Durch die
