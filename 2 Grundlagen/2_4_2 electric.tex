\subsection{Batterie-Antrieb}
"In der Luftfahrt ist es möglich, direkten Strom zu nutzen. Die direkte Nutzung von
Elektrizität erfordert Elektromotoren und Stromspeicher an Bord" \cite{dahal2021techno}, wie Batterie oder Brennstoffzellen \cite{dalmia2022powering}.

Funktionsweise Elektromotor: Durch Potenzialdifferenz und einen Stromfluss wird die elektrische Energie in die mechanische umgewandelt.
Im Vergleich zum Verbrennungsmotor ist der einzige beweglicher Teil hier ist der Rotor \cite{donckers2024electric}, 
was am Ende die Wartungskosten verringern kann (da weniger Verschleiß). Außerdem besteht der elektrische Antrieb aus Controller und.
Controller steuert den Energiefluss. Durch Controller wird festgelegt, welche Leistung soll der Motor bringen bzw. wie viel Energie soll von 
einer Batterie genommen werden für die gewünschte Leistung \cite{donckers2024electric}.
Ein Vorteil des elektrischen Flugzeugs ist, dass den Antrieb zulässt, rückwärtszufahren (Quelle). Durch die Umwandlung der elektrischen Energie in die chemische 
speichern Batterie in sich Energie. 
Im Laufe des Fluges, die verändern sein Gewicht nicht, unabhängig davon, ob sie leer oder vollgeladen sind \cite{donckers2024electric}. 
Eine vom öfter im wissenschaftlichen Literatur erwähnten Batterien ist das Li-Ion. Die haben eine hohe Energiedichte
Anhängigkeit von der Temperatur: Die Batterien sind von äußerlichen Bedingungen beeinflussbar. Die kalte Umgebung kann 
der Wirkungsgrad reduzieren (Quelle), und die warme Umgebung kann zu einem schnelleren Auslaufen der Lebensdauer.

Li-ion Batterien sind 100-265 Wh/kg bis heute (Quelle gibt nicht https://www.farasis-energy.com/en/independent-company-confirms-the-performance-of-farasis-energy-cells/). 
Treibstoff Energiedichte (spezifische Energie) 12.000 Wh/kg \cite{dalmia2022powering}
Weitere Batterien wie lithium-sulfur, lithium-air, und werden erforst Solid-state batterien vielversprechend


Es soll bewusst sein, dass die Herstellung der Lithium-Ionen-Batterien umweltschädlich durch Lithium-Produktion (was viel Wasser verbraucht, Leckagen gefährlich) und kostenintenisiv 
sein kann und Wartung \cite{dalmia2022powering}. 

Es gibt drei unterschiedliche Antriebskonfigurationen von elektrischen Flugzeugen: vollelektrisch, funktioniert nur auf der Batterie oder
 Brennstoffzelle als Energiequelle, turboelektrisch und hybrid-elektrisch ist die Mischung von konventionellen 
 Gasturbinentriebwerke mit Kerosin und Batterie oder Brennstoffzelle \cite{dahal2021techno}. In dieser Arbeit werden nur die vollelektrische
 Antriebe betrachtet, da nur sie fast zum 100 \% Reduktion bringen können.



Batterien werden bereits jetzt als sekundäre Leistungsquelle.(Schmidt?)
Ladeleistung ist für die Dauer der Ladung verantwortlich. 

BS ist kompatibler mit der Flugplanungen, aber benötigt mehreren Batterien für den Austausch, was die Logistik schwerer macht 
und die höhere Anschaffungskosten hat. Batterie sollen richtig und sicher gelagert werden. \cite{salucci2020optimal}

Batteriewechsel ist öfters vorkommende in der Literatur Ladung.

Durch schnellere Ladungen wird Lebensdauer der Batterien reduziert. Was mit sich bringt, dass die Batterien schneller ausgetauscht werden müssen
und mehr Kosten dadurch entsteht. (Quelle) Wobei die langsamen Laden ist für die Fluggesellschaften nicht rentabel sein kann, 
da wenn Flugzeug auf dem Boden steht verdienen Fluggesellschaften kein Geld.

Transportkapazität ist nicht so groß, wie bei konventionellen Flugzeugen und auch die Reichweite für Regionalflüge \cite{abrantes2024impact}.
Die Energieproduktion ist nicht emissionsfrei \cite{abrantes2024impact}, dafür muss mehr renewable Quellen hergestellt werden, 
wie Solar- und Windenergie. Wasserstoff braucht kryogene Lagerung, hat Entflammbarkeit und braucht 
Infrastrukturentwicklung.\cite{abrantes2024impact}

Batterieelektische Antriebe haben niederige spezifische Energie, was dazu führt, dass die größeren Batterien gebraucht werden. Damit steigt auch
die Masse vom Flugzeug

nachhaltigkeit nur von Energieerzeugungsweg abhängig

Sicherheit: Das größte Gefahr in batterieantrieben ist chemische Reaktion des "thermal runway", dafür werden stabile
Kühlungssysteme benötigt \cite{donckers2024electric}.

Konfiguration: ES-19 wurde zu einem anderen Typ ES-30 umgerüstet \cite{donckers2024electric}, jedoch diese Konfiguration basiert 
sich auf den hybriden Antrieb und schafft nicht der nötige Vergleich zwischen Brennstoff getriebenen und elektrischen Flugzeug

Batteriemanagementsystem in einem Flugzeug beeinhaltet solche Information, wie State of Health (SOH) gibt an der Unterschied zwischen anfangs Kapazität vorhanden ist und 
State of Charge (SoC) ist wie viel Prozent der verfügbaren Kapazität geladen werden kann. (umschreiben) \cite{donckers2024electric}