Am Flughafen beschäftigt eine Vielzahl an Akteure, die miteinander arbeiten. Durch die neuen Ansätze steht diesen Akteuren eine 
wichtige Aufgabe vor.
\textit{"Flughafen"}
Einer von den Beteiligten am Flughafen ist Flughafen selbst. 
Flughafen stellt die Infrastruktur, wie Terminals und Start- und Landebahnen zur Verfügung für die Fluggerät- und Passagierabfertugung, 
(das gilt als Kernfunktion), wofür die Nutzungsgebühren erhebt werden \cite{conrady2019luftverkehr}. Seite 180, falls ich Entgelte durchzählen will.
Zur Flughafen gehören außer Start- und Landebahnen, Rollwege, Vorfeld, Flugsteige, sowie die Infrastruktur für die Gepäckabfertigung und 
außerdem stellen Flughäfen eine intermodale Verknüpfung dar \cite{conrady2019luftverkehr}, d.h. die Anbindung an anderen Verkehrsmitteln wird hergestellt.

Flughäfen sind ein großer Teil der regionalen Wirtschaft (Quelle) was für Arbeitsstellen sorgt und ein Ausmaß an Lärm und Weltbelastungen mitbringt.
Deshalb verlangt/kassiert der Flughafen die Entgelte für Lärm und Emissionen Kap 4

Direkte Nutzer von Flughäfen sind die Importeure und Exporteure von Dienstleistungen und Waren \cite{schaar2010analysis}. 

\textit{"Fluggesellschaft"}
Fluggesellschaften sind die Dienstleister, die Infrastruktur von einem Flughafen für Abfertigung von Passagiere und Fracht nutzt. 
Sie haben das Ziel, wirtschaftlich stabil und wettbewerbsfähig zu bleiben. Für sie ist wichtig, die Betriebskosten 
eines Flughafen \cite{schaar2010analysis} (welche Erträge müssen die bezahlen). Niedrigere Erträge können durch nachhaltige 
alternative Konzepte erreicht werden, mehr dazu ist im Kapitel XX zu finden.

Es gibt auch weitere Beteiligten am Flughafen, wie Mitfliegende, Arbeitskräfte

Fluggesellschaften und Treibstoff-companies sind für sichere Betankung verantwortlich. Annex 14 (Doc 9137 Teil 8)
Sie sind auf Infrastruktureinrichtungen, wie Gepäckförderanlagen, Betankungsanlage angewiesen.

\textit{"Bodenverkehrsdienste"} S. 183 conrady
Bodenverkehrsdienste sind für die Abfertigung auf dem Boden verantwortlich. Zu ihrer Tätigkeiten gehören: 
Meistens werden die Bodenverkehrsdienste in Deutschland von den Flughäfen übernommen \cite{conrady2019luftverkehr}.
Sie sind auf die Infrastruktureinrichtungen, wie Gepäckförderanlagen, Betankungsanlage angewiesen \cite{conrady2019luftverkehr}.

Bodenverkehrsdienste sind auch für den Transport von Fracht, Post und Gepäck bis zum Flugzeug zusändig. \cite{mensen2013handbuch}
Flugzeugschlepper wird gebraucht, um das Flugzeug von der Parkposition auf den Rollfeld zu bewegen. Und durch eine Brücke können Reisende 
in die Flugzeug gelangen.

Laut OPS 1.175 "The number of ground staff is dependent upon the nature and the scale of operations"

OPS 1.1150 "Handling agent. An agency which performs on behalf of the operator some or all of the latter's functions
including receiving, loading, unloading, transferring or other processing of passengers or cargo;"


In dieser Arbeit werden nur die internationale und regionale Verkehrsflughäfen betrachtet. 
(Es bietet sonstige Serviceleistungen für die Passagiere, wie Parkplätze, Handel Dienstleistungen.)

"Zu ihnen zählen neben Luftverkehrsgesellschaften (LVG), Flugplatzbetreibern, Flugsicherungen, Luftfahrzeugherstellern,
Verladern, Reiseveranstaltern oder staatlichen
Institutionen auch sogenannte Bodenabfertigungs- oder Bodenverkehrsdienstleister
(Ground Handling Service Provider, GHSP)." 

Zu Systempartner am Flughafen zählen auch Luftfahrzeugherstellern, Reiseveranstaltern, staatliche Institutionen \cite{maertens2023neue}. 
Sie nehmen nicht direkt an der Flugzeugabfertigung bzw. an Betrieb am Vorfeld teil, deswegen werden sie weiter nicht betrachtet.


Auf die Kosten eingehen:

Kategorisierung der Flughäfen basiert sich nach Passagiermenge. Anhand dieser Kategorisierung für den Jahr 2023 war in Deutschland 
sieben große Gemeinschaftsflughäfen (einschließlich 2 Hubs) und 16 Regionalflughäfen (kleine und große). 
BAs können weniger überlastete Flughäfen anfliegen und entferne Bereiche, und
nur die geringe Bedarf abdecken.



Europäische Kommission Leitlinien für staatliche Beihilfe für Flughäfen und Luftverkehrsgesellschaften 2014/C 99/03
„Regionalflughafen“: einen Flughafen mit bis zu 3 Millionen Passagieren im Jahr;

„Betriebskosten“: die mit der Erbringung von Flughafendienstleistungen verbundenen Kosten eines Flughafens;
 dazu gehören Kostenkategorien wie Personalkosten, Kosten für fremdvergebene Dienstleistungen, Kommunikation, 
 Abfallentsorgung, Energie, Instandhaltung, Mieten und Verwaltung, jedoch weder Kapitalkosten, Marketingunterstützung 
 bzw. andere Anreize, die der Flughafen den Luftverkehrsgesellschaften bietet, noch Kosten für Aufgaben mit hoheitlichem Bezug;

 	
Der Bedarf an öffentlichen Mitteln zur Betriebskostenfinanzierung variiert unter den derzeitigen Marktbedingungen
 aufgrund der hohen Fixkosten in der Regel je nach Flughafengröße und ist normalerweise bei kleineren Flughäfen
  verhältnismäßig höher. Unter den derzeitigen Marktbedingungen können nach Auffassung der Kommission in Bezug auf
   die jeweilige finanzielle Tragfähigkeit nachstehende Kategorien von Flughäfen abgegrenzt werden:

a)Flughäfen mit bis zu 200 000 Passagieren im Jahr sind wohl nicht in der Lage, ihre Betriebskosten weitgehend selbst zu tragen;

b)Flughäfen mit 200 000 bis 700 000 Passagieren im Jahr sind wohl nicht in der Lage, einen erheblichen 
Teil ihrer Betriebskosten selbst zu tragen;

c)Flughäfen mit 700 000 bis 1 Million Passagieren im Jahr dürften im Allgemeinen in der Lage sein, einen 
größeren Teil ihrer Betriebskosten selbst zu tragen;

d)Flughäfen mit 1 bis 3 Millionen Passagieren im Jahr dürften im Durchschnitt in der Lage sein, ihre Betriebskosten überwiegend selbst zu tragen;

e)Flughäfen mit mehr als 3 Millionen Passagieren im Jahr erzielen in der Regel einen Betriebsgewinn und dürften 
in der Lage sein, ihre Betriebskosten zu decken.

Europaische Kommission definiert die „große Gemeinschaftsflughäfen“ mit über 10 Mio. Passagieren jährlich; 
„nationale Flughäfen“ mit 5 bis 10 Mio. Passagieren jährlich;
„große Regionalflughäfen“ mit 1 bis 5 Mio. Passagieren jährlich; und „kleine Regionalflughäfen“ mit weniger als 1 Mio. Passagieren jährlich.