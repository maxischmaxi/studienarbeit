In diesem Teil ist beschrieben, welche Flugzeugmodelle und -konfigurationen sind in näheren Zukunft in der Planung sind oder.

Durch die geringe Energiedichte in der Batterien, es ist zu erwarten, dass bis zum Jahr 2050 werden keine größere vollelektische Flugzeuge
existieren.

Positive Auswirkungen auf die Emission-Werte können bereits mit bestimmten Flugzeug- und Triebwerkskonfigurationen erreicht werden.
Zum Beispiel "Claire Liners" vom Bauhaus Luftfahrt e. V. München nicht nur aerodynamische Vorteile mitbringt, 
sondern auch Verringerung des Kraftstoffverbrauchs und somit Emissionen verringern.
Jedoch sind mehr Änderungen notwendig, um netto null \cite{CO2} Werte zu erreichen.

mögliche Konfigurationen:

hydrogen

Embraer hat eine Reihe von nachhaltigen Flugzeugen ENERGIA H2 GAS TURBINE E50-H2GT Dual Fuel Gas Turbine Propulsion
"• 100\% Hydrogen energy for typical routes
• SAF or JetA used for reserves and range extension
• Hydrogen storage customized according to operators’ specific
needs
• Rear mounted engines with optimized propeller design to
decrease noise"

Airbus ZeroE: Airbus hat im Jahr 2020 drei unterschiedlichen emissionsfrei ZEROe Konzepte vorgestellt. 
In allen Konzepten ist der Wasserstoff im Einsatz und alles haben Wasserstoffturbinen. Die Reichweite breitet sich ab über 1.850 - 
3700 km und die Anzahl beförderte Passagiere wird von 100 bis 200 geschätzt \cite{airbus_zea_concepts}.

Wrigth Spirit
Universal Hydrogen
ZeroAvia

E-Genius and 
Pipistrel Alpha Electro

Batterieelektische Flugzeuge
Ein vielversprechender Prototyp war die ES-19 von Heart Aerospace. Beförderung von 19 Passagiere über 400 km mit einem BA. 
Das Flugzeug war für die Regionalstrecken konzipiert und somit kann die geringe Nachfrage gedeckt werden und geringe Betriebs- und Wartungskosten wurden versprochen .
Jedoch zu dem Zeitpunkt wurde das Flugzeug auf ES-30 mit einem hybriden Antrieb umgerüstet.