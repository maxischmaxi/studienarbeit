\chapter{Anhang B. Ergebnisse}
\subsubsection{B1.1 Ergebnisse für Flugzeugvergleich}

\begin{table}[h]
    \centering
    \caption{Kostenübersicht für den Vergleich von Batterienatrieb mit SAF und konv.}
    \label{tab:kosten}
    \begin{tabular}{lrrr}
        \toprule
        \textbf{Kosten [EUR]} & \textbf{Referenzflugzeug} & \textbf{Batterieantrieb} & \textbf{SAF} \\
        \midrule
        Wartungskosten & 475,81 & 428,23 & 475,81 \\
        Crewkosten & 262,00 & 316,79 & 262,00 \\
        Treibstoffkosten & 252,20 & 177,84 & 349,35 \\
        Entgelte und Gebühren & 973,97 & 1549,03 & 973,97 \\
        Kapitalbezogene Kosten & 660,97 & 809,85 & 660,97 \\
        \midrule
        Gesamtkosten & 2624,95 & 3281,74 & 2722,10 \\
        \bottomrule
    \end{tabular}
\end{table}
Die Abkürzung \textit{WA} bedeutet Wasserstoffantrieb, \textit{BA} - Batterieantrieb.
\begin{table}[h]
    \caption{Kostenübersicht für 6000 km}
    %\centering
    \begin{tabular}{lcccc}
        \toprule
        \textbf{Kosten [EUR]} & \textbf{Referenzflugzeug} & \textbf{WA} & \textbf{WA 2050} & \textbf{SAF} \\
        \midrule
        Wartungskosten         & 3247,07  & 3441,89  &    -     & 3247,07  \\
        Crewkosten             & 2323,91  & 2541,05  &     -    & 2323,91  \\
        Treibstoffkosten       & 19500,36 & 38327,52 & 16493,40 & 30013,50 \\
        Entgelte und Gebühren  & 8809,59  & 8900,47  &     -    & 8809,59  \\
        Abschreibung           & 12767,01 & 14933,36 &     -    & 12767,01 \\
        Verzinsung             & 10213,61 & 11946,69 &     -    & 10213,61 \\
        Versicherung           & 1021,36  & 1194,67  &     -    & 1021,36  \\
        apitalbezogene Kosten & 24001,98 & 28074,72 &     -    & 24001,98 \\
        \midrule
        Betriebskosten gesamt  & 57882,91 & 81285,65 & 59451,53 & 68396,05 \\
        \bottomrule
    \end{tabular}
\end{table}

\subsubsection{B1.2 Ergebnisse für Betriebsszenarien}
Alle dargestellte Kosten sind in EUR angegeben.
\begin{table}[h]
    \centering
    \caption{Betriebskosten für die Szenarien}
    \begin{tabular}{lcccc}
        \toprule
        \textbf{Szenario} & \textbf{BA} & \textbf{WA} & \textbf{SAF} & \textbf{Gesamtkosten} \\
        \midrule
        Szenario I   & 1 385 683,36  & 3 121 131,50  & 7 100 877,92  & 11 607 692,80 \\
        Szenario II  & 692 841,68    & 6 845 640,00  & 4 549 855,11  & 12 088 336,55 \\
        Szenario III & 951 705,60    & 5 780 104,00  & 5 242 328,48  & 11 974 137,67 \\
        \bottomrule
    \end{tabular}
    \label{tab:betriebsszenarien}
\end{table}

\begin{table}[h]
   \centering
    \caption{Infrastrukturkosten nach Betriebsszenario und jährliche Abschreibungskosten}
    \begin{tabular}{lccc}
        \toprule
        \textbf{Szenario} & \textbf{Gesamt} & \textbf{Infrastruktur BA} & \textbf{Infrastruktur WA} \\
        \midrule
        Szenario I   & 28 068 315,58 & 12 864 480  & 15 203 835,58 \\
        Szenario II  & 35 716 010,65 & 6 494 740   & 29 221 270,65 \\
        Szenario III & 34 033 527,49 & 8 917 636   & 25 115 891,49 \\
        \midrule
        \multicolumn{4}{c}{\textbf{Jährliche Abschreibungskosten}} \\
        \midrule
        Szenario I   & 5 570 020,85  & 4 699 873,93  & 870 146,92  \\
        Szenario II  & 4 640 583,14  & 2 373 085,11  & 2 267 498,03 \\
        Szenario III & 5 117 392,72  & 3 257 504,34  & 1 859 888,38 \\
        \bottomrule
    \end{tabular}
    \label{tab:szenario_analyse}
\end{table}