\chapter{Anhang A. Ergänzende Formeln und Berechnungsgrundlagen}
\subsubsection{A1.1 genutzte Inflationsfaktoren}
Für die Ausrechnung der Inflationsfaktoren wurden die Daten des Statistischen Bundesamts \cite{destatis2025} verwendet. 
\begin{table}[h]
	\begin{center}
    \caption{Abgeleitete Inflationsfaktoren}
	\label{Inflationsfaktoren}
	\begin{tabular}{|l|c|}
		\hline
		\textbf{Jahren} & \textbf{Inflationsfaktor} \\ \hline
		1994 - 2024 & 1,71 \\ \hline
		2018 - 2024 & 1,216 \\ \hline
		2020 - 2024 & 1,193 \\ \hline
        2021 - 2024 & 1,157 \\ \hline
        2023 - 2024 & 1,022 \\ \hline
	\end{tabular}
    \end{center}
\end{table}

$VPI$ bezeichnet Verbraucherpreisindex. Beispiel die Ausrechnung eines Inflationsfaktors $k_{Infl}$:
\begin{equation}
	{k_{Infl}} = \frac{VPI_{2024}}{VPI_{2018}} \\
	\label{inflation}
 \end{equation}

\subsubsection{A1.2 Kapitalbezogene Formeln}
Die Formeln für kapitalbezogenen Kosten von AEA stammen aus \cite{minwoo2019analysis}.
Die Abschreibung $DEP$ ist anhand der Gesamtinvestitionen für das Flugzeug $TI$, der Auslastung des Flugzeugs $U$
und mit einen Verzinsprozentanteil von 5 \% \cite{scholz_design_evaluation_doc} berechnet.
Versicherungskosten $INS$ sind normalerweise anhand des Herstellerpreises berechnet. 
Es wird angenommen, dass der Herstellerpreis nicht stark
vom dem Kaufpreis abweicht und somit werden sich die Versicherungskosten auf die Gesamtinvestitionen $TI$
für das Flugzeug beziehen. Ein Versicherungsprozentanteil von 6 \% wird angenommen.

Die Formeln für die kapitalbezogenen Kosten von AEA stammen aus dieser Quelle.
Die Abschreibung wird auf Grundlage der Gesamtinvestitionen $TI$ für das Flugzeug sowie dessen Auslastung $U$
berechnet und mit einem Zinssatz von 5 \% verzinst \cite{scholz_design_evaluation_doc}.
Die Versicherungskosten werden in der Regel anhand des Herstellerpreises berechnet
und haben eine große Spanne.
Es wird angenommen, dass sich der Herstellerpreis nicht wesentlich vom Kaufpreis unterscheidet, 
sodass sich die Versicherungskosten auf die Gesamtinvestitionen für das Flugzeug beziehen. 
Nach AEA ist ein Versicherungsprozentanteil von 0,5 \% einberechnet \cite{scholz_design_evaluation_doc}.

\begin{equation}
	\begin{split}
	{DEP} = \frac{TI}{14 \cdot U} \\
	{INT} = {0,05} \cdot \frac{TI}{U} \\
	{INS} = {0,005} \cdot \frac{TI}{U} \\
	\label{kapitalkosten}
 \end{split}
\end{equation}

\subsubsection{A1.3 Entgelte und Gebühren}

In der Arbeit wurden folgende Flughafen-Entgelte mitberechnet: Entgelte für Abfertigung, 
Passagierentgelte, Entgelt Landung und Start
und zusätzliche passagierbezogene Entgelte. Die Passagierentgelte wurden im Fall Kurzstrecken angenommen,
dass die Passagiere innerhalb EU reisen, für Mittel- und Langstrecken wurde angenommen, 
Passagiere außer EU-Rahmen reisen.
Es muss beachtet werden, dass es eine Reihe anderen Entgelte an Flughäfen vorhanden, 
wie die sicherheits- oder emissionsabhängige Entgelte. Zur Reduktion der Komplexität in der Berechnung werden die jedoch nicht betrachtet.
Die Berechnung der Kosten erfolgte unter der Annahme, dass nur Passagiere berücksichtigt wurden, 
während die Fracht bei der Berechnung von Entgelten und Gebühren vernachlässigt wurde.

Die Formel der Flugsicherungsgebühr $(C_{FS})$ für jeweils An- und Abfluggebühr stammt aus \cite{dfs_flugsicherungsgebuehren} und ist in der Gleichung \eqref{Flugsicherung}
dargestellt. Im Jahr 2025 liegt der Wert $P_{FS}$ bei 380,71 Euro. 
Der $MTOW$ bezeichnet das Höchstabfluggewicht (Maximum Take-Off Weight) eines Flugzeugs. 
\begin{equation}
	{C_{FS} ~[\text{EUR}]} = (\frac{MTOW}{50})^{0,7} \cdot P_{FS} \\
	\label{Flugsicherung}
 \end{equation}
Die zugrunde liegenden Werte den Flughafenentgelten wurden aus den Daten von \cite{fraport2025entgelte} entnommen.
Außerdem bestehen bei alternativen Antrieben betriebliche Kosten, die durch Infrastrukturbedienung bedingt sind.
So werden bei BA-Flugzeugen die Kosten für Batteriewechsel zu Flughafenentgelten einkalkuliert. Aufgrund hoher Kosten für Batterien
und ausgewählten Batteriewechselsystems am Flughafen können die Fluggesellschaften die Anschaffungskosten nicht leisten. 
Zudem wird es schwer die Ladegeräte für unterschiedliche 
Batterien bereitzustellen. Aus diesem Grund wird ein Konzept vorgeschlagen, dass
Fluggesellschaften die Batterie bei Flughafengesellschaft leihen. Pro 1,5 Stundenflug wird ein Wert von 250 EUR dazugerechnet, da es durch
die Häufigkeit der Ladezyklen und Abschreibungsjahren (siehe Tab. \ref{BA_Infrastrukturtab}) plausibel erscheint.
Dieser Wert ermöglicht für den Flughafen Anschaffungskosten zu decken und unerwartete dazukommende Kosten zu bewältigen.
Pro Flugzeug wird von einem Wert in Höhe von 285 EUR ausgegangen \cite{guo2023infrastructure}.

 
