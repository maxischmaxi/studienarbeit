\chapter{Anhang A. Ergänzende Formeln und Berechnungsgrundlagen}
\subsubsection{A1.1 genutzte Inflationsfaktoren}
Für die Ausrechnung der Inflationsfaktoren wurden die Daten des Statistischen Bundesamts \cite{destatis2025} verwendet. 
\begin{table}[h]
	{\small \textit{Tabelle: Abgeleitete Inflationsfaktoren}}
	\begin{center}
 %   \caption{Abgeleitete Inflationsfaktoren}
	\label{Inflationsfaktoren}
	\begin{tabular}{|l|c|}
		\hline
		\textbf{Jahren} & \textbf{Inflationsfaktor} \\ \hline
		1994 - 2024 & 1,71 \\ \hline
		2018 - 2024 & 1,216 \\ \hline
		2020 - 2024 & 1,193 \\ \hline
        2021 - 2024 & 1,157 \\ \hline
        2023 - 2024 & 1,022 \\ \hline
	\end{tabular}
    \end{center}
\end{table}

$VPI$ bezeichnet Verbraucherpreisindex. Beispiel die Ausrechnung eines Inflationsfaktors $k_{Infl}$:
\begin{equation}
	{k_{Infl}} = \frac{VPI_{2024}}{VPI_{2018}} \\
%	\label{inflation}
 \end{equation}


 
