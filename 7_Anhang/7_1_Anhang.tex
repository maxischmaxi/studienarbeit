\chapter{Anhang A. Ergänzende Formeln und Berechnungsgrundlagen}
A1.1 genutzte Inflationsfaktoren
Für die Ausrechnung der Inflationsfaktoren wurden die Daten des Statistischen Bundesamts verwendet. 
\begin{table}[h]
	\begin{center}
    \caption{abgeleitete Inflationsfaktoren}
	\label{Inflationsfaktoren}
	\begin{tabular}{|l|c|}
		\hline
		Jahren & \textbf{Inflationsfaktor} \\ \hline
		1994 - 2024 & 1,71 \\ \hline
		2018 - 2024 & 1,216 \\ \hline
		2020 - 2024 & 1,193 \\ \hline
        2021 - 2024 & 1,157 \\ \hline
        2023 - 2024 & 1,022 \\ \hline
	\end{tabular}
    \end{center}
\end{table}

$VPI$ ist Verbraucherpreisindex. Beispiel die Ausrechnung eines Inflationsfaktors $k_{Infl}$:
\begin{equation}
	{k_{Infl}} = \frac{VPI_{2024}}{VPI_{2018}} \\
	\label{inflation}
 \end{equation}

A1.2 kostenbezogene Formeln
Die Formeln von AEA stammen aus \cite{minwoo2019analysis} oder scholz

A1.3 Entgelte und Gebühren

In der Arbeit wurden folgende Flughafen-Entgelte mitberechnet: Entgelte für Abfertigung, 
Passagierentgelte, Entgelt Landung und Start
und zusätzliche passagierbezogene Entgelte. Die Passagierentgelte wurden im Fall Kurzstrecken angenommen,
dass die Passagiere innerhalb EU reisen, für Mittel- und Langstrecken wurde angenommen, 
Passagiere außer EU-Rahmen reisen.
Es muss beachtet werden, dass es eine Reihe anderen Entgelte an Flughäfen vorhanden, 
wie die sicherheits- oder emissionsabhängige Entgelte. Zur Reduktion der Komplexität in der Berechnung werden die jedoch nicht betrachtet.
Die Berechnung der Kosten erfolgte unter der Annahme, dass nur Passagiere berücksichtigt wurden, 
während die Fracht bei der Berechnung von Entgelten und Gebühren vernachlässigt wurde.

Die Formel der Flugsicherungsgebühr $(C_{FS})$ für jeweils An- und Abfluggebühr stammt aus \cite{dfs_flugsicherungsgebuehren} und ist in der Gleichung \eqref{Flugsicherung}
dargestellt. Im Jahr 2025 liegt der Wert $P_{FS}$ bei 380,71 Euro. 
Der $MTOW$ bezeichnet das Höchstabfluggewicht (Maximum Take-Off Weight) eines Flugzeugs. 
\begin{equation}
	{C_{FS} ~[\text{EUR}]} = (\frac{MTOW}{50})^{0,7} \cdot P_{FS} \\
	\label{Flugsicherung}
 \end{equation}
Die zugrunde liegenden Werte den Flughafenentgelten wurden aus den Daten von \cite{fraport2025entgelte} entnommen.
Außerdem bestehen bei alternativen Antrieben betriebliche Kosten, die durch Infrastrukturbedienung bedingt sind.
So werden bei BA-Flugzeugen die Kosten für Batteriewechsel zu Flughafenentgelten einkalkuliert. Aufgrund hoher Kosten für Batterien
und ausgewählten Batteriewechselsystems am Flughafen können die Fluggesellschaften die Anschaffungskosten nicht leisten. 
Zudem wird es schwer die Ladegeräte für unterschiedliche 
Batterien bereitzustellen. Aus diesem Grund wird ein Konzept vorgeschlagen, dass
Fluggesellschaften die Batterie bei Flughafengesellschaft leihen. Pro 1,5 Stundenflug wird ein Wert von 250 EUR dazugerechnet, da es durch
die Häufigkeit der Ladezyklen und Abschreibungsjahren (siehe Tab. \ref{BA_Infrastrukturtab}) plausibel erscheint.
Dieser Wert ermöglicht für den Flughafen Anschaffungskosten zu decken und unerwartete dazukommende Kosten zu bewältigen.
Pro Flugzeug wird von einem Wert in Höhe von 285 EUR ausgegangen \cite{guo2023infrastructure}.

 
