\documentclass[
    12pt,
	paper = a4,
	BCOR=5mm, 			% BCOR je nach Seitenzahl setzen
	english, ngerman,	% Give every language used in the document, the main one as last
	twoside,
	numbers=noenddot,
	parskip	= half, 	% separate paragraphs with half a line
	cdgeometry = symmetric,
	cd = barcolor,
	chapterpage	= false,
	cdmath = false,
	slantedgreek=standard,
	captions=tableheading,
    headsepline,        %Kopfzeile_JB
    titlesignature = true
]{tudscrbook}

\counterwithout{footnote}{chapter} %Fußnoten_JB

\usepackage{tudscrsupervisor} % script for creatiing the task description
\usepackage{tudscrcolor}

\usepackage[utf8]{inputenc}
\usepackage[T1]{fontenc}

\usepackage[ngerman]{babel}
\usepackage{csquotes}

\usepackage{scrdate}   %HB:hier ist ein Fehler(isodate), mit scrdate klappt
\usepackage{blindtext}

\usepackage{setspace}
\usepackage{acronym}    %[nohyperlinks, printonlyused, withpage]
\usepackage{scrhack} 	% acronyms result in warning without this
\usepackage{multicol} 	% use multiple columns, used for the acronyms section

\usepackage{enumitem}\setlist{noitemsep} % used for the bullet points in the task section
\usepackage{microtype}	% better spacing

\usepackage{amsmath}
\usepackage{amsfonts}
\usepackage{bm}			% used for making things bold in equations
\usepackage[b]{esvect}

\usepackage{multirow}	% use tables with columns stretching over multiple rows
\usepackage{booktabs}
\setlength\heavyrulewidth{0.25ex}
\usepackage{longtable}

\usepackage{tabularx}   %Tabellen_JB
\usepackage{eurosym}    %Euro_JB

\usepackage[font=footnotesize, format=plain, labelfont=bf]{caption}  % 
\usepackage{subcaption}	% Packages to allow subfigures

\usepackage[bottom, hang, flushmargin]{footmisc}
\renewcommand\hangfootparindent{1em}
%\usepackage{fnpct}

% use modern bib package
\usepackage[
	backend=biber,
	url=false,
	doi=false,
	isbn=false,
	giveninits=true,
	style=numeric,
	citestyle=ieee,
	sorting=none]{biblatex}
\addbibresource{4_References/Quellen.bib} %HB: macht nicht selber eine Datei?

%\usepackage{xcolor}
\usepackage{listings}	% Package for displaying code
\definecolor{KeywordBlue}{cmyk}{0.88,0.77,0,0} %88,77,0,0
\definecolor{CommentGreen}{cmyk}{0.87,0.24,1.0,0.13} %87,24,100,13
\lstset{basicstyle=\scriptsize\ttfamily, language=C, commentstyle=\color{CommentGreen}, keywordstyle=\ttfamily\color{KeywordBlue}, backgroundcolor =\color[rgb]{0.95,0.95,0.95}, breaklines=true,literate={\\\%}{{\textcolor{black}{\\\%}}}1}

% some of the metadata for the pdf are defined in the title-file,
% as there are variables like author and title, whích would appear twice otherwise
% hyperref should always be the last package to be loaded
\usepackage[
colorlinks=true,
urlcolor=.,
citecolor=.,
linkcolor=.,        
pdfstartview=FitV,                          		
pdfdisplaydoctitle=true,
hyperfootnotes=false
]{hyperref}
\urlstyle{same}		% use the same font for URLs as for the text

%%% PARAMS %%%
\pdfminorversion=7	% creates pdfs in the version 1.7, which prevents a warning with the logo

% Allow for triple digit page numbers in the toc
\makeatletter
\renewcommand*\@pnumwidth{2.1em}
\renewcommand*\@tocrmarg{3.1em}
\makeatother

\KOMAoptions{toc=chapterentrydotfill} 	% Add dots in toc for chapters
\setstretch{1.3}						% Adds a bit of space between the lines
\frenchspacing							% Only a single space after a dot


% Parameters to reduce 'Orphans' and 'Widdows'
\clubpenalty 			= 9999
\widowpenalty 			= 9999
\displaywidowpenalty   	= 1602
\brokenpenalty			= 4999	% Parameter for word disjuction on a pagebreak
\pretolerance			= 1100	% Parameter for difference from choosen format
\tolerance 				= 100 	% Parameter for difference from choosen format

% Less coservative parameters for floating objects in LaTeX
% An overview can be found in the book
% The Latex Companions Chapter 6.1
% A good start is
% http://robjhyndman.com/researchtips/latex-floats/

\setcounter{topnumber}{2}
\setcounter{bottomnumber}{2}
\setcounter{totalnumber}{4}
\renewcommand{\topfraction}{0.85}%0.85
\renewcommand{\bottomfraction}{0.85}
\renewcommand{\textfraction}{0.15}
\renewcommand{\floatpagefraction}{0.7}
\renewcommand{\textfraction}{0.1}
\setlength{\floatsep}{5pt plus 2pt minus 2pt}
%\setlength{\textfloatsep}{15pt plus 2pt minus 2pt}
%\setlength{\intextsep}{5pt plus 2pt minus 2pt}





%ab hier beginnt das Dokument
\usepackage{graphicx} % Required for inserting images
\usepackage[version=4]{mhchem}


\begin{document}



%\maketitle

%\title{Studienarbeit}
%\author{Henriieta Bohdanova}
%\date{October 2024}
\iflanguage{ngerman}
{	
	\faculty{Fakultät Verkehrswissenschaften \glqq Friedrich List\grqq}
	\institute{Institut für Luftfahrt und Logistik}
	%\headlogo{logo/IFL Logo.jpeg}
}
{	
	\faculty{Fakultät Verkehrswissenschaften \glqq Friedrich List\grqq}
	\institute{Institut für Luftfahrt und Logistik}
	%\headlogo{logo/IFL Logo.jpeg}
}

\iflanguage{ngerman}{
	\newcommand*{\Title}{Stakeholder-Kostenanalyse an Flughäfen bei Einführung neuer Luftfahrzeugantriebe}
	\newcommand*{\Author}{Henriieta Bohdanova}
}{
	\newcommand*{\Title}{The Title of the work}
	\newcommand*{\Author}{Forename Surname}
}
\date{10.12.2024}	

\title{\Title}


\author{\Author}
\dateofbirth{28.06.1998}
\placeofbirth{Kyiv}
%\matriculationnumber{4755143}

% TODO adapt to type of work
% All possible values can be looked up in the documentation on page 38:
% https://ftp.tu-chemnitz.de/pub/tex/macros/latex/contrib/tudscr/doc/tudscr.pdf

%\subject{bachelor}
%\graduation[B.Sc.]{Bachelor of Science}
%\subject{master}
%\graduation[M.Sc.]{Master of Science}
\subject{student}
%\graduation[M.Sc.]{Master of Science}

\supervisor{Dipl.-Ing. Edgar Böttcher}

\referee{Prof. Dr.-Ing. habil. Hartmut Fricke}

% TODO: Add subject and if needed keywords
% Add title and author to pdf-meta-data
\hypersetup{
	pdftitle    = {\Title},
	pdfsubject  = {},
	pdfauthor   = {\Author},
	pdfkeywords = {}
}

%\makecover

\maketitle

% Change properties for all other pages that use a these information. No institute to avoid three lines and a black instead of a blue logo.
\iflanguage{ngerman}
{	
    \faculty{Fakultät Verkehrswissenschaften \glqq Friedrich List\grqq}
	\institute{Institut für Luftfahrt und Logistik}
	%\headlogo{logo/IFL Logo.jpeg}
}
{	
	\faculty{Fakultät Verkehrswissenschaften \glqq Friedrich List\grqq}
	\institute{Institut für Luftfahrt und Logistik}
	%\headlogo{logo/IFL Logo.jpeg}
}
%\maketitle

\chapter*{Thesen zur Arbeit}
\label{ch:Thesen}
\thispagestyle{empty}

\begin{enumerate}
    \item Die Einführung innovativer Antriebe führt zu höheren Betriebskosten einer Fluggesellschaft im Vergleich zu herkömmlichen
    Jettriebwerken, da andere technische Anforderungen und zusätzlich neue Abfertigungsprozesse notwendig sind.
    %\item These 2: Die Verteilung alternativen Antrieben einer Flotte kann einen großen Einfluss auf die Infrastrukturkosten eines Flughafens haben.
    %\item These 3: Die Infrastrukturkosten eines Flughafenbetreibers bleiben über die Jahre konstant, 
    %da die einmaligen Investitionen über die Abschreibungsdauer verteilt werden...
    %
    %\item Die Wahl der Flottenzusammensetzung mit innovativen Antriebstechnologien in verschiedenen 
    %Betriebsstrategien beeinflusst signifikant die Betriebskosten von Fluggesellschaften. 
    %Interessanterweise könnte in Szenarien mit den größten Betriebskosten (z. B. bei langen Strecken) die Infrastrukturkosten geringer ausfallen, 
    %da manche innovative Antriebe weniger spezifische oder kostspielige Infrastruktur erfordern.
    \item Die Wahl der Flottenzusammensetzung mit innovativen Antriebstechnologien in verschiedenen 
    Betriebsstrategien beeinflusst die Betriebskosten von Fluggesellschaften, wobei Szenarien mit höheren 
    Betriebskosten geringere Infrastrukturkosten erfordern.
    \item Die Abschreibungsmethode bewirkt, dass trotz hoher Investitionskosten die jährlichen Kosten belastbarer 
    und niedriger sind, während bei geringeren Anschaffungskosten eine ineffiziente Kostenverteilung zu 
    höheren jährlichen Belastungen führen kann.
\end{enumerate}

%These 4: Die Infrastrukturkosten sind schwer vorherzusagen, da unerwartete Reparaturen 
%Am Ende der Abschreibungsperiode müssen erhebliche Investitionen getätigt werden, um die Infrastruktur zu erneuern, 
%was zu einem sprunghaften Anstieg der Infrastrukturkosten nach der Abschreibung führt.

%These 5: Obwohl bestimmte Szenarien höhere Anschaffungskosten aufweisen, 
%    wird der finanzielle Effekt dieser höheren Investitionen durch die längere Abschreibungsdauer relativiert, 
%    sodass die jährlichen Belastungen vergleichbar oder sogar geringer bleiben.

\iflanguage{ngerman}{
	\chapter{Abkürzungen, Einheiten und Symbole}}{
	\chapter{Symbols and Acronyms}
}
\label{ch:acronyms}
%
\iflanguage{ngerman}{
	\section*{Abkürzungen}}{
	\section*{Acronyms}
}
%
\label{sec:Abkürzungen}

\begin{acronym}[6LoWPAN]

	\acro{CORSIA}{Carbon Offsetting and Reduction Scheme for International Aviation}
    \acro{DOC}{Direct Operating Costs}
	\acro{EU}{Europäische Union}
	\acro{IOC}{Indirect Operating Costs}
	\acro{TOC}{Total Operating Costs}
	\acro{ETS}{Total Operating Costs}
	\acro{ICAO}{Total Operating Costs}
	\acro{IATA}{Total Operating Costs}
	\acro{SAF}{Sustainable Aviation Fuel}
	\acro{HEFA}{Hydroprocessed Esters and Fatty Acids}
	\acro{BA}{Batterie-Antrieb}

\end{acronym}


\iflanguage{ngerman}{
	\section*{Einheiten}}{
	\section*{Units}
    }

\label{sec:Einheiten}
%
\begin{acronym}[6LoWPAN]


    \acro{bar}[$bar$]{Bar}
	\acro{km}[$km$]{Kilometer}
	\acro{EUR}[$EUR$]{Euro}
	\acro{USD}[$USD$]{Euro}
	\acro{m3}[$m^3$]{Kubikmeter}
	\acro{t}[$t$]{Tonnen}
	\acro{kg}[$kg$]{Kilogramm}

\end{acronym}


\iflanguage{ngerman}{
	\section*{Symbole}}{
	\section*{Symbols}
}

\label{sec:Symbole}
%Formelzeichen?
\begin{acronym}[6LoWPAN]

    \acro{CO2}[$CO_2$]{Kohlenstoffdioxid}
	\acro{LH2}[$LH_2$]{flüssiger Wasserstoff}
	\acro{O2}[$O_2$]{Ozon}
\end{acronym}
%\chapter{Einleitung}
\label{ch:Einleitung}



Laut Verordnung (EU) 2021 soll die EU zum Jahr 2050 klimaneutral sein, aber auch zum Jahr 2030 sollen die Treibhauseffekte um mindestens 55\%
im Vergleich zum Jahr 1990 reduziert werden. 

Ob nachhaltige Alternativen die Emissionswerte mindern oder vermeiden können, ist derzeit ein begehrtes Thema. 

Diese Arbeit widmet sich dem Thema nachhaltige Antriebe, nämlich im Fokus stehen nachhaltige Kraftstoffe (SAF), Wassertreibstoff und 
batterieelektrische Antriebe.
 
Im Jahr 2023 ermittelte Umweltbundesamt, dass die Treibhausgase im Vergleich zum Vorjahr um mehr als zehn Prozent gesunken sind.
Nichtsdestotrotz wurden allein in Deutschland im Jahr 2023 673 Mio. Tonnen Treibhausgase freigesetzt \cite{bundesregierung}.

\ce{CO2} Emissionen sind als eine Ursache für den Klimawandel gesehen. Luftverkehr hat auch seine Rolle in diesen Auswirkungen. 
Etwa 2,5\% von ganzen anthropogenen \ce{CO2} Emissionen weltweit werden vom Luftverkehr
durch die Treibstoffverbrennung verursacht \cite{conrady2019luftverkehr}.


Wenn die Emissionen weiter mit der Wachstumsrate (2002-2022) von ca.1,8\% (Daten aus: Worldatabank https://data.worldbank.org/indicator/EN.GHG.ALL.MT.CE.AR5) 
jährlich erhöhen werden, erreichen die Treibhausgase im Jahr 2050 den Wert von 88 Gt ohne Berücksichtigung von Landnutzung, 
Landnutzungsänderungen und Forstwirtschaft. (64\% Steigung?)

Neben dem Kohlendioxid \ce{CO2} und Wasserstoff \ce{H2O} entstehen bei der Verbrennung des Treibstoffs andere Nebenprodukte, die 
das Klima beeinflussen, wie Stickoxide \ce{NOx}, die für Ozonbildung in der Stratosphäre verantwortlich sind \cite{conrady2019luftverkehr}.

Mittlerweile sind viele Betriebe verpflichtet auf die Umweltneutralität zu achten. Die Flughäfen stellen ihre Bodendienste auf elektrische Antriebe um.

Durch die neuartige Konfiguration und alternative Kraftstoffe und Antriebe existiert die Möglichkeit die unnötige Emissionen zu vermeiden.
Alternative Antriebe, wie Batterie, Wassertreibstoff oder Sustainable Aviation Fuel (SAF) weisen unter nachhaltigen Produktion und Logistik kein Einfluss 
auf der Umwelt auf und somit helfen die Emissionen zu reduzieren. 


Die Frage, welche Kosten für die alle drei Alternativen durch diese Einführung und den Betrieb entstehen, wurde bislang noch nicht systematisch
untersucht. In der wissenschaftlichen Arbeiten sind die bereits getrennte Kostenberechnungen für die einzelne Alternative, meistens für 
elektrische oder wasserstoffbetriebene Flugzeuge, zu finden.

Jedoch eine zusammenfassende Berechnung der Kosten für alle drei Alternativen wurde bis jetzt nicht erforscht.
Dafür interessant anzuschauen, wie sich konventionelle Kraftstoff-Flugzeugen und neuartigen Antriebe unterscheiden 
und welche Veränderungen der Betrieb-, Infrastruktur- und Ausbildungskosten durch alternativen Antriebe entstehen.

Dieses Thema ist für die Praxis relevant, weil alle Fluggesellschaften bestimmten Kriterien unterliegen. 
Geprägt von strengeren und wachsenden Maßnahmen in Bezug auf die Treibhausgase, brauchen die Betriebe neue Technologien, um die  
höhere \ce{CO2} Ausstoß und damit verbunden höheren Kosten zu vermeiden.
Es soll untersucht werden, ob die nachhaltigen Antriebe eine Möglichkeit haben kostengünstig im Markt zu gelangen und Wettbewerb durchstehen oder 
sogar als Ersatz für die konventionellen Kraftstoffe, wie Kerosin, dienen können.

Aufbau der Arbeit: Im Rahmen dieser Studienarbeit...
Kapitel 2 stellt die relevanten Grundlagen zur weiteren Forschung dar, wie Stakeholder am Flughafen und deren Teil an der Abfertigung eines Flugzeugs,
die Betriebskosten, gesetzliche Einflüsse auf der Luftverkehr sowie die zukünftigen Flugzeugkonfigurationen mit neuer Antriebstechnologien.
Darauf aufbauend werden im Kapitel 3 die Methodik für die Kostenberechnung und betriebliche Szenarien für einen Flughafen definiert, 
als auch die getroffenen Annahmen erörtert.
Kapitel 4 begebt sich um die Auswertung der Kostenanalyse für die ... und dazugehörige Sensitivitätsanalyse.

Kapitel 5 enthält eine abschließende Zusammenfassung und einen Ausblick auf die Unsicherheiten in der Arbeit 
und mögliche Richtung für die weiterführende Forschungsarbeiten.


%\section{Introduction}
\chapter{Relevante Grundlagen und Überblick über alternative Antriebe}
\label{ch:Relevante Grundlagen und Überblick über alternative Antriebe}
Für die Analyse der Forschungsfrage es ist wichtig die zentralen theoretischen Begriffe zu definieren. 
Das Kapitel \ref{s:Bodenabfertigung eines Luftfahrzeugs} stellt die Grundlagen der Flugzeugabfertigung und Definition
der beteiligten Stakeholder am Flughafen dar. Zunächst beschäftigt sich das Kapitel \ref{s:Kosten}
mit den bedeutenden Informationen zu Kosten und Emission-Regulierungsinitiativen am Flughafen. 
Anschließend werden im Teil \ref{s:Neuartige Antriebe}
die neuartigen alternativen Antriebe und Konzepte und Flugzeugmodelle mit diesen Antrieben vorgestellt.

\section{Stakeholder am Flughafen}
\label{s:Stakeholder am Flughafen}
%
\begin{figure}[h]
	\centering
	\includegraphics[width=0.8\linewidth]{Bilder/Stakeholder.png}
	\caption[Relevante Stakeholder am Flughafen]{Relevante Stakeholder am Flughafen}
	\label{stakeholder}
\end{figure}

Am Flughafen ist eine Vielzahl an Stakeholdern beschäftigt, die miteinander agieren. 
Durch die neuen Luftfahrzeugantriebe steht diesen Akteuren eine schwierige Aufgabe bevor. 
Gute Zusammenarbeit der Stakeholder fördert die Pünktlichkeit der Abfertigung und 
hilft Verspätungen zu vermeiden \cite{schmidt2016challenges}.
%
\textbf{Flughafen} \\
Einer der Stakeholder am Flughafen ist der Flughafenbetreiber selbst. 
Der Flughafen stellt die Fluggerät- und Passagierabfertigungs-Infrastruktur wie bspw. 
Terminals oder Start- und Landebahnen zur Verfügung (welche als Kernfunktionen gelten), 
wofür Nutzungsgebühren erhoben werden \cite{conrady2019luftverkehr}. %Seite 180, falls ich Entgelte durchzählen will.

Zum Flughafen gehören außer Start- und Landebahnen unter anderem 
Rollwege, Vorfeld, Flugsteige, sowie die Infrastruktur für Gepäckabfertigung. 
Darüber hinaus stellen Flughäfen eine intermodale Verknüpfung dar \cite{conrady2019luftverkehr}. %d.h. die Anbindung an anderen Verkehrsmitteln wird hergestellt.
Direkte Nutzer von Flughäfen sind Im- und Exporteure von Dienstleistungen und Waren \cite{schaar2010analysis}. 

Flughäfen sind ein großer Teil der regionalen Wirtschaft \cite{schaar2010analysis} 
und sorgen für eine Vielzahl von Arbeitsstellen. 
Dennoch verursachen sie ein Ausmaß an Lärm und Umweltbelastungen, 
welche durch die Emissionen der Flugzeuge entstehen.
Demnach verlangt der Flughafen hierfür ebenfalls Entgelte. %Kap 4

Für die Entwicklung der Infrastruktur und Begleichung der Betriebskosten müssen Flughäfen 
gelegentlich finanzielle Unterstützung aus anderen Quellen, wie staatlicher Subventionen, 
in Anspruch nehmen \cite{schaar2010analysis}.
Die Europäische Kommission besagt, dass Flughäfen mit einem Passagieraufkommen von über 3 Millionen 
Passagiere jährlich in der Lage sind, ihre Betriebskosten selbst durch Gewinn zu decken.
\footnote{"Leitlinien für staatliche Beihilfe für Flughäfen und Luftverkehrsgesellschaften" 2014/C 99/03}
%Und die kleineren Flughäfen durch weniger Betrieb mehr auf die Hilfe angewiesen.
Eine Kategorisierung der Flughäfen basiert auf der Passagiermenge. 
Der Europäischen Kommission nach werden die Flughäfen nach jährlichem Passagieraufkommen folgend unterteilt: 
\begin{itemize}
    \item große Gemeinschaftsflughäfen > 10 Mio. Passagieren;
    \item nationale Flüge mit 5 bis 10 Mio. Passagieren;
    \item große Regionalflughäfen mit 1 bis 5 Mio. Passagieren;
    \item kleine Regionalflughäfen < 1 Mio. Passagieren.
\end{itemize}
Aufgrund dieser Kategorisierung in dem Jahr 2023 gab es in Deutschland 
sieben große Gemeinschaftsflughäfen, einschließlich zwei Hubs, und 16 Regionalflughäfen.\footnote{Die Daten stammen aus dem Statistischem Bericht, "Luftverkehr auf Hauptverkehrsflughäfen 2023"}
Ein Hub ausmacht ein großer Flughafen mit mächtigem Anteil an Umsteigeverkehr.\\

\textbf{Fluggesellschaft} \\
Fluggesellschaften sind Dienstleister, welche die Infrastruktur eines Flughafens für die 
Abfertigung von Passagieren und Fracht nutzen. 
Sie sind gewinnorientiert und haben das Ziel wettbewerbsfähig zu bleiben. 
Für eine Fluggesellschaft ist von Relevanz, wie hoch die Betriebskosten (Erträge)
sind, die der Flughafen verlangt \cite{schaar2010analysis}. 
Die Erträge unterscheiden sich sowohl je nach Flughafengröße und -strategie, als auch im Flugzeugtyp.\\
%
%Fluggesellschaften und Treibstoff-Firmen sind für die sichere Betankung verantwortlich. Quelle: Annex 14 (Doc 9137 Teil 8)
\textbf{Bodenverkehrsdienste}\\ %S. 183 conrady

Bodenverkehrsdienste sind für die Abfertigung der Flugzeuge auf dem Boden zuständig.
Nach Conrady \cite{conrady2019luftverkehr} gehört zu ihren Tätigkeiten außerdem:  
die Fluggastabfertigung, administrative Abfertigung sowie Transportdienste.
Sie sind auf Infrastruktureinrichtungen wie Gepäckförderanlagen und Betankungsanlagen 
und weitere Grundausstattung am Vorfeld angewiesen. 
Die Abfertigung kann entweder von einer Fluggesellschaft, einem Flughafen oder einem 
unabhängigen Dienstleister durchgeführt werden. 
Meistens werden die Bodenverkehrsdienste in Deutschland von den Flughäfen übernommen.\\ %oder die externen Firmen (Dritte) werden engagiert.
%
Bodenverkehrsdienste sind auch für den Transport von Fracht, Post und Gepäck 
bis zum Flugzeug zuständig \cite{mensen2013handbuch}.\\
Zu weiteren Vorfelddiensten gehören Betankungs- und Reinigungsdienste.
Betankungsdienste führen nicht nur die Be- und Entladung sowie Lagerung durch, 
sie sind sondern auch für andere Flüssigkeiten (wie z.B. Öl) zuständig.
Wartungsdienste führen die routinemäßige Kontrolle der Flugzeuge vor den Flügen (Line Maintenance) durch.
Reinigungsdienste und der Flugzeugservice sind für die Reinigung eines Flugzeugs 
von Innen und Außen, Wasserservice, die Klimaanlagen in der Kabine und die Enteisung verantwortlich.
%
%OPS 1.1150 "Handling agent. An agency which performs on behalf of the operator some or all of the latter's functions
%including receiving, loading, unloading, transferring or other processing of passengers or cargo;"
%
%Von der Fluggesellschaft werden Handling Agents eingestellt, der die ganze Abfertigung und Kommunikation zwischen Beteiligten am Vorfeld
%kontrollieren.
%In dieser Arbeit werden nur die internationale und regionale Verkehrsflughäfen betrachtet. 
%(Es bietet sonstige Serviceleistungen für die Passagiere, wie Parkplätze, Handel Dienstleistungen.)

Zu den Systempartnern (Stakeholdern) am Flughafen zählen ebenfalls Luftfahrzeughersteller, Flugsicherungen, 
Reiseveranstalter, staatliche Institutionen \cite{maertens2023neue}
sowie Beteiligte wie Passagiere, Arbeitskräfte und Passagierdienstleister. 
Sie nehmen nicht direkt an der Flugzeugabfertigung bzw. am Betrieb am Vorfeld teil, 
deswegen stehen sie nicht im Fokus dieser Arbeit.
Analog hierzu wird die Flugsicherung aufgrund unveränderter Umstände (Bedingungen) 
durch alternative Antriebe nicht betrachtet. 
Die Arbeit wird sich auf die Betriebskosten einer Fluggesellschaft 
und Infrastrukturkosten des Flughafens fokussieren.
%
%Auf die Kosten eingehen:
%Laut OPS 1.175 %"The number of ground staff is dependent upon the nature and the scale of operations"
%Anzahl der benötigten Bodenmitarbeiter ist von dem Maßstab der Operationen am Flughafen anhängig.
%
%BAs können weniger überlastete Flughäfen anfliegen und entferne Bereiche, und
%nur die geringe Bedarf abdecken.
%Europäische Kommission Leitlinien für staatliche Beihilfe für Flughäfen und Luftverkehrsgesellschaften 2014/C 99/03
%
%Als Regionalflughafen definiert E Kommissionen einen Flughafen mit bis zu 3 Millionen Passagieren im Jahr.
%Flughäfen mit mehr als eine Million Passagieren im Jahr decken überwiegend ihre Betriebskosten selbst. %egal?
%
%„Betriebskosten“: die mit der Erbringung von Flughafendienstleistungen verbundenen Kosten eines Flughafens;
% dazu gehören Kostenkategorien wie Personalkosten, Kosten für fremdvergebene Dienstleistungen, Kommunikation, 
% Abfallentsorgung, Energie, Instandhaltung, Mieten und Verwaltung, jedoch weder Kapitalkosten, Marketingunterstützung 
% bzw. andere Anreize, die der Flughafen den Luftverkehrsgesellschaften bietet, noch Kosten für Aufgaben mit hoheitlichem Bezug;
%
%
%Der Bedarf an öffentlichen Mitteln zur Betriebskostenfinanzierung variiert unter den derzeitigen Marktbedingungen
% aufgrund der hohen Fixkosten in der Regel je nach Flughafengröße und ist normalerweise bei kleineren Flughäfen
% verhältnismäßig höher. Unter den derzeitigen Marktbedingungen können nach Auffassung der Kommission in Bezug auf
%   die jeweilige finanzielle Tragfähigkeit nachstehende Kategorien von Flughäfen abgegrenzt werden:
%
%d)Flughäfen mit 1 bis 3 Millionen Passagieren im Jahr dürften im Durchschnitt in der Lage sein, ihre Betriebskosten überwiegend selbst zu tragen;
%
%e)Flughäfen mit mehr als 3 Millionen Passagieren im Jahr erzielen in der Regel einen Betriebsgewinn und dürften 
%in der Lage sein, ihre Betriebskosten zu decken.
\section{Bodenabfertigung eines Luftfahrzeugs}
\label{s:Bodenabfertigung eines Luftfahrzeugs}

Zur Veranschaulichung der Änderungen an der Infrastruktur am Flughafen, 
welche durch neuartige Antriebe vorgenommen werden müssen, ist es notwendig 
wichtige Begriffe einer Abfertigung des konventionellen Flugzeugs hervorzuheben. 
Unter konventionellen Luftfahrzeugen sind die zu verstehen,
die mit fossilen Treibstoffen, wie Kerosin auf Ölbasis, betrieben werden. 
Der Fokus wird auf die gewerblichen Passagierflugzeuge gelegt,
da die Abfertigung von Passagieren besonders strenge Sicherheitsmaßnahmen erfordert. %oder dass sie einen erhebliche Teil an zivile Luftverkehr ausmachen.

Die Blockzeit setzt sich aus der Zeit vom Beginn der Bewegung von der Parkposition 
bis zum Ende der Bewegung zur Parkposition, einschließlich der Flugzeit, zusammen.
An der Parkposition des Flughafens werden die Triebwerke ausgeschaltet 
und der Ablauf eines Turnarounds beginnt. 
Mensen \cite{mensen2013handbuch} definiert den Turnaround, 
wie die Abfertigung der Flüge, die zeitnah zusammen liegen. % satz komisch
Bei einem Turnaround wird das Luftfahrzeug durch viele Akteure am Flughafen, 
wie Flugplatzbetreiber, Fluggesellschaft und Dritte, für den nächsten Flug vorbereitet \cite{mensen2013handbuch}. 
Es muss ausgeladen, kontrolliert, gereinigt, anschließend versorgt 
und für den nächsten Flug beladen werden.\\

Die Abbildung \ref{abfertigung} stellt die Abfertigung eines Flugzeugs an der Parkposition dar.
Nach ICAO Doc 9157 besteht Abfertigung eines Passagierflugzeugs 
insgesamt aus Passagier-, Gepäck- und Frachtabfertigung, Sanitärservice, Wasserbetankung, 
Gepäckabfertigung, Betankung, Stromversorgung, Startluft, Flugzeugschleppen, 
Bordküchenservice, Wartungsservice sowie Bereitstellung einer Klimaanlage und Sauerstoff,
wie in der Abbildung dargestellt. Durch neuartige Antriebe kann es aufgrund anderer 
technischer Grundlagen zu Änderungen in diesen Prozessen kommen.
Laut EU-OPS 1.305 darf das Luftfahrzeug aus Sicherheitsgründen erst betankt werden, 
wenn sich keine Passagiere an Bord befinden. 
%
%Das Flugzeug wird an ein Hilfstriebwerk (auxiliary power unit - APU) angeschlossen \cite{mensen2013handbuch}. 
%Die APU liefert Strom, wenn die Haupttriebwerke nicht laufen (quelle: [Annex 14. Doc 9137 Part 8]).
%Parallel werden Fracht und sonstige Gepäckeinheiten mit dem Hubwagen abgeladen und mit Transporthängern zur Sortieranlage 
%im Terminal gebracht \cite{mensen2013handbuch}. Im Falle, das die Parkposition direkt am Flughafen ist, 
%können Passagiere direkt über die Treppe oder Fluggastbrücke zum Terminalgebäude gelangen. 
%Wenn die Parkposition am Vorfeld liegt, muss auf einen Bus zurückgegriffen werden. 

\begin{figure}[h]
	\centering
	\includegraphics[width=0.8\linewidth]{Bilder/A321_Abfertigung.png}
	\caption[Abfertigung]{Abfertigung eines A321 \cite{airbus2022a321} mit eigenem Hinweis}
	\label{abfertigung}
\end{figure}

Je nach Flugdistanz und Flugzeuggröße kann es zu unterschiedlichen Abfertigungszeiten kommen. 
Bei kleineren Flugzeugen ist die Abfertigungszeit kürzer, als die einer größeren Maschine. 
In Bezug auf die Transportdistanz wird nach Kurz- (ca.2 Stunden oder bis 1000 km) 
und Mittelstreckenflügen (bis 3,5 Stunden oder bis 3000 km) und 
Langstreckenflügen (ab 3,5 Stunden und ab 3000 km) unterschieden \cite{mensen2013handbuch}.
Die Definition von Distanzen variiert teils erheblich, 
z.B. definiert der Flughafen Frankfurt Langstrecken ab 6000 km.% diese Werte werden auch im Kapitel \ref{s:Betriebsszenarien} genutzt.

\textit{Konventionelle Treibstoffe}\\

Zur Zeit werden Treibstoffe auf Basis fossiler Energie, wie Öl, genutzt. 
Die Ölpreise sind sehr instabil. %wenn du quelle findest is okay
Um Schub zu erzeugen, wird der Treibstoff in der Gasturbine verbrannt, 
wodurch mechanische Leistung erzeugt wird und über eine Welle den 
Propeller oder das Strahltriebwerk antreibt. 
Durch die Verbrennung des Treibstoffs entstehen Abgase, wie auf der Abbildung

whereas biofuels produced at stand-alone facilities are
moved by rail or barge/tanker (large volumes) or truck (small volumes)
https://www.osti.gov/servlets/purl/2440801
A tank farm
comprises multiple interconnected pieces of equipment designed to safely receive, store, and
dispense fuel to a hydrant system or truck delivery to aircraft. Although not an all-inclusive list, a
tank farm consists of tanks; pipeline interconnection; equipment to control the flow of fuel and
vapors; meters to measure the volume of fuels into the tank farm and out to aircraft; filters to
remove contaminants; pumps to move fuel throughout the system; safety equipment to prevent,
detect, and contain leaks throughout the system; off-loading racks to fill fuel trucks; and hydrant
systems—underground pipes and hydrants

%\section{Betrieb-, Infrastruktur- und Ausbildungskosten}
\label{s:Kosten}
In diesem Unterkapitel werden die Kostenstrukturen vorgestellt, 
wobei bei Betriebskosten auf die Kosten der Fluggesellschaft
eingegangen wird und bei der Infrastruktur, aufgrund der dazugehörigen Kapitalkosten, auf die Flughäfen.
%
\subsection{Betriebskosten einer Fluggesellschaft}

Die Betriebskosten einer Flugzeugabfertigung werden auf Direct Operating Costs ($DOC$) und Indirect Operating Costs 
($IOC$) aufgeteilt, welche auch Einzel- und\\ Gemeinkosten genannt werden \cite{conrady2019luftverkehr}. 
Nach Mensen \cite{mensen2013handbuch} können $DOC$ einem bestimmten Flugzeug oder einer Strecke zugeordnet 
werden und normalerweise als DOC pro Flugstunde, pro Kilometer, pro Passagierkilometer oder pro Blockstunde 
berechnet werden. IOC werden nicht direkt einem Flug zugewiesen, sondern fallen für den gesamten Betrieb an, 
z.B. für zeitabhängige Instandhaltungs-, Verwaltungs- und Infrastrukturkosten. 
Nach der Beschäftigungsabhängigkeit werden die Kosten auf fixe und variable Kosten aufgeteilt. 
Fixe Kosten stehen unabhängig zu dem Betrieb (z.B. Kapitalkosten, Versicherung, Personalkosten), 
die variablen Kosten hingegen ändern sich Bezug auf die Beschäftigung.
%
Die Kostenstruktur einer Fluggesellschaft kann mit der Abbildung \ref{doc} veranschaulicht werden.
%
\begin{figure}[h]
	\centering
	\includegraphics[width=0.9\linewidth]{Bilder/Systematik der DOC_Berechnung.png}
	\caption[Kostenstruktur einer Fluggesellschaft]{Kostenstruktur einer Fluggesellschaft \cite{mensen2013handbuch}}
	\label{doc}
\end{figure}

Betriebskosten sind von dem Flugzeugtyp abhängig, deswegen ist es wichtig vor der Anschaffung zu untersuchen, 
ob ein Flugzeug mit einem alternativen Antrieb rentabel ist. 
Die neuen Regularien der \ce{CO2}-Reduktion können einen Anreiz oder sogar 
eine Verpflichtung für die Fluggesellschaften schaffen, um eine bestmögliche Lösung für eine Flotte zu finden. 
Weitere Aspekte politischer Anreize werden im folgenden Unterkapitel \ref{s:Klimapolitische Maßnahmen} betrachtet.
%
Es gibt verschiedene bereits vorgestellte Formeln für die Berechnung der DOC \cite{scholz_design_evaluation_doc}. 
Die meisten davon schließen die gleichen Kostenfaktoren ein, 
unterscheiden sich jedoch in der Rechnungsweise für einzelne Kosten.
Einberechnet werden Treibstoffkosten, Crewkosten, Wartungskosten, kapitalbezogene Kosten, sowie Entgelte und Gebühren.\\

%
\textbf{Treibstoffkosten} sind ein erheblicher Teil der Betriebskosten. 
In den USA sind ein Drittel aller Gesamtkosten (TOC) aller Fluggesellschaften 
die Kosten für Treibstoff und Öl, in Korrelation dazu beträgt 
die Abfertigung am Flughafen ein Sechstel \cite{conrady2019luftverkehr}. 
Im Jahr 2023 wurden etwa 92 Milliarden Gallonen Kraftstoff durch der 
Luftfahrindustrie verbraucht und somit betrug die Treibstoffrechnung fast 32 \% 
aller Betriebskosten der Luftfahrt \cite{iata_industry_statistics_2024}.
Die jährlichen Steigerungen der Preise für fossile Rohstoffe können die nachhaltige Initiative fördern. \\
%Kesorinpreis im Jahr 2022 betrug USD 136/bbl, prognostiziert wird jedoch ...
%
%"Betrachtet man den aktuellen Stand der kommerziellen Luftfahrt, so macht fossiles ATF den 
%größten Teil des Energieverbrauchs im Luftverkehr aus, wobei Jet A und Jet A-1 überwiegend verwendet werden" %wo kommt das her?
%
%Technikkosten (Wartung, Reparatur, Instandhaltung MRO) - 
%es gibt unterschiedliche Typen von Wartung, die Kontrolle was am Vorfeld bei einem 
%Turnaround passiert ist die line maintenance (prüfung von Reifendruck und Ölständen), Flughafenentgelte und Handlingkosten: 
%Bodenabfertigung besteht aus Passagierabfertigung, Fracht-, Gepäckabfertigung und VorfelddiensteFlugzeugabfertigung ist direkte Kosten, 
%Passagierkosten indirekte Kosten, Kapital- und Abschreibungskosten \cite{conrady2019luftverkehr}. 
%Conrady bescheibt die Cockpit Crew als auch Cabin Crew, so Gehälter, Reisekosten, Schulungskosten. Es gibt jedoch nur wenige Arbeiten, 
%die die Ausbildungskosten erwähnen. Auch Handling Agents Kosten gibt es nicht viel Information zu finden. Es kann damit zurückgeführt werden, dass
%sind unabhängige von Fluggesellschaften Handling Agents für Abfertigung zuständig und nicht die Fluggesellschaften Kosten dafür übernehmen.
%
\textbf{Crewkosten} sind auch ein Bestandteil der direkten Betriebskosten. 
Zu einer Crew gehören Piloten und Kabinen-Besatzung.
Nach Conrady \cite{conrady2019luftverkehr} bestehen die Besatzungskosten aus 
Gehältern, Reise- und Schulungskosten. Es gibt jedoch nur wenige Arbeiten, 
die bei der Berechnung der Betriebskosten die Ausbildungskosten erwähnen. \\
%
\textbf{Wartungskosten} eines Flugzeugs fassen die Arbeitskosten für Beschäftigte
und benötigten Materialien für die Wartung zusammen.
Außerdem werden die Kosten in Wartung einer Zelle und den Triebwerken unterteilt \cite{wang2021research}. 
Meistens werden diese Komponenten von unterschiedlichen Unternehmen hergestellt (Quelle).
Am Vorfeld bei der Luftfahrzeugabfertigung findet eine \textit{Line Maintenance} statt, 
dabei wird der Reifendruck und Ölstände überprüft \cite{conrady2019luftverkehr}. 
Darüber hinaus finden eine Reihe anderer regelmäßiger Kontrollen statt.
Wartungskosten hängen von Auslastung eines Flugzeugs ab, je mehr sich ein Flugzeug in Betrieb befindet, desto höhere 
Kosten sind zu erwarten. %schnellere Wartung man braucht, desto teurere Komponente man braucht.
%Wartungskosten wachsen mit der Größe des Flugzeugs, da mehr Personal oder mehr Zeit für die Wartung benötigt ist.
%
%Die Kurzstrecken-Flugzeuge sind mehr mit größerer Anzahl an Landungsgebühren und größere Treibstoffverbrauch verbunden sind, da 
%der Start und die Landung sind die Treibstoff aufwendigste Prozesse im ganzen Flug.
%
Je nach Flotte sind Ersparnisse möglich, wenn die Fluggesellschaften mehrere Flugzeuge vom gleichen Typ anschaffen \cite{conrady2019luftverkehr}. 
In diesem Fall sind weniger Schulungen für Techniker notwendig.\\
%
%
%Eine Reihe anderen Kosten, die für die Arbeit vielleicht nicht relevant?: Flugsicherungsgebühren, Versicherungskosten (bleiben), 
%Servicekosten, Marketing- und Vertriebskosten, Kpsten der allgemeinen Verwaltung, 
%Diese werden aufgrund der Komplexität der Berechnungen und der begrenzten Verfügbarkeit von Daten nicht weiter betrachtet.
%
%
%Preis für Treibstoffversorgung für kurzfristige Perioden kann folgend berechnet werden \cite{iata_saf_procurement_2024}: nicht sicher, ob die Formel nehme
%PRA-Bewertung (Risikobewertung??? Durchschnitt der Vorperiode) + Logistikkosten + zusätzliche Gebühren + Lieferantenmarge
%
Unter \textbf{kapitalbezogenen Kosten} versteht man Kosten, die vom Flugzeuganschaffungspreis abhängig sind.
Darunter sind Abschreibung, Verzinsung und Versicherung zu verzeichnen.
Abschreibungskosten sind ein Teil der Kapitalkosten für das Flugzeug, 
die auf einen festgelegten Zeitraum, in welchem Flugzeug genutzt wird, verteilt werden \cite{conrady2019luftverkehr}.
Abschreibungswerte unterscheiden sich je nach Fluggesellschaft.
Die Abschreibungskosten können auch auf die Infrastruktur bezogen werden. 
Flugzeuge werden gegen Rumpfschäden oder andere Arten von Schäden versichert \cite{scholz_design_evaluation_doc}.  \\
%
%Conceptual_Design_and_Operating_Costs_Evaluation_of_a_19-seat_All-Electric_Aircraft_for_Regional_Aviation
%"die Zins- und Abschreibungskosten sinken tendenziell bei hoher Tagesauslastung"
%20 Prozent in Motorwartung verringert DOC um 4%
%
%
%Arten Wartung: DMC are the maintenance costs caused directly by the aircraft; 
%Wartung "line maintenance cost; periodic maintenance cost; workshop direct maintenance cost; and overhaul cost."
In dieser Arbeit wird der Fokus auf die direkten Betriebskosten gelegt und die indirekten Betriebskosten, wie Kosten für 
allgemeine Verwaltung, Marketing- und Servicekosten, werden wegen geringerer Relevanz nicht betrachtet.\\
%
\subsection{Infrastrukturkosten}
%
Durch den Anstieg der Nachfrage nach innovativen Antrieben sind Änderungen an der Flughafeninfrastruktur notwendig.
%
Die Infrastruktur eines Flughafens gibt vor, über welche Kapazitäten einen Flughafen verfügt.
Die Infrastrukturkosten eines Flughafens bestehen aus Kosten für luft- und landseitige Anlagen \cite{fur2003infrastrukturkosten}. 
Landseitige sind die Einrichtungen die zum Flughafen gehören wie bspw. Terminal oder administrative Gebäude. 
Zur luftseitigen Infrastruktur gehören Start-/Landebahn, Rollbahn, Vorfeld, Flugsicherheitsinfrastruktur und -ausrüstung. 
Die Infrastruktur kann unterschiedlich finanziert werden.
Flughafengebühren, wie Lande-, Lärm-, Emissions-, Abstell-, Passagier- und Frachtgebühr 
tragen zur Finanzierung des Flughafens bei.

Infrastrukturkosten setzten sich nicht nur aus Anschaffungs-/Investitionskosten (Kapitalanforderungen), 
sondern auch durch Kosten für die Instandhaltung der Anlagen und Betriebskosten zusammen.
Kapitalkosten (wie Verzinsung und Abschreibung), die mit Infrastrukturinvestitionen zusammenhängen 
machen einen großen Teil der Gesamtkosten eines Flughafens aus \cite{wittmer2011aviation}.
%"Infrastrukturträger im Luftverkehr sind neben den Flughäfen die 
%Bodenabfertigungsdienste, Kommunikationseinrichtungen (z. B. SITA) und Flugsicherungseinrichtungen 
%(Radar-, Funknavigationsanlagen, Flugverkehrskontrolle ATC = Air traffic control)."  %file:///C:/Users/henri/Downloads/Lufrverkehr_10.1524_9783486841848.pdf
%
Flughäfen müssen wirtschaftliche Analysen nutzen, um Entscheidungen über Flughafeninvestitionen treffen zu können. %Bei Flughäfen mit beschränkten Bauschutzbereich muss für Erweiterung
Investitionsbeihilfen werden durch die Passagieranzahl des Flughafens bestimmt. 
Mit mehr als fünf Millionen Passagieren müssen Flughäfen ihre Kapitalkosten selbst tragen können \cite{conrady2019luftverkehr}. 
%"Allein  die Planfeststellungs-  und  Genehmigungsverfahren  für große  Infrastrukturprojekte dauern in  
%Deutschland im Durchschnitt 10  bis 15 Jahre."%https://www.researchgate.net/publication/279512505_Handlungsbedarf_fur_Planung_und_Nutzung_der_Flughafeninfrastruktur_in_Deutschland_Needs_for_Action_in_Planning_and_Use_of_Airport_Infrastructure_in_Germany
%
Über landseitige Anlagen finanzieren sich Flughäfen durch Mieten, Konzessionen und weitere Quellen \cite{fur2003infrastrukturkosten}.
Hat ein Flughafen regionale Bedeutung werden auch anderen Interessentengruppen an der Entwicklung teilnehmen.
Die Planung der Infrastrukturerweiterung oder -neubau muss in enger Zusammenarbeit 
zwischen den Stakeholdern (Regulierungsbehörden, Mitarbeitern, Anteilseignern, Kreditoren usw.) stattfinden \cite{wittmer2011aviation}.
%
Außerdem müssen Infrastrukturentscheidungen mit den Interessen der Gesellschaft übereinstimmen \cite{WissenschaftlicherBeirat2011}. 

Diese Arbeit wird sich auf die Anschaffungskosten für neue Infrastruktur fokussieren und nicht mit 
laufenden Kosten, wie Betriebs-, Unterhalts- und Administrationskosten der Flughäfen arbeiten, 
da sie untergeordnete Relevanz haben.

\subsection{Ausbildungskosten}

Schulungen sind ein wichtiger Teil der Ausbildung. 
%Über was geht in der annex
Nach ICAO Annex 6 muss das Schulungsprogramm eine Kompetenzschulung für alle installierten Geräte umfassen.
Aufgrund zu erwartender neuer Antriebe werden ebenfalls neue Infrastruktur und Geräte benötigt, 
wodurch neue Gefahren im Luftverkehr entstehen können. % MAX: hier stimmt was nicht
Die erforderlichen Kenntnisse variieren je nach Einsatzbereich.
%
Wegen unzureichender Datenlage in dem Bereich ist schwierig die 
Ausbildungsdauer und damit verbundene Kosten präzise zu berechnen.
Aus diesem Grund wird auf eine detaillierte Analyse der Ausbildungskosten verzichtet 
und nur auf allgemein erforderliche Kenntnisse bei der Schulung hinweisen.
Dabei wird in dem Teil \ref{s:Neuartige Antriebe} auf die Sicherheitsaspekte und Gefahren 
beim Umgang mit den Antriebsarten eingegangen und %im Teil \ref{s:Änderungen durch neue Antriebe, Annahmen und Methodik}
die Schlussfolgerung für Ausbildungen zusammengefasst. %unsicher
%Wasserstoff: benötigt zusätzliche Fortbildung % MAX: das noch formulieren?

%Für Wasserstoff: 
%The goals of this training course include:
%• Familiarization with hydrogen safety properties.
%• To identify, evaluate, and address hydrogen system hazards.
%• To teach safe practices for design, materials selection, and hydrogen system operation.
%• physical principles and empirical observations on which these safe practices are based:
%• how to respond to emergency situations involving hydrogen
%• how to visualize safety concepts
%• identify numerous parameters important to hydrogen safety.
Die Ausbildung soll die allgemeinen Charakteristiken des Wasserstoffs, 
den Umgang mit Wasserstoff und möglichen verbundenen Gefahren, 
und die korrekten Reaktionen in Notfallsituationen beinhalten. %(Quelle: https://www.icas.org/icas_archive/icas2024/data/papers/icas2024_1090_paper.pdf)
Die Schulungen sollten für alle Beteiligten an der Luftfahrzeugabfertigung durchgeführt werden.

%
\section{Neuartige Antriebe}
\label{s:Neartige Antriebe}
Obwohl Kraftstoffverbrauch hat sich in der letzten 30 Jahren halbiert \cite{mensen2013handbuch}, der Auswirkungen bleiben hoch.
Bei der Untersuchung den alternativen Antrieben sind bestimmte Überlegungen relevant. Dichte von Energieträger, Kosten 
und Verfügbarkeit des Rohstoffs, Sicherheit in Bezug auf Herstellung und Benutzung, 
direkte und indirekte \ce{CO2} Emissionen \cite{ansell2023review}.

In diesem Kapitel werden folgende vielversprechende Energieträger angeschaut und zusammengefasst: nachhaltige Kraftstoffe (SAF), 
Batterie-Antriebe (elektrochemische) und Wasserstoff.

\subsection{Sustainable Aviation Fuel (SAF)}

Sustainable Aviation Fuel oder nachhaltige Flugtreibstoff sind synthetische flüssige Biotreibstoffe oder erneuerbare nicht biogene Stoffe, %(renewable non-biological sources), 
die mit herkömmlichen Flugkraftstoffen und bestehenden Betankungssysteme kompatibel sind.
Deswegen werden sie auch als Drop-In Treibstoffe bezeichnet \cite{iata_saf_2024}. 
Die SAFs werden zu herkömmlichen Treibstoffen beigemischt. IATA besagt, dass die zulässige Mischrate 
zurzeit bei max. 50 \% liegt. Es existieren bis jetzt elf Verfahrenswege aus unterschiedlichen Rohstoffen für SAF-Produktion und
manche werden gerade für die Nutzung bewertet \cite{icao_saf_conversion_2024}.

Die SAFs haben ähnliche Charakteristiken, wie Kerosin, was die Energiedichte betrifft. Dennoch Großteil von SAF weisen keine Aromaten auf. 
Fehlende aromatische Verbindungen in den SAF kann zu Leckagen in der Dichtung führen \cite{jarin2024emissions}. 
Aus diesem Grund bis jetzt ist kein Flugzeug für das Fliegen mit reinem SAF zertifiziert \cite{iata_saf_2024}.

Im Hinblick auf die Zukunft ist es zu erwarten, dass Reduktion von \ce{CO2} mit dem reinen SAF realisiert werden kann.
Ein praxisnahes Beispiel dafür ist den ersten transatlantischen Demonstrationsflug im November 2023 von der Fluggesellschaft Virgin Atlantic,
welche mit 100 \% SAF durchgeführt wurde \cite{virginatlantic_saf_2023}. 
%(davon 88 \%HEFA und 12\% Synthetic Aromatic Kerosene (SAK))

Diese Arbeit wird auf reine SAF ohne Beimischung beschränkt, da nur so kann das Ziel 2050 erreicht werden.

Es gibt kein SAF-Flugtreibstoff, die Emissionen komplett vermeidet, jedoch durch bestimmte Verfahren können sie
bis 95 \% reduziert werden \cite{icao_saf_conversion_2024}.

% "Current engine and propulsion systems are not compatible with 100 \% bio-jet 
%fuels which require retrofitting and development of new engine propulsion systems. "

%Die vier wichtigsten Technologie sind Hydroprocessed Esters and Fatty Acids (HEFA), Fischer-Tropsch (FT), Alcohol to Jet (ATJ) 
%und Power to Liquid (PtL). 
 
Aufgrund der kommerziellen Verfügbarkeit ist Hydroprocessed Esters and Fatty Acids (HEFA) eine der wichtigsten Verfahren.
Die HEFA wird aus tierischen und pflanzlichen Ölen und Fetten mittels Hydroprocessing hergestellt \cite{bauen2020sustainable}. 
Die Haupteinschränkung von HEFA ist begrenzte Anzahl an Rohstoffen \cite{bauen2020sustainable}.
Der Anteil der SAF-Nutzung (HEFA) ist bei ...
Vielversprechend ist auch das Verfahren Power-to-liquid (PtL).
Dieser katalytische Verfahren nach Fischer-Tropsch nutzt für die Herstellung eine Kombination von Kohlenmonoxid (\ce{CO}) und durch Elektrolyse produzierter Wasserstoff (\ce{H2}) \cite{bauen2020sustainable}.
Das Verfahren bringt die höchste \ce{CO2}-Emissionseinsparungen \cite{de2017life}, jedoch befindet sich im früheren Stadium \cite{bauen2020sustainable}.

In Bezug auf die Infrastruktur sind manche davon überzeugt, dass es keine Änderungen im Flugzeug oder Betankungssystem notwendig sind \cite{sky2020hydrogen} %https://www.iata.org/en/pressroom/2023-releases/2023-06-04-03/.
Wobei Dahal et al. \cite{dahal2021techno} geht jedoch davon aus, dass neue Antriebe und Triebwerke für die Nutzung des reinen SAF entwickelt werden müssen.
Das reine SAF wurde noch nicht zertifiziert, um in das Treibstofflager vom Flughafen zu gelangen \cite{iata_saf_2024}.

%In der Abbildung XX sind die SAF Produktionswege aus unterschiedlichen Quellen aufgelistet und den Teil des Kohlenstoffdioxids,
%die mit diesen Wegen reduziert werden. Bemerkenswert ist, dass die Herstellung
%aus einer Kombination aus Kohlenmonoxid (CO) und Wasserstoff (H2) durch das katalytische Verfahren nach Fischer-Tropsch (existing renewables), 
%auch PtL genannt, fast bis 100 \% CO2-Ausstoß reduziert, gefolgt von Municipal Solid Waste (MSW). 

%PtL: brauche ich das überhaupt?
%CO2 kann durch Direct Air Capture (DAC) aus der Luft gewonnen werden und dann mittels 
%Reverse-Water-Gas-Shift-Reaktion (RWGS) mit Wasserstoff zu CO2 umgewandelt werden. 
%PtL ist sehr energieintensiv und braucht erneuerbare Stromquelle (hängt davon ab).
%\cite{ansell2023review}
%Eine Herausforderung im Blick auf Ressourcen- und Flächenbedarf bleibt bei biogenen oder PtL-Synthese \cite{ansell2023review}.

%MSW ist Abfall aus nicht biogenen Quellen, wie Kunststoffe. \cite{icao_saf_conversion_2024} PtL SAF "Trotz der erheblichen Herausforderungen 
%bietet sich PtL SAF an, langfristig einer der stärksten Beiträge zur Energiewende der Fluggesellschaften zu werden."

%Kosten:
%https://www.icao.int/environmental-protection/Pages/SAF_RULESOFTHUMB.aspx
%https://theicct.org/sites/default/files/publications/Alternative_jet_fuels_cost_EU_20190320_1.pdf
%"Overall, estimates of capital spending on renewable diesel/HEFA facilities
%ra n g e f ro m a ro u n d € 0. 4 0 to €1.50 per liter of annual capacity, averaging around €0.60 per liter,
%with larger facilities generally having lower per-liter capital costs due to economies of scale"
%ICAO hat eine Reihe von Heuristiken rausgebracht, um Preisschätzungen unter SAF-Kraftstoffen zu ermitteln. (für USA)
%FT mit Feedstock CO2 from Direct Air Capture, H2 - Feedstock Price \$300/t, \$6/kg, Total capacity 1000 mill L/year.
%FT aus MSW - \$30/ton 

Durch Produktion und Lieferung können es zu Emissionen-Ausstoß kommen, aus diesem Grund 

%Im März 2024 hat International Aero Engines AG (IAE) bei der MTU Maintenance ein V2500-Triebwerk mit 100 \% nachhaltigem 
%Flugkraftstoff HEFA-SPK getestet.

Die Preise für nachhaltige Flugtreibstoffe sind zwei- bis zu fünfmal höher als für herkömmliche Kerosin \cite{iata_saf_2024} %unsicher wg Quelle.

%Einbindung in Regularien
In EU-Richtlinien sowie in CORSIA sind die Kriterien für SAF-Qualität festgelegt.
Im Rahmen EU-ETS gelten SAFs als emissionsfrei und bei der richtigen Zertifizierung sind vor der Abgabe von \ce{CO2}-Zertifikaten 
befreit \cite{icao_saf_conversion_2024}. Die Preise für nachhaltige Flugtreibstoffe sind zwei- bis zu fünfmal höher als für herkömmliche Kerosin \cite{iata_saf_2024} %unsicher wg Quelle.
Um Fluggesellschaften für die Nutzung der nachhaltigen Kraftstoffe zu motivieren, hat EU-ETS
20 Mio. Zertifikaten zur Verfügung gestellt \cite{icao_saf_conversion_2024}. 
In ReFuelEU sind vor allem die verpflichtete Beimischungsanteile von nachhaltigen Stoffen festgelegt.

\begin{figure}[h]
	\centering
	\includegraphics[width=0.4\linewidth]{Bilder/Preise SAF.png}
	\caption[Durchschnittlicher IATA-Mindestverkaufspreis (MSP) der wichtigsten SAF-Pfade über den Zeitraum 2020 bis 2050]{Durchschnittlicher IATA-Mindestverkaufspreis \cite{icao_saf_conversion_2024}}
	\label{fuelcell}
\end{figure}

%Nach ASTM D7566 müssen Kraftstoffe einen bestimmten Gehalt an Aromaten aufweisen, um kompatibel mit ben bestehenden Flugzeugen zu sein. (Quelle?)

%Quelle: Synthetic aromatic kerosene property prediction improvements with isomer specific characterization via GCxGC 
%and vacuum ultraviolet spectroscopy: 
%SAK besteht grundlegend aus Aromaten und unterscheidet sich deutlich von den anderen SAFs und wird für Aufschwellung von Dichtungen. 
%"Aromaten sind organische Verbindungen, die die Schmierfähigkeit, Dichte und Materialverträglichkeit des Flugkraftstoffs verbessern"

Die sind gleichzeitig für Kondensstreifen verantwortlich (Quelle), die klimanegativ wirken (Quelle). 

%Derzeit wird von ASTM evaluiert \cite{icao_saf_conversion_2024}.
%HEFA und SAK Mischung können die 100\% SAF-Flüge ermöglichen, dabei reduziert die Mischung die Rußpartikeln.

Laut IATA durch Drop-In SAF können die Emissionen um 62 \% reduziert werden. https://www.iata.org/en/pressroom/2023-releases/2023-06-04-03/
"As a drop-in solution, SAF is expected to deliver about 62\% of carbon mitigation needed to achieve net zero by 2050"

%Manche berichten, dass Flugdistanz bei SAF gleich(Quelle), und dazu zum Schluss gekommen, dass 
%Treibstoffverbrauch verringert werden kann(Quelle).
%Annahme: Verbrauch bei konventionellen und SAF gleich

%\subsection{Batterie-Antrieb}
%"In der Luftfahrt ist es möglich, direkten Strom zu nutzen. Die direkte Nutzung von
%Elektrizität erfordert Elektromotoren und Stromspeicher an Bord" \cite{dahal2021techno}, wie Batterie oder Brennstoffzellen \cite{dalmia2022powering}.
Eine andere Möglichkeit, Emissionen zu reduzieren, ist direkten Strom als Antrieb mittels Elektromotoren und Stromspeicher, wie Batterien oder Brennstoffzellen, zu nutzen.
Eine einfache Darstellung des Batterie-Antriebs (BA) ist in der Abbildung \ref{ba} gezeigt.
\begin{figure}[h]
	\centering
	\includegraphics[width=0.7\linewidth]{Bilder/BA.png}
	\caption[Einfachster Batterie-Antrieb]{ \cite{hepperle2012electric}}
	\label{ba}
\end{figure}

Es gibt drei unterschiedliche Antriebskonfigurationen von elektrischen Flugzeugen: vollelektrisch, funktioniert nur auf der Batterie oder
Brennstoffzelle als Energiequelle, turboelektrisch und hybrid-elektrisch ist eine Mischung von konventionellen 
Gasturbinentriebwerken mit Kerosin und Batterie oder Brennstoffzellen \cite{dahal2021techno}. %Turboelelektrisches Antrieb 
Im Folgenden wird ein vollelektrischer Antrieb beschrieben.
%
Funktionsweise Elektromotor: Durch Potenzialdifferenz und einem Stromfluss wird die elektrische Energie in mechanische umgewandelt.
Im Vergleich zum Verbrennungsmotor ist der einzige bewegliche Teil bei BA der Rotor \cite{donckers2024electric}, 
was die Wartungskosten verringern kann. Außerdem besteht der elektrische Antrieb aus einem Controller, welcher den Energiefluss steuert. 
Durch den Controller wird festgelegt, welche Leistung der Motor erzeugen bzw. wie viel Energie von 
einer Batterie genutzt werden soll, um die gewünschte Leistung zu erzeugen \cite{donckers2024electric}. 

Das Batteriemanagementsystem in einem Flugzeug verfügt über Informationen wie State of Health (SOH), welche den Unterschied zwischen Anfangs-
und Bestandskapazität einer Batterie angibt, und State of Charge (SoC), welche zeigt, wie viel Prozent der verfügbaren Kapazität geladen werden kann \cite{donckers2024electric}.

%Ein Vorteil des elektrischen Flugzeugs ist, dass den Antrieb zulässt, rückwärtszufahren (Quelle) und somit auf den Schlepper-Einsatz verzichtet werden kann.
Durch die Umwandlung der elektrischen in chemische Energie kann diese in der Batterie gespeichert werden.
Im Laufe des Fluges verändern die Batterien ihr Gewicht nicht, unabhängig davon, ob sie leer oder vollständig geladen sind \cite{donckers2024electric}. 
Eine in der wissenschaftlichen Literatur weit verbreitete Batterie ist die Lithium-Ion-Batterie. Diese haben eine hohe Energiedichte im Vergleich zu anderen vorhandenen Batterien. %ist es so?
Heutige Li-ion Batterien haben eine Energiedichte von 100-265 Wh/kg (Quelle). Werden diese Werte mit der spezifischen Energiedichte von 12 kWh/kg von Kerosin verglichen \cite{dalmia2022powering},
ergibt sich eine Differenz von ca. 45-facher Steigerung gegenüber einer Li-Ion-Batterie. Das weist darauf hin, dass Batterien ein viel höheres Gewicht für die gleiche Energie verursachen. 
Somit steigt auch die Masse des Flugzeugs.

Batterien sind von äußerlichen Bedingungen beeinflussbar. Kalte Umgebungen können 
den Wirkungsgrad reduzieren (Quelle), warme Umgebungen können zu einem schnelleren Auslaufen der Lebensdauer führen.
Die Herstellung einer Lithium-Ionen-Batterie ist durch Lithium-Produktion umweltschädlich (hoher Wasserverbrauch, gefährliche Leckagen) und 
kostenintensiv in der Wartung \cite{dalmia2022powering}. 

In Forschung befinden sicht weitere Arten von Batterien wie Lithium-Sulfur, Lithium-Air, sowie Solid-state Batterien, welche vielversprechend wirken (Quelle).
%
%
%Batterien werden bereits jetzt als sekundäre Leistungsquelle.(Schmidt?)

Der Batteriewechsel ist kompatibler mit der Flugplanung, benötigt aber mehrere Batterien für den Austausch, was die Logistik erschwert 
und höhere Anschaffungskosten verursacht. Batterien müssen ordnungsgemäß und sicher gelagert werden. \cite{salucci2020optimal}

Ein Batteriewechsel ist öfters vorkommende in der Literatur Ladung.

Ladeleistung ist für die Dauer der Ladung verantwortlich. Durch schnellere Ladungen wird Lebensdauer der Batterien reduziert. Was mit sich bringt, dass die Batterien schneller ausgetauscht werden müssen
und mehr Kosten dadurch entsteht. (Quelle) Wobei die langsamen Laden ist für die Fluggesellschaften nicht rentabel sein kann, 
da wenn Flugzeug auf dem Boden steht verdienen Fluggesellschaften kein Geld.

Transportkapazität ist nicht so groß, wie bei konventionellen Flugzeugen und auch die Reichweite für Regionalflüge \cite{abrantes2024impact}.
Die Energieproduktion ist nicht emissionsfrei \cite{abrantes2024impact}, dafür muss mehr erneuerbaren Quellen hergestellt werden, 
wie Solar- und Windenergie. 
%Wasserstoff braucht kryogene Lagerung, hat Entflammbarkeit und braucht Infrastrukturentwicklung.\cite{abrantes2024impact}

In Bezug auf Sicherheit, das größte Gefahr bei BA ist chemische Reaktion des thermischen Durchgehens, wofür stabile
Kühlungssysteme benötigt sind \cite{donckers2024electric}. Thermisches Durchgehen verursacht einen starken Anstieg der 
Innentemperatur von Batterie, was zu kompletten Ausfall der Batterie oder Freisetzung der brennbaren Gase führen kann \cite{shahid2022review}.
%Das thermische Durchgehen von Lithium-Ionen-Batterien ist das Phänomen exothermer Kettenreaktionen innerhalb der Batterie.
% Diese Reaktionen verursachen normalerweise einen starken Anstieg der internen Batterietemperatur, wodurch die inneren Strukturen 
% der Batterie destabilisiert und abgebaut werden, was zu einem Totalausfall der Batterie führen kann. Das thermische Durchgehen kann 
% durch verschiedene Formen mechanischer, elektrischer und thermischer Misshandlung auftreten. All dies führt zu einem internen Kurzschluss 
% der Batterie, da der Separator zwischen Anode und Kathode entweder zusammenbricht, abreißt oder durchbohrt wird. Dadurch wird eine große 
% Menge Wärme erzeugt, die wiederum den Grad der elektrochemischen Reaktionen verstärkt und eine übermäßige Wärmeentwicklung verursacht. 
% Dieser Zyklus setzt sich fort, wodurch die Temperatur der Batterie stark ansteigt und große Mengen brennbarer Gase freigesetzt werden.

%\subsection{Wasserstoffantrieb}
\label{ss:Wasserstoff-Antrieb}

In vielen Forschungsarbeiten wird Wasserstoff als die Lösung für umweltfreundliche Luftfahrt dargestellt.
Dieses Energiemedium wird jedoch von Kosten, Sicherheit und öffentlicher Akzeptanz behindert \cite{ansell2023review}.
Durch die Nutzung von Wasserstoff werden keine \ce{CO2}-Emissionen verursacht, jedoch können andere Abgase 
wie Stickstoffoxid \ce{NO_x} bei der Verbrennung in Wasserstoffturbinen oder Wasserdampf emittiert werden, was zur Bildung von Kondensstreifen
führt \cite{hepperle2012electric}.\\
%was zur Bildung von Kondensstreifen führt \cite{conrady2019luftverkehr}. %überprüfen

\subsubsection{Herstellung}
Es existieren verschiedene Wege zur Herstellung von Wasserstoff. 
Die gängigsten sind Dampfreformierung (Steam Methane Reforming - SMR) und die Elektrolyse. %(Fully Renewable). 
Bei SMR trifft Wasserdampf in der Reaktion zusammen mit Methan aus Erdgas, infolgedessen entsteht 
Wasserstoff \ce{H2} und Kohlenmonoxid \ce{CO} bzw. -dioxid \cite{mulder2019outlook}. Bei der Elektrolyse wird das Wasser mithilfe von Elektrizität 
in Wasserstoff \ce{H2O} und Sauerstoff \ce{O2} gespalten \cite{mulder2019outlook}. Durch diesen Herstellungsweg können \ce{CO2}-Emissionen 
vollständig vermieden werden \cite{dalmia2022powering}. 
Mulder et al. \cite{mulder2019outlook} schätzt die Investitionskosten für die Produktion des Wasserstoffs durch Elektrolyse deutlicher günstiger als ein Kohlekraftwerk.
Die Nachhaltigkeit der Elektrolyse ist außerdem, genau wie bei dem Batterieantrieb, von der Stromquelle abhängig.
%
Wenn für die Produktion von Wasserstoff erneuerbare Energiequellen (wie Solar- und Windanlagen) genutzt wurden, 
wird dieser als grüner Wasserstoff bezeichnet \cite{mulder2019outlook}. 
Durch Elektrolyse produzierter grüner Wasserstoff ist kostenintensiv \cite{dalmia2022powering}.
Wird der genutzte Strom aus fossilen Energieträgern erzeugt, kommt es zu indirekten Emissionen.
Bei anderen Herstellungswegen (bspw. grauer und blauer Wasserstoff) kommt es hingegen zum Ausstoß von Kohlenstoff, 
wobei bei blauem Wasserstoff das \ce{CO2} gesammelt und gespeichert wird \cite{mulder2019outlook}.
%Chardonnet et al. schätzt die Investitionskosten für die Produktion in 15 Mio. € (nach \cite{mulder2019outlook}). wie viel kilo?
%
%
\subsubsection{Zustände}
Wasserstoff kann in mehreren Zuständen genutzt werden. Die in der Verkehrsbranche am weitesten verbreiteten
sind einerseits der gasförmige \ce{GH2}, und andererseits der kryogene flüssige Wasserstoff \ce{LH2}. 
Um mehr Energie speichern zu können und dabei weniger Platz zu verbrauchen, muss das gasförmige \ce{H2} stark komprimiert und 
bei einem Druck von 350 oder 700 bar gespeichert werden \cite{colpan2022fuel}.
Allerdings hat gasförmiger Wasserstoff auch bei einem Druck von 700 Bar eine geringere Energiedichte
als flüssiger Wasserstoff \cite{eichlseder2012hydrogen}.
Flüssiger Wasserstoff wird durch das Abkühlen und Verdichten von gasförmigem Wasserstoff gewonnen.
%"LH2 hat die höchste spezifische Energie- und Kühlleistung aller herkömmlichen Substanzen." \cite{ansell2023review}
In der Tabelle \ref{wasserstoff_energie} sind die Vergleichswerte für Kerosin und Wasserstoff im flüssigen Zustand dargestellt.
Die Gravimetrische Energiedichte bei flüssigem Wasserstoff ist deutlich höher als bei Kerosin, 
jedoch ist die volumetrische Energiedichte $E_V$ wesentlich geringer.
Das bedeutet, dass \ce{LH2} zwar bessere Gewichtsverhältnisse als Kerosin hat, 
für die gleiche Menge Energie wird allerdings 3,5-mal so viel Platz benötigt.
Aufgrund seiner Eigentschafen ist der Wasserstoff für die Nutzung auf längeren Flugdistanzen geeignet.
%(flüssiger Wasserstoff - geringere Gesamtspeichermasse-größere Flugzeuge und volumenanforderungen; für kleinere regionale - GH2)
\begin{table}[h]
	\begin{center}
    \caption{Vergleich von flüssigem Wasserstoff energiebezogenen Eigenschaften mit anderen konventionellen Treibstoffen}
	\label{wasserstoff_energie}
	\begin{tabular}{|l|c|c|c|}
		\hline
		& \textbf{$E_V$ in $[kWh/l]$} & \textbf{$E_G$ in $[kWh/kg]$} & \textbf{$Dichte$ $[kg/m^3]$}  \\ \hline
		Wasserstoff LH2 \cite{colpan2022fuel} & 2,6 & 37,0 & 65 \\ \hline
		Kerosin \cite{colpan2022fuel} & ~9,5 & ~11.9 &  \\ \hline % energiedichte von kerosin fehlt oder?
	\end{tabular}
    \end{center}
\end{table}
%
Flüssiger kryogener Wasserstoff ist wesentlich besser als gasförmiger für den Transport per LKW geeignet,
dafür benötigt er aber einen höheren Energieaufwand \cite{colpan2022fuel}. 

Die Konzepte schlagen verschiedene Platzierungen des Wasserstofftanks vor, unter anderem in Form von halbkugelförmigen Endkappen auf dem Flugzeugrumpf \cite{dahal2021techno} 
oder am Ende des Flugzeugrumpfes zwischen der Fracht und den hinteren Notausgängen \cite{rietdijk2024architecture}.
Bei Flugzeugen, welche mit Wasserstoff betrieben werden, ist die Tankisolierung von großer Bedeutung. 
Das flüssige \ce{LH2} muss bei -253 °C gelagert werden \cite{colpan2022fuel} 
und zusätzlich kann die Nutzung zu Versprödung der Materialien führen \cite{dahal2021techno}.
Durch Wärme verdampft der Wasserstoff, was zum Anstieg des Drucks und der Temperatur im Tank führt. Heutige Tankanlagen 
haben tägliche Abdampfverluste in Höhe von 0,3 \% bis 3 \% \cite{eichlseder2012hydrogen}.\\

%Wasserstoff führt schnell zu Versprödung durch zyklische Belastungen, 
%die bei der Wärmeausdehnung und -kontraktion während des Nachfüllens 
%sowie durch den Kraftstoffverbrauch verursacht werden \cite{dahal2021techno}.
%Das kann die Abfertigungszeiten beeinflussen. (bei schlechte Isolierung - schneller starten, bei guten - mehr Flexibilität)
%
%
%
%\cite{mulder2019outlook} - Annahme: anstatt Windkraftanlage - Stromnetz

\subsubsection{Antrieb}
Wasserstoff kann in zwei Ansätzen als Antrieb genutzt werden: 
einerseits als Treibstoff für die Verbrennung im \ce{H2}-Verbrennungsmotor,
andererseits in der Brennstoffzelle, um den elektrischen Motor anzutreiben \cite{sky2020hydrogen}. 
Zudem gibt es einen hybriden Antrieb, bei welchem die Brennstoffzelle zusammen mit einer Batterie genutzt wird.
%
%Eine andere Methode, um Wasserstoff in der Luftfahrt zu nutzen, ist die Brennstoffzellentechnologie.
Brennstoffzellen haben ein hohes Gewicht \cite{hepperle2012electric} und 
benötigen den gasförmigen Wasserstoff als Antrieb \cite{colpan2022fuel}.
Dabei wird durch die chemische Reaktion aus gasförmigem Wasserstoff \ce{H2} und Sauerstoff \ce{O2} Strom produziert \cite{dalmia2022powering}, 
wodurch der Propeller des Flugzeugs angetrieben wird. Aufgrund der besseren Speicherung in flüssiger Form muss bedacht werden, wann der Wasserstoff in den Gaszustand überführt wird. 
Brennstoffzellen haben, ebenso wie Batterieantrieb, weniger bewegende Teile als konventionelle Antriebe \cite{dalmia2022powering} was weniger Wartungskosten verursachen könnte,
allerdings könnte der technisch anspruchsvolle Wasserstofftank häufigere Wartungszyklen erfordern.
%
%Brennstoffzellen erzeugen viel Wärme, weshalb Kühlsysteme benötigt werden, um die Leistung aufrechtzuerhalten.
%https://ieeexplore.ieee.org/document/9794396
%
%Brennstoffzellen (wie die Polymerelektrolytbrennstoffzelle) haben eine höhere Energiedichte (als was?) und 
%können somit für Mittel- oder Langstrecken genutzt werden \cite{dalmia2022powering}.  ist es so?
%
%
%"Die Wasserstofftriebwerke werden in ihrer Architektur den bestehenden Düsentriebwerken ähneln, 
%jedoch mit einigen Ergänzungen, wie z. B. Kraftstoffpumpen und -steuergeräten, Brennkammern und
% einem zusätzlichen Wärmetauscher zur Verdampfung von flüssigem Wasserstoff (LH2)" \cite{colpan2022fuel}
Für die Verbrennung des Wasserstoffs sind Änderungen in der Brennkammer benötigt, 
um höhere Temperaturen zu vermeiden \cite{khandelwal2013hydrogen}.
Colpan et al. \cite{colpan2022fuel} ist jedoch der Meinung, 
dass sich die Wasserstofftriebwerke konventionellen Düsentriebwerken ähneln werden. 
Allerdings werden zusätzliche Komponenten wie Kraftstoffpumpen und Wärmetauscher 
für den flüssigen Wasserstoff benötigt.\\
%
%. Das vorhergeht von % der satz ist komisch, bitte überprüfen
%Gewichtsanteil von LH2 Tank in Anhängigkeit von der gelagerten Menge. 
%Große Flugzeuge bieten bessere Speichereffizienz und Wasserstoffmengen-Verhältnis.
%"(da niedrige Tankgewichte erforderlich sind, um den hohen spezifischen Energievorteil von Wasserstoffkraftstoff voll auszuschöpfen)"
%\cite{ansell2023review}
%
%Aufgrund des schweren und massiven Wasserstofftanks werden 
%unterschiedliche wasserstoffkonfugurationen vorgeschlagen.
%Es gibt Konzepte, wo der Tank am Rumpf, am Ende des Rumpfes 
%oder vorne, hinter der Pilotenkabine sitzt. % MAX: was ist hier?
%Sobald die Brennstoffzellen in dem Flugzeug integriert sind, werden Generatoren/Generatoren das 
%Auxiliary Power Unit (APU)-System des Flugzeugs mit Strom versorgen. Zum Einsatz kommen Brennstoffzellen 
%mit Protonenaustauschmembran, deren Strom die Motoren antreibt. Die Motoren werden über Propeller verfügen,
%die dann dem Flugzeug Schub verleihen, da sie sich mit unterschiedlichen Geschwindigkeiten im Verhältnis zur Drosselklappe drehen.\cite{dalmia2022powering}
%
%
%Betriebsleergewicht bleibt konstant, im Betracht Startbruttogewicht um 30 \%, jedoch wegen großen Wasserstofftank
%\textit{Sicherheit beim Umgang mit Wasserstoff}\\

Wasserstoff wird als hochentzündlich skaliert \cite{dalmia2022powering}. Aufgrund seiner Natur breitet sich die Flamme eher 
vertikal aus und die Brenndauer von \ce{LH2} ist kürzer als die des Kerosins \cite{colpan2022fuel}.
Der Wasserstoff hat eine hohe Flammengeschwindigkeit und es besteht die große Gefahr eines Flammenrückschlags bei der Flammenausbreitung \cite{khandelwal2013hydrogen}.
Dennoch ist Wasserstoff innerhalb der richtigen Infrastruktur nicht gefährlicher als andere brennbare konventionelle Treibstoffe und in manchen 
Fällen sogar sicherer \cite{khandelwal2013hydrogen}. 
Wird der flüssige Wasserstoff verschüttet, wird er aufgrund seiner Leichtigkeit vertikal nach oben verdampfen \cite{colpan2022fuel}. 
Direkter Kontakt mit kryogenem Wasserstoff führt zu Erfrierungen.
Die Forschung des Wasserstoffs muss sich mit Themen wie Explosionsgefahr, Materialgefahr, Betankung und 
dem Umgang in der Abfertigung auseinandersetzen. 

%
%Zusätzliche Ausbildungen für Wartungsmitarbeiter und Schulungen für Bodenabfertigungspersonal
%werden benötigt, weil sich die Charakteristiken von Wasserstoff 
%von herkömmlichen Treibstoffen stark unterscheiden.
Aufgrund des starken Unterschiedes der Charakteristiken von Wasserstoff 
im Vergleich zu herkömmlichen Treibstoffen werden zusätzliche Schulungen für 
Wartungsmitarbeiter und für Bodenabfertigungspersonal benötigt, 
um mögliche Gefahren zu erkennen und diese zu vermeiden. 
Die Ausbildung soll die allgemeinen Charakteristiken des Wasserstoffs, 
den Umgang mit Wasserstoff und möglichen verbundenen Gefahren, 
und die korrekten Reaktionen in Notfallsituationen beinhalten \cite{rietdijk2024architecture}.
%
%Die Ausbildung soll die allgemeinen Charakteristiken des Wasserstoffs, 
%den Umgang mit Wasserstoff und möglichen verbundenen Gefahren, 
%und die korrekten Reaktionen in Notfallsituationen beinhalten. %(Quelle: https://www.icas.org/icas_archive/icas2024/data/papers/icas2024_1090_paper.pdf)
%Die Schulungen sollten für alle Beteiligten an der Luftfahrzeugabfertigung durchgeführt werden.
%
%Wegen des großen Unterschiedes zu herkömmlichen Treibstoffen und 
%Ausrüstungen müssen die Mitarbeiter neu geschult werden, 
%um mögliche Gefahren zu erkennen und zu vermeiden \cite{gu2023hydrogen}.
%
%\cite{mulder2019outlook}?
%
%
%not sure:
%\cite{dahal2021techno} In der Dahal et el. Studie wurde ein Konzept vorgeschlagen, wo die Tanks auf dem Fuselage sich befinden.
%"Die kryogenen Kraftstofftanks erfordern eine angemessene Isolierung sowie die Verwendung von Materialien, die 
%gegen Versprödung und zyklische Belastungen schützen, die durch die thermische Ausdehnung und Kontraktion beim Nachfüllen 
%und den Kraftstoffverbrauch verursacht werden"
%
%etwa 64\% less Specific Fuel Consumption (SFC) \cite{colpan}

%\chapter{Änderungen durch neue Antriebe, Annahmen und Methodik}
\label{ch:Änderungen durch neue Antriebe, Annahmen und Methodik}

Konzepte mit neuen Antrieben befinden sich im Entwicklungsprozess und bis jetzt es ist ratsam, wie zukünftige Flughäfen aussehen werden
und welche Ausstattung für die Flugzeug-Abfertigung ausgesucht wird. Wasserstoff-Flugzeuge werden erst ab dem Jahr 2035 in 
auf den Markt eintreten, wobei die BA-Flugzeuge schon in den nächsten Jahren erwartet werden.
Der Wechsel zu nachhaltigen Antrieben kann es zu deutlichen Änderungen in der Infrastruktur und Abläufen am Vorfeld führen, die
in diesem Kapitel beschrieben werden.
Außerdem auf der Grundlage der unterschiedlichen Quellen und vernünftigen Behauptungen wird eine Reihe der Annahmen für diese Arbeit getroffen.
%Dieser Teil der Arbeit beschäftigt sich mit der vorhandenen Infrastruktur-Optionen. %und im Fall des Wasserstoffs Lieferketten.
%
%Wenn die Infrastruktur auf den Regionalflughäfen gemacht wird, 
%macht es nicht so ein Ausmaß wie auf größeren Flughäfen, wo viele Abfertigungsplätze umgerüstet werden müssen.

\section{Änderungen an der Abfertigung und dazugehörige Kosten von alternativen Antrieben}
\label{s:Änderungen an der Abfertigung und dazugehörige Kosten von alternativen Antrieben}

Infrastrukturkosten sind von der Größe des Flughafens abhängig. Größere Flughäfen können mehr Flugzeuge als Regionalflughäfen 
abfertigen, was dazu führt, dass mehr Abfertigungsplätze umgerüstet und versorgt werden müssen und mehr Arbeitskräfte geschult werden müssen. 
In diesem Teil wird näher auf die Änderungen in der Infrastruktur durch die Einführung von neuen Antrieben eingegangen und 
die Forschungsrichtung ausgesucht.

\subsection{SAF}
SAF ist zwar nicht die beste langfristige Lösung wegen vorhandenen Emissionen, aber wegen benötigter Entwicklung der anderen nachhaltigen
Antrieben stellt eine gute Option dar. In der nahen Zukunft werden vor allem die großen Flugzeuge mit BA-Antrieb nicht entwickelt, 
deswegen können SAF für die Langstreckenflüge benutzt werden \cite{dalmia2022powering}.
Diese Arbeit wird sich auf das reine SAF ohne Beimischung beschränken, da nur so das gesetzte Ziel bis zum Jahr 2050 erreicht werden kann.

SAF benötigt keine Infrastrukturänderung und darf in bestehenden Systemen und Flugzeugen benutzt werden \cite{dalmia2022powering}.
Dadurch, dass SAF zu herkömmlichen Treibstoffen beigemischt wird, ist zurzeit einen zusätzlichen Treibstofftank 
für das gemischten Kraftstoff gebraucht. Bis jetzt Transport von SAF mit einer Pipeline nicht zugelassen (Quelle?).
Bei der intensiven Recherche wurde jedoch keine Information gefunden, die besagt, dass es verboten wird reine SAF nicht als
Drop-In zu benutzen. Aus diesem Grund gilt für die Arbeit, dass die Lieferung von SAF mit bestehenden Pipelines 
zertifiziert und genauso wie bei Betankung mit herkömmlichen Treibstoffen zugelassen wird.

\subsection{Batterie-Antrieb}
Batteriegetriebene Flugzeuge brauchen größere Veränderung am Flughafen als bei der Nutzung von SAF.
Bis zum Jahr 2050 sind die BAs auf die kleineren Flugzeuge und damit auf Kurz- und Regionalverkehr beschränkt. 

In der Literatur werden zwei Batterien-Lademöglichkeiten diskutiert, die Batterien zu wechseln (Swap-Methode), 
wo die Batterie aus dem Flugzeug herausgenommen werden und 
an einer Ladestation geladen, oder Ladekabel in das Flugzeug einzustecken (Plug-In), wie bei etablierten E-Autos.
Zudem ist ein modulares System möglich, wo beide Ansätze benutzt werden \cite{salucci2020optimal}.
%
Bei diesem Antrieb muss beachtet werden, welche Lebensdauer eine Batterie hat und wie die Batterien geladen werden. 
Je länger die Ladung dauert, desto mehr Kosten auf dem Boden verursacht werden. Nichtsdestotrotz kann schnelle Ladung 
zur Stagnation von Lebensdauer einer Batterie führen (Quelle).

Ladeleistung ist für die Dauer der Ladung verantwortlich. Reduzierung der Lebensdauer der Batterien 
hat zur Folge, dass die Batterien schneller ausgetauscht werden müssen
und mehr Kosten dadurch entstehen. 
%Wobei die langsamen Laden ist für die Fluggesellschaften nicht rentabel sein kann, 
%da wenn Flugzeug auf dem Boden steht verdienen Fluggesellschaften kein Geld.

%
\textit{Plug-In} Methode benötigt ein schnelles Laden, damit Flugzeuge weniger Zeit auf Boden verbringen müssen.
Jedoch ist ein Anstieg in Turnaround-Zeiten aufgrund nicht zurzeit möglichen Schnellladung möglich \cite{avogadro2024demystifying}.

Bei \textit{Swap-Methode} kann Aus- und Einbau der Batterie aus dem/in das Flugzeug lange dauern \cite{dalmia2022powering}. 
Guo et al. \cite{guo2020aviation} ist zum Schluss gekommen, dass Batteriewechsel effizienter und ökonomischer ist, 
wenn die batteriebetriebenen Flugzeuge nur ein kleiner Teil (unter 10 \%) der Flotte ist, in anderem Fall lohnt sich eine Plug-In-Ladung. 
Für die Batteriewechsel müssen auch Transport und Hebegeräte gestellt werden, um die Batterien bewegen zu können \cite{reimers2018introduction}.
Jedoch mit Batteriewechsel können die Abfertigungszeiten reduziert werden (Quelle), was an einem großen Flughafen von der Bedeutung ist. 

Batteriewechsel bietet gleichmäßigere Deckung der Nachfrage \cite{guo2020aviation} und kompatibler mit der Flugplanung \cite{salucci2020optimal}, 
da der Austausch einer Batterie viel schneller ist, als Dauer einer Plug-In Ladung. 
Dennoch werden mehrere Batterien benötigt, die zudem ordnungsgemäß und sicher gelagert werden müssen \cite{salucci2020optimal}.
Außerdem ermöglicht der Swap-Methode langsameres Laden und macht das Laden mit geringer Leistung möglich \cite{avogadro2024demystifying}.
Aus diesen Gründen werden in nächsten Teilen die Kosten für diese Option ausgewertet. 

Würde der Austausch der Batterien parallel zu anderen Prozessen, wie Deboarding kann die kürzere Turnaround-Zeit im 
Vergleich zu konventionellen Flugzeugen erreicht werden \cite{schmidt2016challenges}.

\textit{Annahmen für die BA-Analyse}\\
%
Für weitere Betrachtung wurde es die Swap-Methode entschieden. Zudem wird es angenommen, 
dass die Batterien für Flugzeuge zur Flughafen-Infrastruktur gehören.
Das bedeutet, dass diese Anschaffungskosten für Flughafen anfallen. Danach werden die Batterien in Form von Leasing 
von für Fluggesellschaften ausgeliehen.

Für die Ladevorgänge wird Strom benötigt. 
Die Strompreise und die vorhandene Leistung sind normalerweise von Tag und Nacht abhängig \cite{salucci2020optimal}, 
aus praktischen Gründen wird die konstante Spitzenleistung von Batteriewechselsystem von 250 kW wird angenommen. 
Bei einer Batteriekapazität von 900 kWh, würde die Ladung bei so einer Leistung ohne Beachtung der Verluste 3,6 Stunden dauern.

%%%%%%% kommt später in 2_4_2
Für die Nachhaltigkeit des Batterieantriebes sind die Energiequellen von der Bedeutung. 
Der Strom aus dem Stromnetz kann sein Ursprung aus den Kraftwerken und Verteilerzentren haben \cite{dalmia2022powering}. 
Was dazu führt, dass die fossilen Brennstoffe für die Verbrennung benutzt werden und dadurch zu Emissionen beitragen. 
Als Alternative wäre Nutzung der erneuerbaren Energiequellen, wie Windenergie oder Solarenergie. Diese Energie ist normalerweise teurer
und die Produktionsmenge ist bis jetzt nicht ausreichend, um die Luftverkehr-Nachfrage zu decken (Quelle).

%Für die Ladung den batteriegetriebene-Flugzeugen Infrastruktur:
%Infrastrukturmodell nach Guo et al. "Die EA-Aufladung wird teilweise von einer flughafenbasierten Solar-PV-Anlage geliefert.
%Jährliche Betriebskosten (OPEX)= Strombezugskosten aus dem Netz in der Sommer und Wintersaison (typische Tage) basierend auf der Nachfrage. \cite{guo2020aviation}

\subsection{Wasserstoff}
Logistik ist ein wichtiger Teil der Produktionskette. 
Die Nutzung des Wasserstoffs am Flughafen erfordert Austausch der Betankungsanlagen und Anschaffung neuen Lieferketten, 
um Wasserstoff als Treibstoff benutzen zu können. Diese Investitionskosten werden die Flughafenbetreiber beeinträchtigen. \\
\textit{Transport}\\
Die Lieferketten und die Produktion des Wasserstoffs haben ein großes Gewicht in der vorhandenen Literatur.
Der Transport ist durch Pipelines im gasförmigen Zustand, LKW und Zügen sowohl im gasförmigen, als auch im flüssigen Zustand möglich. 
Kapitel \ref{ss:Wasserstoff-Antrieb} stellte die möglichen physischen Formen von Wasserstoff dar und am Ende darlegte,
dass flüssiger Wasserstoff für den Transport als im Gasform vorteilhafter ist. 
Außerdem kann es auch in einer chemischen Verbindung, wie Ammoniak und Methanol, gebunden und somit transportiert werden.
%
%
%Wasserstoff kann hochentzündlich sein \cite{dalmia2022powering}. %(prüfen ob entzündlicher als Kerosin)
%
% 
Die Effizienz der Produktion- und Lieferkosten ist geografisch determiniert. 
Die Lieferoptionen müssen nach Flughafenposition gewählt werden. Für die Flughäfen, die nahe einer Wasserstoff-Pipeline liegen ist sinnvoller
damit der Wasserstoff zu besorgen, als mit einem LKW transportieren zu lassen.
%
Für europäische Distanzen sind die Wasserstoff-Pipelines günstiger als Transport mit chemischen Verbindungen, 
welcher bei längeren Distanzen in Betracht kommt \cite{undertaking2022strategic}. 
Bei Änderung von Erdgasleitung für Wasserstoff können Kosten gespart
werden und muss keine neue Infrastruktur gebaut werden, sondern die Leitungen für Wasserstoff umgerüstet werden können \cite{undertaking2022strategic}.
%
Jedoch der Transfer von \ce{LH2} mit vakuumisolierte Pipeline beschränkt sich auf die kurzen Distanzen
wegen proportionalen Skalierung der Verluste zu Leitungslänge \cite{colpan2022fuel}.
Obwohl die hohen Kapitalkosten von Pipeline-Anlage zu erwarten, die Betriebskosten niedriger sein werden \cite{mulder2019outlook}.

Mulder et al. \cite{mulder2019outlook} vertritt der Ansicht, dass bei größerem Umfang an Wasserstoff die Pipelines 
vorteilhafter als LKW - Lieferung sind. 
Colpan et al.\cite{colpan2022fuel} ist jedoch der Meinung im Fall, wenn große Menge an Wasserstoff benötigt wird, 
ist der Lieferung weder mit LKW noch Pipeline sinnvoll. 

Dennoch nach Schenke et al. \cite{schenke2024lh2} kann Lieferung den flüssigen Wasserstoff mit einem LKW bei einer hohen Anzahl an Flügen 
kostengünstiger als andere Lieferalternativen sein. Allerdings Transport von \ce{LH2} erfordert speziell konstruierte Tanks \cite{mulder2019outlook}.

%Lieferung von gasförmigem Wasserstoff mit niedrigem Druck mit dem Straßenverkehr bis jetzt nur für ungenügende Menge möglich \cite{undertaking2022strategic}
%

\textit{Speicherung}\\
Gasförmiger Wasserstoff kann unterirdisch in Salzkavernen und in erschöpften Gasfelder gespeichert werden \cite{undertaking2022strategic}, 
Sie müssen sich in der unmittelbaren Nähe zum Flughafen befinden. Da es sich je nach Flughafenstandort variiert, wird dieser Speicheroption nicht weiter behandelt.
Außerdem kann es für die Lagerung ein oberirdischer Druckzylinder, wo flüssiger Wasserstoff oder als festen Materialien (wie Metallhybriden) gespeichert wird.
Aufgrund tiefen Temperaturen müssen diese Zylinder oder Tankern müssen gut isoliert und kryogen sein \cite{undertaking2022strategic}.
Andernfalls wird flüssiger Wasserstoff bei der Lagerung verdampft, was zum Verlust der Menge kommt \cite{undertaking2022strategic}. 
Die Verdampfung wird mit größeren Lager kleiner \cite{colpan2022fuel}.

Dennoch der größte Teil der Verdampfung findet durch Transferphase statt \cite{undertaking2022strategic}.
%Wasserstoff kann in Hochdrucktanks und Salzkavernen gelagert werden \cite{mulder2019outlook}
Somit muss der Weg zwischen Betankung und dem Speicher kurz sein, damit Verdampfungsverluste minimiert werden können \cite{colpan2022fuel}


Eine weitere mögliche Betankungsoption ist der Austausch des Flugzeugtanks als Kapseln. Dabei werden die leeren Kapseln an die 
Wasserstoffproduktionsstelle zurückgegeben, wo die wieder nachgefüllt werden können \cite{colpan2022fuel}. 
Diese Möglichkeit kann vor allem für die kleineren Flughäfen als Alternative sein, da kein Wasserstoffspeicher und 
sonstige Anlagen bereitgestellt werden müssen.
Für längeres Parken am Flughafen werden kalte Tanks eine sichere Verbindung mit der Wasserstoffinfrastruktur benötigen \cite{colpan2022fuel} %weiß nicht

%"current FCEV system costs higher than 200 €/kW for passenger cars but need 
%to fall below 50 €/kW for mass market. " \cite{undertaking2022strategic}


%neue terminals: In dem Fall, falls die Flugzeuge länger als übliche Modelle werden, die Sicherheitssperzone für die Betankung 20 Meter betragen kann \cite{gu2023hydrogen}
%
%
%"Flüssiger Wasserstoff muss jedoch bei Minusgraden gelagert werden, was eine Verbesserung der Speichertechnologien sowohl im Flugzeug selbst als auch auf Flughäfen erfordert."
%Wasserstoff kann entweder als Brennstoff für Verbrennung genutzt werden oder als Wasserbrennstoffzelle für die elektrischen Flugzeuge. \cite{dalmia2022powering}

In dem Unterkapitel \ref{Wasserstoff-Antrieb} wurde angeführt, dass die Produktion des Wasserstoffs, 
viel Platz Energie und hohe Kosten benötigt. Produktion und Verflüssigung des Wasserstoffs am Flughafen würde viel zusätzliche Infrastrukturkosten \cite{dalmia2022powering}.
Deswegen für die Flughäfen wäre es die Alternative besser, das Wasserstoff, woanders einzukaufen und zum Flughafen 
mit LKW oder Pipelines liefern zu lassen \cite{gu2023hydrogen}.
Aufgrund aufwendigeren und kostenintensiven Infrastrukturprozessen für die Produktion und Verflüssigung von 
Wasserstoff werden wahrscheinlich Flughäfen, 
besonders kleineren, anfangs auf die "On-Site" Produktion verzichten.

Aufgrund hohen Unterschiedes zu herkömmlichen Treibstoffen und neuen Ausrüstungen müssen die Mitarbeiter neu geschult werden, 
um mögliche Gefahren zu erkennen und zu vermeiden \cite{gu2023hydrogen}.

%In der Arbeit Dalmia et al. \cite{dalmia2022powering} wird die Produktion am Flughafen diskutiert. 
%Dazu wird einen Elektrolyseur für Erzeugung des gasförmigen Wasserstoffs, einen Kompressor und einen Tankwagen oder
%eine Pipeline mit modularem Tanksystem für Betankung der Flugzeuge benötigt. 
%Modulares Tanksystem kann in bestehenden Flugzeugen eingesetzt werden
%und wird als wie Fracht in das Flugzeug geladen. %(nochmal nachlesen)
%
%Für die Wasserstofferzeugung wäre die ideale Methode der Elektrolyse, die mit Strom betrieben würde, der vor Ort aus 
%erneuerbaren Ressourcen erzeugt würde, wie z. B. Solarenergie aus nicht reflektierenden Paneelen (um eine Blendung der 
%Solarmodule für Piloten zu vermeiden). Das Wasser, das für den Prozess verwendet wird, wird aus nahe gelegenen 
%Wasserquellen stammen, einschließlich Süßwasserflüssen und Seen. Diese Wasserstoffproduktion wird vor Ort an Flughäfen 
%stattfinden. Die benötigte Infrastruktur umfasst einen Elektrolyseur, einen Kompressor und einen Tankwagen. 
%Der Elektrolyseur wird gasförmigen Wasserstoff erzeugen. Der Kompressor erhöht dann den Druck des Wasserstoffs, 
%indem er sein Volumen für die Speicherung reduziert, und der Tankwagen transportiert den komprimierten Wasserstoff 
%zum Elektroflugzeug. Eine weitere Option für den Transport wäre die Installation einer landseitigen bis luftseitigen Pipeline,
% die den komprimierten gasförmigen Wasserstoff in einem modularen Tanksystem transportiert. Dieses System ermöglicht 
% die Nachrüstung von bereits im Einsatz befindlichen Flugzeugen. Die Wasserstofftanks würden ähnlich wie Fracht 
% in ein Flugzeug geladen, festgeschnallt und sicher mit dem Flugzeug verbunden. Wenn das Ziel erreicht ist, 
% wird der leere Tank durch einen neuen Tank ersetzt, der dann zurückgeschickt wird, um in der Wasserstoffproduktionsanlage 
% am Flughafen wieder aufgefüllt zu werden. Dieses System ermöglicht den Einsatz von Wasserstoff in allen Bereichen des Fluges,
% wodurch die Effizienz maximiert, das Gesamtgewicht reduziert und die Nutzlast und Reichweite verbessert werden.²³\cite{dalmia2022powering}
%
%Gerade wird Wasserstoff durch die Pipelines transportiert \cite{mulder2019outlook}. ist es so?
%Der Transport ist durch Pipelines im gasförmigen Zustand, LKW und Zügen sowohl im gasförmigen,
% als auch im flüssigen Zustand möglich. 
% Transport von LH2 erfordert speziell konstruierte Tanks \cite{mulder2019outlook}

%"Eine Kryopumpe 
%bringt den Wasserstoff auf den benötigten Druck von 1000 bar. " ?file:///C:/Users/henri/Downloads/765438.pdf

\textit{Annahmen für die Analyse}
Die Gesamtinvestitionen sind von der Wahl der Produktion, Speicherung, Lieferketten als auch Betankungsentscheidung abhängig.
Da derzeit nicht einsehbar ist, welcher Technologie umgesetzt wird, fokussiert sich die Arbeit auf einen bestimmten Versorgungsweg.

Die externe Produktion ist am Anfang sinnvoll \cite{colpan2022fuel}, aus diesem Grund wurde angenommen, dass die Produktion 
des Wasserstoffes und Verflüssigung nicht am Flughafen stattfindet, sondern eingekauft und zum Flughafen mit LKW transportiert.
Am Flughafen wird der Wasserstoff in kryogenen Tanks gespeichert und mit Betankungswagen werden die Flugzeuge mit Kraftstoff befüllt.
%
Lieferkosten für flüssigen Wasserstoff \ce{LH2} werden nicht explizit ausgerechnet,
da die schon in Betriebskosten von Wasserstoffbetriebenen Flugzeuge eingeschlossen sind.
%
Die Abbildung \ref{supply_wasserstoff} stellt so eine Variante von Produktion- und Lieferketten für flüssigen Wasserstoff dar.
\begin{figure}[h]
	\centering
	\includegraphics[width=0.9\linewidth]{Bilder/Supply_hydrogen.png}
	\caption[Lieferkette von flüssigem Wasserstoff mit externer Herstellung und interner Lagerung bzw. die Betankung]{Lieferkette von flüssigem Wasserstoff mit externer Herstellung und interne Lagerung bzw. die Betankung \cite{schenke2024lh2}}
	\label{supply_wasserstoff}
\end{figure}

Zusätzlich wird es davon ausgegangen, dass Sicherheitsradius beim Wasserstoffbetankung nicht erweitert wird.

%\section{Flugzeuge und Annahmen}
\subsection{Verglichene Flugzeuge und relevante Flugzeugdaten}
\label{ss:Relevante Flugzeugdaten}
%
Um den betrieblichen Unterschied zwischen konventionellen und neuartigen Antrieben aufzuzeigen, 
werden die Referenz-Flugzeuge mit neuen Konzepten verglichen. 
Der Wahl eines Antriebes ist von der Reichweite abhängig.\\
%
\textit{Konventionelle Flugzeuge}\\
%
Für den Vergleich mit elektrischen Batteriebetriebenen Flugzeugen wurde eine L410 festgelegt. 
Die L410 ist ein Zubringer-Flugzeug mit 19 Plätzen der Firma Aircraft Industries. 
Die moderne Version L410NG verfügt über neue Avionik und ist mit zwei GE H85-200 Triebwerken 
mit einer Wellenleistung von 850 (SHP) ausgestattet (Quellen).
Der Verbrauch einer L410 beträgt 240 kg/h \cite{let2016l410}. 
Sonstige für die Methodik wichtige Flugzeugdaten wurden in der Tabelle \ref{Flugzeuge} zusammengefasst.
Unter $V$ ist die \\ Reisegeschwindigkeit und unter $R$ ist die Reichweite eines Flugzeugs zu verstehen. 
$MTOW$ ist das Höchstabfluggewicht und $EOW$ (Empty Operating Weigth) ist die Betriebsleermasse eines Flugzeugs.

Für den Vergleich von größeren Distanzen wurde eine A321LR festgelegt. 
Die A321LR ist ein Schmalrumpfflugzeug von Airbus und ist eine 
Version der A321neo mit einer höheren Reichweite.
Das Flugzeug ist mit zwei Triebwerken ausgestattet, 
die einem maximalen Schub ($T_{T/O}$) von 33 kN haben \cite{eurocontrol_a321}.

\textit{Alternative Flugzeuge und Annahmen}\\
Die ES-19 von Heart Aerospace dient als Vergleich zur L410. 
Das Konzept hat einen rein elektrischen, batteriebetriebenen Antrieb.
Heart Aerospace hat die ES-19 zwar auf eine hybride Wasserstoffversion, die ES-30, umgerüstet, 
das Konzept der ES-19 wurde allerdings breit diskutiert und oft in wissenschaftlichen Arbeiten erwähnt. 
Das Flugzeug hat vier Triebwerke und sollte über eine Reichweite von 400 km verfügen, 
hierbei wird eine Reisegeschwindigkeit von 330 km/h erreicht  \cite{anker2023feasibility} \cite{heart_aerospace_es19}.
Für die ES-19 wird eine Batterie mit einer Kapazität von 720 kWh benötigt,
zuzüglich 30 \% der Reserveenergie resultiert das in 900 kWh \cite{donckers2024electric}. \\
Mit der Leistung heutiger Batterien wäre es unmöglich bei diesem Gewicht und dieser Distanz zu bleiben.
Deswegen wird angenommen, dass die Batterien sich positiv im Gewicht-zu-Leistungs-Verhältnis 
entwickeln und ein Kapazitätswert von 450 kWh/kg erreicht wird.
Manche Studien gehen davon aus, dass die Einsparungen der Wartungskosten 
von BA-Flugzeugen 10-15 \% erreichen können \cite{wangsness2021fremskyndet,avogadro2024demystifying}. 
Deswegen wird in dieser Arbeit eine Verminderung von 10 \% zu dem Referenzflugzeug einberechnet.
%
Da es bis jetzt nur wenig ausgearbeitete größere Konzepte für Wasserstoff-Antriebe gibt, 
wird der Betriebsvergleich auf Basis von einer A321LR stattfinden. 
Dabei wird angenommen, dass das Flugzeug mit Wasserstoffturbine betrieben ist.
Im Vergleich zu konventionellen Flugzeugen werden die Wasserstoff-Flugzeuge, 
die für Mittelstrecken geeignet, 14 \% höheres MTOW haben und die Kapitalkosten 
für das Kurzstrecken-Flugzeug um 7 \% sowie Wartungskosten um 6 \% steigen \cite{sky2020hydrogen}. 
Diese Anteile wirken sicht zwar positiv auf Mittel- und Langstrecken Flugzeuge aus, 
werden in dieser Arbeite dennoch angenommen.

Die Änderungen in den Abfertingsprozessen können erheblich sein \cite{ati_hydrogen_infrastructure}. 

Es ist zu erwarten, dass Wasserstoff-Flugzeuge länger als konventionelle Flugzeuge werden
und dass die Kraftstoffsicherheitszone bei Anschließen und der Trennung 
von der Wasserstoff-Betankung zum Jahr 2030 auf 20 Meter reduziert wird \cite{hoelzen2022h2}.

Die Flugzeit nimmt aufgrund des Wasserstofftank-Gewichts zwischen 5 und 15 \% zu \cite{sky2020hydrogen}. 
Aus diesem Grund in dieser Arbeit einen Wert von 10 \% angenommen.
Die Auslastung eines Flugzeugs mit Wasserstoff-Antrieb kann im Vergleich zu konventionellen Flugzeugen sinken, 
da WA-Flugzeuge wegen der Betankungsprozesse potenziell mehr Zeit auf dem Boden verbringen werden (Quelle). %Gu?
Der Vergleich zu SAF-Betriebskosten findet auch mit der A321LR statt. 
Es wird davon ausgegangen, dass der Unterschied nur bei den Treibstoffkosten entsteht. 
Für die A321LR wird die Passagieranzahl von 220 und Verbrauch von 1,7 kg pro Kilometer und pro PAX angenommen \cite{fonseca2022doc}.
%
In der Tabelle \ref{Flugzeuge} sind relevante charakteristische Werte 
und Annahmen für die Vergleichsflugzeuge zusammengefasst.
Anhand dieser Daten ist die ES-19 langsamer als ein L410, 
das bedeutet für die gleiche Strecke wird mehr Zeit benötigt, 
was am Ende die Auslastung eines Flugzeugs und somit die Betriebskosten verändern kann. 
Aufgrund des Batteriegewichts ist das BA Flugzeug schwerer als konventionelle Alternativen.
Die beiden Flugzeuge können die gleiche Anzahl an Passagieren befördern. 
Obwohl sich die Reisegeschwindigkeiten bei Referenz- und BA-Flugzeugen unterscheiden werden, 
für Kurzstrecken-Flüge ergibt sich keine erhebliche Differenz.
Deswegen wird angenommen, dass die batteriebetriebenen Flugzeuge ähnliche 
Auslastung wie konventionelle Flugzeuge aufweisen.
Auch eine Änderung der Lebensdauer von BA-Flugzeugen wird nicht erwartet \cite{reimers2018introduction}.
Eine A321LR erreicht mit oben genannten Annahmen höhrere Geschwindigkeiten und Reichweiten, sowie kleinere MTOW und EOW.
%
%
%%%% kommt in die 2_4_3 Wartung:
%Brennstoffzellen haben ebenfalls weniger bewegende Teile \cite{dalmia2022powering} 
%weshalb es in diesem Bereich zu weniger Wartungskosten kommen kann,
%allerdings hat der benötigte Wasserstofftank kürzere Wartungsintervalle (Quelle).

\begin{table}[h]
	\begin{center}
    \caption{Bewertete Flugzeuge: Werte und Annahmen}
	\label{Flugzeuge}
	\begin{tabular}{|l|c|c|c|c|c|c|}
		\hline
		 & \textbf{V} ~[\text{km/h}] & \textbf{R} ~[\text{km}] & \textbf{MTOW} ~[\text{kg}] & \textbf{EOW} ~[\text{kg}] & \textbf{PAX-Anzahl} 
		 & \textbf{Quellen} \\ \hline
		L410  & 417 & 2 570 & 7 000 & 4 120 & 19 & \cite{let_l410ng}\\ \hline
		ES-19 &  330 & 400 & 8 618 & - & 19 & \cite{anker2023feasibility} \cite{heart_aerospace_es19}\\ \hline
		A321LR & 1104 & 7 400 & 97 000 & 52 060 & max. 244 & \cite{airbus_a321neo} \cite{fonseca2022doc} \\ \hline
		WA & ~913 & - & - & 110 580 & 220 &\\ \hline
	\end{tabular}
    \end{center}
\end{table}

Dass die Anschaffungspreise die Betriebskosten beeinflussen, 
wurde bereits in \ref{s:Kosten} angeführt. 
Die Tabelle \ref{Flugzeugpreise} stellt die Verkaufspreise 
für konventionelle Referenz-Flugzeuge dar.
Da der Kaufpreis einer A321LR nicht zur Verfügung steht, 
wird auf den Listenpreis einer A321neo zurückgegriffen. 
Da sie aus einer Flugzeug-Reihe kommen, kann davon ausgegangen werden, 
dass die Preise ähnlich sind. Der Inflationsfaktor wurde einbezogen.
%Die Preise für alternative Antriebe sind wegen  nicht dargestellt

\begin{table}[h]
	\begin{center}
    \caption{Flugzeugpreise}
	\label{Flugzeugpreise}
	\begin{tabular}{|l|c|c|c|}
		\hline
		 & \textbf{L410} & \textbf{A321neo}  & \textbf{Quelle}  \\ \hline
		 Verkaufspreis ~[\text{EUR}] & 6 455 884 & 129,5 Mio &  \cite{marksel2023comparative} \cite{aerotelegraph_airbus}\\ \hline
	\end{tabular}
    \end{center}
\end{table}


%\section{Aufstellung der Formeln für Kosten}
\label{s:Aufstellung der Formeln für Kosten}

In dem Kapitel \ref{s:Kosten} sind die entstehenden Kosten beim Betrieb der Fluggesellschaft definiert.
Schließlich sind in diesem Unterkapitel die dazugehörigen Formeln anhand anderer Modelle vorgestellt und teilweise angepasst. 

Es gibt eine Reihe von Methoden, um DOC zu berechnen. 
Diese wurden in dem Teil \ref{s:Kosten} erwähnt.
Als Grundlage für ein Modell wurde die Association of European Airlines (AEA) 1989 gewählt, 
da sie häufige Anwendung in der akademischen Welt hat, 
sehr umfassend ist und Berechnungswerte für sowohl Kurz-, als auch Langstrecken hat. 
Falls die Quelle abweicht, wurde sie explizit angegeben. % MAX: was meinst du damit?

Die Betriebskosten werden mit konstanter Reisegeschwindigkeit und Verbrauch ohne 
Berücksichtigung erheblicher Energieverluste während des Starts und der Landung berechnet. 
Die Inflationsfaktoren stammen aus dem Verbraucherpreisindex Deutschland des Statistischen Bundesamts, 
entsprechend eingesetzte Werte sind in der Anlage XX zu finden. % MAX: anlage?
Alle Werte in USD werden mit dem Wechselkurs\footnote{Wechselkurs vom 14.01.2025} (1 EUR = 1,0245 USD)
umgerechnet, alle Werte in Pfund mit dem Kurs (1 GBP = 1,20 EUR).
Aufgrund der hohen Unsicherheit der Preisprojektionen wird der Preis für Kerosin 
konstant gehalten, es wird von einem Wert von 0,688 EUR pro Liter\footnote{Es wurde der } ausgegangen \cite{iata_industry_statistics_2024}. 
Stromkosten von 0,1976 € pro kWh werden ebenfalls als konstant angenommen \cite{eurostat_nrg_pc_205}.
In Anbetracht der Kostensenkung für nachhaltige Kraftstoffe in der Zukunft, 
wird der minimale Preis für HEFA-Treibstoff von 1,07 EUR pro Liter in den Berechnungen verwendet \cite{watson2024sustainable}.
Wegen der starken Schwankungen des Wasserstoffpreises wird der Durchschnittspreis für 
elektrolytisch hergestellten Wasserstoff von 4,88 EUR/kg eingesetzt \cite{hoelzen2022hydrogen}.
%
%Der Reservebedarf beträgt 30 \% der Gesamtkapazität.
%
\subsection{Betriebskosten einer Fluggesellschaft}

Die Gleichung \eqref{DOC} stellt die Betriebskosten $DOC$ der Flugzeugnutzung dar. 
Es werden sowohl variablen als auch ein Teil der Fixkosten betrachtet.
Diese bestehen aus Treibstoff-/Energiekosten $C_T$, 
Wartungskosten $C_W$, Entgelten und Gebühren $C_{EG}$, 
Kosten für Personal $C_{Crew}$ und kapitalgebundenen Kosten $C_{KK}$.
%
\begin{equation}
     {DOC ~[\text{EUR}]} = C_T + C_W + C_{Crew} + C_{KK} + C_{EG} \\
     \label{DOC}
  \end{equation}
%
Die Betriebskosten werden auf Basis von Blockstunden kalkuliert. 
Blockstunden setzen sich aus Flugzeit $t_{F,h}$ sowie der kumulierten Rollzeit $t_{R}$ 
von und zur Parkposition $t_{R}$ zusammen. 
Für Kurz- und Mittelstrecken beträgt $t_{R}$ {0,25 h} und für Langstrecken 0,42 h \cite{scholz_design_evaluation_doc}.

\begin{equation}
   t_{B} ~[\text{h}] = t_{F,h} + t_{R} \\
   \label{blockzeit}
\end{equation}

\textbf{Treibstoff-/ Energiepreise} hängen vom Treibstoff- bzw. Energiepreis selbst $P_{T/E}$ 
und vom Verbrauch eines Flugzeugs pro Blockstunde $m_{V,B}$ ab (vgl. \eqref{fuel}) ab.

\begin{equation}
   {C_T ~[\text{EUR}]} = (P_{T/E} \cdot m_{V,B}) \\
   \label{fuel}
\end{equation}

Die Treibstoff- bzw. Energiepreise $P_{T/E}$ sind in der Tabelle zusammengeführt.

Die \textbf{Wartungskosten} werden nach dem Jenkinson 1999 Modell berechnet. 
Das Modell ermöglicht es, grob aber schnell die Wartungskosten 
für ein Flugzeug abzuschätzen \cite{bruge2018wartungskosten}.
Da dieses Modell auf konventionelle Flugzeuge ausgearbeitet wurde, 
berechnet man die Wartungskosten für alternative Antriebe als Prozentanteil des Referenz-Flugzeugs, 
die in \ref{ss:Relevante Flugzeugdaten} angeführt sind.
Die Formeln stammen aus \cite{bruge2018wartungskosten} und beziehen sich auf das Jahr 1994, 
somit muss der Inflationsfaktor $k_{Infl}$ einkalkuliert werden. 
Die Berechnung liefert die Ergebnisse in USD, 
aus diesem Grund wird ein Wechselkursfaktor $k_{WK}$ in die Berechnung implementiert.
Wartungskosten werden normalerweise auf die Wartung der Flugzeugzelle $C_{W,Zelle,B}$ 
und der Triebwerke $C_{W,Triebwerk,B}$ aufgeteilt, 
wie in den Gleichungen \eqref{maintenance}-\eqref{maintenance engine} dargestellt. 
Die Wartungskosten der Zelle sind von dem leeren Betriebsgewicht 
(Operating Empty Mass $m_{OE}$) abhängig. 
Die Kosten des Triebwerks sind vom erzeugten Schub beim Start (Take-Off Thrust $T_{T/O}$) abhängig.

\begin{equation}
   {C_{W,B} ~[\text{EUR}]} = (C_{W,Zelle,B} + n_{T} \cdot C_{W,Triebwerk,B} ) \cdot t_{B} \cdot n_{F, Jahr} \cdot k_{WK} \cdot k_{Infl}\\
   \label{maintenance}
\end{equation}

\begin{equation}
   {C_{M,AF,b} ~[\text{EUR}]} = (175 \frac{USD}{h} + 0,0041 \frac{USD}{h} \cdot m_{OE} )\cdot k_{Infl}\\
   \label{maintenance Zelle}
\end{equation}

\begin{equation}
   {C_{M,E,L,b} ~[\text{EUR}]} = (0,00029 \frac{USD}{h} \cdot T_{T/O} )\cdot k_{Infl}\\
   \label{maintenance engine}
\end{equation}
%
Diese Formeln sind für Triebwerke mit einem Nebenstromverhältnis von 5:1 ausgelegt \cite{bruge2018wartungskosten}. %warum hier die 5?
Trotz der höheren Nebenstromverhältnisse die bei moderneren und größeren Flugzeugen gegeben sind,
wird aus Gründen der Vereinfachung diese Formel genutzt.\\
%
%

Zu \textbf{Entgelten und Gebühren} $C_{EG}$ gehören Flughafenentgelte 
(Passagier-, Lande- und Startentgelte, Sicherheitsentgelte sowie Abfertigung am Vorfeld)
und die Flugsicherungsgebühr. 
Die detaillierte Beschreibung der benutzten Entgelten und Gebühren sind in der Anhang A1.3 zu finden.\\ % MAX: anhang?

\textbf{Crewkosten} setzen sich aus Lohnkosten für Piloten $L_{Pilot}$ und Besatzung $L_{crew}$ zusammen. 
Die Anzahl der Besatzungsmitglieder $n_{crew}$ ist von der Anzahl der Passagiere abhängig. 
Gemäß den luftfahrtrechtlichen Bestimmungen ist pro 50 Passagiere ein Flugbegleiter notwendig \cite{conrady2019luftverkehr}.
Die Besatzungskosten sind mit einem Stundenlohn von 37 EUR für Flugbegleiter 
und 90 EUR für Piloten pro Blockstunden berechnet \cite{discover_airlines_cabin}.
Die Erklärung für die Kosten sind im Anhang XX zu finden. % MAX: anhang?
Die Werte setzen sich aus Grundgehalt und 75 Blockstunden pro Monat zusammen.

\begin{equation}
   {C_{crew} ~[\text{EUR}]} = ({L}_{Pilot} \cdot 2 + {L}_{crew} \cdot n_{crew} ) \cdot t_{B} \\
   \label{crew cost}
\end{equation}

%Analog zum \cite{marksel2023comparative} werden die Kosten für den Wasserstoff und Batterie Antrieb getrennt ausgerechnet, 
%wobei für Batterie-Antrieb anstatt eines Wasserstofftanks und eine Brennstoffzelle, basiert sich auf einer Batterie, Motor 
%und Leistungselektronik

%\begin{equation}
%   {P_{E,f} ~[\text{€}]} = P_{FC} \cdot + P_{EM} \cdot + m_F \\
%   \label{engine group price hydrogen}
%\end{equation}
Zu \textbf{kapitalbezogenen Kosten} gehören Abschreibungs-, Versicherungs- und Verzinsungskosten. 
Abschreibungskosten sind von den Anschaffungskosten bzw. dem Kaufpreis, 
der Abschreibungsdauer und den Blockstunden pro Jahr abhängig \cite{conrady2019luftverkehr}.
Die Abschreibungsdauer nach AEA beträgt jeweils 14 Jahre für 
Kurz- und Mittelstecken und 16 Jahre für Langstrecken.

Da keine öffentlich zugänglichen Marktpreise für Luftfahrzeuge mit alternativen Antriebssystemen vorliegen, 
wird die Kalkulation der Anschaffungskosten für elektrisch betriebene Flugzeuge 
nach der Methodik von \cite{monjon2020conceptual} durchgeführt. 
In der Arbeit wurde das Regressionsmodell \eqref{price BA} mit Abhängigkeit von Passagieranzahl $n_{PAX}$ 
anhand einer Marktanalyse für die Berechnung erstellt. 
%
In der Formel sind aufgrund der Einführung neuer Technologien 10 \% höhere Anschaffungskosten mitbetrachtet. 
Die Studie ist aus dem Jahr 2020, Preise wurden in USD berechnet, 
weshalb der Inflationsfaktor $k_{Infl}$ und der Wechselkurs $k_{WK}$ 
ebenfalls betrachtet werden.

\begin{equation}
   {C_{BA,ac} ~[\text{EUR}]} = (407408 \cdot n_{PAX} - 2967.4) \cdot k_{WK} \cdot {k_{Infl}}\\
   \label{price BA}
\end{equation}

Ein weiterer wichtiger Faktor der kapitalgebundenen Kosten ist die Auslastung $U$ eines Flugzeuges. 
Er wird durch die Anzahl der verfügbaren Stunden pro Jahr $t_{verf}$, 
der Block- $t_B$ und der Turnaround-Zeit $t_{TA}$ berechnet. 
Die jährliche Verfügbarkeitszeit $t_{verf}$ beträgt für Kurzstreckenflugzeuge 3750 h, 
während für Mittel- und Langstreckenflugzeuge ein Wert von 4800 h eingesetzt wird \cite{scholz_design_evaluation_doc}. 
Für die Kurzstrecken-Flugzeuge wurde eine Turnaround-Zeit von 1,5 h gewählt (Quelle).
Aufgrund der Größe und aufwendigeren Abfertigung wird für Mittel- und Langstrecken ein Wert von 2 h gewählt.

\begin{equation}
   U = \frac{t_{verf}}{t_B + t_{TA}} \\
   \label{utilisation}
\end{equation}

Verzinsungskosten $C_{Zins}$ sind durch den Prozentanteil der Anschaffungskosten bedingt, für diese gelten ca. 5 \%. 
Versicherungskosten sind hingegen von dem Kaufpreis eines Flugzeugs abhängig (inklusive Rabatte beim Kauf), 
in dieser Arbeit werden ebenfalls 5 \% angenommen \cite{scholz_design_evaluation_doc}. 
Weitere Formeln für kapitalgebundene Kosten sind in dem Anhang XX zu finden. % MAX: anhang?
%sind Anschaffungskosten geteilt durch 
%
%Für wasserstoffbetriebene Flugzeuge werden in der Quelle vorgestellte Formeln benutzt, um Kaufpreis einschätzen zu können.
%Außerdem bestehen bei alternativen Antrieben betriebliche Kosten, die durch Infrastruktur bedingt sind.
\subsection{Infrastrukturkosten}

\subsubsection{Batterie-Antrieb}
Kapitalkosten der Batterieantrieb-Infrastruktur sind in folgender Formel zusammengestellt:

\begin{equation}
     {CAPEX ~[\text{EUR}]} = C_{Bat} + C_{BSS} \\ % C_{Strom} + C_{W,Bat}
     \label{BA_Infrastruktur}
  \end{equation}

\begin{equation}
   \begin{split}
  {C_{Bat}} = P_{Bat} \cdot n_{Bat}  \\
 % {C_{Strom}} = n_{BSS} \cdot P_{Strom} \cdot N_{menge}\\ %brauche ich das?
  {C_{BSS}} = P_{BSS} \cdot n_{BSS} \\
  %(\frac{30}{Lebensdauer})\\
  %{C_{Hubwagen}} = P_{Hubwagen} \cdot n_{Hubwagen}
  \label{BA}
   \end{split}
  \end{equation}
%
$P_{Bat}$ stellt der Preis einer Batterie und $n_{Bat}$ die Anzahl der Batterien dar, die benötigt werden. 
Dazu werden noch 20 \% Reserve-Batterien einberechnet, um Einschränkungen im Betrieb zu umgehen. 
%$N_{menge}$ ist die Menge an Strom, der für das Laden einer Batterie nötig ist.
$P_{BSS}$ ist der Preis des Ladegeräts der Batteriewechselstation und $n_{BSS}$ ist der Anzahl an Ladegeräten. 
Stromkosten werden hier nicht betrachtet, da diese bereits in Betriebskosten 
pro Flug einberechnet werden und somit durch die Fluggesellschaft gedeckt werden. 
%Zusätzlich werden die Kosten für Batteriewechsel $P_{W,Bat}$ und Preis pro Hubwagen $P_{Hubwagen}$ mitbetrachtet.
Für die Berechnung wird eine lineare Beschreibung genutzt. 
Es ist nicht auszuschließen, dass für die ordnungsgemäße Lagerung 
von Batterien zusätzliche Anlagen wie bspw. Lagergebäude nötig sind.
  
%Energiekosten = Menge mal Preis
%
% Stromkosten = Anzahl der BSS mal die Leistung mal Kosten pro Einheit Spitzleistung mal Anzahl der Tage/30
%
% Anschaffungskosten BSS = Anzahl BSS mal Kosten mal Tage/Lebensdauer
 % An Optimization Model for Airport Infrastructures in Support to Electric Aicraft:
 % Um Batterienachfrage zu decken, wäre es möglich der Anzahl der Ersatz-Batterien zu erhöhen, die außerhalb der Spitzzeiten geladen werden 
 % oder der Anzahl der Batteriewechselstationen zu erhöhen. Es wird angenommen, dass für Flughafen 800 kW - Ladestationen zur Verfügung haben, 
 % (Studie, wo die Leistungen für
 % 3 hybride-Flugzeuge ausgerechnet wurden - insgesamt für 12 Flugzeuge sind 21 Batterien und 4 Ladestationen benötigt, 77,5 € pro Station, Peak power 3,2 MW, Batteriewechsel jede 1.51 Jahre
 % also eine Station für 3 Flugzeuge?). Natürlich in der Zukunft werden die Flugzeuge unterschiedliche Leistungen
 % und auch Batterien brauchen, was am Ende die Ladedauer und Anzahl der Ladestationen ändern kann
%
%10 \% für die Wartung der Systeme (wie bei OPTIMAL RECHARGING INFRASTRUCTURE SIZING AND OPERATIONS 
%FOR A REGIONAL AIRPORT)
%
%
\begin{table}[h]
	\begin{center}
    \caption{Werte und Annahmen der BA-Infrastruktur}
	\label{BA_Infrastrukturtab}
	\begin{tabular}{|l|c|c|}
		\hline
		 & \textbf{Werte} & \textbf{Quelle} \\ \hline
		$P_{BSS} ~[\text{EUR}]$ &  11 974  & \cite{guo2020aviation} \\ \hline
      Abschreibung $BSS ~[\text{Jahre}]$&  10  & \cite{salucci2020optimal} \\ \hline
		$P_{Bat} ~[\text{EUR/kWh}]$ & 125000 & \cite{guo2020aviation} \\ \hline
      Lebenszyklen $Bat$ & 5000 & \cite{reimers2018introduction} \\ \hline
      Abschreibung $Bat ~[\text{Jahre}]$& 2,7 & -\\ \hline
		%$P_{W,Bat} ~[\text{EUR}] $ & 285  & \cite{guo2023infrastructure}\\ \hline
    %  $P_{HW} ~[\text{EUR}] $ &  &\\ \hline
	\end{tabular}
    \end{center}
\end{table}
Die Anzahl der Batterien ist von der Gesamtanzahl der Abfertigungen $N_{Abfertigung}$ 
und Ladezyklen $c_{Batterie}$ einer Batterie abhängig. 
Zeitgleich ist die Anzahl der Zyklen, die eine Batterie an einem Tag geladen werden kann, 
von den Betriebsstunden des Flughafens und der Ladedauer der Batterie abhängig. 
In Zukunft ist eine Lebensdauer einer Batterie von bis zu 5000 und mehr Zyklen zu erwarten \cite{reimers2018introduction}, 
weshalb ein Wert von 5000 Zyklen in der Berechnung angenommen wird.
Die Ladedauer ist von der Leistung der Ladestationen und Kapazität einer Batterie abhängig.
Für die Ladegeräte wurde eine konstante Spitzenleistung von 250 kW angenommen \cite{salucci2020optimal}. 
%Der Strom ist normalerweise von Tag und Nacht abhängig und wie viel parallel genutzt wird.  Die Verluste beim Laden werden ignoriert. 
Zusätzlich wird der Puffer von 20 \% für die Batterieanzahl implementiert, um mögliche Engpässe zu vermeiden. 
Die Anzahl der Ladegeräte $n_{BSS}$ ist von der Gesamtzahl der Batterien und Ladezyklen abhängig.

Mit einer angenommenen Batteriekapazität von 900 kWh und Ladegeräten 
mit der Leistung von 250 kW, entspricht das 3,6 Stunden Ladedauer.
Bei einem 18 Stunden-Betrieb entspricht das insgesamt 5 Ladezyklen pro Tag für jede Batterie. 
Daraus kann abgeleitet werden, dass jede Batterie 1000 Tage genutzt werden kann,
bis ihre Leistung auf 80 \% sinkt.
Die Einbeziehung der Abschreibungswerte unterstützt die dritte Hypothese.
%
\begin{equation}
  {n_{Bat}} = {1,2} \cdot \frac{N_{Abfertigung}}{c_{Bat}}\\
  \label{BatAnzahl}
  \end{equation}
%
\subsubsection{Wasserstoff-Antrieb}
%
%Lieferung
%falls in flüssigform: LKW/Schiff/Bahn - Distanz und von der Menge abhängig.

Für Wasserstoff-Antriebe wird ein oberirdischer Tank für kryogenen flüssigen Wasserstoff \ce{LH2}, 
eine kryogene Pumpe zum Befüllen und Entleeren des Lagers ${kP}$ 
und ein Tankwagen ${BW}$ gebraucht (vgl. \eqref{WA_Infrastruktur}). 
Die Preise für die Anlage sind in der Tabelle \ref{BA_Infrastrukturtab} zusammengefasst. 
%
Der benötigte spezifische Energiebedarf der Infrastrukturelemente wie sowohl Betankungswägen 
oder kryogener Pumpen, als auch Wirkungsgrad der Technologien, wird dabei nicht betrachtet.

\begin{equation}
   {CAPEX ~[\text{EUR}]} = {P_{Lagertank}} + P_{kP} \cdot N_{kP} + P_{BW} \cdot N_{BW}  \\
   \label{WA_Infrastruktur}
\end{equation}

\begin{table}[h]
	\begin{center}
    \caption{Werte und Annahmen der Wasserstoffinfrastruktur}
	\label{WA_Infrastrukturtab}
	\begin{tabular}{|l|c|c|c|}
		\hline
		 & \textbf{Werte}& \textbf{Einheit}& \textbf{Quellen} \\ \hline
		Preis Lagertank $P_{Lagertank}$ & 41,9 & EUR/kg \ce{LH2}  & \cite{schenke2024lh2}\\ \hline
      Abschreibung Lagertank & 20  & Jahre  & \cite{hoelzen2023h2}\\ \hline
      Volumen Lagertank ~[\text{$m^3$}] & 4 732 & $m^3$ & \cite{fesmire2021lh2}\\ \hline
		Preis kryogene Pumpe $P_{kP}$ & 250171 & EUR/kg/h & \cite{hoelzen2022h2} \\ \hline
      Abschreibung ${kP}$ & 10 & Jahre & \cite{hoelzen2023h2} \\ \hline
		Preis Betankungswagen $P_{BW}$ & 87848 & EUR & \cite{hoelzen2022h2} \\ \hline
      Abschreibung ${BW}$ & 12  & Jahre  & \cite{hoelzen2022h2} \\ \hline
	\end{tabular}
    \end{center}
\end{table}

Wie bereits im Grundlagen-Kapitel erläutert wurde, 
hat der kryogene Wasserstoff eine Dichte von ca. 65 $kg/m^3$ \cite{colpan2022fuel},
also können 307,58 t in einem kugelförmigen Lagertank gespeichert werden.
Es wird davon ausgegangen, dass die Vorgänge der Betankung des Wagens % MAX: was ist A?
und die folgende Betankung des Flugzeugs jeweils 30 Minuten dauern \cite{hoelzen2022h2}. 
Zudem wird zur Vereinfachung angenommen, dass pro Flugzeug ein Tankwagen benötigt wird.
%
%Außerdem müssen diese Teile mit Spannung versorgt werden: Kontrollraum, Wartungseinrichtugen % MAX: ist das ein Satz?


\section{Betriebsszenarien}
\label{s:Betriebsszenarien}
Betriebsszenarien helfen die vorher beschriebene Betrieb- und Infrastrukturkosten in Anwendung zu bringen 
und eine mögliche Entwicklung in der nahen Zukunft zu zeigen. 
Die Größe des Flughafens beeinflusst die Infrastrukturkosten. 
Bei einem größeren Flughafen werden die Betriebsdifferenzen deutlicher, da das Verkehrsaufkommen wesentlich höher ist.
Größere Flughäfen abfertigen täglich mehr Flugzeuge als Regionalflughäfen, 
was dazu führt, dass mehr Abfertigungsplätze umgerüstet/versorgt werden müssen und mehr Arbeitskräfte geschult werden müssen.
%
Deshalb wird für die Betriebsszenarien der Flughafen Frankfurt gewählt, der fungiert als bedeutendes Luftverkehrsdrehkreuz
und zudem der größte Verkehrsflughafen Deutschlands.
Der Fraport meldete im Jahr 2023 insgesamt 423764 gewerbliche Flugbewegungen, was im Durchschnitt 1160 Flugbewegungen pro Tag ausmacht. 
Es wird angenommen, dass die Hälfte davon Abflüge sind, also müssen 580 Flugzeuge am Tag abgefertigt werden.
%
Die Gesamtbewegungen teilen sich nach Entfernungen folgend auf \cite{fraport2023frankfurt}:
\begin{itemize}
    \item Kurzstrecken (bis 2500 km) sind bei 72,8 \%;
    \item Mittelstrecken (bis 6000 km) sind 9,3 \%;
    \item Langstrecken (ab 6000 km) die restlichen 17,9 \%. 
    \end{itemize}
Da ist nicht explizit definiert wird, welche Entfernungen Flugzeug fliegt, werden die Betriebskosten anhand vorher beschriebene Distanzen berechnet.
Nämlich für Kurzstrecken wird eine Entfernung von 400 km genommen und für Langstrecken eine Distanz von 6000 km. Für Mittelstrecken
werden die Betriebskosten mit einer Distanz von 4000 Kilometer berechnet, jedoch es wird davon ausgegangen, dass Treibstoffverbrauch sich nicht ändert.
%
Anhand dessen wird eine Flotte mit 580 Flugzeugen aufgestellt, wo die alternativen Antriebe im Einsatz sind.
Aufgrund der Flugeinschränkungen in der Nacht wird es angenommen, dass die Flüge von 6 bis 24 Uhr gleichmäßig stattfinden. 
Wie bereits diskutiert wurde, können Kurzstrecken-Flüge durch den Einsatz von batteriebetriebenen Flugzeugen ersetzt werden. Hier wird
auch als Ersatz SAF mitberechnet. Es ist nennenswert, dass nur Teil der tatsächlichen Nachfrage des Kurzstrecken-Bedarfs 
dadurch gedeckt werden kann. Die Mittel- und Langstrecken werden durch Flugzeuge mit Wasserstoffturbine und SAF versorgt/erfüllt.
%
In Betrachtung des tatsächlichen Flugplans sind die Spitzenstunden im Laufe des Tages zu finden, wo
der Verkehrsfluss stärker als im Durschschnitt ist. In diesem Fall werden höhere Infrastruktur- und Betriebskosten zu erwarten.
Indessen um die Interpretation zu erleichtern, wird in dieser Arbeit angenommen, dass stündlich die gleiche Anzahl an Flugzeugen 
am Flughafen abgewickelt werden. 
%
Die Aufteilung den Antrieben für jedes Szenario ist in der Abbildung \cite{betriebsszenarien} dargestellt.
%
\begin{figure}[h]
	\centering
	\includegraphics[width=1.0\linewidth]{Bilder/Betriebsszenarien.png}
	\caption[Betriebsszenarien]{Aufteilung der Flugzeugflotte nach Antriebsart}
	\label{betriebsszenarien}
\end{figure}

In dem \textbf{ersten Betriebsszenario} wird angenommen, dass die Kurzstrecken durch die BA komplett ersetzt werden.
Die Mittelstrecken werden vollkommen durch WA und die Langstrecken durch die SAF bedient.

Das \textbf{zweite Betriebsszenario} wird mit folgender Aufteilung berechnet. 
Die Hälfte der Kurzstrecken wird durch BA versorgt und die andere Hälfte durch SAF; 
die Mittelstrecken werden genauso, wie im ersten Szenario komplett durch die Wasserstoffflugzeuge bedient und 
bei Langstrecken sind 10 \% SAF und den restlichen Anteil Wasserstoff.

Das \textbf{dritte Szenario} 50 \% der Kurzstrecken sind von Batterie-Antrieb, die restlichen 22,8 \% sind mit SAF betrieben.
Mittelstrecken: die Hälfte der Mittelstrecken sind mit WA und die andere Hälfte mit SAF
Langstrecken: die Hälfte der Mittelstrecken sind mit WA und die andere Hälfte mit SAF


\chapter{Auswertung der Ergebnisse und kritische Auseinandersetzung}
\label{ch:Auswertung der Ergebnisse}
Anhand der in dem Teil \ref{ch:Änderungen durch neue Antriebe, Annahmen und Methodik} 
vorgeschlagenen Methodik wurde in diesem Kapitel der Vergleich zwischen 
Referenz-Flugzeugen und alternativen Antrieben geschaffen.
Außerdem werden aufgestellte Betriebsszenarien sowohl ausgewertet, 
schließlich diskutiert und mit anderen Arbeiten verglichenen, 
als auch auf die Vorschläge für andere Arbeiten eingegangen.

\section{Vergleich von Referenzflugzeugen und neuen Antrieben}
\label{s:Ergebnisse_Flugzeuge}
Folgende Erkenntnisse ermöglichen Diskussionen der ersten Hypothese. 
Die Ergebnisse sind nach Flugdistanzen aufgeteilt.\\
%
\subsubsection{Kurzstreckenvergleich: Batterieantrieb und SAF vs. konventioneller Treibstoff}
%
In der Abbildung \ref{vergleichBA_Ref} sind die Ergebnisse der 
batteriebetriebenen ES-19 und der konventionellen L410 dargestellt.
Der Vergleich wurde für 400 Kilometer Flug durchgeführt.
%
Die Gesamtbetriebskosten der ES-19 sind ca. 22 \% höher als die der konventionellen L410, 
wobei der Antrieb mit SAF ein nur 3,6 \% höhere Ergebnis liefert. 
Entgelte und Gebühren bewirken den größten Teil der Betriebskosten aller verglichenen Antriebe, 
gefolgt von kapitalbezogenen Kosten. 
Beim ersten ist ein signifikanter Unterschied zum konventionellen Antrieb zu erkennen, 
und zwar 56 \%, beim zweiten sind das 36 \% höhere Kosten bei Batterieantrieb. 
Der Anteil der Treibstoff- bzw. Energiekosten ist am niedrigsten im Vergleich zu anderen Kosten 
und die Treibstoffkosten sind bei einem konventionellen Flugzeug 30 \% höher. 
Die Treibstoffkosten für SAF sind wesentlich höher (38 \%).

\begin{figure}[h]
	\centering
	\includegraphics[width=0.9\linewidth]{Bilder/VergleichBA_Ref.png}
	\caption[Betriebskosten]{Vergleich der Referenz und Flugzeug mit der Batterieantrieb und SAF}
	\label{vergleichBA_Ref}
\end{figure}

Zusätzlich zu den Entgelten und Gebühren fallen weitere Abfertigungskosten an,
etwa für den Batteriewechsel oder das Leasing der Batterie für den Flug,
die in die Flughafenentgelte einfließen.
Dieser Wert wird auch in der Sensitivitätsanalyse \ref{s:Sensitivitätsanalyse} überprüft. 
Der Einflusswert für kapitalbezogene Kosten ist der Anschaffungspreis eines Flugzeugs. 
Dieser Wert wurde als weiterer Parameter für die Sensitivitätsanalyse ausgewählt.\\

\paragraph{Langstreckenvergleich: Wasserstoffantrieb und SAF vs. konventioneller Treibstoff}
In den weiteren Abschnitten wird eine Gegenüberstellung zwischen Flugzeugen mit herkömmlichen Treibstoffen, 
SAF und wasserstoffbetriebener Turbine für einen 6000 Kilometer-Flug durchgeführt. %satz übelstschlecht MAX, nein, war okay :)
Aufgrund der derzeit hohen Wasserstoffpreise wurde zum Vergleich der Wasserstoffmindestpreis 
von 2,1 $EUR/kg$ für das Jahr 2050 herangezogen \cite{hoelzen2022hydrogen}, 
um die mögliche Entwicklung der Wasserstofftechnologien zu beobachten.
%
Das Ergebnis zeigt, dass SAF- und Wasserstoffbetriebene Flugzeuge höhere Betriebskosten haben (siehe \ref{vergleichWA_Ref}).
Die Betriebskosten der Wasserstoffturbine sind rund 40 \% höher, während sie bei SAF um etwa 19 \% höher liegen.
Treibstoff- und kapitalbezogene Kosten haben den größten Einfluss auf die Gesamtkosten, 
Wartungs- und Crewkosten entgegen den geringsten.
Der Einfluss der Entgelte und Gebühren auf die Gesamtkosten ist nicht so groß, 
wie bei den kleineren verglichenen Flugzeugen.
Die Treibstoffkosten für WA sind fast doppelt so hoch (ca. 197 \%), als der für herkömmliche Triebwerke. 
Die Entgelte und Gebühren, Crew- und Wartungskosten haben einen nicht so 
drastischen Unterschied verglichen mit konventionellen Antrieben. 
%
Wird die Entwicklung zukünftiger Preise für Wasserstoff mitbetrachtet, 
kann es sogar zu geringeren Treibstoffkosten im Vergleich zu konventionellen Antrieben führen. 
Die Gesamtbetriebskosten werden weiterhin höher liegen.

%mögliche Hypothese: zukünftige Preisniveau ermöglicht die günstigere Preiswerte
\begin{figure}[h]
	\centering
	\includegraphics[width=0.9\linewidth]{Bilder/VergleichWA_SAF.png}
	\caption[Betriebskosten]{Vergleich der Referenz und Flugzeug mit der Wasserstoffantrieb und SAF für einen 6000 km Flug}
	\label{vergleichWA_Ref}
\end{figure}

Da die Treibstoffkosten einer der größten Teile in den Ergebnissen ausmachen, 
wird der bedeutsame Parameter Treibstoffpreis für die Sensitivitätsanalyse ausgewählt.

Werden Betriebskosten pro Passagierkilometer ausgerechnet und verglichen, 
haben die großen Flugzeuge geringere Kosten als die kleinen Flugzeuge.
Wie auch vorher resultiert wurde, haben hier konventionelle Treibstoffe 
einen Vorteil sowohl gegenüber den kleinen, als auch großen Flugzeugen.

\section{Ergebnisse der Betriebsszenarien}
In dem Kapitel \ref{s:Betriebsszenarien} wurden drei Szenarien für den Vergleich aufgestellt, 
in diesem Abschnitt werden sie ausgewertet und miteinander verglichen. 
Erstens werden die Betriebskosten unter sich verglichen, 
zweitens die benötigten Infrastrukturkosten für jedes Szenario und schließlich 
werden die Szenarien unter Berücksichtigung beider Aspekte betrachtet. 
Die detaillierten Ergebnisse für die Betriebsszenarien sind in Anhang B1.2 zu finden.
%
\subsubsection{Betriebskosten}
Obwohl der Unterschied der Betriebskosten gleichmäßig ist, 
hat unter allen Betriebsszenarien das zweite die höchsten Gesamtbetriebskosten (siehe Abb. \ref{res_betriebsszenarien}). 
Das Erste hat die geringsten Kosten und ist nur etwa 4 \% geringer als das Zweite.
Erkennbar ist auch, dass die Betriebskosten, die durch Batterieantrieb verursacht werden, 
am geringsten unter allen Szenarien sind.
In dem \textbf{zweiten Szenario} entsteht Großteil (57 \%) der Kosten durch wasserstoffbetriebene-Flugzeuge. 
Weitere 38 \% verursacht das SAF.
\textbf{Erstes Szenario} hat den niedrigsten Betriebswert. 
%Die Kosten des ersten Szenarios teilen sich folgend auf. 
Der Betrieb mit SAF verursacht in diesem Szenario die meisten Kosten, nämlich 61 \%.
Etwa ein Drittel der Kosten sind mit dem Betrieb durch Wasserstoff verbunden. Batteriebetriebene 
Flugzeuge haben in diesem Szenario die geringsten verbundenen Betriebskosten (12 \%).
Das \textbf{dritte Szenario} hat 1 \% geringere Kosten als das zweite und somit 
keine signifikante Differenz aufweist. 
Der stärkste Einfluss in diesem Fall geht, ebenso wie im ersten Szenario, 
vom Betrieb mit Wasserstoff aus (48 \%), gefolgt von SAF (44 \%).
%
\begin{figure}[h]
	\centering
	\includegraphics[width=0.8\linewidth]{Bilder/betriebssz_res.png}
	\caption[Betriebskosten in Abhängigkeit von Szenarien mit Gesamtkostentrend]{Betriebskosten in Abhängigkeit von Szenarien mit Gesamtkostentrend}
	\label{res_betriebsszenarien}
\end{figure}
%
\subsubsection{Infrastrukturkosten}
In der Tabelle \ref{Infrastrukturwerte_res} sind die benötigten 
Infrastrukturanschaffungswerte für jedes Szenario zusammengefasst. 
Die Werte wurden anhand der vorgeschlagenen Methodik ermittelt.\\
%
\begin{table}[h]
	\begin{center}
    \caption{Infrastrukturwerte für Wasser- und Batterieantrieb für alle Szenarien}
	\label{Infrastrukturwerte_res}
	\begin{tabular}{|l|c|c|c|}
		\hline
		 & \textbf{Szenario I}& \textbf{Szenario II}& \textbf{Szenario III} \\ \hline
		Anzahl Ladestationen $n_{BSS}$ & 20 & 10& 14\\ \hline
		Anzahl Batterien $n_{Bat}$ & 101 & 51& 70 \\ \hline
		Anzahl Betankungswagen $n_{BW}$ & 4 & 7 & 5\\ \hline
		Anzahl Pumpen $n_{kP}$  & 5 & 8 & 6\\ \hline
	\end{tabular}
    \end{center}
\end{table}

Nennenswert ist, dass in der Berechnung von allen Betriebsszenarien erstmal nur einmalige Infrastrukturausgaben 
ohne jährliche Abschreibungen ausgerechnet sind (siehe Abb. \ref{res_infr_betriebsszenarien}). 
Das erste Szenario hat die geringsten Ausgaben, wobei das zweite Szenario die höchsten (27 \% höher als das Erste) hat.
Die Gesamtkosten für das zweite Szenario liegen bei über 35 Tausend Euro. %Die konkreten Ergebnisszahlen sind in der Anhang XX zu finden.
Im ersten Szenario zeigt sich, dass die Preisaufteilung für Batterie- und Wasserstoffinfrastruktur nahezu gleiche Anteile aufweist.
In den anderen Szenarien führen die Kosten für die Wasserstoffinfrastruktur.\\
\begin{figure}[h]
	\centering
	\includegraphics[width=0.8\linewidth]{Bilder/Infr_Szenarien.png}
	\caption[Vergleich der einmaligen Infrastrukturausgaben zwischen den Betriebsszenarien]{Vergleich der einmaligen Infrastrukturausgaben zwischen den Betriebsszenarien. Die schwarzen Linien
	zeichnen minimale und maximale Reduzierungen basierend auf den aufgestellten Szenarien}
	\label{res_infr_betriebsszenarien}
\end{figure}

Wird die lineare jährliche Abschreibung mitbetrachtet, 
zeigen die Ergebnisse eine andere Entwicklung (siehe Tab. \ref{res_abschr_betriebsszenarien}]. 
\begin{figure}[h]
	\centering
	\includegraphics[width=0.8\linewidth]{Bilder/infr_abschreibung.png}
	\caption[Vergleich der jährlichen Infrastrukturausgaben zwischen den Betriebsszenarien]{Vergleich der jährlichen Infrastrukturausgaben zwischen den Betriebsszenarien}
	\label{res_abschr_betriebsszenarien}
\end{figure}
Die erheblichsten Kosten bei den Abschreibungswerten entstehen im ersten Szenario, 
wobei dieses Szenario bei einmaligen Ausgaben das beste Ergebnis erzeugt.
Die geringsten jährlichen Abschreibungskosten würde das zweite Szenario liefern, 
obwohl es die höchsten einmaligen Ausgaben hat. 
Demnach kann eine ungünstige Wahl der Abschreibungsverteilung die anfangs günstigeren Ausgaben 
im Vergleich zu anderen Szenarien steigen lassen, womit die dritte Hypothese bewiesen wurde.\\


%

Um die Ergebnisse zu vertiefen und die Arbeit detaillierter zu erforschen, können die je Kosten je Entfernung berechnet werden.
Das ermöglicht auf die tatsächliche Flotte rüberzutragen.
Als anderes Aspekt für die Vertiefung wäre interessant anzuschauen, welche konkrete Unterschiede in Emissionen durch alternative Antriebe entstehen
und welcher Unterschied in Flughafen-Entgelten und möglich zukünftige politischen.
%\documentclass[11pt]{article}
%\usepackage{cite}

%\begin{document}

%\title{Bibtex}

%\maketitle

%jzjzjz \cite{conrady2019luftverkehr}

%\bibliography{Quellen.bib}
%\bibliographystyle{plain}
%\end{document}

\phantomsection\addcontentsline{toc}{chapter}{Literaturverzeichnis}
% Einbinden der Bibliothek

% cite everything in the bib with:
\nocite{*}

{
	\small % reduces fontsize
	\printbibliography
	\normalsize % resets fontsize back to normal
}



\end{document}
