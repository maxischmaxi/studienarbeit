\section{Bewertung der Ergebnisse}
\label{s:Bewertung der Ergebnisse}

\subsubsection{Interpretation der Ergebnisse}
%Betriebskosten
Die Ergebnisse dieser Studie liefern wichtige Erkenntnisse für den Betrieb mit alternativen Antrieben.
Aus der Analyse der Betriebskosten, in der die konventionellen Flugzeuge mit neuen Antrieben verglichen wurden, 
lässt sich zusammenfassen, dass neuartige Antriebe zu höheren Betriebskosten für eine Fluggesellschaft führen werden.
Demzufolge kann die Hypothese, dass die Einführung innovativer Antriebe zu höheren 
Betriebskosten einer Fluggesellschaft im Vergleich zu herkömmlichen Jettriebwerken führt, bestätigt werden. 
Das liegt vor allem an den technischen Anforderungen und neuen zusätzlichen Abfertigungsprozessen, 
die mit neuartigen Kraftstoffen eingeführt werden müssen. 
Avogadro et al. \cite{avogadro2024demystifying} kommt zu gleichen Ergebnis, 
dass batteriebetriebene Flugzeuge in der näheren Zukunft höhere Betriebskosten aufweisen werden. 
Marksel et al. \cite{marksel2023comparative} geht in ihrem Vergleich vom konventionellen Flugzeug 
und der Wasserstoffbrennzelle davon aus, 
dass die Betriebskosten sogar unter den marktüblichen Kerosin-Jets vordringen werden. 
Jedoch ist das nicht in den nächsten zehn Jahren zu erwarten. 
Die Entwicklung von Ölpreisen ist hier von Bedeutung, 
da dadurch die Attraktivität der neuen Technologien beeinflusst werden kann.
%
Höhere Kosten für elektrische Flugzeuge sind vor allem von Anschaffungspreisen für die Flugzeuge abhängig. 
Die Sensitivitätsanalyse zu diesem Parameter zeigte, dass das Modell den Änderungen gegenüber robust reagiert.
In der Arbeit wurde das Konzept des Batterie-Leasings am Flughafen vorgeschlagen. 
Unter realen Bedingungen ist es möglich, dass die Batteriemodule bei Anschaffung 
der Flugzeuge ebenfalls erworben werden müssen, 
was letztendlich die Antriebskosten der Fluggesellschaften steigern können. 

%
Die Analyse der Betriebskosten in unterschiedlichen Betriebsszenarien vermittelt, 
dass sich Verteilung der Betriebskosten je nach ausgewähltem Antrieb unterscheidet. 
Sollte man die Flottengröße oder die Zusammensetzung modifizieren, 
würden unterschiedliche Ergebnisse zum Vorschein kommen.
Darauf basiert die zweite Hypothese. 

Während größere Fluggesellschaften eher eine heterogene Flotte zusammenstellen, 
setzen Low-Cost-Carrier eher auf eine homogene Flotte.
Somit werden sich die Kosten je nach Strategie verschieben.
Werden die Betriebskosten zusammen mit Infrastrukturkosten für jedes Szenario angeschaut, fällt auf, 
dass die Infrastrukturkosten sich proportional zu Betriebskosten verhalten. 
Für beide hat das zweite Szenario die höchsten Kosten und das erste die geringsten. 
Die Betriebskosten unterscheiden sich prozentual um 3,5 \%, 
während die Infrastrukturkosten 27 \% höher ausfallen.
Obwohl kein direkter Zusammenhang zwischen Betriebskosten und Infrastrukturkosten besteht, 
kann die Auswahl der eingesetzten Technologien seitens der Flughafenbetreiber einen Einfluss darauf haben, 
welche Antriebsarten von Fluggesellschaften gewählt werden. Somit wurde die zweite Hypothese widerlegt.
%Details
Die Batterieantriebe verursachen in den Szenarien zwar die geringsten Kosten,
jedoch ist nicht zu vernachlässigen, dass der Antrieb ungenügendes Sitzangebot bietet
und die Batterien eingeschränkte Reichweite haben, 
womit die Nachfrage nach Kurzstrecken nicht vollständig gedeckt werden kann.

%Betriebsszenarien - Infrastrukturkosten

Wie bereits in den Ergebnissen ausgewertet wurde, sind im zweiten Szenario die größten einmaligen Infrastrukturkosten zu erwarten.
Das liegt daran, dass dieses Szenario die größten Anschaffungseinheiten für den Wasserstoffantrieb hat. 
Das erste Szenario liefert hingegen das kleinste Ergebnis, dabei ist die Kostenverteilung zwischen Batterienantrieb 
und Wasserstoffantrieb etwa gleich groß.
Wird die lineare Abschreibung zusätzlich betrachtet, ist eine Differenz in den Ergebnissen zu sehen. 
Die jährlichen Abschreibungskosten sind in dem ersten Szenario am höchsten 
und hingegen zu den Gesamtinfrastrukturkosten erzeugt hier das zweite die geringsten jährlichen Infrastrukturkosten.
Das ist bedingt durch eine für das erste Szenario ungünstige Abschreibungsstruktur. 
In diesem Fall muss beobachtet werden, ob sich die anderen Abschreibungsmethoden optimaler entwickeln.
Folglich ist eine weitere Hypothese, dass Abschreibungsmethode bewirkt, 
dass die jährlichen Kosten trotz hoher Investitionskosten belastbarer und niedriger sind, 
während sich bewiesen hat, dass eine ineffiziente Kostenverteilung bei geringeren Anschaffungskosten zu 
höheren jährlichen Belastungen führen kann.

%\textit{Beschränkungen der Forschung}
%
Trotz der Ergebnisse unterliegt diese Arbeit einigen Beschränkungen. %Aufgrund begrenzter Zugang zu relevanten Daten.
Diese Arbeit hat sich nur auf Betriebskosten der Fluggesellschaften und Anschaffungskosten für die Infrastruktur,
jedoch nicht auf die Betriebskosten der Infrastruktur konzentriert. 
Außerdem sind am Flughafen zahlreiche weitere Stakeholder beschäftigt, 
was insbesondere für Bodendienstleister zusätzliche Herausforderungen mit sich bringt.

Die Komplexität der Technologien des realen Systems wurde durch vereinfachte Annahmen reduziert, 
um eine fokussierte Analyse zu ermöglichen. 
Wegen nicht vorhandener Daten in der technologischen Entwicklung von neuen Antrieben wurden einige Annahmen getroffen.
Zudem hat diese Arbeit in begrenztem Untersuchungszeitraum stattgefunden, 
wodurch manche Variablen vernachlässigt werden mussten.
%
Angesichts des Mangels an relevanten Daten war es nicht möglich die Ausbildungskosten bei der Einführung neuer Antrieben zu berechnen.
Es lässt sich eindeutig feststellen, dass hier eine Forschungslücke vorhanden ist, welche noch zu erkunden ist.
Trotz gesendeter Anfragen zu diesem Thema an einige bedeutende Unternehmen blieb eine Antwort bislang aus.
%

%\textit{Vorschläge für eventuelle weiterführende Forschungen}
Um die Ergebnisse zu vertiefen und die Arbeit detaillierter zu erforschen, können die Kosten für jede Entfernung berechnet werden.
Das ermöglicht, die Berechnungen auf eine tatsächliche Flotte zu übertragen und Preisanstiege nach Entfernung zu ermitteln.
Als weiteren Aspekt für die Vertiefung wäre interessant zu betrachten, welche konkreten Unterschiede in Emissionen durch 
alternative Antriebe entstehen, welcher Unterschied sich in Flughafen-Entgelten bildet und welche Kosten dadurch
für Luftfahrzeugbetreiber ergeben. Vor allem durch mögliche zukünftige politische Entscheidungen.

%Im Vergleich zu anderen Untersuchungen haben sich die Ergebnisse...
%Es gibt Werke, die sich detaillierter mit Infrastruktur auseinandersetzen und mehr Komponenten, als auch z.B. Installationsprozesse
%mitbeziehen. %(wie file:///C:/Users/henri/Downloads/765438.pdf)

