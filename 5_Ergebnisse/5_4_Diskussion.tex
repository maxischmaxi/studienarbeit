
\section{Bewertung der Ergebnisse}
\label{s:Bewertung der Ergebnisse}

\textit{Interpretation der Ergebnisse}\\
%Betriebskosten
Die Ergebnisse dieser Studie liefern wichtige Erkenntnisse für den Betrieb mit alternativen Antrieben.
Aus der Analyse der Betriebskosten, in der die konventionellen Flugzeuge mit neuen Antrieben verglichen wurden, 
lässt sich zusammenfassen, dass die neuartigen Antriebe höhere Betriebskosten für eine Fluggesellschaft bringen werden.
Demzufolge kann die Hypothese, dass die Einführung innovativer Antriebe zu höheren Betriebskosten einer Fluggesellschaft im Vergleich zu herkömmlichen
Jettriebwerken führt, bestätigt werden. Das liegt vor allem am technischen Anforderungen und neuen zusätzlichen Abfertigungsprozessen, die mit neuartigen
Kraftstoffen dazukommen. Avogadro et al.\cite{avogadro2024demystifying} kommt zu gleichem Ergebnis, 
dass in der näheren Zukunft batteriebetribene Flugzeuge höhere Betriebskosten aufweisen werden. Und Marksel et al. \cite{marksel2023comparative} in ihrem
Vergleich von konventionellem Flugzeug und Wasserstoffbrennzelle, geht davon aus, dass die Betriebskosten sogar unter dem marktüblichen Kerosinjets
Flugzeuge vordringen werden. Jedoch die nächsten 10 Jahre ist das nicht zu erwarten. 
Die Entwicklung von Ölpreisen sind von der Bedeutung und beeinflussen
die Attraktivität der neuen Technologien.
%
Die höheren Kosten für elektrische Flugzeuge sind vor allem von Anschaffungspreise für
die Flugzeuge abhängig. Die Sensitivitätsanalyse zu diesem Parameter zeigte, dass das Modell den Änderungen gegenüber sensibel reagiert.
In der Arbeit wurde der Konzept des Batterie-Leasing am Flughafen vorgeschlagen. Unter realen Bedingungen ist es möglich, 
dass die Batteriemodule bei Anschaffung der Flugzeugen mitbesorgt werden müssen, was endlich die Antriebskosten der Fluggesellschaften steigern
können. 


Betriebsszenarien - Betriebskosten
Die Analyse der Betriebskosten unter unterschiedlichen Betriebsszenarien vermittelt, dass Verteilung der Betriebskosten sich je nach
ausgewählte Antrieb unterscheidet sich. Sollte Flottengröße oder Zusammensetzung sich modifizieren, würden unterschiedliche Ergebnisse zum Vorschein kommen.
Darauf wurde die zweite Hypothese basiert. 

Wobei die größeren Fluggesellschaften sich eine heterogene Flotte aussuchen werden, werden Low-Cost-Carrier eher eine homogene zusammenstellen.
Somit je nach Strategie werden sich die Kosten verschieben.
Werden die Betriebskosten zusammen mit Infrastrukturkosten für jedes Szenario angeschaut, fällt auf, dass die Infrastrukturkosten sich proportional 
zu Betriebskosten verhalten. Für die beiden hat das zweite Szenario die höchsten Kosten und das erste die geringsten. 
Die Betriebskosten unterscheiden sich prozentual um 3,5 \%, während die Infrastrukturkosten um 27 \% höher ausfallen.
Obwohl kein direkter Zusammenhang zwischen Betriebskosten und Infrastrukturkosten besteht, 
kann die Auswahl der eingesetzten Technologien seitens der Flughafenbetreiber einen Einfluss darauf haben, 
welche Antriebsarten von Fluggesellschaften gewählt werden. Somit wurde sich die zweite Hypothese widerlegt.
%Details
Die Batterieantriebe verursachen zwar in der Szenarien die geringsten Kosten. 
Jedoch darf es nicht vergessen werden, dass die Batterien eingeschränkte Reichweite haben, als auch
ungenügende Sitzangebot und damit können nicht vollständig die Nachfrage von Kurzstrecken decken.

Betriebsszenarien - Infrastrukturkosten

Wie bereits in den Ergebnissen ausgewertet wurde, sind bei zweitem Szenario die größten einmaligen Infrastrukturkosten zu erwarten.
Das liegt daran, dass dieses Szenario die größten Anschaffungseinheiten für den Wasserstoffantrieb haben. Das erste Szenario liefert
hingegen das kleinste Ergebnis, dabei ist die Kostenverteilung zwischen BA und WA ist etwa gleich groß.
Wird die lineare Abschreibung zusätzlich betrachtet, ist eine Differenz in der Ergebnissen zu sehen. 
Die jährlichen Abschreibungskosten
sind in dem ersten Szenario am höchsten und hingegen zu Gesamtinfrastrukturkosten bringt hier das zweite die geringsten jährlichen Infrastrukturkosten.
Das ist bedingt durch ungünstige für erstes Szenario Abschreibungsstruktur. In diesem Fall soll beobachtet werden, ob die andere Abschreibungsmethode optimaler werden.
Folglich ist weitere Hypothese, dass Abschreibungsmethode bewirkt, dass trotz hoher Investitionskosten die jährlichen Kosten belastbarer 
und niedriger sind, während bei geringeren Anschaffungskosten eine ineffiziente Kostenverteilung zu 
höheren jährlichen Belastungen führen kann, hat sich bewiesen.

\textit{Beschränkungen der Forschung}

Trotz der Ergebnisse unterliegt diese Arbeit einigen Beschränkungen. %Aufgrund begrenzter Zugang zu relevanten Daten.
Die Arbeit hat sich nur auf Betriebskosten der Fluggesellschaften und Anschaffungskosten für die Infrastruktur,
jedoch nicht auf die Betriebskosten der Infrastruktur. Außerdem sind am Flughafen viele andere Stakeholder beschäftigt, vor allem
welche Herausforderungen bei Bodendienstleister dazu kommen werden.

Die Komplexität des realen Systems und Technologien wurde durch die vereinfachte Annahmen reduziert, um eine fokussierte Analyse zu ermöglichen.
Wegen fehlender Daten in der technologischen Entwicklungen von neuen Antrieben wurden einige Annahmen.
Zudem wurde die Arbeit in begrenzter Untersuchungszeitraum stattgefunden, wodurch manche Variablen vernachlässigt werden müssten.
%
Angesichts der Mangel an relevanten Daten war es unmöglich die Ausbildungskosten bei der Einführung den neuen Antrieben zu berechnen.
Es lässt sich eindeutig feststellen, dass da eine Forschungslücke vorhanden ist, welche noch erkundet werden muss. 
Trotz einer gesendeten Anfrage zu dem Thema an ein bedeutendes Unternehmen blieb eine Antwort bislang aus.

Die Ausrechnung der Kosten wurde nur mit der Betracht den Passagieren und der Fracht außer Acht gelassen hat. 

\textit{Vorschläge für eventuelle weiterführende Forschungen}
Um die Ergebnisse zu vertiefen und die Arbeit detaillierter zu erforschen, können die Kosten für jede Entfernung berechnet werden.
Das ermöglicht die Berechnungen auf die tatsächliche Flotte rübergetragen und Preisanstiege je Entfernung rauszusuchen.
Als anderes Aspekt für die Vertiefung wäre interessant anzuschauen, welche konkrete Unterschiede in Emissionen durch 
alternative Antriebe entstehen, welcher Unterschied sich in Flughafen-Entgelten bildet und welche Kosten dadurch
für Luftfahrzeugbetreibern ergeben. Vor allem durch möglich zukünftige politische Entscheidung.

Vergleich zu anderen Arbeiten:
„Im Vergleich zu bisherigen Untersuchungen wird deutlich, dass…“
„Im Kontext der bestehenden Literatur deuten die Ergebnisse darauf hin, dass…“

Im Vergleich zu anderen Untersuchungen haben sich die Ergebnisse...
Es gibt Werke, die sich detaillierter mit Infrastruktur auseinandersetzen und mehr Komponenten, als auch z.B. Installationsprozesse
mitbeziehen. %(wie file:///C:/Users/henri/Downloads/765438.pdf)

