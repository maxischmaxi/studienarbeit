
\section{Bewertung der Ergebnisse}
\label{s:Bewertung der Ergebnisse}

\textit{Interpretation der Ergebnisse}\\
%Betriebskosten
Die Ergebnisse dieser Studie liefern wichtige Erkenntnisse für den Betrieb mit alternativen Antrieben.
Aus der Analyse der Betriebskosten in der die konventionellen Flugzeuge mit neuen Antrieben verglichen wurden, 
lässt sich zusammenfassen, dass die neuartigen Antriebe höhere Betriebskosten für eine Fluggesellschaft bringen werden.
Demzufolge hat sich die erste aufgestellte Hypothese bestätigt. Die höheren Kosten für elektrische Flugzeuge sind vor allem von Anschaffungspreise für
die Flugzeuge abhängig. Die Sensitivitätsanalyse zu diesem Parameter zeigte, dass das Modell den Änderungen gegenüber sensibel reagiert.



Betriebsszenarien - Betriebskosten
Die Analyse der Betriebskosten unter unterschiedlichen Betriebsszenarien bringt die Kenntnisse bei, dass die Betriebskosten werden sich je nach
ausgewählte Antriebe ändern. Sollte Flottengröße oder Zusammensetzung sich ändern, würden unterschiedliche Ergebnisse rauskommen.
Wobei die Größere Fluggesellschaften sich, werden Low-Cost-Carrier eher
 die geringste Betriebskosten
liefert das erste Szenario

Die Verteilung der Betriebskosten in den Szenarien ist definitiv von der Wahl den Antrieben anhängig. Die Batterieantriebe verursachen zwar
in der Szenarien die geringsten Kosten. Jedoch darf es nicht vergessen werden, dass die Batterien eingeschränkte Reichweite, als auch
ungenügende Sitzangebot und können nicht vollständig die Nachfrage von Kurzstrecken decken.

Betriebsszenarien - Infrastrukturkosten

\textit{Beschränkungen der Forschung}
Trotz den Ergebnissen unterliegt diese Arbeit einigen Beschränkungen. Aufgrund begrenzter Zugang zu relevanten Daten.
Die Arbeit hat sich nur auf Betriebskosten der Fluggesellschaften und Anschaffungskosten für die Infrastruktur,
jedoch nicht auf die Betriebskosten der Infrastruktur.

Die Komplexität des realen Systems und Technologien wurde durch die vereinfachte Annahmen reduziert, um eine fokussierte Analyse zu ermöglichen.
Wegen fehlender Daten in der technologischen Entwicklungen von neuen Antrieben wurden einige Annahmen.
Zudem wurde die Arbeit in begrenzter Untersuchungszeitraum stattgefunden, wodurch manche Variablen vernachlässigt werden müssten.
%
Angesichts der Mangel an relevanten Daten war es unmöglich die Ausbildungskosten bei der Einführung den neuen Antrieben zu berechnen.
Es lässt sich eindeutig feststellen, dass da eine Forschungslücke vorhanden ist, welche noch erkundet werden muss. 
Trotz einer gesendeten Anfrage zu dem Thema an ein bedeutendes Unternehmen blieb eine Antwort bislang aus.

Die Ausrechnung der Kosten wurde nur mit der Betracht den Passagieren und der Fracht außer Acht gelassen hat. 

\textit{Vorschläge für eventuelle weiterführende Forschungen}
Um die Ergebnisse zu vertiefen und die Arbeit detaillierter zu erforschen, können die je Kosten je Entfernung berechnet werden.
Das ermöglicht die Berechnungen auf die tatsächliche Flotte rübergetragen.
Als anderes Aspekt für die Vertiefung wäre interessant anzuschauen, welche konkrete Unterschiede in Emissionen durch 
alternative Antriebe entstehen, welcher Unterschied sich in Flughafen-Entgelten bildet und welche Kosten dadurch
für Luftfahrzeugbetreibern ergeben. Vor allem durch möglich zukünftige politische Entscheidung.

Vergleich zu anderen Arbeiten:
„Im Vergleich zu bisherigen Untersuchungen wird deutlich, dass…“
„Im Kontext der bestehenden Literatur deuten die Ergebnisse darauf hin, dass…“

Es gibt eine Menge anderen Arbeiten, die Betriebskosten entstehenden durch neuartige Antriebe verglichen mit Konventionellen.
Im Vergleich zu anderen Untersuchungen haben sich die Ergebnisse...
Es gibt Werke, die sich detaillierter mit Infrastruktur auseinandersetzen und mehr Komponenten, als auch z.B. Installationsprozesse
mitbeziehen. %(wie file:///C:/Users/henri/Downloads/765438.pdf)


Die Ergebnisse werden interpretiert und Erkenntnisse werden erläutert.

Die Beschränkungen der Forschung und deren Auswirkungen werden dargelegt.

Es wird beschrieben, inwiefern die Erwartungen erfüllt worden sind.

Eventuelle Ursachen und Folgen für die Ergebnisse werden besprochen.

Vorschläge für eventuelle weiterführende Forschungen werden gemacht.

Ergebnisse werden interpretiert.

Es werden keine zu vagen Empfehlungen für weiterführende Forschungen gemacht.