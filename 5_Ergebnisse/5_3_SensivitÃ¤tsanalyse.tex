\section{Sensitivitätsanalyse}
\label{s:Sensitivitätsanalyse}
%
Die Sensitivitätsanalyse ermöglicht die Robustheit einer Berechnung zu überprüfen und die Parameter mit größtem Einfluss herauszufinden. 
Dafür wurde die Teilkosten aus dem \ref{s:Ergebnisse_Flugzeuge} ausgesucht, die größter Einfluss auf die Gesamtkosten haben. 
Bei den Vergleich von Batterie, SAF und herkömmlichen Treibstoff sind die Entgelte und Gebühren, als auch kapitalbezogene Kosten.

Werden Entgelte um 10 \% erhöht, erhöhen sich die Betriebskosten moderat ($S$=0,15), auch der Preissenkung für die Anschaffungspreise
um 10 \% nicht so einen großen Einfluss aus das Gesamtsystem hat ($S$=0,28).

Dafür wurden folgende Parametern ausgesucht: 1. der Preis für Leasing einer Batterie, 2. Flugzeugpreis
Der ersten Wert kann je nach Leistung der Batterie und Häufigkeit der Nutzung unterschiedlich ausfallen, deswegen wäre es wichtig anzuschauen,
welcher Unterschied im Betriebskosten entsteht, wenn ich Preis für Leasing 10 \% teurer wäre. Die Preise für neue Flugzeugantriebe
werden durch die steigende Nachfrage mit der Zeit sinken, aus diesem Grund wurde der 10 \% Rückgang wurde angeschaut.


Beim Vergleich von Wasserstoffbetrieben und SAF mit konventionellem Antrieb entsteht der größte Kostenanteil
von Treibstoffkosten und kapitalbezogenen Kosten. Deswegen wurde für Parametern wie: 1. Preis für Treibstoff; 
2. Anschaffungskosten eines Flugzeugs entschieden. Dabei Preis für alternative Antriebe soll in der Zukunft senken und Preis für herkömmlichen
Treibstoffen aufgrund Wachstum von Ölpreisen wachsen. Es wird auch den Wert von 10 \% genommen.
Wie bereits angesprochen wird der Anschaffungspreis für neue Technologien mit der Zeit sinken.

\begin{table}[h]
	\begin{center}
    \caption{Sensitivitätsanalyse: Unterschied der Betriebskosten vom Parameter Treibstoffpreis}
	\label{BA_Infrastrukturtab}
	\begin{tabular}{|l|c|c|c|}
		\hline
		Antriebe & \textbf{Referenz}& \textbf{Wasserstoff}& \textbf{SAF} \\ \hline
		Ausgang $EUR$ & 56340,3 & 78853,1 & 66853,5  \\ \hline
        Preis Szenario (+/- 10 \%) & 58290,4  & 75020,3 & 63852,1 \\ \hline
        Abweichung $\%$ & 3,5 & -4,9 & -4,5 \\ \hline
		$S$ & 0,35 & 0,49 & 0,45 \\ \hline
	\end{tabular}
    \end{center}
\end{table}

Der Wasserstoffantrieb reagiert der Preisänderung (S=0,49) gegenüber am stärksten. Das System reagiert relativ 
empfindlich auf die Preisänderung von Wasserstoff, die Preise für Kerosin und SAF sind weniger sensibel.
