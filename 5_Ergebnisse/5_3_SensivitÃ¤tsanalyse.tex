\section{Sensitivitätsanalyse}
\label{s:Sensitivitätsanalyse}
%
Die Sensitivitätsanalyse ermöglicht die Robustheit einer Berechnung zu überprüfen und die Parameter mit größtem Einfluss herauszufinden. 
Dafür wurde die Teilkosten aus dem \ref{s:Ergebnisse_Flugzeuge} ausgesucht, die größter Einfluss auf die Gesamtkosten haben. 
Bei dem Vergleich von Batterie, SAF und herkömmlichen Treibstoff sind die Entgelte und Gebühren, als auch kapitalbezogene Kosten.

Dafür wurden folgende Parametern für BA ausgesucht: 1. der Preis für Leasing einer Batterie, 2. Anschaffungspreis für das Flugzeug.
Der erste Wert kann je nach Leistung der Batterie und Häufigkeit der Nutzung unterschiedlich ausfallen, deswegen wäre es wichtig anzuschauen,
welcher Unterschied im Betriebskosten entsteht, wenn der Preis für Leasing 10 \% teurer wäre. Die Preise für neue Flugzeugantriebe
werden durch die steigende Nachfrage mit der Zeit sinken, aus diesem Grund wurde der 10 \% Rückgang den Anschaffungspreisen angeschaut.

Werden Entgelte um 10 \% erhöht, erhöhen sich die Betriebskosten moderat ($S$ = 0,15), auch der Senkung den Anschaffungspreisen
um 10 \% nicht so einen großen Einfluss aus das Gesamtsystem hat ($S$ = 0,28).

Beim Vergleich von Wasserstoffbetrieben und SAF mit konventionellem Antrieb entsteht der größte Kostenanteil
von Treibstoffkosten. Deswegen wurde der Parameter "Treibstoffpreis" für Sensitivitätsanalyse ausgewählt.
Der Preis für alternative Antriebe und Treibstoffe soll in der Zukunft senken und Preis für herkömmlichen
Treibstoffen aufgrund Wachstum von Ölpreisen wachsen. Zudem wird ein Wert von 10 \% zugrunde gelegt.

\begin{table}[h]
	\begin{center}
    \caption{Sensitivitätsanalyse: Unterschied der Betriebskosten vom Parameter Treibstoffpreis}
	\label{BA_Infrastrukturtab}
	\begin{tabular}{|l|c|c|c|}
		\hline
		Antriebe & \textbf{Referenz}& \textbf{Wasserstoff}& \textbf{SAF} \\ \hline
		Ausgang $EUR$ & 56340,3 & 78853,1 & 66853,5  \\ \hline
        Preis Szenario (+/- 10 \%) & 58290,4  & 75020,3 & 63852,1 \\ \hline
        Abweichung $\%$ & 3,5 & -4,9 & -4,5 \\ \hline
		$S$ & 0,35 & 0,49 & 0,45 \\ \hline
	\end{tabular}
    \end{center}
\end{table}

Der Wasserstoffantrieb reagiert der Preisänderung ($S$ = 0,49) gegenüber am stärksten. Das System reagiert relativ 
empfindlich auf die Preisänderung von Wasserstoff, die Preise für Kerosin und SAF sind weniger sensibel.
