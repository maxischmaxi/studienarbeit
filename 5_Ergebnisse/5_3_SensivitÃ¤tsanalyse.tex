\section{Sensitivitätsanalyse}
\label{s:Sensitivitätsanalyse}
%
Die Sensitivitätsanalyse ermöglicht die Robustheit einer Berechnung zu überprüfen 
und die Parameter mit dem größten Einfluss zu ermitteln. 
Dafür wurden die Teile der Kosten aus \ref{s:Ergebnisse_Flugzeuge} gewählt, 
die den größten Einfluss auf die Gesamtkosten haben. 
Bei dem Vergleich von Batterie, SAF und herkömmlichen Treibstoff 
machen die Entgelte und Gebühren, als auch kapitalbezogene Kosten den größten Anteil der Betriebskosten aus. % MAX: sind die gebühren was ? höher?

Dafür wurden folgende Parameter für BA ausgesucht: der Preis für das Leasing einer Batterie 
und der Anschaffungspreis für das Flugzeug.
Der erste Wert kann je nach Leistung der Batterie und Häufigkeit der Nutzung unterschiedlich ausfallen, 
deswegen wäre es wichtig, zu betrachten, welcher Unterschied in den Betriebskosten entsteht, 
wenn der Preis für das Leasing 10 \% teurer wäre. 
Die Preise für neue Flugzeugantriebe werden durch die steigende Nachfrage mit der Zeit sinken, 
aus diesem Grund wurde ein Rückgang in Höhe von 10 \% der Anschaffungspreise betrachtet.

Werden Entgelte um 10 \% erhöht, erhöhen sich die Betriebskosten moderat ($S$ = 0,15), 
auch die Senkung der Anschaffungspreise um 10 \% hat geringen Einfluss auf das Gesamtsystem ($S$ = 0,28).

Beim Vergleich von Wasserstoff- und SAF-Antrieben mit konventionellen Antrieben 
entsteht der größte Unterschied zwischen den Kosten für Treibstoff. 
Deswegen wurde der Parameter \glqq Treibstoffpreis\grqq{} für die Sensitivitätsanalyse ausgewählt.
Der Preis für alternative Antriebe und Treibstoffe soll in Zukunft sinken und der Preis 
für herkömmliche Treibstoffe aufgrund des Anstiegs von Ölpreisen wachsen. 
Hierfür wird ein Wert von 10 \% festgelegt.

\begin{table}[h]
	\begin{center}
    \caption{Sensitivitätsanalyse: Unterschied der Betriebskosten vom Parameter Treibstoffpreis}
	\label{sensiv}
	\begin{tabular}{|l|c|c|c|}
		\hline
		\textbf{Antriebe} & \textbf{Referenz}& \textbf{Wasserstoff}& \textbf{SAF} \\ \hline
		Ausgang $EUR$ & 56340,3 & 78853,1 & 66853,5  \\ \hline
        Preis Szenario (+/- 10 \%) & 58290,4  & 75020,3 & 63852,1 \\ \hline
        Abweichung $\%$ & 3,5 & -4,9 & -4,5 \\ \hline
		$S$ & 0,35 & 0,49 & 0,45 \\ \hline
	\end{tabular}
    \end{center}
\end{table}

Der Wasserstoffantrieb reagiert der Preisänderung ($S$ = 0,49) gegenüber am stärksten. 
Das System reagiert relativ empfindlich auf die Preisänderung von Wasserstoff, 
die Preise für Kerosin und SAF sind weniger sensibel.
