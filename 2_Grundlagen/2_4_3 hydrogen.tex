\subsection{Wasserstoff-Antrieb}
\label{Wasserstoff-Antrieb}

In vielen Forschungsarbeiten wird der Wasserstoff als die Lösung für umweltfreundliche Luftfahrt dargestellt.
Diese Alternative wird jedoch von Kosten, Sicherheit und öffentlicher Akzeptanz behindert \cite{ansell2023review}.
Durch Nutzung von Wasserstoff werden keine \ce{CO2}-Emissionen verursacht, 
jedoch können andere Abgase wie Stickoxid \ce{NOx} bei der Verbrennung in Wasserstoffturbinen oder Wasserdampf emittiert werden, 
was zur Bildung von Kondensstreifen führt(Quelle).

Zündungstemperatur?

\begin{table}[h]
	\begin{center}
    \caption{Vergleich von unkomprimiertem Wasserstoffgas hinsichtlich energiebezogenen Eigenschaften mit anderen marktüblichen Treibstoffen bei 26 \acs{Celsius} }
	\label{wasserstoff_energie}
	\begin{tabular}{|l|c|c|c|}
		\hline
		& \textbf{volumetrische Energiedichte in \acs{kWh/l}} & \textbf{gravimetrische Energiedichte in \acs{kWh/kg}} & \textbf{Zündungstemperatur in \acs{kWh/kg}} \\ \hline
		Wasserstoff & 2,6 \cite{colpan2022fuel} & 37,0 \cite{colpan2022fuel} & \\ \hline
		Kerosin & ~9,5 \cite{colpan2022fuel} & ~11.9 \cite{colpan2022fuel}& \\ \hline
	\end{tabular}
    \end{center}
\end{table}

In der Tabelle \ref{wasserstoff_energie} sind die Vergleichswerte für Kerosin und Wasserstoff im flüssigen Zustand dargestellt. 
Das \ce{LH2} hat zwar bessere Gewichtsverhältnisse als Kerosin, jedoch wird für die gleiche Menge Energie 3,5-mal so viel Platz gebraucht. 
%(flüssiger Wasserstoff - geringere Gesamtspeichermasse-größere Flugzeuge und volumenanforderungen; für kleinere regionale - GH2)

Für Flugzeuge ist die Tankisolierung von großer Bedeutung. Der \ce{LH2} muss bei -253 °C Grad gelagert werden, 
ansonsten geht es zur gasförmigen Form über \cite{colpan2022fuel}. In dieser Form muss der Wasserstoff in einem Speicher
bei einem Druck von 350 oder 700 Bar gelagert werden \cite{colpan2022fuel}.
%Das kann die Abfertigungszeiten beeinflussen. (bei schlechte Isolierung - schneller starten, bei guten - mehr Flexibilität)
%
Es existieren unterschiedliche Wege zur Herstellung von Wasserstoff. 
Die gängigsten Herstellungswege sind Steam Methane Reforming (SMR) und die Elektrolyse. %(Fully Renewable). 
Bei SMR trifft Wasserdampf in der Reaktion zusammen mit Methan aus Erdgas, infolgedessen entstehet der
Wasserstoff \ce{H2} und Kohlenmonoxid bzw. -dioxid \cite{mulder2019outlook}. Bei der Elektrolyse wird das Wasser mithilfe von Elektrizität 
in Wasserstoff und Sauerstoff \ce{O2} gespaltet \cite{mulder2019outlook}. Durch diesen Herstellungsweg können \ce{CO2}-Emissionen 
vollständig vermieden werden \cite{dalmia2022powering}. 

Die Nachhaltigkeit der Elektrolyse ist außerdem von der Stromquelle abhängig. 
Wenn für die Produktion von Wasserstoff erneuerbare Energiequellen (wie Solar- und Windanlagen) genutzt worden, 
wird dieser als grüner Wasserstoff bezeichnet \cite{mulder2019outlook}. 
Wird der Strom aus fossilen Energieträgern erzeugt, kommt es zu indirekten Kohlenstoffemissionen.
Durch Elektrolyse produzierter grüner Wasserstoff ist kostenintensiv \cite{dalmia2022powering} und von den Stromkosten abhängig. 
%Chardonnet et al. schätzt die Investitionskosten für die Produktion in 15 Mio. € (nach \cite{mulder2019outlook}). wie viel kilo?
Bei anderen Herstellungswegen (grauer und blauer Wasserstoff)
kommt es hingegen zum Ausstoß von Kohlenstoff, wobei bei blauem Wasserstoff das \ce{CO2} gesammelt und gespeichert \cite{mulder2019outlook} wird.
%
%\cite{mulder2019outlook} - Annahme: anstatt Windkraftanlage - Stromnetz

Wasserstoff kann in zwei unterschiedlichen Weisen als Antrieb genutzt werden: 
erstens als Treibstoff für die Verbrennung im \ce{H2}-Verbrennungsmotor,
zweitens als Brennstoffzelle, um den elektrischen Motor anzutreiben \cite{sky2020hydrogen}.

"Die Wasserstofftriebwerke werden in ihrer Architektur den bestehenden Düsentriebwerken ähneln, 
jedoch mit einigen Ergänzungen, wie z. B. Kraftstoffpumpen und -steuergeräten, Brennkammern und
 einem zusätzlichen Wärmetauscher zur Verdampfung von flüssigem Wasserstoff (LH2)" \cite{colpan2022fuel}
Colpan et al. \cite{colpan2022fuel} ist der Meinung, dass sich die Wasserstofftriebwerke im Vergleich zu konventionellen Triebwerken ähneln werden. 
Allerdings werden Komponenten, wie Kraftstoffpumpen, 

Flüssiger Wasserstoff

%"LH2 hat die höchste spezifische Energie- und Kühlleistung aller herkömmlichen Substanzen." \cite{ansell2023review}
Flüssiger Wasserstoff entsteht durch die Verflüssigung des Wasserstoffs.

Senkt man die Temperatur von Wasserstoff auf -253 °C \cite{colpan2022fuel}, entsteht flüssiger Treibstoff.

Die Nutzung von Wassertreibstoff ist für größere Flugzeuge geeignet und somit für längere Stecken. Das vorhergeht von % der satz ist komisch, bitte überprüfen
Gewichtsanteil von LH2 Tank in Anhängigkeit von der gelagerten Menge. 
Große Flugzeuge bieten bessere Speichereffizienz und Wasserstoffmengen-Verhältnis.
"(da niedrige Tankgewichte erforderlich sind, um den hohen spezifischen Energievorteil von Wasserstoffkraftstoff voll auszuschöpfen)"
\cite{ansell2023review}


Gasförmiger Wasserstoff/ Brennstoffzellentechnologie

Eine andere Methode, um Wasserstoff in der Luftfahrt zu nutzen ist der Wasserstoff im gasförmigen Zustand. 
Diese Technologie wird Brennstoffzellentechnologie genannt.
Durch die chemische Reaktion aus gasförmigem Wasserstoff \ce{H2} und Sauerstoff \ce{O2} wird Strom produziert \cite{dalmia2022powering} und 
dadurch der Propeller vom Flugzeug angetrieben.
Brennstoffzellen (wie die Polymerelektrolytbrennstoffzelle) haben eine höhrere Energiedichte (als was?) und 
können somit für Mittel- oder Langstrecken genutzt werden \cite{dalmia2022powering}. 
Um mehr Energiedichte speichern zu können, muss \ce{H2} stark komprimiert werden. 

hydrogen-Electric Antrieb

Sobald die Brennstoffzellen in dem Flugzeug integriert sind, werden Generatoren/Generatoren das 
Auxiliary Power Unit (APU)-System des Flugzeugs mit Strom versorgen. Zum Einsatz kommen Brennstoffzellen 
mit Protonenaustauschmembran, deren Strom die Motoren antreibt. Die Motoren werden über Propeller verfügen,
die dann dem Flugzeug Schub verleihen, da sie sich mit unterschiedlichen Geschwindigkeiten im Verhältnis zur Drosselklappe drehen.\cite{dalmia2022powering}

Wasserstoff kann in einer chemischen Verbindung gebunden werden, wie Ammoniak und Methanol, und somit auch transportiert.

not sure:

\cite{dahal2021techno} In der Dahal et el. Studie wurde ein Konzept vorgeschlagen, wo die Tanks auf dem Fuselage sich befinden.
"Die kryogenen Kraftstofftanks erfordern eine angemessene Isolierung sowie die Verwendung von Materialien, die 
gegen Versprödung und zyklische Belastungen schützen, die durch die thermische Ausdehnung und Kontraktion beim Nachfüllen 
und den Kraftstoffverbrauch verursacht werden"

Die zusätzliche Ausbildung für Wartungsmitarbeiter und die Schulungen für Bodenabfertigen
werden benötigt, weil die Charakteristiken von Wasserstoff 
sich von herkömmlichen Treibstoffen stark unterscheiden. \cite{mulder2019outlook}?

Betriebsleergewicht bleibt konstant, im Betracht Startbruttogewicht um 30 \%, jedoch wegen großen Wasserstofftank

