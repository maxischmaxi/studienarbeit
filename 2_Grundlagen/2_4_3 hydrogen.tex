\subsection{Wasserstoffantrieb}
\label{ss:Wasserstoff-Antrieb}

In vielen Forschungsarbeiten wird Wasserstoff als die Lösung für umweltfreundliche Luftfahrt dargestellt.
Dieses Energiemedium wird jedoch von Kosten, Sicherheit und öffentlicher Akzeptanz behindert \cite{ansell2023review}.
Durch die Nutzung von Wasserstoff werden keine \ce{CO2}-Emissionen verursacht, jedoch können andere Abgase 
wie Stickstoffoxid \ce{NO_x} bei der Verbrennung in Wasserstoffturbinen oder Wasserdampf emittiert werden, was zur Bildung von Kondensstreifen
führt \cite{hepperle2012electric}.\\
%was zur Bildung von Kondensstreifen führt \cite{conrady2019luftverkehr}. %überprüfen

\subsubsection{Herstellung}
Es existieren verschiedene Wege zur Herstellung von Wasserstoff. 
Die gängigsten sind Dampfreformierung (Steam Methane Reforming - SMR) und die Elektrolyse. %(Fully Renewable). 
Bei SMR trifft Wasserdampf in der Reaktion zusammen mit Methan aus Erdgas, infolgedessen entsteht 
Wasserstoff \ce{H2} und Kohlenmonoxid \ce{CO} bzw. -dioxid \cite{mulder2019outlook}. Bei der Elektrolyse wird das Wasser mithilfe von Elektrizität 
in Wasserstoff \ce{H2O} und Sauerstoff \ce{O2} gespalten \cite{mulder2019outlook}. Durch diesen Herstellungsweg können \ce{CO2}-Emissionen 
vollständig vermieden werden \cite{dalmia2022powering}. 
Mulder et al. \cite{mulder2019outlook} schätzt die Investitionskosten für die Produktion des Wasserstoffs durch Elektrolyse deutlicher günstiger als ein Kohlekraftwerk.
Die Nachhaltigkeit der Elektrolyse ist außerdem, genau wie bei dem Batterieantrieb, von der Stromquelle abhängig.
%
Wenn für die Produktion von Wasserstoff erneuerbare Energiequellen (wie Solar- und Windanlagen) genutzt wurden, 
wird dieser als grüner Wasserstoff bezeichnet \cite{mulder2019outlook}. 
Durch Elektrolyse produzierter grüner Wasserstoff ist kostenintensiv \cite{dalmia2022powering}.
Wird der genutzte Strom aus fossilen Energieträgern erzeugt, kommt es zu indirekten Emissionen.
Bei anderen Herstellungswegen (bspw. grauer und blauer Wasserstoff) kommt es hingegen zum Ausstoß von Kohlenstoff, 
wobei bei blauem Wasserstoff das \ce{CO2} gesammelt und gespeichert wird \cite{mulder2019outlook}.
%Chardonnet et al. schätzt die Investitionskosten für die Produktion in 15 Mio. € (nach \cite{mulder2019outlook}). wie viel kilo?
%
%
\subsubsection{Zustände}
Wasserstoff kann in mehreren Zuständen genutzt werden. Die in der Verkehrsbranche am weitesten verbreiteten
sind einerseits der gasförmige \ce{GH2}, und andererseits der kryogene flüssige Wasserstoff \ce{LH2}. 
Um mehr Energie speichern zu können und dabei weniger Platz zu verbrauchen, muss das gasförmige \ce{H2} stark komprimiert und 
bei einem Druck von 350 oder 700 bar gespeichert werden \cite{colpan2022fuel}.
Allerdings hat gasförmiger Wasserstoff auch bei einem Druck von 700 Bar eine geringere Energiedichte
als flüssiger Wasserstoff \cite{eichlseder2012hydrogen}.
Flüssiger Wasserstoff wird durch das Abkühlen und Verdichten von gasförmigem Wasserstoff gewonnen.
%"LH2 hat die höchste spezifische Energie- und Kühlleistung aller herkömmlichen Substanzen." \cite{ansell2023review}
In der Tabelle \ref{wasserstoff_energie} sind die Vergleichswerte für Kerosin und Wasserstoff im flüssigen Zustand dargestellt.
Die Gravimetrische Energiedichte bei flüssigem Wasserstoff ist deutlich höher als bei Kerosin, 
jedoch ist die volumetrische Energiedichte $E_V$ wesentlich geringer.
Das bedeutet, dass \ce{LH2} zwar bessere Gewichtsverhältnisse als Kerosin hat, 
für die gleiche Menge Energie wird allerdings 3,5-mal so viel Platz benötigt.
Aufgrund seiner Eigentschafen ist der Wasserstoff für die Nutzung auf längeren Flugdistanzen geeignet.
%(flüssiger Wasserstoff - geringere Gesamtspeichermasse-größere Flugzeuge und volumenanforderungen; für kleinere regionale - GH2)
\begin{table}[h]
	\begin{center}
    \caption{Vergleich von flüssigem Wasserstoff energiebezogenen Eigenschaften mit anderen konventionellen Treibstoffen}
	\label{wasserstoff_energie}
	\begin{tabular}{|l|c|c|c|}
		\hline
		& \textbf{$E_V$ in $[kWh/l]$} & \textbf{$E_G$ in $[kWh/kg]$} & \textbf{$Dichte$ $[kg/m^3]$}  \\ \hline
		Wasserstoff LH2 \cite{colpan2022fuel} & 2,6 & 37,0 & 65 \\ \hline
		Kerosin \cite{colpan2022fuel} & ~9,5 & ~11.9 &  \\ \hline % energiedichte von kerosin fehlt oder?
	\end{tabular}
    \end{center}
\end{table}
%
Flüssiger kryogener Wasserstoff ist wesentlich besser als gasförmiger für den Transport per LKW geeignet,
dafür benötigt er aber einen höheren Energieaufwand \cite{colpan2022fuel}. 

Die Konzepte schlagen verschiedene Platzierungen des Wasserstofftanks vor, unter anderem in Form von halbkugelförmigen Endkappen auf dem Flugzeugrumpf \cite{dahal2021techno} 
oder am Ende des Flugzeugrumpfes zwischen der Fracht und den hinteren Notausgängen \cite{rietdijk2024architecture}.
Bei Flugzeugen, welche mit Wasserstoff betrieben werden, ist die Tankisolierung von großer Bedeutung. 
Das flüssige \ce{LH2} muss bei -253 °C gelagert werden \cite{colpan2022fuel} 
und zusätzlich kann die Nutzung zu Versprödung der Materialien führen \cite{dahal2021techno}.
Durch Wärme verdampft der Wasserstoff, was zum Anstieg des Drucks und der Temperatur im Tank führt. Heutige Tankanlagen 
haben tägliche Abdampfverluste in Höhe von 0,3 \% bis 3 \% \cite{eichlseder2012hydrogen}.\\

%Wasserstoff führt schnell zu Versprödung durch zyklische Belastungen, 
%die bei der Wärmeausdehnung und -kontraktion während des Nachfüllens 
%sowie durch den Kraftstoffverbrauch verursacht werden \cite{dahal2021techno}.
%Das kann die Abfertigungszeiten beeinflussen. (bei schlechte Isolierung - schneller starten, bei guten - mehr Flexibilität)
%
%
%
%\cite{mulder2019outlook} - Annahme: anstatt Windkraftanlage - Stromnetz

\subsubsection{Antrieb}
Wasserstoff kann in zwei Ansätzen als Antrieb genutzt werden: 
einerseits als Treibstoff für die Verbrennung im \ce{H2}-Verbrennungsmotor,
andererseits in der Brennstoffzelle, um den elektrischen Motor anzutreiben \cite{sky2020hydrogen}. 
Zudem gibt es einen hybriden Antrieb, bei welchem die Brennstoffzelle zusammen mit einer Batterie genutzt wird.
%
%Eine andere Methode, um Wasserstoff in der Luftfahrt zu nutzen, ist die Brennstoffzellentechnologie.
Brennstoffzellen haben ein hohes Gewicht \cite{hepperle2012electric} und 
benötigen den gasförmigen Wasserstoff als Antrieb \cite{colpan2022fuel}.
Dabei wird durch die chemische Reaktion aus gasförmigem Wasserstoff \ce{H2} und Sauerstoff \ce{O2} Strom produziert \cite{dalmia2022powering}, 
wodurch der Propeller des Flugzeugs angetrieben wird. Aufgrund der besseren Speicherung in flüssiger Form muss bedacht werden, wann der Wasserstoff in den Gaszustand überführt wird. 
Brennstoffzellen haben, ebenso wie Batterieantrieb, weniger bewegende Teile als konventionelle Antriebe \cite{dalmia2022powering} was weniger Wartungskosten verursachen könnte,
allerdings könnte der technisch anspruchsvolle Wasserstofftank häufigere Wartungszyklen erfordern.
%
%Brennstoffzellen erzeugen viel Wärme, weshalb Kühlsysteme benötigt werden, um die Leistung aufrechtzuerhalten.
%https://ieeexplore.ieee.org/document/9794396
%
%Brennstoffzellen (wie die Polymerelektrolytbrennstoffzelle) haben eine höhere Energiedichte (als was?) und 
%können somit für Mittel- oder Langstrecken genutzt werden \cite{dalmia2022powering}.  ist es so?
%
%
%"Die Wasserstofftriebwerke werden in ihrer Architektur den bestehenden Düsentriebwerken ähneln, 
%jedoch mit einigen Ergänzungen, wie z. B. Kraftstoffpumpen und -steuergeräten, Brennkammern und
% einem zusätzlichen Wärmetauscher zur Verdampfung von flüssigem Wasserstoff (LH2)" \cite{colpan2022fuel}
Für die Verbrennung des Wasserstoffs sind Änderungen in der Brennkammer benötigt, 
um höhere Temperaturen zu vermeiden \cite{khandelwal2013hydrogen}.
Colpan et al. \cite{colpan2022fuel} ist jedoch der Meinung, 
dass sich die Wasserstofftriebwerke konventionellen Düsentriebwerken ähneln werden. 
Allerdings werden zusätzliche Komponenten wie Kraftstoffpumpen und Wärmetauscher 
für den flüssigen Wasserstoff benötigt.
%
%. Das vorhergeht von % der satz ist komisch, bitte überprüfen
%Gewichtsanteil von LH2 Tank in Anhängigkeit von der gelagerten Menge. 
%Große Flugzeuge bieten bessere Speichereffizienz und Wasserstoffmengen-Verhältnis.
%"(da niedrige Tankgewichte erforderlich sind, um den hohen spezifischen Energievorteil von Wasserstoffkraftstoff voll auszuschöpfen)"
%\cite{ansell2023review}
%
%Aufgrund des schweren und massiven Wasserstofftanks werden 
%unterschiedliche wasserstoffkonfugurationen vorgeschlagen.
%Es gibt Konzepte, wo der Tank am Rumpf, am Ende des Rumpfes 
%oder vorne, hinter der Pilotenkabine sitzt. % MAX: was ist hier?
%Sobald die Brennstoffzellen in dem Flugzeug integriert sind, werden Generatoren/Generatoren das 
%Auxiliary Power Unit (APU)-System des Flugzeugs mit Strom versorgen. Zum Einsatz kommen Brennstoffzellen 
%mit Protonenaustauschmembran, deren Strom die Motoren antreibt. Die Motoren werden über Propeller verfügen,
%die dann dem Flugzeug Schub verleihen, da sie sich mit unterschiedlichen Geschwindigkeiten im Verhältnis zur Drosselklappe drehen.\cite{dalmia2022powering}
%
%
%Betriebsleergewicht bleibt konstant, im Betracht Startbruttogewicht um 30 \%, jedoch wegen großen Wasserstofftank
\textit{Sicherheit beim Umgang mit Wasserstoff}\\
Wasserstoff wird als hochentzündlich beschrieben \cite{dalmia2022powering}. Aufgrund seiner Natur breitet sich die Flamme eher 
vertikal aus und die Brenndauer von \ce{LH2} ist kürzer als die des Kerosins \cite{colpan2022fuel}.
Der Wasserstoff hat eine hohe Flammengeschwindigkeit und es besteht die große Gefahr eines Flammenrückschlags bei der Flammenausbreitung \cite{khandelwal2013hydrogen}.
Dennoch ist Wasserstoff innerhalb der richtigen Infrastruktur nicht gefährlicher als andere brennbare konventionelle Treibstoffe und in manchen 
Fällen sogar sicherer \cite{khandelwal2013hydrogen}. 
Wird der flüssige Wasserstoff verschüttet, wird er aufgrund seiner Leichtigkeit vertikal nach oben verdampfen \cite{colpan2022fuel}. 
Direkter Kontakt mit kryogenem Wasserstoff führt zu Erfrierungen.
Die Forschung des Wasserstoffs muss sich mit Themen wie Explosionsgefahr, Materialgefahr, Betankung und 
dem Umgang in der Abfertigung auseinandersetzen. 

%
%Zusätzliche Ausbildungen für Wartungsmitarbeiter und Schulungen für Bodenabfertigungspersonal
%werden benötigt, weil sich die Charakteristiken von Wasserstoff 
%von herkömmlichen Treibstoffen stark unterscheiden.
Aufgrund des starken Unterschiedes der Charakteristiken von Wasserstoff 
im Vergleich zu herkömmlichen Treibstoffen werden zusätzliche Schulungen für 
Wartungsmitarbeiter und für Bodenabfertigungspersonal benötigt, 
um mögliche Gefahren zu erkennen und diese zu vermeiden.
%
%Die Ausbildung soll die allgemeinen Charakteristiken des Wasserstoffs, 
%den Umgang mit Wasserstoff und möglichen verbundenen Gefahren, 
%und die korrekten Reaktionen in Notfallsituationen beinhalten. %(Quelle: https://www.icas.org/icas_archive/icas2024/data/papers/icas2024_1090_paper.pdf)
%Die Schulungen sollten für alle Beteiligten an der Luftfahrzeugabfertigung durchgeführt werden.
%
%Wegen des großen Unterschiedes zu herkömmlichen Treibstoffen und 
%Ausrüstungen müssen die Mitarbeiter neu geschult werden, 
%um mögliche Gefahren zu erkennen und zu vermeiden \cite{gu2023hydrogen}.
%
%\cite{mulder2019outlook}?
%
%
%not sure:
%\cite{dahal2021techno} In der Dahal et el. Studie wurde ein Konzept vorgeschlagen, wo die Tanks auf dem Fuselage sich befinden.
%"Die kryogenen Kraftstofftanks erfordern eine angemessene Isolierung sowie die Verwendung von Materialien, die 
%gegen Versprödung und zyklische Belastungen schützen, die durch die thermische Ausdehnung und Kontraktion beim Nachfüllen 
%und den Kraftstoffverbrauch verursacht werden"
%
%etwa 64\% less Specific Fuel Consumption (SFC) \cite{colpan}
