\subsection{Wasserstoff-Antrieb}
\label{ss:Wasserstoff-Antrieb}

In vielen Forschungsarbeiten wird der Wasserstoff als die Lösung für umweltfreundliche Luftfahrt dargestellt/vorgestellt.
Dieses Energiemedium wird jedoch von Kosten, Sicherheit und öffentlicher Akzeptanz behindert \cite{ansell2023review}.
Durch Nutzung von Wasserstoff werden keine \ce{CO2}-Emissionen verursacht, jedoch können andere Abgase 
wie Stickstoffoxid \ce{NOx} bei der Verbrennung in Wasserstoffturbinen oder Wasserdampf emittiert werden, was zur Bildung der Kondensstreifen
führt \cite{hepperle2012electric}.\\
%was zur Bildung von Kondensstreifen führt \cite{conrady2019luftverkehr}. %überprüfen
%
\textit{Herstellung}\\
Es existieren verschiedene Wege zur Herstellung von Wasserstoff. 
Die gängigsten davon sind Dampfreformierung (Steam Methane Reforming - SMR) und die Elektrolyse. %(Fully Renewable). 
Bei SMR trifft Wasserdampf in der Reaktion zusammen mit Methan aus Erdgas, infolgedessen entsteht der
Wasserstoff \ce{H2} und Kohlenmonoxid \ce{CO} bzw. -dioxid \cite{mulder2019outlook}. Bei der Elektrolyse wird das Wasser mithilfe von Elektrizität 
in Wasserstoff \ce{H2O} und Sauerstoff \ce{O2} gespaltet \cite{mulder2019outlook}. Durch diesen Herstellungsweg können \ce{CO2}-Emissionen 
vollständig vermieden werden \cite{dalmia2022powering}. 
Mulder et al. \cite{mulder2019outlook} schätzt die Investitionskosten für Produktion des Wasserstoffs durch Elektrolyse deutlicher günstiger als ein Kohlekraftwerk.
Die Nachhaltigkeit der Elektrolyse ist außerdem genauso wie bei BA von der Stromquelle abhängig.
%
Wenn für die Produktion von Wasserstoff erneuerbare Energiequellen (wie Solar- und Windanlagen) genutzt worden, 
wird dieser als grüner Wasserstoff bezeichnet \cite{mulder2019outlook}. 
Durch Elektrolyse produzierter grüner Wasserstoff ist kostenintensiv \cite{dalmia2022powering}.
Wird der genutzte Strom aus fossilen Energieträgern erzeugt, kommt es zu indirekten Emissionen.
Beispielhaft bei anderen Herstellungswegen (grauer und blauer Wasserstoff)
kommt es hingegen zum Ausstoß von Kohlenstoff, wobei bei blauem Wasserstoff das \ce{CO2} gesammelt und gespeichert wird \cite{mulder2019outlook}.
%Chardonnet et al. schätzt die Investitionskosten für die Produktion in 15 Mio. € (nach \cite{mulder2019outlook}). wie viel kilo?

%
\textit{Zustände}\\
Der Wasserstoff kann in mehreren Zuständen benutzt werden. Die am weitesten verbreitete in
Verkehrsbranche sind der gasförmige \ce{GH2} und kryogener flüssige Wasserstoff \ce{LH2}. 
Um mehr Energie speichern zu können und dabei weniger Platz zu verbrauchen, muss der gasförmigen \ce{H2} stark komprimiert werden, 
nämlich bei einem Druck von 350 oder 700 bar gespeichert werden \cite{colpan2022fuel}.
Allerdings hat gasförmiger Wasserstoff auch bei einem Druck von 700 bar geringere Energiedichte
als flüssiger Wasserstoff \cite{eichlseder2012hydrogen}.
Flüssiger Wasserstoff entsteht durch die Verflüssigung des Wasserstoffs.
%"LH2 hat die höchste spezifische Energie- und Kühlleistung aller herkömmlichen Substanzen." \cite{ansell2023review}
In der Tabelle \ref{wasserstoff_energie} sind die Vergleichswerte für Kerosin und Wasserstoff im flüssigen Zustand dargestellt.
Gravimetrische Energiedichte bei flüssigem Wasserstoff ist deutlich höher als bei Kerosin, aber volumetrische Energiedichte $E_V$ ist viel geringer.
Das bedeutet, dass \ce{LH2} zwar bessere Gewichtsverhältnisse als Kerosin hat, jedoch wird für die gleiche Menge Energie 3,5-mal so viel Platz gebraucht.
Aufgrund seiner Stoffeigenschaft ist der Wasserstoff für die Nutzung auf längeren Flugdistanzen geeignet.
%(flüssiger Wasserstoff - geringere Gesamtspeichermasse-größere Flugzeuge und volumenanforderungen; für kleinere regionale - GH2)
\begin{table}[h]
	\begin{center}
    \caption{Vergleich von flüssigen Wasserstoff energiebezogenen Eigenschaften mit anderen konventionellen Treibstoffen }
	\label{wasserstoff_energie}
	\begin{tabular}{|l|c|c|c|}
		\hline
		& \textbf{$E_V$ in $[kWh/l]$} & \textbf{$E_G$ in $[kWh/kg]$} & \textbf{$Dichte$ $[kg/m^3]$}  \\ \hline
		Wasserstoff LH2 & 2,6 \cite{colpan2022fuel} & 37,0 \cite{colpan2022fuel} & 70,8 \cite{eichlseder2012hydrogen}\\ \hline
		Kerosin & ~9,5 \cite{colpan2022fuel} & ~11.9 \cite{colpan2022fuel}&  \\ \hline
	\end{tabular}
    \end{center}
\end{table}
%
Flüssiger kryogener Wasserstoff ist viel besser für den Transport mit LKW geeignet als gasförmigen,
aber dafür benötigt mehr Energieaufwand \cite{colpan2022fuel}. 

Die Konzepte schlagen vor die Platzierung des Wasserstofftanks im Form halbkugelförmogen Endkappen auf dem Flugzeugrumpf entlang \cite{dahal2021techno} 
oder am Ende des Rumpfes zwischen Fracht und hinteren Notausgängen \cite{rietdijk2024architecture}
Bei den Flugzeuge, die mit Wasserstoff betrieben werden, und der Tankspeicher ist die Tankisolierung von großer Bedeutung. 
Der flüssigen \ce{LH2} muss bei -253 °C gelagert werden \cite{colpan2022fuel}. 
Durch die Wärme verdampft der Wasserstoff, was zum Anstieg von Druck und Temperatur in dem Tank führt. Heutige Tankanlagen 
haben die täglichen Abdampfverluste in der Höhe von 0,3 \% bis 3 \% \cite{eichlseder2012hydrogen}.\\

Wasserstoff führt schnell zu Versprödung \cite{dahal2021techno}
"zyklische Belastungen, die durch die Wärmeausdehnung und -kontraktion beim Nachfüllen und den Kraftstoffverbrauch verursacht werden"
%Das kann die Abfertigungszeiten beeinflussen. (bei schlechte Isolierung - schneller starten, bei guten - mehr Flexibilität)
%
%
%
%\cite{mulder2019outlook} - Annahme: anstatt Windkraftanlage - Stromnetz
\textit{Antrieb}\\
Wasserstoff kann in zwei Ansätzen als Antrieb genutzt werden: 
erstens als Treibstoff für die Verbrennung im \ce{H2}-Verbrennungsmotor,
zweitens in der Brennstoffzelle, um den elektrischen Motor anzutreiben \cite{sky2020hydrogen}. 
Zudem gibt es einen hybriden Antrieb, wo die Brennstoffzelle zusammen mit einer Batterie genutzt wird.
%
Eine andere Methode, um Wasserstoff in der Luftfahrt zu nutzen ist die Brennstoffzellentechnologie.
Brennstoffzellen sind ziemlich schwer von Gewicht und der Struktur \cite{hepperle2012electric} und 
benötigen den gasförmigen Wasserstoff für den Antrieb \cite{colpan2022fuel}.
Dabei wird durch die chemische Reaktion aus gasförmigem Wasserstoff \ce{H2} und Sauerstoff \ce{O2} wird Strom produziert \cite{dalmia2022powering} und 
dadurch der Propeller vom Flugzeug angetrieben. Aufgrund besseren Speicherung in flüssigen Form muss überlegt werden, wann der Wasserstoff in Gaszustand überführt wird. 
%Brennstoffzellen (wie die Polymerelektrolytbrennstoffzelle) haben eine höhere Energiedichte (als was?) und 
%können somit für Mittel- oder Langstrecken genutzt werden \cite{dalmia2022powering}.  ist es so?
%
%
%"Die Wasserstofftriebwerke werden in ihrer Architektur den bestehenden Düsentriebwerken ähneln, 
%jedoch mit einigen Ergänzungen, wie z. B. Kraftstoffpumpen und -steuergeräten, Brennkammern und
% einem zusätzlichen Wärmetauscher zur Verdampfung von flüssigem Wasserstoff (LH2)" \cite{colpan2022fuel}

Für die Verbrennung des Wasserstoffs werden Änderung in der Brennkammer benötigt, damit höhere Temperaturen vermieden werden können \cite{khandelwal2013hydrogen}
Colpan et al. \cite{colpan2022fuel} ist jedoch der Meinung, dass sich die Wasserstofftriebwerke im Vergleich zu konventionellen Düsentriebwerken ähneln werden. 
Allerdings werden zusätzliche Komponenten, wie Kraftstoffpumpen und Wärmetauscher von flüssigem Wasserstoff benötigt.


%. Das vorhergeht von % der satz ist komisch, bitte überprüfen
%Gewichtsanteil von LH2 Tank in Anhängigkeit von der gelagerten Menge. 
%Große Flugzeuge bieten bessere Speichereffizienz und Wasserstoffmengen-Verhältnis.
%"(da niedrige Tankgewichte erforderlich sind, um den hohen spezifischen Energievorteil von Wasserstoffkraftstoff voll auszuschöpfen)"
%\cite{ansell2023review}


%Sobald die Brennstoffzellen in dem Flugzeug integriert sind, werden Generatoren/Generatoren das 
%Auxiliary Power Unit (APU)-System des Flugzeugs mit Strom versorgen. Zum Einsatz kommen Brennstoffzellen 
%mit Protonenaustauschmembran, deren Strom die Motoren antreibt. Die Motoren werden über Propeller verfügen,
%die dann dem Flugzeug Schub verleihen, da sie sich mit unterschiedlichen Geschwindigkeiten im Verhältnis zur Drosselklappe drehen.\cite{dalmia2022powering}


%Betriebsleergewicht bleibt konstant, im Betracht Startbruttogewicht um 30 \%, jedoch wegen großen Wasserstofftank
\textit{Sicherheit beim Umgang mit Wasserstoff}\\
Wasserstoff wird als hochentzündlich skaliert \cite{dalmia2022powering}. Aufgrund seiner Natur breitet sich die Flamme eher 
vertikal aus und die Brenndauer von \ce{LH2} ist kürzer als bei Kerosin \cite{colpan2022fuel}.
Der Wasserstoff hat zwar hohe Flammengeschwindigkeit und es besteht große Gefahr an Flammenrückschlag bei der Flammenausbreitung \cite{khandelwal2013hydrogen}.
Dennoch bei der richtigen Infrastruktur ist der Wasserstoff nicht gefährlicher als andere brennbare konventionelle Treibstoffe und in manchen 
Fällen sogar sicherer \cite{khandelwal2013hydrogen}. 
Wird der flüssige Wasserstoff verschüttet, wird es aufgrund seiner Leichtigkeit vertikal nach oben verdampft \cite{colpan2022fuel}. 
Direkter Kontakt mit kryogen Wasserstoff wird zu Erfrierung führen.
Die Forschung der Wasserstoff muss sich mit den Themen wie Explosionsgefahr, Materialgefahr, Betankung und 
bei der Umgang bei Abfertigung auseinandersetzen. 


Die zusätzliche Ausbildung für Wartungsmitarbeiter und die Schulungen für Bodenabfertigen
werden benötigt, weil die Charakteristiken von Wasserstoff 
sich von herkömmlichen Treibstoffen stark unterscheiden. \cite{mulder2019outlook}?


%not sure:
%\cite{dahal2021techno} In der Dahal et el. Studie wurde ein Konzept vorgeschlagen, wo die Tanks auf dem Fuselage sich befinden.
%"Die kryogenen Kraftstofftanks erfordern eine angemessene Isolierung sowie die Verwendung von Materialien, die 
%gegen Versprödung und zyklische Belastungen schützen, die durch die thermische Ausdehnung und Kontraktion beim Nachfüllen 
%und den Kraftstoffverbrauch verursacht werden"

Brennstoffzelle erzeugen viel Wärme, die Leistung aufrechtzuerhalten werden Kühlsysteme benötigt werden.
%https://ieeexplore.ieee.org/document/9794396