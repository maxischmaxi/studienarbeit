Laut sind die meisten Komponenten von direkten Kosten höher als bei herkömmlichen BAe 146, aber die Kosten für den 
Treibstoff bzw. Energie ist deutlich höher als für elektrisch-angetriebene Wrigth Spirit. Es wurde einen Anstieg 
der Preise von 48 \% (für 2-stündigen Flug) für elektrische Alternative gefunden. Die Gesamtkosten werden stark von Batterie-Lebensdauer 
beeinflusst. Hier wurde ohne Batterie-Swap berechnet. Betrachten nicht die Ausbildungskosten für den Besatzung und werden 
nicht explizit Abfertigungskosten definiert, obwohl vielleicht ist bereits in Entgelten mitbetrachtet. \cite{goodge2024analysis}

Außerdem darf man nicht vergessen entstehende Kosten für die Infrastruktur und die Ausbildung des Bodenpersonals.

In der Arbeit von Avogadro und Redondi wurde herausgefunden gesamt Betriebskosten höher erwartet, jedoch Treibstoff und Wartungskosten sinken,
jedoch die Ausbildungskosten von Piloten nicht betrachtet. Größere elektrische Flugzeuge ca. 20 \% teurer und dass der Markt für elektrische 
Flugzeuge in der näheren Zukunft wird auf regionale Flüge beschränkt.
