\subsection{Klimapolitische Maßnahmen}
\label{s:Klimapolitische Maßnahmen}


Luftverkehr erholt dich von der Pandemie COVID-19 und die Emissionswerte haben noch nicht die Emissionswerte vom Jahr geschafft.
 (https://www.umweltbundesamt.de/daten/klima/der-europaeische-emissionshandel#luftverkehr-im-emissionshandel-)

Durch klimapolitischen Initiativen können die Verminderung der Emissionen erreicht werden. 
So legt ICAO Annex 16 weltweite Grenzwerte für Umweltstandards und technische Anforderung für die Flugzeuge fest, um Emissionen zu reduzieren.

Auch die Verordnung (EU) 2018/1139 etabliert die gesetzlichen Instrumente für die Sicherheit und den Umweltschutz in der Zivilluftfahrt innerhalb der EU 
und fixiert die Rolle der Europäischen Agentur für Flugsicherheit (EASA). EASA als Luftfahrtbehörde verpflichtet die 
Umweltleistung der Luftfahrt zu überwachen und zu fördern, wozu die Gebühren erhebt werden. Außerdem macht die Verordnung die Zulassung von
neuen Flugzeugen leichter und dadurch fördert Innovationen und Anregungen für nachhaltige Luftfahrttechnologien.

Kurze Bescheibung EASA und Tätigkeiten

Die Verordnung ermöglicht Gebührenanpassung und Anreize für die Projekte, die ökologische Vorteile bringen und 
zu einem hohen Umweltschutzniveau beizutragen.

Die konkreten Maßnahmen dürfen auf nationalen Ebenen gesteuert werden, aber die die Mitgliedstaaten verpflichtet sind, 
die Sicherheit und Umweltschutz betrachten. Nach Artikel 126 kann die EASA (nach Konsultation der Kommission) die detaillierten 
Vorschriften für die Gebühren bestimmen, um nachhaltige Technologien und innovative Projekte zu fördern.

%"Artikel 86
%und Agentur (EASA) unterstützt die Kommission und die Mitgliedstaaten bei der Ermittlung der wichtigsten 
%Forschungsthemen im Bereich der Zivilluftfahrt im Hinblick auf die Gewährleistung der Kohärenz und Koordinierung
%zwischen öffentlich finanzierter Forschung und Entwicklung und den Maßnahmen, die unter diese Verordnung fallen.

%Die Agentur kann Forschungstätigkeiten entwickeln und finanzieren, soweit sie sich ausschließlich auf die
%Verbesserung hinsichtlich Tätigkeiten in ihrem Zuständigkeitsbereich beziehen. Der Forschungsbedarf und die Tätigkeiten
%der Agentur werden in ihrem jährlichen Arbeitsprogramm aufgeführt. 

%Artikel 87
%Die für den Zweck der Zertifizierung der Konstruktion von Erzeugnissen gemäß Artikel 11 von der Agentur
%getroffenen Emissions- und Lärmschutzmaßnahmen sollen maßgebliche schädliche Auswirkungen auf das Klima, die
%Umwelt und die menschliche Gesundheit durch die betreffenden Erzeugnisse der Zivilluftfahrt verhindern, wobei die
%internationalen Richtlinien und Empfehlungen, die Vorteile für die Umwelt, die technische Machbarkeit und die
%wirtschaftlichen Auswirkungen gebührend zu berücksichtigen sind."


Solche politischen Entscheidungen können die Fluggesellschaften zu motivieren, Emissionen zu vermeiden und somit einsparen.

Ab Jahr 2012 wurde in der Europäische Union (EU) EU-Emissionsrechthabndel (EU ETS) für Luftverkehr eingeführt.
Ein Großteil der Zertifikate wird unentgeltlich für die Fluggesellschaften zugeteilt, den Rest wird versteigert. 
Das Ziel ist die begrenzte Zertifikate unter Gesellschaften zu verteilen und somit die Treibhauseffekte zu senken.
Die Einnahmen von der Versteigerung werden dafür eingesetzt.\cite{conrady2019luftverkehr}

Im Jahr 2016 hat Internationale Zivilluftfahrtorganisation (ICAO) die Resolution A39-3 herausgebracht, 
wo Carbon Offsetting and Reduction Scheme for International Aviation (CORSIA) als globale Maßnahme für den Markt vorgestellt wurde. 
Die Einführung erfolgt in 3 Phasen: 
Pilotphase 2023; ab Jahr 2024 verläuft die Erste Phase und schließlich die Dritte Phase, wo Mitglieder verpflichtet
\cite{conrady2019luftverkehr}. 
Werden die Emissionen von dem Jahr 2020 überschritten, müssen die Beteiligte in anderen Bereichen die Kohlenstoffdioxide ausgleichen, z.B.
durch Beforstung oder anderen Klimainitiativen. 

Ab Jahr 2027 werden alle Betreiber aus dem Europäischen Wirtschaftsraum (EWR) entweder CORSIA oder EU ETS unterliegen \cite{uba_aviation_2023}.
Im Raum der Europäischen Wirtschaftsraum gilt EU ETS und hat anspruchsvollere Klimaschutzmaßnahmen als CORSIA \cite{uba_aviation_2023}.
CORSIA-Regeln gelten hingegen für internationale Flüge außerhalb Europäischen Wirtschaftsraum (EWR).

Politische Anreize?
Mit dem Gesetzgebungspaket "Fit to 55" will die EU bis zum Jahr 2030 die Emissionen, unter anderem vom Luftverkehr, um mindestens 55 \% senken.
Laut Verordnung RefuelEU legt fest, welche Anteil der nachhaltigen Kraftstoffe (SAF) soll im Verbrauch sein, Verpflichtungen für 
Kraftstofflieferanten, Flughäfen in der Union, sowie Luftfahrzeugbetreiber.

Ab Jahr 2028 neuer Standard für \ce{CO2}-Emissionen Grenzwerte.