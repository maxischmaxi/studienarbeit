\section{Klimapolitische Maßnahmen}
\label{s:Klimapolitische Maßnahmen}

%Luftverkehr hat zwar noch die Emissionswerte vom vor Pandemie COVID-19 erreicht, jedoch erholt sich langsam
% (https://www.umweltbundesamt.de/daten/klima/der-europaeische-emissionshandel#luftverkehr-im-emissionshandel-)

Durch klimapolitische Initiativen kann die Verminderung von Emissionen erreicht werden. 
So legt ICAO Annex 16 weltweite Grenzwerte für Umweltstandards und technische Anforderung 
für die Flugzeuge fest, um Emissionen zu reduzieren.

Auch die Verordnung (EU) 2018/1139\footnote{Verordnung (EU) 2018/1139 des Europäischen Parlaments und des Rates vom 4. Juli 2018 
zur Festlegung gemeinsamer Vorschriften für die Zivilluftfahrt und zur Errichtung einer Agentur
der Europäischen Union für Flugsicherheit sowie zur Änderung der Verordnungen (EG) Nr. 2111/2005, 
(EG) Nr. 1008/2008, (EU) Nr. 996/2010, (EU) Nr. 376/2014 und der Richtlinien 2014/30/EU und 2014/53/EU.} 
%
etabliert die gesetzlichen Instrumente 
für Sicherheit und den Umweltschutz in der Zivilluftfahrt innerhalb der EU 
und fixiert die Rolle der Europäischen Agentur für Flugsicherheit (EASA). 
Die EASA ist als Luftfahrtbehörde verpflichtet die Umweltleistung der Luftfahrt 
zu überwachen und zu fördern, wofür Gebühren erhoben werden. 
Außerdem macht die Verordnung die Zulassung von neuen Flugzeugen leichter 
wodurch Innovationen und Anregungen für nachhaltige Luftfahrttechnologien gefördert werden.

Die Verordnung ermöglicht Gebührenanpassungen und Anreize für die Projekte, 
die ökologische Vorteile bringen und zu einem hohen Umweltschutzniveau beitragen.
%
Die konkreten Maßnahmen dürfen auf nationaler Ebene gesteuert werden, 
aber die die Mitgliedstaaten sind verpflichtet, die Sicherheit und den Umweltschutz zu betrachten. 
Nach Artikel 126 kann die EASA die detaillierten 
Vorschriften für Gebühren bestimmen, um nachhaltige Technologien und innovative Projekte zu fördern.
%(nach Konsultation der Kommission) 
%"Artikel 86
%und Agentur (EASA) unterstützt die Kommission und die Mitgliedstaaten bei der Ermittlung der wichtigsten 
%Forschungsthemen im Bereich der Zivilluftfahrt im Hinblick auf die Gewährleistung der Kohärenz und Koordinierung
%zwischen öffentlich finanzierter Forschung und Entwicklung und den Maßnahmen, die unter diese Verordnung fallen.
%
%Die Agentur kann Forschungstätigkeiten entwickeln und finanzieren, soweit sie sich ausschließlich auf die
%Verbesserung hinsichtlich Tätigkeiten in ihrem Zuständigkeitsbereich beziehen. Der Forschungsbedarf und die Tätigkeiten
%der Agentur werden in ihrem jährlichen Arbeitsprogramm aufgeführt. 
%
%Artikel 87
%Die für den Zweck der Zertifizierung der Konstruktion von Erzeugnissen gemäß Artikel 11 von der Agentur
%getroffenen Emissions- und Lärmschutzmaßnahmen sollen maßgebliche schädliche Auswirkungen auf das Klima, die
%Umwelt und die menschliche Gesundheit durch die betreffenden Erzeugnisse der Zivilluftfahrt verhindern, wobei die
%internationalen Richtlinien und Empfehlungen, die Vorteile für die Umwelt, die technische Machbarkeit und die
%wirtschaftlichen Auswirkungen gebührend zu berücksichtigen sind."
Solche politischen Entscheidungen können die Fluggesellschaften motivieren, 
Emissionen zu vermeiden und somit wirtschaftliche Einsparungen zu erzielen.

Ab dem Jahr 2012 wurde in der Europäischen Union (EU) EU-Emissionsrechthandel (EU ETS) für den Luftverkehr eingeführt.
Dabei können Zertifikate erworben werden, um eine bestimmte Anzahl an Emissionen freisetzen zu können.
Ein Großteil der Zertifikate wird unentgeltlich den Fluggesellschaften zugeteilt, den Rest wird versteigert. 
Das Ziel ist die begrenzte Zertifikate unter Gesellschaften zu verteilen und somit die Treibhauseffekte zu senken.
Einnahmen der Versteigerung werden für die Bekämpfung von Einflüssen eingesetzt \cite{conrady2019luftverkehr}.
%andere Quellen?
Die Internationale Zivilluftfahrtorganisation (ICAO) hat im Jahr 2016 die Resolution A39-3\footnote{Resolution A39-3. ICAO zur „Consolidated Statement of Continuing ICAO Policies and Practices Related to Environmental Protection – Global Market-based Measure (MBM) Scheme“, 2016.} veröffentlicht, 
in welcher das Carbon Offsetting and Reduction Scheme for International Aviation (CORSIA) 
als globale Maßnahme für den Markt vorgestellt wurde. 
Werden die Emissionen des Jahres 2020 überschritten, müssen Beteiligte in anderen Bereichen 
die Kohlenstoffdioxide ausgleichen, z.B. durch Aufforstung oder anderen Klimainitiativen. 
Die Einführung erfolgt in 3 Phasen: 
bis zum Jahr 2023 verlief die Pilotphase; ab Jahr 2024 bis 2026 verläuft die Erste Phase und schließlich die Zweite Phase, 
in der die Mitglieder verpflichtet sind, teilzunehmen \cite{conrady2019luftverkehr}. 


Ab dem Jahr 2027 werden alle Betreiber aus dem Europäischen Wirtschaftsraum 
entweder CORSIA oder EU ETS unterliegen \cite{uba_aviation_2023}.
Im Europäischen Wirtschaftsraum gilt EU ETS und hat anspruchsvollere 
Klimaschutzmaßnahmen als CORSIA \cite{uba_aviation_2023}.
CORSIA-Regeln gelten hingegen für internationale Flüge außerhalb des Europäischen Wirtschaftsraums.

Mit dem Gesetzgebungspaket "Fit to 55" will die EU bis zum Jahr 2030 die Emissionen, 
unter anderem die des Luftverkehrs, um mindestens 55 \% senken.
Die Verordnung RefuelEU legt fest, welchen Anteil nachhaltige Kraftstoffe (SAF) am Verbrauch haben müssen, 
und enthält zudem Verpflichtungen für Kraftstofflieferanten, Flughäfen in der Union sowie Luftfahrzeugbetreiber.
%Ab Jahr 2028 neuer Standard für \ce{CO2}-Emissionen Grenzwerte.
