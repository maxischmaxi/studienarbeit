\subsection{Bestehende Konzepte und zukünftige Flugzeugmodelle}
In diesem Teil ist beschrieben, welche Flugzeugmodelle und -konfigurationen sind in näheren Zukunft zu erwarten in der Planung sind oder.
Anhand der neuen Antriebe müssen die neuen Flugzeugkonzepte und -modelle angeschaut werden, 
aus diesem Grund stellt dieser Unterkapitel vorhandene Konzepte.
Aufgrund ihrer Drop-In Fähigkeit werden die SAFs keine neuen Luftfahrzeugkonfigurationen brauchen.
Positive Auswirkungen auf die Emission-Werte können bereits mit bestimmten Flugzeug- und Triebwerkskonfigurationen erreicht werden.
Zum Beispiel "Claire Liners" vom Bauhaus Luftfahrt e. V. München nicht nur aerodynamische Vorteile mitbringt, 
sondern auch Verringerung des Kraftstoffverbrauchs und damit Reduktion der Emissionen.
Jedoch sind mehr Änderungen notwendig, um netto null \cite{CO2}-Werte zu erreichen.
\\
\textbf{Konfigurationen mit Batterie-Antrieb}\\
Wie bereits erwähnt wurden, haben die zurzeit bestehenden Batterien die geringe Energiedichte. Dadurch es ist zu erwarten, dass bis zum Jahr 
2050 keine große vollelektrische Flugzeuge hergestellt werden, sondern werden die Regional- und Kurzstrecken in den Mittelpunkt gestellt.

ES-19\\
Ein vielversprechender Prototyp war die ES-19 von Heart Aerospace. Das Unternehmen versprach die Beförderung von 19 Passagiere über 400 km mit einem BA. 
Das Flugzeug war für die Regionalstrecken konzipiert und somit kann die geringe Nachfrage gedeckt werden. 
Außerdem waren geringe Betriebs- und Wartungskosten erwartet (Quelle).
Jedoch zu dem Zeitpunkt wurde das Flugzeug auf ES-30 mit einem hybriden Antrieb umgerüstet.


\textbf{Konfigurationen mit Wasserstoff-Antrieb}

Embraer hat eine Reihe von nachhaltigen Flugzeugen ENERGIA H2 GAS TURBINE E50-H2GT Dual Fuel Gas Turbine Propulsion
"• 100\% Hydrogen energy for typical routes
• SAF or JetA used for reserves and range extension
• Hydrogen storage customized according to operators’ specific
needs
• Rear mounted engines with optimized propeller design to
decrease noise"
Embraer zeigte eine Reihe von nachhaltigen Flugzeugen ENERGIA. Die Flugzeuge haben unterschiedliche Antriebe, unter anderem 
hybrid-elektrisch, Wasserbrennstoffzelle und Wasserstoffturbine. Bei Wasserstoffturbine wurde das Konzept von Dualen-Treibstoff vorgeschlagen, 
wo entweder JetA/SAF oder Wasserstoff benutzt werden kann.

Airbus hat im Jahr 2020 drei unterschiedlichen emissionsfrei ZEROe Konzepte vorgestellt. 
In allen Konzepten ist der Wasserstoff im Einsatz und mit Wasserstoffturbinen. Die Reichweite breitet sich ab über 1.850 - 
3700 km und die Anzahl beförderte Passagiere wird von 100 bis 200 geschätzt \cite{airbus_zea_concepts}.

Wright Spirit \cite{wright_electric_website} hat ein Konzept auf der Basis der konventionellen Flugzeug BAe 146 vorgestellt, aber mit einem Wasserstoff-Antrieb.
Das Flugzeug soll mit 4 Triebwerken, 2,5 MW Motoren und vorgestellter Batterie mit 800 Wh/kg eine Reichweite von 1000 
km erreichen und 100 Passagiere transportieren.


Universal Hydrogen

ZeroAvia stellt ihrer hybrid Wasserstoff-elektrischen Antriebe mit 3 unterschiedlichen Leistungen und Kapazitäten vor. Das kleinste davon
ist Antrieb ZA600 mit einer Leistung von 600 kW, mit der Möglichkeit bis 20 Passagieren über 555 km zu befördern. 
Geplante Eintrittzeit (Entry-in-System EIS) ist im Jahr 2025. Der Antrieb ist mit gasförmigem Wasserstoff angetrieben.
%Flugzeug mit 80 Menschen verbraucht bis zu 80% weniger Treibstoff pro STD/kg (5 mal weniger)

E-Genius and 
Pipistrel Alpha Electro


Es werden unterschiedliche wasserstoffkonfugurationen vorgeschlagen aufgrund des schweren und massiven Wasserstofftanks.
Es gibt Konzepte, wo der Tank am Rumpf, am Ende des Rumpfes, vorne hinter Pilotenkabine