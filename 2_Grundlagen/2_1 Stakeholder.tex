\chapter{Relevante Grundlagen und Überblick über alternative Antriebe}
\label{ch:Relevante Grundlagen und Überblick über alternative Antriebe}
Für die Analyse der Forschungsfrage es ist wichtig die zentralen theoretischen Begriffe zu definieren. 
Das Kapitel \ref{s:Bodenabfertigung eines Luftfahrzeugs} stellt die Grundlagen der Flugzeugabfertigung und Definition
der beteiligten Stakeholder am Flughafen dar. Zunächst beschäftigt sich das Kapitel \ref{s:Kosten}
mit den bedeutenden Informationen zu Kosten am Flughafen und Emission-Regulierungsinitiativen. 
Anschließend werden im Teil \ref{s:Neartige Antriebe}
die neuartigen alternativen Antriebe und Konzepte und Flugzeugmodelle mit diesen Antrieben vorgestellt.

\section{Stakeholder am Flughafen}
\label{s:Stakeholder am Flughafen}
%
\begin{figure}[h]
	\centering
	\includegraphics[width=0.8\linewidth]{Bilder/Stakeholder.png}
	\caption[Relevante Stakeholder am Flughafen]{Relevante Stakeholder am Flughafen}
	\label{stakeholder}
\end{figure}

Am Flughafen ist eine Vielzahl an Stakeholdern beschäftigt, die miteinander agieren. Durch die neuen Luftfahrzeugantriebe 
steht diesen Akteuren eine schwierige Aufgabe vor. Gute Zusammenarbeit der Stakeholdern fördert 
die Pünktlichkeit der Abfertigung und hilft die Verspätungen zu vermeiden \cite{schmidt2016challenges}.\\

\textbf{Flughafen} \\
Einer der Stakeholder am Flughafen ist der Flughafenbetreiber selbst. 
Der Flughafen stellt der Fluggerät- und Passagierabfertigung Infrastruktur wie bspw. Terminals oder Start- und Landebahnen zur Verfügung (das gilt als Kernfunktion), 
wofür Nutzungsgebühren erhoben werden \cite{conrady2019luftverkehr}. %Seite 180, falls ich Entgelte durchzählen will.

Zum Flughafen gehören außer Start- und Landebahnen unter anderem Rollwege, Vorfeld, Flugsteige, sowie die Infrastruktur für die Gepäckabfertigung. 
%
Darüber hinaus stellen Flughäfen eine intermodale Verknüpfung dar \cite{conrady2019luftverkehr}. %d.h. die Anbindung an anderen Verkehrsmitteln wird hergestellt.
Direkte Nutzer von Flughäfen sind die Im- und Exporteure von Dienstleistungen und Waren \cite{schaar2010analysis}. 

Flughäfen sind ein großer Teil der regionalen Wirtschaft \cite{schaar2010analysis} und sorgen für eine Vielzahl an Arbeitsstellen. 
Dennoch verursachen sie ein Ausmaß an Lärm und Umweltbelastungen, die durch die Emissionen der Flugzeuge entstehen.
Demnach verlangt der Flughafen hierfür ebenfalls Entgelte.\\ %Kap 4
%
Für die Entwicklung der Infrastruktur und Begleichung der Betriebskosten müssen Flughäfen gelegentlich finanzielle Unterstützung aus anderen
Quellen, wie staatliche Subventionen, in Anspruch nehmen \cite{schaar2010analysis}.
Die Europäische Kommission besagt, dass Flughäfen mit einem Passagieraufkommen von über 3 Millionen 
Passagiere jährlich in der Lage sind, ihre Betriebskosten selbst durch Gewinn zu decken.\footnote{"Leitlinien für staatliche Beihilfe für Flughäfen und Luftverkehrsgesellschaften" 2014/C 99/03}
%Und die kleineren Flughäfen durch weniger Betrieb mehr auf die Hilfe angewiesen.
Eine Kategorisierung der Flughäfen basiert auf der Passagiermenge. 
Der Europäischen Kommission nach werden die Flughäfen nach jährlichem Passagieraufkommen folgend unterteilt: 
\begin{itemize}
    \item große Gemeinschaftsflughäfen > 10 Mio. Passagieren;
    \item nationale Flüge mit 5 bis 10 Mio. Passagieren;
    \item große Regionalflughäfen mit 1 bis 5 Mio. Passagieren;
    \item kleine Regionalflughäfen < 1 Mio. Passagieren.
\end{itemize}
Aufgrund dieser Kategorisierung in dem Jahr 2023 gab es in Deutschland 
sieben große Gemeinschaftsflughäfen, einschließlich zwei Hubs, und 16 Regionalflughäfen.\footnote{Die Daten stammen aus dem Statistischem Bericht, "Luftverkehr auf Hauptverkehrsflughäfen 2023"}
Ein Hub ausmacht ein großer Flughafen mit mächtigem Anteil an Umsteigeverkehr.\\

\textbf{Fluggesellschaft} \\
Fluggesellschaften sind Dienstleister, welche die Infrastruktur eines Flughafens für die Abfertigung von Passagieren und Fracht nutzen. 
Sie sind gewinnorientiert und haben das Ziel wettbewerbsfähig zu bleiben. 
Für eine Fluggesellschaft ist von Relevanz, wie hoch die Betriebskosten (Erträge)
sind, die der Flughafen verlangt \cite{schaar2010analysis}. Die Erträge unterscheiden sich sowohl je nach Flughafengröße und -strategie,
als auch von dem Flugzeugtyp.\\
%
%Fluggesellschaften und Treibstoff-Firmen sind für die sichere Betankung verantwortlich. Quelle: Annex 14 (Doc 9137 Teil 8)
\textbf{Bodenverkehrsdienste} \\ %S. 183 conrady

Bodenverkehrsdienste sind für die Abfertigung der Flugzeuge auf dem Boden zuständig.
Nach Conrady \cite{conrady2019luftverkehr} gehört zu ihren Tätigkeiten außerdem:  
die Fluggastabfertigung, administrative Abfertigung sowie Transportdienste.
Sie sind auf Infrastruktureinrichtungen wie Gepäckförderanlagen und Betankungsanlagen und weitere Grundausstattung am Vorfeld angewiesen. 
Die Abfertigung kann entweder von einer Fluggesellschaft, einem Flughafen oder einem unabhängigen Dienstleister durchgeführt werden. 
Meistens werden die Bodenverkehrsdienste in Deutschland von den Flughäfen übernommen.\\ %oder die externen Firmen (Dritte) werden engagiert.
%
Bodenverkehrsdienste sind auch für den Transport von Fracht, Post und Gepäck bis zum Flugzeug zuständig \cite{mensen2013handbuch}.\\
Zu anderen Vorfelddienste gehören Betankungsdienste und Reinigungsdienste.
Betankungsdienste führen nicht nur die Be- und Entladung und Lagerung durch, sondern auch für andere Flüssigkeiten (wie z.B. Öl) zuständig.
Wartungsdienste führen die routinemäßige Kontrolle den Flugzeugen vor den Flügen (Line Maintenance).
Die Reinigungsdienste und der Flugzeugservice sind Reinigung von Innen und Außen eines Flugzeugs verantwortlich, Wasserservice, 
Klimaanlagen in der Kabine und Enteisung.
%
%OPS 1.1150 "Handling agent. An agency which performs on behalf of the operator some or all of the latter's functions
%including receiving, loading, unloading, transferring or other processing of passengers or cargo;"
%
%Von der Fluggesellschaft werden Handling Agents eingestellt, der die ganze Abfertigung und Kommunikation zwischen Beteiligten am Vorfeld
%kontrollieren.
%In dieser Arbeit werden nur die internationale und regionale Verkehrsflughäfen betrachtet. 
%(Es bietet sonstige Serviceleistungen für die Passagiere, wie Parkplätze, Handel Dienstleistungen.)

Zu den Systempartnern (Stakeholdern) am Flughafen zählen ebenfalls Luftfahrzeughersteller, Flugsicherungen, 
Reiseveranstaltern, staatliche Institutionen \cite{maertens2023neue},
sowie Beteiligte wie Passagiere, Arbeitskräfte und Passagierdienstleister. 
Sie nehmen nicht direkt an der Flugzeugabfertigung bzw. an Betrieb am Vorfeld teil, deswegen werden sie außerhalb des Fokus dieser Arbeit bleiben.
Analog hierzu wird die Flugsicherung aufgrund unveränderter Umstände (Bedingungen) durch alternative Antriebe nicht betrachtet. 
Die Arbeit wird sich auf die Betriebskosten einer Fluggesellschaft und Infrastrukturkosten des Flughafens fokussieren.
%
%Auf die Kosten eingehen:
%Laut OPS 1.175 %"The number of ground staff is dependent upon the nature and the scale of operations"
%Anzahl der benötigten Bodenmitarbeiter ist von dem Maßstab der Operationen am Flughafen anhängig.
%
%BAs können weniger überlastete Flughäfen anfliegen und entferne Bereiche, und
%nur die geringe Bedarf abdecken.
%Europäische Kommission Leitlinien für staatliche Beihilfe für Flughäfen und Luftverkehrsgesellschaften 2014/C 99/03
%
%Als Regionalflughafen definiert E Kommissionen einen Flughafen mit bis zu 3 Millionen Passagieren im Jahr.
%Flughäfen mit mehr als eine Million Passagieren im Jahr decken überwiegend ihre Betriebskosten selbst. %egal?
%
%„Betriebskosten“: die mit der Erbringung von Flughafendienstleistungen verbundenen Kosten eines Flughafens;
% dazu gehören Kostenkategorien wie Personalkosten, Kosten für fremdvergebene Dienstleistungen, Kommunikation, 
% Abfallentsorgung, Energie, Instandhaltung, Mieten und Verwaltung, jedoch weder Kapitalkosten, Marketingunterstützung 
% bzw. andere Anreize, die der Flughafen den Luftverkehrsgesellschaften bietet, noch Kosten für Aufgaben mit hoheitlichem Bezug;
%
%
%Der Bedarf an öffentlichen Mitteln zur Betriebskostenfinanzierung variiert unter den derzeitigen Marktbedingungen
% aufgrund der hohen Fixkosten in der Regel je nach Flughafengröße und ist normalerweise bei kleineren Flughäfen
% verhältnismäßig höher. Unter den derzeitigen Marktbedingungen können nach Auffassung der Kommission in Bezug auf
%   die jeweilige finanzielle Tragfähigkeit nachstehende Kategorien von Flughäfen abgegrenzt werden:
%
%d)Flughäfen mit 1 bis 3 Millionen Passagieren im Jahr dürften im Durchschnitt in der Lage sein, ihre Betriebskosten überwiegend selbst zu tragen;
%
%e)Flughäfen mit mehr als 3 Millionen Passagieren im Jahr erzielen in der Regel einen Betriebsgewinn und dürften 
%in der Lage sein, ihre Betriebskosten zu decken.
\section{Bodenabfertigung eines Luftfahrzeugs}
\label{s:Bodenabfertigung eines Luftfahrzeugs}

Zur Veranschaulichung der Änderungen an der Infrastruktur am Flughafen die durch neuartige Antriebe vorgenommen werden müssen, 
ist es notwendig wichtige Begriffe einer Abfertigung des konventionellen Flugzeugs hervorzuheben. 
Unter konventionellen Luftfahrzeugen sind die zu verstehen,
die mit fossilen Treibstoffen, wie Kerosin auf der Ölbasis, betrieben werden. Der Fokus wird auf die gewerblichen Passagierflugzeuge gelegt,
weil die Abfertigung von Passagieren besonders strengere Sicherheitsmaßnahmen erfordert. %oder dass sie einen erhebliche Teil an zivile Luftverkehr ausmachen.

Die Blockzeit setzt sich aus der Zeit vom Beginn der Bewegung von der Parkposition bis zum Ende der Bewegung zur Parkposition, 
einschließlich der Flugzeit, zusammen.
An der Parkposition des Flughafens werden die Triebwerke ausgeschaltet und der Ablauf eines Turnaround beginnt. 
Mensen \cite{mensen2013handbuch} definiert den Turnaround, wie die Abfertigung der Flüge, die zeitnah zusammen liegen.
Bei einem Turnaround wird das Luftfahrzeug durch viele Akteure am Flughafen, wie Flugplatzbetreiber, Fluggesellschaft und die Dritte, für 
den nächsten Flug vorbereitet \cite{mensen2013handbuch}. Es muss ausgeladen, kontrolliert, gereinigt, anschließend versorgt 
und für den nächsten Flug beladen werden. \\

Die Abbildung \ref{abfertigung} stellt die Abfertigung eines Flugzeugs an der Parkposition dar.
Nach ICAO Doc 9157 besteht Abfertigung eines Passagierflugzeugs insgesamt aus Passagier-, Gepäck- und Frachtabfertigung, 
Sanitärservice, Wasserbetankung, Gepäckabfertigung, Betankung, Stromversorgung,
Startluft, Flugzeugschleppen, Bordküchenservice, Wartungsservice sowie Bereitstellung einer Klimaanlage und Sauerstoff,
wie in der Abbildung dargestellt. Durch neuen Antriebe kann es zu Änderung diesen Prozessen kommen aufgrund andere technischen Grundlagen.
Laut EU-OPS 1.305 darf das Luftfahrzeug aus Sicherheitsgründen erst betankt werden, wenn die Passagiere sich nicht an Bord befinden. 
%
%Das Flugzeug wird an ein Hilfstriebwerk (auxiliary power unit - APU) angeschlossen \cite{mensen2013handbuch}. 
%Die APU liefert Strom, wenn die Haupttriebwerke nicht laufen (quelle: [Annex 14. Doc 9137 Part 8]).
%Parallel werden Fracht und sonstige Gepäckeinheiten mit dem Hubwagen abgeladen und mit Transporthängern zur Sortieranlage 
%im Terminal gebracht \cite{mensen2013handbuch}. Im Falle, das die Parkposition direkt am Flughafen ist, 
%können Passagiere direkt über die Treppe oder Fluggastbrücke zum Terminalgebäude gelangen. 
%Wenn die Parkposition am Vorfeld liegt, muss auf einen Bus zurückgegriffen werden. 

\begin{figure}[h]
	\centering
	\includegraphics[width=0.8\linewidth]{Bilder/A321_Abfertigung.png}
	\caption[Abfertigung]{Abfertigung eines A321 \cite{airbus2022a321} mit eigenem Hinweis}
	\label{abfertigung}
\end{figure}

Je nach Flugdistanz und nach Flugzeuggröße kann es zu unterschiedlichen Abfertigungszeiten kommen. Bei den kleineren Flugzeugen ist die Dauer 
kürzer als bei einer größeren Maschine. 
In Bezug auf die Transportdistanz wird nach Kurz- (ca.2 Stunden oder bis 1000 km) 
und Mittelstreckenflüge (bis 3,5 Stunden oder bis 3000 km), Langstreckenflüge (ab 3,5 Stunden und ab 3000 km) unterschieden \cite{mensen2013handbuch}.
Die Definition von Distanzen variiert teils erheblich, z.B. Flughafen Frankfurt definiert die Langstrecken ab 6000 km.% diese Werte werden auch im Kapitel \ref{s:Betriebsszenarien} genutzt.

\textit{Konventionelle Treibstoffe}\\
Zurzeit werden die Treibstoffe auf der Basis der fossilen Energie, wie Öl benutzt. Die Ölpreise sind ziemlich instabil. %wenn du quelle findest is okay
Um den Schub zu erzeugen, wird in der Gasturbine der Treibstoff verbrannt, wodurch die mechanische Leistung entsteht 
und über eine Welle den Propeller oder Strahltriebwerk antreibt. Durch Verbrennung des Treibstoffs entstehen die Abgase, wie auf der Abbildung
%habe Quelle nicht mehr parat

\cite{schmidt2016challenges}