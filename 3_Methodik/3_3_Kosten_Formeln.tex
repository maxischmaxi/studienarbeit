\section{Aufstellung der Formeln für Kosten}
\label{s:Aufstellung der Formeln für Kosten}

In dem Kapitel \ref{s:Kosten} sind die entstehenden Kosten beim Betrieb der Fluggesellschaft definiert.
Schließlich sind in diesem Unterkapitel die dazugehörigen Formeln anhand anderer Modelle vorgestellt und teilweise angepasst. 

Es gibt eine Reihe von Methoden, um DOC zu berechnen. 
Diese wurden in dem Teil \ref{s:Kosten} erwähnt.
Als Grundlage für ein Modell wurde die Association of European Airlines (AEA) 1989 gewählt, 
da sie häufige Anwendung in der akademischen Welt hat, 
sehr umfassend ist und Berechnungswerte für sowohl Kurz-, als auch Langstrecken hat. 
Die Formeln für AEA wurden von \cite{} entnommen.
Falls die Quelle abweicht, wurde sie explizit angegeben. % MAX: was meinst du damit?

Die Betriebskosten werden mit konstanter Reisegeschwindigkeit und Verbrauch ohne 
Berücksichtigung erheblicher Energieverluste während des Starts und der Landung berechnet. 
Die Inflationsfaktoren stammen aus dem Verbraucherpreisindex Deutschland des Statistischen Bundesamts, 
entsprechend eingesetzte Werte sind in der Anlage XX zu finden. % MAX: anlage?
Alle Werte in USD werden mit dem Wechselkurs\footnote{Wechselkurs vom 14.01.2025} (1 EUR = 1,0245 USD)
umgerechnet, alle Werte in Pfund mit dem Kurs (1 GBP = 1,20 EUR).
Aufgrund der hohen Unsicherheit der Preisprojektionen wird der Preis für Kerosin 
konstant gehalten, es wird von einem Wert von 0,688 EUR pro Liter\footnote{Der Wert von 112,2 USD/bbl 
wurde verwendet, anschließend in Liter umgerechnet und mit dem aktuellen Wechselkurs angepasst.} 
ausgegangen \cite{iata_industry_statistics_2024}. 
Stromkosten von 0,1976 € pro kWh werden ebenfalls als konstant angenommen \cite{eurostat_nrg_pc_205}.
In Anbetracht der Kostensenkung für nachhaltige Kraftstoffe in der Zukunft, 
wird der minimale Preis für HEFA-Treibstoff von 1,07 EUR pro Liter in den Berechnungen verwendet \cite{watson2024sustainable}.
Wegen der starken Schwankungen des Wasserstoffpreises wird der Durchschnittspreis für 
elektrolytisch hergestellten Wasserstoff von 4,88 EUR/kg eingesetzt \cite{hoelzen2022hydrogen}.
%
%Der Reservebedarf beträgt 30 \% der Gesamtkapazität.
%
\subsection{Betriebskosten einer Fluggesellschaft}

Die Gleichung \eqref{DOC} stellt die Betriebskosten $DOC$ der Flugzeugnutzung dar. 
Es werden sowohl variablen als auch ein Teil der Fixkosten betrachtet.
Diese bestehen aus Treibstoff-/Energiekosten $C_T$, 
Wartungskosten $C_W$, Entgelten und Gebühren $C_{EG}$, 
Kosten für Personal $C_{Crew}$ und kapitalgebundenen Kosten $C_{KK}$.
%
\begin{equation}
     {DOC ~[\text{EUR}]} = C_T + C_W + C_{Crew} + C_{KK} + C_{EG} \\
     \label{DOC}
  \end{equation}
%
Die Betriebskosten werden auf Basis von Blockstunden kalkuliert. 
Blockstunden setzen sich aus Flugzeit $t_{F,h}$ sowie der kumulierten Rollzeit $t_{R}$ 
von und zur Parkposition $t_{R}$ zusammen. 
Für Kurz- und Mittelstrecken beträgt $t_{R}$ {0,25 h} und für Langstrecken 0,42 h \cite{scholz_design_evaluation_doc}.

\begin{equation}
   t_{B} ~[\text{h}] = t_{F,h} + t_{R} \\
   \label{blockzeit}
\end{equation}

\textbf{Treibstoff-/ Energiepreise} hängen vom Treibstoff- bzw. Energiepreis selbst $P_{T/E}$ 
und vom Verbrauch eines Flugzeugs pro Blockstunde $m_{V}$ ab (vgl. \eqref{fuel}) ab.

\begin{equation}
   {C_T ~[\text{EUR}]} = (P_{T/E} \cdot m_{V} \cdot t_B) \\
   \label{fuel}
\end{equation}
%
%Die Treibstoff- bzw. Energiepreise $P_{T/E}$ sind in der Tabelle zusammengeführt.

Die \textbf{Wartungskosten} werden nach dem Jenkinson 1999 Modell berechnet. 
Das Modell ermöglicht es, grob aber schnell die Wartungskosten 
für ein Flugzeug abzuschätzen \cite{bruge2018wartungskosten}.
Da dieses Modell auf konventionelle Flugzeuge ausgearbeitet wurde, 
berechnet man die Wartungskosten für alternative Antriebe als Prozentanteil des Referenz-Flugzeugs, 
die in \ref{ss:Relevante Flugzeugdaten} angeführt sind.
Die Formeln stammen aus \cite{bruge2018wartungskosten} und beziehen sich auf das Jahr 1994, 
somit muss der Inflationsfaktor $k_{Infl}$ einkalkuliert werden. 
Die Berechnung liefert die Ergebnisse in USD, 
aus diesem Grund wird ein Wechselkursfaktor $k_{WK}$ in die Berechnung implementiert.
Wartungskosten werden normalerweise auf die Wartung der Flugzeugzelle $C_{W,Zelle,B}$ 
und der Triebwerke $C_{W,Triebwerk,B}$ aufgeteilt, 
wie in den Gleichungen \eqref{maintenance} dargestellt. 
Die Wartungskosten der Zelle sind von dem leeren Betriebsgewicht 
(Operating Empty Mass $m_{OE}$) abhängig. 
Die Kosten des Triebwerks sind vom erzeugten Schub beim Start (Take-Off Thrust $T_{T/O}$) abhängig.

\begin{equation}
   \begin{split}
   {C_{W,B} ~[\text{EUR}]} = (C_{W,Zelle,B} + n_{T} \cdot C_{W,Triebwerk,B} ) \cdot t_{B} \cdot n_{F, Jahr} \cdot k_{WK} \cdot k_{Infl}\\
%
   {C_{M,AF,b} ~[\text{USD}]} = (175 \frac{USD}{h} + 0,0041 \frac{USD}{h} \cdot m_{OE} )\cdot k_{Infl}\\
%
   {C_{M,E,L,b} ~[\text{USD}]} = (0,00029 \frac{USD}{h} \cdot T_{T/O} )\cdot k_{Infl}\\
   \label{maintenance}
   \end{split}
\end{equation}
%
Diese Formeln sind für Triebwerke mit einem Nebenstromverhältnis von 5:1 ausgelegt \cite{bruge2018wartungskosten}. %warum hier die 5?
Trotz der höheren Nebenstromverhältnisse die bei moderneren und größeren Flugzeugen gegeben sind,
wird aus Gründen der Vereinfachung diese Formel genutzt.\\
%
%

Zu \textbf{Entgelten und Gebühren} $C_{EG}$ gehören Flughafenentgelte 
(Passagier-, Lande- und Startentgelte, Sicherheitsentgelte sowie Abfertigung am Vorfeld)
und die Flugsicherungsgebühr. 
Die detaillierte Beschreibung der benutzten Entgelten und Gebühren sind in der Anhang A1.3 zu finden.\\ % MAX: anhang?

\textbf{Crewkosten} setzen sich aus Lohnkosten für Piloten $L_{Pilot}$ und Besatzung $L_{crew}$ zusammen. 
Die Anzahl der Besatzungsmitglieder $n_{crew}$ ist von der Anzahl der Passagiere abhängig. 
Gemäß den luftfahrtrechtlichen Bestimmungen ist pro 50 Passagiere ein Flugbegleiter notwendig \cite{conrady2019luftverkehr}.
Die Besatzungskosten sind mit einem Durchschnittslohn von 37 EUR für Flugbegleiter 
und 90 EUR für Piloten pro Blockstunde berechnet \cite{discover_airlines_cabin}.
Die Werte setzen sich aus Grundgehalt und 75 Blockstunden pro Monat zusammen.

\begin{equation}
   {C_{crew} ~[\text{EUR}]} = ({L}_{Pilot} \cdot 2 + {L}_{crew} \cdot n_{crew} ) \cdot t_{B} \\
   \label{crew cost}
\end{equation}

%Analog zum \cite{marksel2023comparative} werden die Kosten für den Wasserstoff und Batterie Antrieb getrennt ausgerechnet, 
%wobei für Batterie-Antrieb anstatt eines Wasserstofftanks und eine Brennstoffzelle, basiert sich auf einer Batterie, Motor 
%und Leistungselektronik

%\begin{equation}
%   {P_{E,f} ~[\text{€}]} = P_{FC} \cdot + P_{EM} \cdot + m_F \\
%   \label{engine group price hydrogen}
%\end{equation}
Zu \textbf{kapitalbezogenen Kosten} gehören Abschreibungs-, Versicherungs- und Verzinsungskosten. 
Abschreibungskosten sind von den Anschaffungskosten bzw. dem Kaufpreis, 
der Abschreibungsdauer und den Blockstunden pro Jahr abhängig \cite{conrady2019luftverkehr}.
Die Abschreibungsdauer nach AEA beträgt jeweils 14 Jahre für 
Kurz- und Mittelstecken und 16 Jahre für Langstrecken \cite{scholz_design_evaluation_doc}.

Da keine öffentlich zugänglichen Marktpreise für Luftfahrzeuge mit alternativen Antriebssystemen vorliegen, 
wird die Kalkulation der Anschaffungskosten für elektrisch betriebene Flugzeuge 
nach der Methodik von \cite{monjon2020conceptual} durchgeführt. 
In der Arbeit wurde das Regressionsmodell \eqref{price BA} mit Abhängigkeit von Passagieranzahl $n_{PAX}$ 
anhand einer Marktanalyse für die Berechnung erstellt. 
%
In der Formel sind aufgrund der Einführung neuer Technologien 10 \% höhere Anschaffungskosten mitbetrachtet. 
Die Studie ist aus dem Jahr 2020, Preise wurden in USD berechnet, 
weshalb der Inflationsfaktor $k_{Infl}$ und der Wechselkurs $k_{WK}$ 
ebenfalls betrachtet werden.

\begin{equation}
   {C_{BA,ac} ~[\text{EUR}]} = (407408 \cdot n_{PAX} - 2967.4) \cdot k_{WK} \cdot {k_{Infl}}\\
   \label{price BA}
\end{equation}

Ein weiterer wichtiger Faktor der kapitalgebundenen Kosten ist die Auslastung $U$ eines Flugzeuges. 
Er wird durch die Anzahl der verfügbaren Stunden pro Jahr $t_{verf}$, 
der Block- $t_B$ und der Turnaround-Zeit $t_{TA}$ berechnet \cite{minwoo2019analysis}. 
Die jährliche Verfügbarkeitszeit $t_{verf}$ beträgt für Kurzstreckenflugzeuge 3750 h, 
während für Mittel- und Langstreckenflugzeuge ein Wert von 4800 h eingesetzt wird \cite{scholz_design_evaluation_doc}. 
Für die Kurzstrecken-Flugzeuge wurde eine Turnaround-Zeit von 1,5 h gewählt (Quelle).
Aufgrund der Größe und aufwendigeren Abfertigung wird für Mittel- und Langstrecken ein Wert von 2 h gewählt.

\begin{equation}
   U = \frac{t_{verf}}{t_B + t_{TA}} \\
   \label{utilisation}
\end{equation}
 
Weitere Formeln für kapitalgebundene Kosten sind in dem Anhang A1.2 zu finden.
%sind Anschaffungskosten geteilt durch 
%
%Für wasserstoffbetriebene Flugzeuge werden in der Quelle vorgestellte Formeln benutzt, um Kaufpreis einschätzen zu können.
%Außerdem bestehen bei alternativen Antrieben betriebliche Kosten, die durch Infrastruktur bedingt sind.
\subsection{Infrastrukturkosten}

\subsubsection{Batterieantrieb}
Kapitalkosten der Batterieantrieb-Infrastruktur sind in folgender Formel zusammengestellt:
%
\begin{equation}
     {CAPEX ~[\text{EUR}]} = C_{Bat} + C_{BSS} \\ % C_{Strom} + C_{W,Bat}
     \label{BA_Infrastruktur}
  \end{equation}

\begin{equation}
   \begin{split}
  {C_{Bat}} = P_{Bat} \cdot N_{Bat}  \\
 % {C_{Strom}} = n_{BSS} \cdot P_{Strom} \cdot N_{menge}\\ %brauche ich das?
  {C_{BSS}} = P_{BSS} \cdot N_{BSS} \\
  %(\frac{30}{Lebensdauer})\\
  %{C_{Hubwagen}} = P_{Hubwagen} \cdot n_{Hubwagen}
  \label{BA_zusatz}
   \end{split}
  \end{equation}
%
$P_{Bat}$ stellt den Preis einer Batterie und $N_{Bat}$ die Anzahl der Batterien dar, die benötigt werden. 
Dazu werden noch 20 \% Reserve-Batterien einberechnet, um Einschränkungen im Betrieb zu umgehen \eqref{BatInfAnzahl}. 
%$N_{menge}$ ist die Menge an Strom, der für das Laden einer Batterie nötig ist.
$P_{BSS}$ ist der Preis des Ladegeräts der Batteriewechselstation und $N_{BSS}$ ist der Anzahl an Ladegeräten. 
Stromkosten werden hier nicht betrachtet, da diese bereits in Betriebskosten 
pro Flug einberechnet werden und somit durch die Fluggesellschaft gedeckt werden. 
%Zusätzlich werden die Kosten für Batteriewechsel $P_{W,Bat}$ und Preis pro Hubwagen $P_{Hubwagen}$ mitbetrachtet.
Für die Berechnung wird eine lineare Abschreibung genutzt. 
Es ist nicht auszuschließen, dass für die ordnungsgemäße Lagerung 
von Batterien zusätzliche Anlagen wie bspw. Lagergebäude nötig sind. %Geli: sein werden?
In der Tabelle \ref{BA_Infrastrukturtab} sind die genutzten Werte für die batteriegebundene Infrastruktur aufgeführt.
%
%Energiekosten = Menge mal Preis
%
% Stromkosten = Anzahl der BSS mal die Leistung mal Kosten pro Einheit Spitzleistung mal Anzahl der Tage/30
%
% Anschaffungskosten BSS = Anzahl BSS mal Kosten mal Tage/Lebensdauer
 % An Optimization Model for Airport Infrastructures in Support to Electric Aicraft:
 % Um Batterienachfrage zu decken, wäre es möglich der Anzahl der Ersatz-Batterien zu erhöhen, die außerhalb der Spitzzeiten geladen werden 
 % oder der Anzahl der Batteriewechselstationen zu erhöhen. Es wird angenommen, dass für Flughafen 800 kW - Ladestationen zur Verfügung haben, 
 % (Studie, wo die Leistungen für
 % 3 hybride-Flugzeuge ausgerechnet wurden - insgesamt für 12 Flugzeuge sind 21 Batterien und 4 Ladestationen benötigt, 77,5 € pro Station, Peak power 3,2 MW, Batteriewechsel jede 1.51 Jahre
 % also eine Station für 3 Flugzeuge?). Natürlich in der Zukunft werden die Flugzeuge unterschiedliche Leistungen
 % und auch Batterien brauchen, was am Ende die Ladedauer und Anzahl der Ladestationen ändern kann
%
%10 \% für die Wartung der Systeme (wie bei OPTIMAL RECHARGING INFRASTRUCTURE SIZING AND OPERATIONS 
%FOR A REGIONAL AIRPORT)
%
%
\begin{table}[h]
	\begin{center}
    \caption{Werte und Annahmen der BA-Infrastruktur}
	\label{BA_Infrastrukturtab}
	\begin{tabular}{|l|c|c|}
		\hline
		 & \textbf{Werte} & \textbf{Quelle} \\ \hline
		$P_{BSS} ~[\text{EUR}]$ &  11 974  & \cite{guo2020aviation} \\ \hline
      Abschreibung $BSS ~[\text{Jahre}]$&  10  & \cite{salucci2020optimal} \\ \hline
		$P_{Bat} ~[\text{EUR/kWh}]$ & 125 000 & \cite{guo2020aviation} \\ \hline
      Lebenszyklen $c_{Bat}$ & 5 000 & \cite{reimers2018introduction} \\ \hline
      Abschreibung $Bat ~[\text{Jahre}]$& 2,7 & -\\ \hline
		%$P_{W,Bat} ~[\text{EUR}] $ & 285  & \cite{guo2023infrastructure}\\ \hline
    %  $P_{HW} ~[\text{EUR}] $ &  &\\ \hline
	\end{tabular}
    \end{center}
\end{table}
Die Anzahl der Batterien ist von der Gesamtanzahl der Abfertigungen $N_{Abfertigung}$ 
und Ladezyklen $c_{Bat}$ einer Batterie abhängig. 
Zeitgleich ist die Anzahl der Zyklen, die eine Batterie an einem Tag geladen werden kann, 
von den Betriebsstunden des Flughafens und der Ladedauer der Batterie abhängig. 
In Zukunft ist eine Lebensdauer einer Batterie von bis zu 5000 und mehr Zyklen zu erwarten \cite{reimers2018introduction}, 
weshalb ein Wert von 5000 Zyklen in der Berechnung angenommen wird.
Die Ladedauer ist von der Leistung der Ladestationen und Kapazität einer Batterie abhängig.
Für die Ladegeräte wurde eine konstante Spitzenleistung von 250 kW angenommen \cite{salucci2020optimal}. 
%Der Strom ist normalerweise von Tag und Nacht abhängig und wie viel parallel genutzt wird.  Die Verluste beim Laden werden ignoriert. 
Zusätzlich wird der Puffer von 20 \% für die Batterieanzahl implementiert, um mögliche Engpässe zu vermeiden. 
Die Anzahl der Ladegeräte $N_{BSS}$ ist von der Gesamtzahl der Batterien und Ladezyklen abhängig.

Mit einer angenommenen Batteriekapazität von 900 kWh und Ladegeräten 
mit der Leistung von 250 kW, entspricht das 3,6 Stunden Ladedauer.
Bei einem 18 Stunden-Betrieb entspricht das insgesamt 5 Ladezyklen pro Tag für jede Batterie. %Geli: resultiert insgesamt in 5 ..? was denkst du
Daraus kann abgeleitet werden, dass jede Batterie 1000 Tage genutzt werden kann,
bis ihre Leistung auf 80 \% sinkt.
Die Einbeziehung der Abschreibungswerte unterstützt die dritte Hypothese.
%
\begin{equation}
   \begin{split}
  {N_{Bat}} = {1,2} \cdot \frac{N_{Abfertigung}}{c_{Bat}}\\
   {N_{BSS}} = \frac{N_{Bat}}{c_{Bat}}\\
   \label{BatInfrAnzahl}
    \end{split}
   \end{equation}
%
\subsubsection{Wasserstoffantrieb}
%
%Lieferung
%falls in flüssigform: LKW/Schiff/Bahn - Distanz und von der Menge abhängig.

Für Wasserstoffantriebe wird ein oberirdischer Tank für kryogenen flüssigen Wasserstoff \ce{LH2}, 
eine kryogene Pumpe zum Befüllen und Entleeren des Lagers ${kP}$ 
und ein Betankungsfahrzeug ${BF}$ gebraucht (vgl. \eqref{WA_Infrastruktur}). 
Die Preise und Abschreibungswerte für die Infrastruktur des Wasserstoffantriebs 
sind in der Tabelle \ref{WA_Infrastrukturtab} zusammengefasst. 
%
Der benötigte spezifische Energiebedarf der Infrastrukturelemente wie sowohl Betankungsfahrzeuge 
oder kryogener Pumpen, als auch Wirkungsgrad der Technologien, wird dabei nicht betrachtet.

\begin{equation}
   {CAPEX ~[\text{EUR}]} = {P_{Lagertank}} + P_{kP} \cdot N_{kP} + P_{BF} \cdot N_{BF}  \\
   \label{WA_Infrastruktur}
\end{equation}
%
Die Anzahl der Betankungsfahrzeugen $N_{BF}$ hängt von der Anzahl der stündlichen Abfertigungen ab. 
Zudem wird ein Sicherheitsfaktor
von 1,2 dazugerechnet, um reibungslose Prozesse sicherzustellen.
Die Gesamtzahl den Betankungsfahrzeugen bestimmt die benötigte Anzahl an Kryopumpen $N_{kP}$. 
Es wird angenommen, dass pro Fahrzeug eine Kryopumpe benötigt \cite{}.
Der Konstante 1 beschreibt eine Kryopumpe,
die permanent mit dem Wasserstoffspeicher verbunden ist.
Zudem wird zur Vereinfachung angenommen, dass pro Flugzeug ein Tankwagen benötigt wird.
\begin{equation}
   \begin{split}
   {N_{BF}} = {1,2} \cdot {N_{Abfertigung,h}}\\
   {N_{kP}} = N_{BF} + 1 \\
   \label{WAInfrAnzahl}
   \end{split}
   \end{equation}

\begin{table}[h]
	\begin{center}
    \caption{Werte und Annahmen der Wasserstoffinfrastruktur}
	\label{WA_Infrastrukturtab}
	\begin{tabular}{|l|c|c|c|}
		\hline
		 & \textbf{Werte}& \textbf{Einheit}& \textbf{Quellen} \\ \hline
		Preis Lagertank $P_{Lagertank}$ & 41,9 & EUR/kg \ce{LH2}  & \cite{schenke2024lh2}\\ \hline
      Abschreibung Lagertank & 20  & Jahre  & \cite{hoelzen2023h2}\\ \hline
      Volumen Lagertank ~[\text{$m^3$}] & 4 732 & $m^3$ & \cite{fesmire2021lh2}\\ \hline
		Preis kryogene Pumpe $P_{kP}$ & 250 171 & EUR/kg/h & \cite{hoelzen2022h2} \\ \hline
      Abschreibung ${kP}$ & 10 & Jahre & \cite{hoelzen2023h2} \\ \hline
		Preis Betankungswagen $P_{BW}$ & 87 848 & EUR & \cite{hoelzen2022h2} \\ \hline
      Abschreibung ${BW}$ & 12  & Jahre  & \cite{hoelzen2022h2} \\ \hline
	\end{tabular}
    \end{center}
\end{table}

Wie bereits im Grundlagen-Kapitel erläutert wurde, 
hat der kryogene Wasserstoff eine Dichte von ca. 65 $kg/m^3$ \cite{colpan2022fuel},
also können 307,58 t in einem kugelförmigen Lagertank gespeichert werden.
Es wird davon ausgegangen, dass die Vorgänge der Betankung des Wagens
und die folgende Betankung des Flugzeugs jeweils 30 Minuten dauern \cite{hoelzen2022h2}. 
%
%Außerdem müssen diese Teile mit Spannung versorgt werden: Kontrollraum, Wartungseinrichtugen % MAX: ist das ein Satz?
