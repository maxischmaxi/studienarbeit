\section{Aufstellung der Formeln für Kosten}
\label{s:Aufstellung der Formeln für Kosten}

Wie bereits im Kapitel 2 beschrieben wurde die Kosten einer Fluggesellschaft werden auf direkte und indirekte Betriebskosten verteilt.
In diesem Kapitel werden die dazugehörigen Formeln vorgestellt und teilweise angepasst. 

Es gibt eine Reihe von Methoden, um DOC zu berechnen, sie sind ähnlich.
Als Grundlage für ein Modell wurde von Association of European Airlines (AEA) 1989 gewählt, da sie hohen Anwendung in der akademischen Welt hat,
umfassend ist und Berechnungswerte für Kurz-, als auch Langstecken hat.

Reservebedarf beträgt 30 \% von Gesamtkapazität.



\subsection{Betriebskosten einer Fluggesellschaft}

Die Gleichung \eqref{DOC} stellt die Betriebskosten $DOC$ für Flugzeugbenutzung dar. Die bestehen aus Treibstoff-/Energiekosten $C_T$, 
Wartungskosten $C_W$, Entgelte und Gebühren $C_{EG}$, Kosten für Personal $C_{Crew}$, kapitalverbundene Kosten $C_{KK}$.


\begin{equation}
     {DOC ~[\text{EUR}]} = C_T + C_W + C_{Crew} + C_{KK} + C_{EG} \\
     \label{DOC}
  \end{equation}

\textbf{Treibstoff-/ Energiepreise} hängt von der Treibstoff- bzw. Energiepreis selbst \\ $P_{T/E}$ und vom Verbrauch 
von einem Flugzeug pro Blockstunde $m_{Verbrauch_pro_Flug}$ ab (vgl. \eqref{fuel}).



\begin{equation}
   {C_T ~[\text{EUR}]} = (P_{T/E} \cdot n_{F,Jahr} \cdot m_{Verbrauch,B}) \\
   \label{fuel}
\end{equation}

Die \textbf{Wartungskosten} werden nach Jenkinson 1999 Modell berechnet. Das Modell ermöglicht schnell und grob die Wartungskosten 
für ein Flugzeug abzuschätzen \cite{bruge2018wartungskosten}.
Da dieses Modell für konventionelle Flugzeuge ausgearbeitet wurde, werden die Wartungskosten für alternative Antriebe
als Prozentanteil von Referenz-Flugzeug berechnet. 
Die Formeln stammen aus \cite{bruge2018wartungskosten} und beziehen sich auf das Jahr 1994, somit muss den Inflationsfaktor $k_{Infl}$
einkalkuliert werden. Die Berechnung liefert die Ergebnisse in USD, deswegen ein Wechselkursfaktor $k_{WK}$ dazugerechnet wird.
Die Wartungskosten werden normalerweise auf die Wartung der Flugzeugzelle $C_{W,Zelle,B}$ und Triebwerken 
$C_{W,Triebwerk,B}$ aufgeteilt, wie in der \eqref{maintenance} dargestellt. Die Wartungskosten der Zelle sind von dem leeren Betriebsgewicht 
(Operating Empty Mass $m_{OE}$) abhängig. Kosten von Triebwerk ist von erzeugten Schub beim Start (Take-Off Thrust $T_{T/O}$) abhängig.

\begin{equation}
   {C_{W,B} ~[\text{EUR}]} = (C_{W,Zelle,B} + n_{T} \cdot C_{W,Triebwerk,B} ) t_{B} \cdot n_{F, Jahr} \cdot k_{WK} \cdot k_{Infl}\\
   \label{maintenance}
\end{equation}

\begin{equation}
   {C_{M,AF,b} ~[\text{EUR}]} = (175 \frac{USD}{h} + 0,0041 \frac{USD}{h} \cdot m_{OE} )\cdot k_{Infl}\\
   \label{maintenance Zelle}
\end{equation}

\begin{equation}
   {C_{M,E,L,b} ~[\text{EUR}]} = (0,00029 \frac{USD}{h} \cdot T_{T/O} )\cdot k_{Infl}\\
   \label{maintenance engine}
\end{equation}

Diese Formeln sind für Triebwerke ausgelegt mit Nebenstromverhältnis 5. Da die neuen größeren Flugzeuge höhere Nebenstromverhältnis hat,
für die Vereinfachung wird trotzdem diese Formel benutzt.

%Wartungskosten wachsen mit der Größe des Flugzeugs, da mehr Personal oder mehr Zeit für die Wartung benötigt ist.

Unter \textbf{Entgelten und Gebühren} sind Flughafenentgelte (Passagier-, Lande- und Startentgelte, Sicherheitsentgelt, Abfertigung am 
Vorfeld) und Flugsicherungsgebühr gemeint.//
Formel für Flugsicherungsgebühr $(C_{FS})$ jeweils An- und Abfluggebühr stammt aus \cite{dfs_flugsicherungsgebuehren} und in \eqref{Flugsicherung}
dargestellt. Im Jahr 2025 liegt den Wert $P_{FS}$ bei 380,71 Euro. $MTOW$ bezeichnet Höchstabfluggewicht (Maximum Take-Off Weight) 
von einem Flugzeug. Die Werte für Flughafenentgelte wurden aus den \cite{fraport2025entgelte} genommen.
Außerdem müssen bei BA-Flugzeugen die Kosten für
Batteriewechsel $P_{W,Bat}$ einkalkuliert werden, pro Flugzeug wird ein Wert von 285 EUR genommen \cite{guo2023infrastructure}.

\begin{equation}
   {C_{FS} ~[\text{EUR}]} = (\frac{MTOW}{50})^{0,7} \cdot P_{FS} \\
   \label{Flugsicherung}
\end{equation}

Die Betriebskosten werden auf Blockstunden bezogen. Blockstunden bestehen aus Flug- $t_{F,h}$ und Rollzeit $t_{R}$ 
zum und von Parkposition $t_{R}$. 
Für Kurs- und Mittelstrecken beträgt $t_{R}$ {0,25 h} und für Langstrecken 0,42 h \cite{scholz_design_evaluation_doc}.

\begin{equation}
   t_{B} = t_{f,h,j} + t_{R} \\
   \label{blockzeit}
\end{equation}

\textbf{Crewkost} definieren sich aus der Lohn für Piloten $L_{Pilot}$ und Besatzung $L_{crew}$. Anzahl der Besatzung $n_{crew}$ sind 
von der Anzahl der Passagiere abhängig. Jeweils je 50 Passagier ist einen Flugbegleiter notwendig \cite{conrady2019luftverkehr}.

\begin{equation}
   {C_{crew} ~[\text{EUR}]} = ({L}_{Pilot} \cdot 2 + {L}_{crew} \cdot n_{crew} ) \cdot t_{B} \\
   \label{crew cost}
\end{equation}

%Der Preis für ganzes Flugzeug außer Antrieb-System wird nach folgend berechnet

%\begin{equation}
%   {P_{i,f} ~[\text{€}]} = \frac{m_{i,f}}{m_{i,c}} \cdot p_{i,c} \cdot P_{A,C} \\
%   \label{aicraft price hydrogen}
%\end{equation}



 
%Analog zum \cite{marksel2023comparative} werden die Kosten für den Wasserstoff und Batterie Antrieb getrennt ausgerechnet, 
%wobei für Batterie-Antrieb anstatt eines Wasserstofftanks und eine Brennstoffzelle, basiert sich auf einer Batterie, Motor 
%und Leistungselektronik

%\begin{equation}
%   {P_{E,f} ~[\text{€}]} = P_{FC} \cdot + P_{EM} \cdot + m_F \\
%   \label{engine group price hydrogen}
%\end{equation}
Zu \textbf{kapitalbezogenen Kosten} gehören Abschreibungs-, Versicherungs- und Verzinsungskosten. 
Abschreibungskosten sind von Anschaffungskosten (Kaufpreis), Abschreibungsdauer und Blockstunden pro Jahr abhängig \cite{conrady2019luftverkehr}.
Abschreibungsdauer nach AEA ist jeweils 14 Jahre für Kurs- und Mittelstecken und 16 Jahre Langstrecken.

Aufgrund nicht Zugänglichkeit zu den Preisen für alternativen Antrieben wird der Verkaufspreis von elektrischem Flugzeug nach 
\cite{monjon2020conceptual} berechnet, 
wo das Regressionsmodell anhand Marktanalyse erstellt wurde. In der Formel sind 10 \% höhere Anschaffungskosten mitbetrachtet, 
aufgrund der Einführung der neuen Technologie auf den Markt.
Die Studie ist aus dem Jahr 2020 und die Preise wurden im USD berechnet, deswegen Inflationsrate $k_{Infl}$ von 1,193 und
der Wechselkurs $k_{WK}$ mitbetrachtet werden sollen.

\begin{equation}
   {C_{BA,ac} ~[\text{EUR}]} = (407408 \cdot n_{PAX} - 2967.4) \cdot k_{WK} \cdot {k_{Infl}}\\
   \label{price BA}
\end{equation}

Noch ein wichtiger Faktor bei der kapitalverbundenen Kosten ist die Auslastung $U$ eines Flugzeuges. Es wird durch Anzahl 
verfügbaren Stunden pro Jahr $t_{verf}$, Blockzeit $t_B$ und Turnaround Zeit $t_{TA}$. $t_{verf}$ für Kurzstrecken wurde 
in der Höhe von 3750 h und für Mittel- und Langstrecken 4800 h \cite{scholz_design_evaluation_doc}. $t_{TA}$ wurde in Höhe von 1,5
Stunden gewählt (Quelle).

\begin{equation}
   U = \frac{t_{verf}}{t_B + t_{TA}} \\
   \label{utilisation}
\end{equation}

Verzinsungskosten sind durch den Prozentanteil von Anschaffungskosten bedingt, für dieser gilt ca. 5 \%. Versicherungskosten sind 
hingegen von dem Kaufpreis eines Flugzeugs abhängig (inklusive Rabatte beim Kauf) und für diese Arbeit werden 5 \% angenommen 
\cite{scholz_design_evaluation_doc}.

%Für wasserstoffbetriebene Flugzeuge werden in der Quelle vorgestellte Formeln benutzt, um Kaufpreis einschätzen zu können.
Außerdem bestehen bei alternativen Antrieben betriebliche Kosten, die durch Infrastrukturbedienung bedingt sind.
\subsection{Infrastrukturkosten}

\textbf{Batterie-Antrieb}\\
Kapitalkosten für BA-Infrastruktur sind in folgender Formel zusammengestellt:

\begin{equation}
     {CAPEX ~[\text{EUR}]} = C_{Bat} + C_{Strom} + C_{BSS} + C_{W,Bat} + C_{Hubwagen}\\
     \label{BA_Infrastruktur}
  \end{equation}

\begin{equation}
   \begin{split}
  {C_{Bat}} = P_{Bat} \cdot n_{Bat}\\
  {C_{Strom}} = n_{BSS} \cdot P_{Strom} \cdot N_{menge}\\
  {C_{BSS}} = P_{BSS} \cdot n_{BSS} \\
  %(\frac{30}{Lebensdauer})\\
  {C_{Hubwagen}} = P_{Hubwagen} \cdot n_{Hubwagen}
  \label{BA}
   \end{split}
  \end{equation}

, wobei $P_{Bat}$ ist der Preis für eine Batterie und $n_{Bat}$ ist der Anzahl Batterien, die benötigt sind.
$N_{menge}$ ist die Menge an Strom, der für das Laden einer Batterie benötigt ist.
$P_{BSS}$ ist der Preis für Ladegerät der Batteriewechselstation und $n_{BSS}$ ist wie immer der Anzahl an Ladegeräten. 
Zusätzlich werden die Kosten für
Batteriewechsel $P_{W,Bat}$ und Preis für einen Hubwagen $P_{Hubwagen}$ mitbetrachtet.
  
%Energiekosten = Menge mal Preis

 % Stromkosten = Anzahl der BSS mal die Leistung mal Kosten pro Einheit Spitzleistung mal Anzahl der Tage/30

  %Anschaffungskosten BSS = Anzahl BSS mal Kosten mal Tage/Lebensdauer

 % An Optimization Model for Airport Infrastructures in Support to Electric Aicraft:
 % Um Batterienachfrage zu decken, wäre es möglich der Anzahl der Ersatz-Batterien zu erhöhen, die außerhalb der Spitzzeiten geladen werden 
 % oder der Anzahl der Batteriewechselstationen zu erhöhen. Es wird angenommen, dass für Flughafen 800 kW - Ladestationen zur Verfügung haben, 
 % (Studie, wo die Leistungen für
 % 3 hybride-Flugzeuge ausgerechnet wurden - insgesamt für 12 Flugzeuge sind 21 Batterien und 4 Ladestationen benötigt, 77,5 € pro Station, Peak power 3,2 MW, Batteriewechsel jede 1.51 Jahre
 % also eine Station für 3 Flugzeuge?). Natürlich in der Zukunft werden die Flugzeuge unterschiedliche Leistungen
 % und auch Batterien brauchen, was am Ende die Ladedauer und Anzahl der Ladestationen ändern kann

%10 \% für die Wartung der Systeme (wie bei OPTIMAL RECHARGING INFRASTRUCTURE SIZING AND OPERATIONS 
%FOR A REGIONAL AIRPORT)


\begin{table}[h]
	\begin{center}
    \caption{Werte und Annahmen für BA-Infrastruktur}
	\label{BA_Infrastrukturtab}
	\begin{tabular}{|l|c|c|}
		\hline
		 & \textbf{Werte} & \textbf{Quelle} \\ \hline
		$P_{BSS} ~[\text{EUR}]$ &  11 974  & \cite{guo2020aviation} \\ \hline
		$P_{Bat} ~[\text{EUR/kWh}]$ & 125000 & \cite{guo2020aviation} \\ \hline
		$P_{W,Bat} ~[\text{EUR}] $ & 285  & \cite{guo2023infrastructure}\\ \hline
      $P_{HW} ~[\text{EUR}] $ &  &\\ \hline
	\end{tabular}
    \end{center}
\end{table}

Anzahl Batterien ist von der Gesamtanzahl der Abfertigungen $N_{Abfertigung}$ und Zyklen $c_{Batterie}$ einer Batterie abhängig. 
Für die Ladegeräte wurde die Leistung
von 250 kW genommen. Die Verluste beim Laden werden ignoriert. Zusätzlich wird der Puffer von 20 \% dazu berechnet, um 
Engpässe zu vermeiden. In dieser Arbeit wird eine Batterie Kapazität von 900 kWh genommen, bei einem 18 Stunden-Betrieb 
das entspricht 3,6 Stunden Ladedauer für eine Batterie mit insgesamt 5 Ladezyklen pro Tag. 
Anzahl der Ladegeräten $n_{BSS}$ ist von der Gesamtzahl der Batterien und Ladezyklen abhängig.

\begin{equation}
  {N_{Bat}} = {1,2} \cdot \frac{N_{Abfertigung}}{c_{Batterie}}\\
  \label{BatAnzahl}
  \end{equation}

\textbf{Wasserstoff-Antrieb}

%Lieferung
%falls in flüssigform: LKW/Schiff/Bahn - Distanz und von der Menge abhängig.

Für die Wasserstoff-Antriebe wird einen oberirdischen Tank für kryogene flüssigen Wasserstoff \ce{LH2}, eine kryogene Pumpe 
zum Befüllen und Entleeren des Lagers ${kP}$ und ein Betankungswagen ${BW}$ gebraucht (vgl. \eqref{WA_Infrastruktur}). 
Die Preise für die Anlage ist in der Tabelle \ref{BA_Infrastrukturtab} aufgeführt. Es werden zwei Pumpensysteme benötigt.
Eine davon ist zum Befüllen des Betankungswagens an den \ce{LH2}-Speicher und andere beim Betanken des Flugzeuges \cite{hoelzen2022h2}.

\begin{equation}
   {CAPEX ~[\text{EUR}]} = {P_{Tank}} + P_{kP} + P_{BW} \cdot N_{BW}  \\
   \label{WA_Infrastruktur}
\end{equation}

\begin{table}[h]
	\begin{center}
    \caption{Werte und Annahmen}
	\label{BA_Infrastrukturtab}
	\begin{tabular}{|l|c|c|c|}
		\hline
		 & \textbf{Werte}& \textbf{Einheit}& \textbf{Quellen} \\ \hline
		$P_{Lagertank}$ & 34 & EUR/kgGH2 stored  & sek \cite{hoelzen2023h2}\\ \hline
      Abschreibung Lagertank & 20  & Jahre  & sek \cite{hoelzen2023h2}\\ \hline
      Größe Lagertank ~[\text{$m^3$}] & 4 732 &  & \cite{fesmire2021lh2}\\ \hline
		$P_{kP}$ & 346 & EUR/kg/h & sek \cite{hoelzen2023h2} \\ \hline
      Abschreibung ${kP}$ & 10 & Jahre & sek \cite{hoelzen2023h2} \\ \hline
		$P_{BW}$ & 87848 & EUR & \cite{hoelzen2022h2} \\ \hline
	\end{tabular}
    \end{center}
\end{table}

Wie bereits in Grundlagen-Kapitel erläutert wurde, hat den kryogenen Wasserstoff die Dichte von 70 $kg/m^3$ (Quelle), 
also können 331,24 t davon in einem kugelförmigen Lagertank gespeichert werden.
Es wird davon ausgegangen, dass die Vorgänge der Betankung des Wagens und folgende Betankung des Flugzeugs jeweils 
30 Minuten dauern \cite{hoelzen2022h2}. (1 Flugzeug pro Tankwagen, also alles von - Ankunftsrate abhängig)



