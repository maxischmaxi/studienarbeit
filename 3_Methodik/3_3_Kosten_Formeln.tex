\section{Aufstellung der Formeln für Kosten}
\label{s:Aufstellung der Formeln für Kosten}

Wie bereits im Kapitel 2 beschrieben wurde, werden die Kosten einer Fluggesellschaft auf direkte und indirekte Betriebskosten verteilt.
In diesem Kapitel werden die dazugehörigen Formeln vorgestellt und teilweise angepasst. 

Es gibt eine Reihe von Methoden, um DOC zu berechnen, wobei diese oft ähnlich sind.
Als Grundlage für ein Modell wurde %hier fehlt das modell oder?
der Association of European Airlines (AEA) 1989 gewählt, da sie häufige Anwendung in der akademischen Welt hat,
sehr umfassend ist und Berechnungswerte für sowohl Kurz-, als auch Langstecken hat.

Der Reservebedarf beträgt 30 \% der Gesamtkapazität.



\subsection{Betriebskosten einer Fluggesellschaft}

Die Gleichung \eqref{DOC} stellt die Betriebskosten $DOC$ der Flugzeugnutzung dar. Diese bestehen aus Treibstoff-/Energiekosten $C_T$, 
Wartungskosten $C_W$, Entgelten und Gebühren $C_{EG}$, Kosten für Personal $C_{Crew}$ und kapitalgebundenen Kosten $C_{KK}$.


\begin{equation}
     {DOC ~[\text{EUR}]} = C_T + C_W + C_{Crew} + C_{KK} + C_{EG} \\
     \label{DOC}
  \end{equation}

\textbf{Treibstoff-/ Energiepreise} hängen vom Treibstoff- bzw. Energiepreis selbst \\ $P_{T/E}$ und vom Verbrauch 
eines Flugzeugs pro Blockstunde $m_{Verbrauch_pro_Flug}$ ab (vgl. \eqref{fuel}).



\begin{equation}
   {C_T ~[\text{EUR}]} = (P_{T/E} \cdot n_{F,Jahr} \cdot m_{Verbrauch,B}) \\
   \label{fuel}
\end{equation}

Die \textbf{Wartungskosten} werden nach dem Jenkinson 1999 Modell berechnet. Das Modell ermöglicht es, grob aber schnell die Wartungskosten 
für ein Flugzeug abzuschätzen \cite{bruge2018wartungskosten}.
Da dieses Modell auf konventionelle Flugzeuge ausgearbeitet wurde, berechnet man die Wartungskosten für alternative Antriebe
als Prozentanteil des Referenz-Flugzeugs. 
Die Formeln stammen aus \cite{bruge2018wartungskosten} und beziehen sich auf das Jahr 1994, somit muss der Inflationsfaktor $k_{Infl}$
einkalkuliert werden. Die Berechnung liefert die Ergebnisse in USD, aus diesem Grund wird ein Wechselkursfaktor $k_{WK}$ in die Berechnung implementiert.
Wartungskosten werden normalerweise auf die Wartung der Flugzeugzelle $C_{W,Zelle,B}$ und der Triebwerke 
$C_{W,Triebwerk,B}$ aufgeteilt, wie in Gleichung \eqref{maintenance} dargestellt. Die Wartungskosten der Zelle sind von dem leeren Betriebsgewicht 
(Operating Empty Mass $m_{OE}$) abhängig. Kosten des Triebwerks sind vom erzeugten Schub beim Start (Take-Off Thrust $T_{T/O}$) abhängig.

\begin{equation}
   {C_{W,B} ~[\text{EUR}]} = (C_{W,Zelle,B} + n_{T} \cdot C_{W,Triebwerk,B} ) t_{B} \cdot n_{F, Jahr} \cdot k_{WK} \cdot k_{Infl}\\
   \label{maintenance}
\end{equation}

\begin{equation}
   {C_{M,AF,b} ~[\text{EUR}]} = (175 \frac{USD}{h} + 0,0041 \frac{USD}{h} \cdot m_{OE} )\cdot k_{Infl}\\
   \label{maintenance Zelle}
\end{equation}

\begin{equation}
   {C_{M,E,L,b} ~[\text{EUR}]} = (0,00029 \frac{USD}{h} \cdot T_{T/O} )\cdot k_{Infl}\\
   \label{maintenance engine}
\end{equation}

Diese Formeln sind für Triebwerke mit Nebenstromverhältnis ausgelegt 5. %warum hier die 5?
Trotz der höheren Nebenstromverhältnisse die bei moderneren und größeren Flugzeugen gegeben sind,
wird aus Gründen der Vereinfachung trotzdem diese Formel genutzt.

%Wartungskosten wachsen mit der Größe des Flugzeugs, da mehr Personal oder mehr Zeit für die Wartung benötigt ist.

Unter \textbf{Entgelte und Gebühren} sind Flughafenentgelte (Passagier-, Lande- und Startentgelte, Sicherheitsentgelte sowie Abfertigung am 
Vorfeld) und die Flugsicherungsgebühr gemeint. 
Die Formel der Flugsicherungsgebühr $(C_{FS})$ für jeweils An- und Abfluggebühr stammt aus \cite{dfs_flugsicherungsgebuehren} und ist in der Gleichung \eqref{Flugsicherung}
dargestellt. Im Jahr 2025 liegt der Wert $P_{FS}$ bei 380,71 Euro. 
Der $MTOW$ bezeichnet das Höchstabfluggewicht (Maximum Take-Off Weight) eines Flugzeugs. 
Die zugrunde liegenden Werte der Flughafenentgelte basieren auf den Daten von \cite{fraport2025entgelte}.
Außerdem müssen bei BA-Flugzeugen die Kosten für
Batteriewechsel $P_{W,Bat}$ einkalkuliert werden, pro Flugzeug wird ein Wert von einem Wert in Höhe von 285 EUR ausgegangen \cite{guo2023infrastructure}.

\begin{equation}
   {C_{FS} ~[\text{EUR}]} = (\frac{MTOW}{50})^{0,7} \cdot P_{FS} \\
   \label{Flugsicherung}
\end{equation}

Die Betriebskosten werden auf Basis von Blockstunden kalkuliert. 
Blockstunden setzen sich aus Flugzeit $t_{F,h}$ sowie der kumulierten Rollzeit $t_{R}$ von und zur Parkposition $t_{R}$ zusammen. 
Für Kurz- und Mittelstrecken beträgt $t_{R}$ {0,25 h} und für Langstrecken 0,42 h \cite{scholz_design_evaluation_doc}.

\begin{equation}
   t_{B} = t_{f,h,j} + t_{R} \\
   \label{blockzeit}
\end{equation}

\textbf{Crewkosten} setzen sich aus Lohnkosten für Piloten $L_{Pilot}$ und Besatzung $L_{crew}$ zusammen. Die Anzahl der Besatzungmitglieder $n_{crew}$ ist 
von der Anzahl der Passagiere abhängig. Gemäß den luftfahrtrechtlichen Bestimmungen ist pro 50 Passagiere ein Flugbegleiter notwendig \cite{conrady2019luftverkehr}.

\begin{equation}
   {C_{crew} ~[\text{EUR}]} = ({L}_{Pilot} \cdot 2 + {L}_{crew} \cdot n_{crew} ) \cdot t_{B} \\
   \label{crew cost}
\end{equation}

%Der Preis für ganzes Flugzeug außer Antrieb-System wird nach folgend berechnet

%\begin{equation}
%   {P_{i,f} ~[\text{€}]} = \frac{m_{i,f}}{m_{i,c}} \cdot p_{i,c} \cdot P_{A,C} \\
%   \label{aicraft price hydrogen}
%\end{equation}



 
%Analog zum \cite{marksel2023comparative} werden die Kosten für den Wasserstoff und Batterie Antrieb getrennt ausgerechnet, 
%wobei für Batterie-Antrieb anstatt eines Wasserstofftanks und eine Brennstoffzelle, basiert sich auf einer Batterie, Motor 
%und Leistungselektronik

%\begin{equation}
%   {P_{E,f} ~[\text{€}]} = P_{FC} \cdot + P_{EM} \cdot + m_F \\
%   \label{engine group price hydrogen}
%\end{equation}
Zu \textbf{kapitalbezogenen Kosten} gehören Abschreibungs-, Versicherungs- und Verzinsungskosten. 
Abschreibungskosten sind von Anschaffungskosten (Kaufpreis), Abschreibungsdauer und Blockstunden pro Jahr abhängig \cite{conrady2019luftverkehr}.
Die Abschreibungsdauer nach AEA beträgt jeweils 14 Jahre für Kurz- und Mittelstecken und 16 Jahre für Langstrecken.

Da keine öffentlich zugänglichen Marktpreise für Luftfahrzeuge mit alternativen Antriebssystemen vorliegen, 
wird die Kalkulation der Anschaffungskosten für elektrisch betriebene Flugzeuge nach der Methodik von \cite{monjon2020conceptual} durchgeführt, 
in welcher das Regressionsmodell anhand einer Marktanalyse erstellt wurde. 
In der Formel sind aufgrund der Einführung neuer Technologien 10 \% höhere Anschaffungskosten mitbetrachtet. 
Die Studie ist aus dem Jahr 2020, Preise wurden in USD berechnet, weshalb die Inflationsrate $k_{Infl}$ 1,193 und
der Wechselkurs $k_{WK}$ ebenfalls betrachtet werden sollen.

\begin{equation}
   {C_{BA,ac} ~[\text{EUR}]} = (407408 \cdot n_{PAX} - 2967.4) \cdot k_{WK} \cdot {k_{Infl}}\\
   \label{price BA}
\end{equation}


Ein weiterer wichtiger Faktor der kapitalgebundenen Kosten ist die Auslastung $U$ eines Flugzeuges. Es wird durch die Anzahl 
der verfügbaren Stunden p.a. $t_{verf}$, Blockzeit $t_B$ und Turnaround Zeit $t_{TA}$ berechnet. 
Die jährliche Verfügbarkeitszeit $t_{verf}$ beträgt für Kurzstreckenflugzeuge 3750 h, 
während für Mittel- und Langstreckenflugzeuge ein Wert von 4800 h angesetzt wird \cite{scholz_design_evaluation_doc}.

\begin{equation}
   U = \frac{t_{verf}}{t_B + t_{TA}} \\
   \label{utilisation}
\end{equation}

Verzinsungskosten $C_{Zins}$ sind durch den Prozentanteil von Anschaffungskosten bedingt, für diese gelten ca. 5 \%. 
Versicherungskosten sind hingegen von dem Kaufpreis eines Flugzeugs abhängig (inklusive Rabatte beim Kauf), 
in dieser Arbeit werden ebenfalls 5 \% angenommen \cite{scholz_design_evaluation_doc}.

%Für wasserstoffbetriebene Flugzeuge werden in der Quelle vorgestellte Formeln benutzt, um Kaufpreis einschätzen zu können.
Außerdem bestehen bei alternativen Antrieben betriebliche Kosten, die durch Infrastrukturbedienung bedingt sind.
\subsection{Infrastrukturkosten}

\textbf{Batterie-Antrieb}\\
Kapitalkosten für BA-Infrastruktur sind in folgender Formel zusammengestellt:

\begin{equation}
     {CAPEX ~[\text{EUR}]} = C_{Bat} + C_{Strom} + C_{BSS} + C_{W,Bat} + C_{Hubwagen}\\
     \label{BA_Infrastruktur}
  \end{equation}

\begin{equation}
   \begin{split}
  {C_{Bat}} = P_{Bat} \cdot n_{Bat}\\
  {C_{Strom}} = n_{BSS} \cdot P_{Strom} \cdot N_{menge}\\
  {C_{BSS}} = P_{BSS} \cdot n_{BSS} \\
  %(\frac{30}{Lebensdauer})\\
  {C_{Hubwagen}} = P_{Hubwagen} \cdot n_{Hubwagen}
  \label{BA}
   \end{split}
  \end{equation}


wobei $P_{Bat}$ der Preis für eine Batterie darstellt und $n_{Bat}$ die Anzahl der Batterien, die benötigt sind.
$N_{menge}$ ist die Menge an Strom, der für das Laden einer Batterie nötig ist.
$P_{BSS}$ ist der Preis des Ladegeräts der Batteriewechselstation und $n_{BSS}$ ist wie immer der Anzahl an Ladegeräten. %ich weiß nicht was du mit wie immer meinst
Außerdem werden die Kosten für Batteriewechsel $P_{W,Bat}$ und Preis pro Hubwagen $P_{Hubwagen}$ mitbetrachtet.
  
%Energiekosten = Menge mal Preis

% Stromkosten = Anzahl der BSS mal die Leistung mal Kosten pro Einheit Spitzleistung mal Anzahl der Tage/30

% Anschaffungskosten BSS = Anzahl BSS mal Kosten mal Tage/Lebensdauer

 % An Optimization Model for Airport Infrastructures in Support to Electric Aicraft:
 % Um Batterienachfrage zu decken, wäre es möglich der Anzahl der Ersatz-Batterien zu erhöhen, die außerhalb der Spitzzeiten geladen werden 
 % oder der Anzahl der Batteriewechselstationen zu erhöhen. Es wird angenommen, dass für Flughafen 800 kW - Ladestationen zur Verfügung haben, 
 % (Studie, wo die Leistungen für
 % 3 hybride-Flugzeuge ausgerechnet wurden - insgesamt für 12 Flugzeuge sind 21 Batterien und 4 Ladestationen benötigt, 77,5 € pro Station, Peak power 3,2 MW, Batteriewechsel jede 1.51 Jahre
 % also eine Station für 3 Flugzeuge?). Natürlich in der Zukunft werden die Flugzeuge unterschiedliche Leistungen
 % und auch Batterien brauchen, was am Ende die Ladedauer und Anzahl der Ladestationen ändern kann

%10 \% für die Wartung der Systeme (wie bei OPTIMAL RECHARGING INFRASTRUCTURE SIZING AND OPERATIONS 
%FOR A REGIONAL AIRPORT)


\begin{table}[h]
	\begin{center}
    \caption{Werte und Annahmen der BA-Infrastruktur}
	\label{BA_Infrastrukturtab}
	\begin{tabular}{|l|c|c|}
		\hline
		 & \textbf{Werte} & \textbf{Quelle} \\ \hline
		$P_{BSS} ~[\text{EUR}]$ &  11 974  & \cite{guo2020aviation} \\ \hline
		$P_{Bat} ~[\text{EUR/kWh}]$ & 125000 & \cite{guo2020aviation} \\ \hline
		$P_{W,Bat} ~[\text{EUR}] $ & 285  & \cite{guo2023infrastructure}\\ \hline
      $P_{HW} ~[\text{EUR}] $ &  &\\ \hline
	\end{tabular}
    \end{center}
\end{table}

Die Anzahl der Batterien ist von der Gesamtanzahl der Abfertigungen $N_{Abfertigung}$ und Zyklen $c_{Batterie}$ einer Batterie abhängig. 
Für die Ladegeräte wurde die Leistung von 250 kW angenommen. 
Die Verluste beim Laden werden ignoriert. 
Zusätzlich wird der Puffer von 20 \% implementiert, um Engpässe zu vermeiden. 
In dieser Arbeit wird eine Batteriekapazität von 900 kWh angenommen, bei einem 18 Stunden-Betrieb 
entspricht das 3,6 Stunden Ladedauer für eine Batterie mit insgesamt 5 Ladezyklen pro Tag. 
Die Anzahl der Ladegeräte $n_{BSS}$ ist von der Gesamtzahl der Batterien und Ladezyklen abhängig.

\begin{equation}
  {N_{Bat}} = {1,2} \cdot \frac{N_{Abfertigung}}{c_{Batterie}}\\
  \label{BatAnzahl}
  \end{equation}

\textbf{Wasserstoff-Antrieb}

%Lieferung
%falls in flüssigform: LKW/Schiff/Bahn - Distanz und von der Menge abhängig.

Für Wasserstoff-Antriebe wird ein oberirdischer Tank für kryogenen flüssigen Wasserstoff \ce{LH2}, 
eine kryogene Pumpe zum Befüllen und Entleeren des Lagers ${kP}$ und ein Tankwagen ${BW}$ gebraucht (vgl. \eqref{WA_Infrastruktur}). 
Die Preise für die Anlage sind in der Tabelle \ref{BA_Infrastrukturtab} zusammengefasst.

\begin{equation}
   {CAPEX ~[\text{EUR}]} = {P_{Tank}} + P_{kP} + P_{BW} \cdot N_{BW}  \\
   \label{WA_Infrastruktur}
\end{equation}

\begin{table}[h]
	\begin{center}
    \caption{Werte und Annahmen}
	\label{BA_Infrastrukturtab}
	\begin{tabular}{|l|c|c|c|}
		\hline
		 & \textbf{Werte}& \textbf{Einheit}& \textbf{Quellen} \\ \hline
		$P_{Lagertank}$ & 34 & EUR/kgGH2 stored  & sek \cite{hoelzen2023h2}\\ \hline
      Abschreibung Lagertank & 20  & Jahre  & sek \cite{hoelzen2023h2}\\ \hline
      Größe Lagertank ~[\text{$m^3$}] & 4 732 &  & \cite{fesmire2021lh2}\\ \hline
		$P_{kP}$ & 346 & EUR/kg/h & sek \cite{hoelzen2023h2} \\ \hline
      Abschreibung ${kP}$ & 10 & Jahre & sek \cite{hoelzen2023h2} \\ \hline
		$P_{BW}$ & 87848 & EUR & \cite{hoelzen2022h2} \\ \hline
	\end{tabular}
    \end{center}
\end{table}

Wie bereits im Grundlagen-Kapitel erläutert wurde, hat der kryogene Wasserstoff eine Dichte von 70 $kg/m^3$ (Quelle),  %quelle?
also können 331.240 t in einem kugelförmigen Lagertank gespeichert werden.
Es wird davon ausgegangen, dass die Vorgänge der Betankung des Wagens und folgende Betankung des Flugzeugs jeweils 
30 Minuten dauern \cite{hoelzen2022h2}. (1 Flugzeug pro Tankwagen, also alles von - Ankunftsrate abhängig)
