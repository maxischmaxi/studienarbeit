%\subsection{Annahmen für die Berechnung der Kosten}
%\label{s:Annahmen für die Berechnung der Kosten}

Auf der Grundlage der unterschiedlichen Quellen und vernünftigen Behauptungen wurde eine Reihe der Annahmen getroffen.

Die Betriebskosten werden mit Reisegeschwindigkeit ohne Beachtung von Start und Landung, wo auch mehr Energie verbraucht wird.
Batterie: Mit jetziger Leistung der Batterien wäre es unmöglich bei so einem Gewicht und der Distanz zu bleiben.
Deswegen wird es angenommen, dass die Batterien sich positiv in Gewicht/Leistung-ratio entwickeln und Kapazitätswert von 450 kWh/kg genommen.

Stromkosten von 0.1976 € per kWh werden auch als konstante angenommen \cite{eurostat_nrg_pc_205}.


Manche Studien gehen davon aus, dass die Einsparungen in der Wartungskosten von BA-Flugzeugen 10-15 \% erreichen können 
\cite{wangsness2021fremskyndet,avogadro2024demystifying}. Deswegen in dieser Arbeit wird ein Wert von 10 \% zu dem Referenzflugzeug genommen.

Es wird davon ausgegangen, dass Sicherheitsradius beim Wasserstoffbetankung wird nicht erweitert. 
%Diese Daten in der Literatur sind noch nicht verfügbar.

Infrastrukturkosten

Da sich die Flughäfen Position topografisch unterscheidet, wird die Annahme getroffen, dass die Lieferung mit LKW stattfindet 
und die Lieferkosten sind bereits in dem Preis von flüssigem Wasserstoff inbegriffen.

Alle Werte in USD werden mit dem Wechselkurs \footnote{Wechselkurs vom 14.01.2025} (EUR 1 = USD 1.0245) umgerechnet und beim Pfund (1,20 = EUR 1).

%Nach AEA wird die Versicherung als einem Wert von 0,5 \%  des Flugzeugpreises berechnet und durchschnittliche Verzinsungsrate ca.5 \%

%\cite{scholz_design_evaluation_doc}.
%Die Lebensdauer mach AEA für Kurz- und Mittelstrecken erreicht 14 Jahre und für die Langstrecken 16 Jahre \cite{scholz_design_evaluation_doc}.

%Beechcraft ist mit 2 Triebwerken PT6A-67D ausgestattet, Take-off power (5 Min) - 954 kW, sonst die Leistung 906 kW.\cite{easa_tcds_pt6a67}

Obwohl die maximalen Geschwindigkeiten unter Referenz- und BA-Flugzeugen sich unterscheiden werden, 
für die Kurzstrecken-Flügen ergibt es sich nicht eine sehr große Differenz.
Deswegen es wird angenommen, dass die batteriebetriebenen Flugzeuge ähnliche Auslastung wie konventionelle Flugzeuge aufweisen.

%Lebensdauer von 1000 Zyklen bis die Kapazität bis 80 \% reduziert wird für eine Li-Ion-Batterie wird angenommen, da folgende Werte bei Auto-Industrie bekannt für.
In der Zukunft ist die Lebensdauer einer Batterie bis 5000 und mehr Zyklen zu erwarten \cite{reimers2018introduction}, deswegen wurde auch dieser
Wert in der Berechnung angenommen.

Aufgrund der hohen Unsicherheit der Preisprojektionen wird der Preis für Kerosin konstant gehalten, nämlich im Wert von
0,688 EUR pro Liter \cite{iata_industry_statistics_2024}. 


%Kosten für den Markt?
%Kosten für SAF (HEFA) (Produktionskosten) Jahr 2030 - Durchschnittspreis 1.0 USD/l, 
%minimal 0.8 USD/l, 1.2 USD/l. 
%Fossil jet range with CO2 price 0.4-0.9 USD/l für Jahr 2021 (Carbon price = USD 50/tonne.)\cite{iea_biojet_2021}
In Anbetracht der Kostensenkung für nachhaltige Kraftstoffe in der Zukunft, wird der minimale Preis für HEFA-Treibstoff 
von 1,07 EUR pro Liter wird in den Rechnungen verwendet \cite{watson2024sustainable}.

Lieferkosten für flüssigen Wasserstoff und HEFA werden nicht explizit ausgerechnet,
da die schon in Betriebskosten von Wasserstoff-getriebene Flugzeuge eingeschlossen sind. Es wird vor allem angenommen, dass die Produktion 
des Wasserstoffes und Verflüssigung nicht am Flughafen stattfindet, sondern eingekauft und zum Flughafen transportiert.

Es wird angenommen, dass die Batterien für BA zur Flughafen-Infrastruktur gehören, 
d.h. die Anschaffungskosten Flughafen anfallen und danach werden die für Fluggesellschaften ausgeliehen.
Die Inflationsfaktoren stammen aus Verbraucherpreisindex Deutschland aus Statistischem Bundesamt.

Die Besatzungskosten sind mit Lohn von 37 EUR für Flugbegleiter und 90 EUR für Piloten pro Blockstunden berechnet \cite{discover_airlines_cabin}.

Die konstante Spitzenleistung von Batteriewechselsystem von 250 kW wird angenommen. Der Strom ist normalerweise von Tag und Nacht abhängig
und wie viel parallel genutzt wird. Angenommen ist die Batteriekapazität, wie bei der ES-19, dauert es 3,6 h bis die Batterie geladen wird.
d.h. 10 fzg jeder braucht batterie, dann diese batterie kann erst in 4 std benutzt werden.