%\subsection{Annahmen für die Berechnung der Kosten}
%\label{s:Annahmen für die Berechnung der Kosten}

Infrastrukturkosten
%
Die Inflationsfaktoren stammen aus Verbraucherpreisindex Deutschland aus Statistischem Bundesamt und genutzte Werte sind in der Anlage XX zu finden.
Alle Werte in USD werden mit dem Wechselkurs \footnote{Wechselkurs vom 14.01.2025} (EUR 1 = USD 1.0245) umgerechnet und mit Pfund (1,20 = EUR 1).
Aufgrund der hohen Unsicherheit der Preisprojektionen wird der Preis für Kerosin konstant gehalten, nämlich im Wert von
0,688 EUR pro Liter \cite{iata_industry_statistics_2024}. Stromkosten von 0.1976 € per kWh werden auch als konstante angenommen \cite{eurostat_nrg_pc_205}.
In Anbetracht der Kostensenkung für nachhaltige Kraftstoffe in der Zukunft, wird der minimale Preis für HEFA-Treibstoff 
von 1,07 EUR pro Liter wird in den Rechnungen verwendet \cite{watson2024sustainable}.
%Nach AEA wird die Versicherung als einem Wert von 0,5 \%  des Flugzeugpreises berechnet und durchschnittliche Verzinsungsrate ca.5 \%

%\cite{scholz_design_evaluation_doc}.
%Die Lebensdauer mach AEA für Kurz- und Mittelstrecken erreicht 14 Jahre und für die Langstrecken 16 Jahre \cite{scholz_design_evaluation_doc}.



%Lebensdauer von 1000 Zyklen bis die Kapazität bis 80 \% reduziert wird für eine Li-Ion-Batterie wird angenommen, da folgende Werte bei Auto-Industrie bekannt für.
%In der Zukunft ist die Lebensdauer einer Batterie bis 5000 und mehr Zyklen zu erwarten \cite{reimers2018introduction}, deswegen wurde auch dieser
%Wert in der Berechnung angenommen.



%Kosten für den Markt?
%Kosten für SAF (HEFA) (Produktionskosten) Jahr 2030 - Durchschnittspreis 1.0 USD/l, 
%minimal 0.8 USD/l, 1.2 USD/l. 
%Fossil jet range with CO2 price 0.4-0.9 USD/l für Jahr 2021 (Carbon price = USD 50/tonne.)\cite{iea_biojet_2021}
%In Anbetracht der Kostensenkung für nachhaltige Kraftstoffe in der Zukunft, wird der minimale Preis für HEFA-Treibstoff 
%von 1,07 EUR pro Liter wird in den Rechnungen verwendet \cite{watson2024sustainable}.
