\section{Flugzeugcharakteristika und zugrunde gelegte Annahmen}
\label{ss:Relevante Flugzeugdaten}
%
Um den betrieblichen Unterschied zwischen konventionellen und neuartigen Antrieben aufzuzeigen, 
werden die Referenz-Flugzeuge mit neuen Konzepten verglichen. 
Die Wahl eines Antriebes ist von der Reichweite abhängig.\\
%
\subsubsection{Konventionelle Flugzeuge}
%
Für den Vergleich mit batteriebetriebenen Flugzeugen wurde eine L410 festgelegt. 
Die L410 ist ein Zubringer-Flugzeug mit 19 Plätzen der Firma Aircraft Industries. 
Die moderne Version L410NG verfügt über neue Avionik und ist mit zwei GE H85-200 Triebwerken 
mit einer Wellenleistung von 850 (SHP) ausgestattet \cite{GEAerospace_H85_2025}.
Der Verbrauch einer L410 beträgt 240 kg/h \cite{let2016l410}. 
Sonstige für die Methodik wichtige Flugzeugdaten wurden in der Tabelle \ref{Flugzeuge} zusammengefasst.
Unter $V$ ist die \\ Reisegeschwindigkeit und unter $R$ ist die Reichweite eines Flugzeugs zu verstehen. 
$MTOW$ ist das Höchstabfluggewicht und $EOW$ (Empty Operating Weigth) ist die Betriebsleermasse eines Flugzeugs.

Für den Vergleich von größeren Distanzen wurde eine A321LR festgelegt. 
Die A321LR ist ein Schmalrumpfflugzeug von Airbus und ist eine 
Version der A321neo mit einer höheren Reichweite.
Das Flugzeug ist mit zwei Triebwerken ausgestattet, 
die einen maximalen Schub ($T_{T/O}$) von 33 kN haben \cite{eurocontrol_a321}.

\subsubsection{Alternative Flugzeuge und Annahmen}
Die ES-19 von Heart Aerospace dient als Vergleich zur L410. 
Das Konzept hat einen rein elektrischen, batteriebetriebenen Antrieb.
Heart Aerospace hat die ES-19 zwar auf eine hybride Wasserstoffversion, die ES-30, umgerüstet, 
das Konzept der ES-19 wurde allerdings breit diskutiert und oft in wissenschaftlichen Arbeiten erwähnt. 
Das Flugzeug hat vier Triebwerke und sollte über eine Reichweite von 400 km verfügen, 
hierbei wird eine Reisegeschwindigkeit von 330 km/h erreicht  \cite{anker2023feasibility}, \cite{heart_aerospace_es19}.
Für die ES-19 wird eine Batterie mit einer Kapazität von 720 kWh benötigt,
zuzüglich 30 \% der Reserveenergie resultiert das in 900 kWh \cite{donckers2024electric}. \\
Mit der Leistung heutiger Batterien wäre es unmöglich bei diesem Gewicht und dieser Distanz zu bleiben.
Deswegen wird angenommen, dass die Batterien sich positiv im Gewicht-zu-Leistungs-Verhältnis 
entwickeln und ein Kapazitätswert von 450 kWh/kg erreicht wird.
Manche Studien gehen davon aus, dass die Einsparungen der Wartungskosten 
von batteriebetriebenen Flugzeugen 10-15 \% erreichen können \cite{wangsness2021fremskyndet,avogadro2024demystifying}. 
Deswegen wird in dieser Arbeit eine Verminderung von 10 \% zu dem Referenzflugzeug einberechnet.
%
Da es bis jetzt nur wenig ausgearbeitete größere Konzepte für Wasserstoffantriebe gibt, 
wird der Betriebsvergleich auf Basis einer A321LR stattfinden. 
Dabei wird angenommen, dass das Flugzeug mit Wasserstoffturbine betrieben ist.
Im Vergleich zu konventionellen Flugzeugen werden die Wasserstoff-Flugzeuge, 
welche für Mittelstrecken geeignet sind, ein 14 \% höheres MTOW haben und die Kapitalkosten 
für das Kurzstrecken-Flugzeug um 7 \% sowie Wartungskosten um 6 \% steigen \cite{sky2020hydrogen}. 
Diese Anteile wirken sich zwar positiv auf Mittel- und Langstrecken Flugzeuge aus, 
werden in dieser Arbeit dennoch angenommen.
Der spezifische Treibstoffverbrauch eines Wasserstoffantriebs beträgt nur 35 \% 
des eines Kerosinstrahltriebwerks \cite{scholz2021parameterselection}, dieser Wert wird für die Arbeit verwendet.

Die Änderungen in den Abfertigungsprozessen können erheblich sein \cite{ati_hydrogen_infrastructure}. 
%
Es ist zu erwarten, dass Wasserstoff-Flugzeuge länger als konventionelle Flugzeuge werden
und dass die Kraftstoffsicherheitszone bei Anschließen und der Trennung 
der Wasserstoff-Betankung zum Jahr 2030 auf 20 Meter reduziert wird \cite{hoelzen2022h2}.

Die Flugzeit nimmt aufgrund des Wasserstofftank-Gewichts zwischen 5 und 15 \% zu \cite{sky2020hydrogen}. 
Deshalb wird in dieser Arbeit einen Wert von 10 \% angenommen.
Die Auslastung eines Flugzeugs mit Wasserstoffantrieb kann im Vergleich zu konventionellen Flugzeugen sinken, 
da wasserstoffbetriebene Flugzeuge wegen der Betankungsprozesse potenziell mehr Zeit auf dem Boden verbringen werden \cite{gu2023hydrogen}.
Der Vergleich zu SAF-Betriebskosten findet auch mit der A321LR statt. 
Es wird davon ausgegangen, dass der Unterschied nur bei den Treibstoffkosten entsteht. 
Für die A321LR wird die Passagieranzahl von 220 und ein Verbrauch von 1,7 kg PAXkm angenommen \cite{fonseca2022doc}.
%
In der Tabelle \ref{Flugzeuge} sind relevante charakteristische Werte 
und Annahmen für die Vergleichsflugzeuge zusammengefasst.
Anhand dieser Daten ist die ES-19 langsamer als ein L410, 
das bedeutet für die gleiche Strecke wird mehr Zeit benötigt, 
was am Ende die Auslastung eines Flugzeugs und somit die Betriebskosten verändern kann. 
Aufgrund des Batteriegewichts ist das batteriebetriebene Flugzeug schwerer als konventionelle Alternativen.
Beide Flugzeuge können die gleiche Anzahl an Passagieren befördern. 
Obwohl sich die Reisegeschwindigkeiten bei Referenz- und BA-Flugzeugen unterscheiden werden, 
ergibt sich für Kurzstrecken-Flüge keine erhebliche Differenz.
Deswegen wird angenommen, dass die batteriebetriebenen Flugzeuge ähnliche 
Auslastungen wie konventionelle Flugzeuge aufweisen.
Auch eine Änderung der Lebensdauer von batteriebetriebenen Flugzeugen wird nicht erwartet \cite{reimers2018introduction}.
Eine A321LR erreicht mit oben genannten Annahmen höhere Geschwindigkeiten und Reichweiten, 
sowie ein geringeres MTOW und EOW.
%
%
%%%% kommt in die 2_4_3 Wartung:
%Brennstoffzellen haben ebenfalls weniger bewegende Teile \cite{dalmia2022powering} 
%weshalb es in diesem Bereich zu weniger Wartungskosten kommen kann,
%allerdings hat der benötigte Wasserstofftank kürzere Wartungsintervalle (Quelle). 
%\cite{gu2023hydrogen}

\begin{table}[h]
	\begin{center}
    \caption{Bewertete Flugzeuge: Werte und Annahmen}
	\label{Flugzeuge}
	\begin{tabular}{|l|c|c|c|c|c|c|}
		\hline
		 & \textbf{V} ~[\text{km/h}] & \textbf{R} ~[\text{km}] & \textbf{MTOW} ~[\text{kg}] & \textbf{EOW} ~[\text{kg}] & \textbf{PAX-Anzahl} 
		 & \textbf{Quellen} \\ \hline
		L410  & 417 & 2 570 & 7 000 & 4 120 & 19 & \cite{let_l410ng}\\ \hline
		ES-19 &  330 & 400 & 8 618 & - & 19 & \cite{anker2023feasibility} \cite{heart_aerospace_es19}\\ \hline
		A321LR & 1104 & 7 400 & 97 000 & 52 060 & max. 244 & \cite{airbus_a321neo} \cite{fonseca2022doc} \\ \hline
		WA & ~913 & - & 110 580 & - & 220 & -\\ \hline
	\end{tabular}
    \end{center}
\end{table}

Dass die Anschaffungspreise die Betriebskosten beeinflussen, 
wurde bereits in \ref{s:Kosten} angeführt. 
Die Tabelle \ref{Flugzeugpreise} stellt die Verkaufspreise 
für konventionelle Referenz-Flugzeuge dar.
Da der Kaufpreis einer A321LR nicht zur Verfügung steht, 
wird auf den Listenpreis einer A321neo zurückgegriffen. 
Da sie aus einer Flugzeug-Reihe kommen, kann davon ausgegangen werden, 
dass die Preise ähnlich sind. Zusätzlich wird der Inflationsfaktor einbezogen. %%%%% klingt das nicht komisch?
%Die Preise für alternative Antriebe sind wegen  nicht dargestellt

\begin{table}[h]
	\begin{center}
    \caption{Flugzeugpreise}
	\label{Flugzeugpreise}
	\begin{tabular}{|l|c|c|}
		\hline
		 & \textbf{L410} & \textbf{A321neo}  \\ \hline
		 Verkaufspreis ~[\text{EUR}] & 6,46 Mio \cite{marksel2023comparative} & 129,5 Mio \cite{aerotelegraph_airbus} \\ \hline
	\end{tabular}
    \end{center}
\end{table}

