\chapter{Änderungen durch neue Antriebe, Annahmen und Methodik}
\label{ch:Änderungen durch neue Antriebe, Annahmen und Methodik}

Konzepte mit neuen Antrieben sind im Entwicklungsprozess. Wasserstoff-Flugzeuge werden im Jahr 2035 erwartet, wobei die BA-Flugzeuge
schon in den nächsten Jahren erwartet werden.
Durch den Wechsel zu nachhaltigen Antrieben kann es zu Änderungen in der Infrastruktur und Abläufen am Vorfeld kommen.


Die Größe des Flughafens beeinflusst die Infrastrukturkosten. Größere Flughäfen können mehr Flugzeuge als Regionalflughäfen 
abfertigen, was dazu führt, dass viele Abfertigungsplätze umgerüstet werden müssen und 
mehr Arbeitskräfte geschult werden müssen. In diesem Teil wird näher auf die Änderungen in der Infrastruktur durch die Einführung 
von neuen Antrieben eingegangen.
%Wenn die Infrastruktur auf den Regionalflughäfen gemacht wird, 
%macht es nicht so ein Ausmaß wie auf größeren Flughäfen, wo viele Abfertigungsplätze umgerüstet werden müssen.

\section{Änderungen an der Abfertigung und dazugehörige Kosten von alternativen Antrieben}
\label{s:Änderungen an der Abfertigung und dazugehörige Kosten von alternativen Antrieben}

\subsection{SAF}

SAF ist zwar nicht die beste langfristige Lösung wegen vorhandenen Emissionen. 
Während Elektroflugzeuge entwickelt werden, sind SAFs eine gute Option, um die Klimaauswirkungen des Luftverkehrs zu reduzieren.
In der nahen Zukunft werden die großen BA nicht entwickelt, deswegen können SAFs für die Langstreckenflüge benutzt werden \cite{dalmia2022powering}.


SAF benötigt keine Infrastrukturänderung und darf in bestehenden Systemen und Flugzeugen benutzt werden \cite{dalmia2022powering}.
Dadurch, dass SAF zu herkömmlichen Treibstoffen beigemischt wird, ist zurzeit einen zusätzlichen Treibstofftank 
für das gemischten Kraftstoff gebraucht. Bis jetzt Transport von SAF mit einer Pipeline nicht zugelassen (Quelle?).
Es wurde keine Information gefunden, die besagt, dass es verboten wird reine SAF nicht als
Drop-In zu benutzen, deswegen für die Arbeit, wird gelten, dass die SAF-Transport mit Pipelines und genauso wie Kerosin-Betankung zugelassen wird.

\subsection{Batterie-Antrieb}


Batteriegetriebene Flugzeuge brauchen größere Veränderung am Flughafen als bei SAF Nutzung.
Bis Jahr 2050 sind die BAs auf die kleineren Flugzeuge und damit auf Kurz- und Regionalverkehr beschränkt (Quelle). 

Wartungsverfahren unterscheidet sich stark von konventionellen Flugzeugen (CA). 

Es ist zu erwarten, dass die Wartungskosten niedriger sein können durch weniger bewegende Teile in dem Antrieb. 
Jedoch bei einem Fehler muss den ganzen Motor ausgetauscht werden \cite{dalmia2022powering}, 
was die hohen Kosten mit sich bringt.
Die elektrischen Motoren werden keine Hot-Section Inspection benötigt, jedoch die Teile wie Lager ausgetauscht werden 
müssen und mögliche Schäden durch die Fremdpartikeln in den Motor \cite{reimers2018introduction}. 
Die Änderung der Lebensdauer wird auch nicht erwartet \cite{reimers2018introduction}.
Für die Batteriewechsel müssen auch Transport und Hebegeräte gestellt werden, um die Batterien bewegen zu können \cite{reimers2018introduction}.

Bei BAs sind die Energiequellen von der Bedeutung. 
Der Strom aus dem Stromnetz kann sein Ursprung aus den Kraftwerken und Verteilerzentren haben \cite{dalmia2022powering}. 
Was dazu führt, dass die fossilen Brennstoffe für die Verbrennung benutzt werden und dadurch zu Emissionen beizutragen. 
Als Alternative wäre Nutzung der erneuerbaren Energiequellen, wie Windenergie oder Solarenergie. Diese Energie ist normalerweise teurer
und die Produktionsmenge ist bis jetzt nicht ausreichend, um die Luftverkehr-Nachfrage zu decken (Quelle).

Flugzeugabstellplätze sollen umgerüstet werden. Es gibt die Möglichkeit, die Batterie zu wechseln oder Ladekabel 
in das Flugzeug einzustecken (Plug-In).
Je länger die Ladung dauert, desto mehr Kosten auf dem Boden verursacht werden. 
Bei BA muss beachtet werden, welche Lebensdauer eine Batterie hat und wie die Batterien geladen werden. Schnelle Ladung kann zur Stagnation von
Lebensdauer einer Batterie bringen.

Aus- und Einbau der Batterie aus/in dem Flugzeug kann lange dauern \cite{dalmia2022powering}. Batteriewechsel ist effizienter und ökonomischer, 
wenn die batteriebetriebenen Flugzeuge nur ein kleiner Teil (unter 10\%) der Flotte ist, in anderem Fall lohnt sich eine Plug-In-Ladung \cite{guo2020aviation}. 

Jedoch mit Batteriewechsel können die Abfertigungszeiten reduziert werden (Quelle), was an einem großen Flughafen wichtig ist. 
Batteriewechsel bietet gleichmäßigere Deckung der Nachfrage, 
da der Austausch einer Batterie viel schneller ist, als Dauer einer Plug-In Ladung \cite{guo2020aviation}.
Aus diesen Gründen werden die Kosten für diese Möglichkeit ausgewertet.

%Für die Ladung den batteriegetriebene-Flugzeugen Infrastruktur:

%Infrastrukturmodell nach Guo et al. "Die EA-Aufladung wird teilweise von einer flughafenbasierten Solar-PV-Anlage geliefert.
%Jährliche Betriebskosten (OPEX)= Strombezugskosten aus dem Netz in der Sommer und Wintersaison (typische Tage) basierend auf der Nachfrage. \cite{guo2020aviation}

\subsection{Wasserstoff}

Wasserstoff kann hochentzündlich sein \cite{dalmia2022powering}. (prüfen ob entzündlicher als Kerosin)

Die Effizienz der Produktion- und Lieferkosten sind von geografischen Aspekten abhängig. 
Für europäische Distanzen sind die Wasserstoff-Pipelines günstiger als Transport mit chemischen Verbindungen, 
welcher bei längeren Distanzen in Frage kommt. Bei Änderung von Erdgasleitung für Wasserstoff können Kosten gespart
werden und keine neue Infrastruktur muss gebaut werden.\cite{undertaking2022strategic}

Lieferung den flüssigen Wasserstoff kann mit einem LKW bei einer großen Zahl an Flügen kostengünstiger als andere 
Lieferalternativen sein \cite{schenke2024lh2}.


Wasserstoff kann unterirdische in Salzkavernen und in erschöpften Gasfelder gespeichert werden \cite{undertaking2022strategic}, 
jedoch in der Flughafen nähe würde diese Option aus 
platzgründen nicht passen und man braucht
die oberirdische Druckzylinder, wo flüssiger Wasserstoff oder in festen Materialien (wie Metallhybriden) gespeichert wird.

Bei der Lagerung wird flüssiger Wasserstoff verdampt, was zur Verlust der Menge kommt, 
größte Teil wird jedoch durch Transferphase verdampft \cite{undertaking2022strategic}.

Diese Zylinder oder Tankern müssen gut isoliert und kryogen sein \cite{undertaking2022strategic}.
Transportkosten sind von der geografische Lage des Flughafens abhängig, befindet sich ein Flughafen nahe einer Wasserstoff-Pipeline...
"current FCEV system costs higher than 200 €/kW for passenger cars but need 
to fall below 50 €/kW for mass market. " \cite{undertaking2022strategic}

Auslastung eines Flugzeugs mit Wasserstoff-Antrieb kann weniger Stunden bedeuten. (Auslastung weniger, da möglicherweise mehr Zeit auf dem Boden
wg Betankung) 

Die Betankungsstruktur am Flughafen muss ausgetauscht werden, diese Investitionskosten werden auch die Betreibern (Fluggesellschaften)
beeinträchtigen. 

Lieferung von gasförmigen Wasserstoff mit niedrigen Druck mit dem Straßenverkehr bis jetzt nur für ungenügende Menge möglich \cite{undertaking2022strategic}

Brennstoffzellen haben auch weniger bewegende Teile \cite{dalmia2022powering}
"Flüssiger Wasserstoff muss jedoch bei Minusgraden gelagert werden, was eine Verbesserung der Speichertechnologien sowohl im Flugzeug selbst als auch auf Flughäfen erfordert."
Wasserstoff kann entweder als Brennstoff für Verbrennung genutzt werden oder als Wasserbrennstoffzelle für den elektrischen Flugzeuge. \cite{dalmia2022powering}

Produktion des Wasserstoff braucht viel am Kosten, Platz und Energie. Deswegen für die Flughäfen wäre es vielleicht besser das Wasserstoff, woanders einzukaufen und zum Flughafen 
mit LKW oder Pipelines liefern lassen \cite{gu2023hydrogen}. Die Änderung für Abfertingsprozesse können erheblich sein \cite{ati_hydrogen_infrastructure}. Es ist zu erwarten, dass die Wasserstoff-Flugzeuge länger als konventionelle Flugzeuge werden
und dass Kraftstoffsicherheitszone bei Anschließen und Trennung von Wasserstoff-Betankung zum Jahr 2030 auf 20 m reduziert werden (Quelle 39)

Aufgrund hohen Unterschiedes zu herkömmlichen Treibstoffen und neuen Ausrüstungen müssen die Mitarbeiter neu geschult werden, 
um mögliche Gefahren zu erkennen und zu vermeiden \cite{gu2023hydrogen}.

Für die Wasserstofferzeugung wäre die ideale Methode der Elektrolyse, die mit Strom betrieben würde, der vor Ort aus 
erneuerbaren Ressourcen erzeugt würde, wie z. B. Solarenergie aus nicht reflektierenden Paneelen (um eine Blendung der 
Solarmodule für Piloten zu vermeiden). Das Wasser, das für den Prozess verwendet wird, wird aus nahe gelegenen 
Wasserquellen stammen, einschließlich Süßwasserflüssen und Seen. Diese Wasserstoffproduktion wird vor Ort an Flughäfen 
stattfinden. Die benötigte Infrastruktur umfasst einen Elektrolyseur, einen Kompressor und einen Tankwagen. 
Der Elektrolyseur wird gasförmigen Wasserstoff erzeugen. Der Kompressor erhöht dann den Druck des Wasserstoffs, 
indem er sein Volumen für die Speicherung reduziert, und der Tankwagen transportiert den komprimierten Wasserstoff 
zum Elektroflugzeug. Eine weitere Option für den Transport wäre die Installation einer landseitigen bis luftseitigen Pipeline,
 die den komprimierten gasförmigen Wasserstoff in einem modularen Tanksystem transportiert. Dieses System ermöglicht 
 die Nachrüstung von bereits im Einsatz befindlichen Flugzeugen. Die Wasserstofftanks würden ähnlich wie Fracht 
 in ein Flugzeug geladen, festgeschnallt und sicher mit dem Flugzeug verbunden. Wenn das Ziel erreicht ist, 
 wird der leere Tank durch einen neuen Tank ersetzt, der dann zurückgeschickt wird, um in der Wasserstoffproduktionsanlage 
 am Flughafen wieder aufgefüllt zu werden. Dieses System ermöglicht den Einsatz von Wasserstoff in allen Bereichen des Fluges,
  wodurch die Effizienz maximiert, das Gesamtgewicht reduziert und die Nutzlast und Reichweite verbessert werden.²³\cite{dalmia2022powering}

