
\section{Betriebsszenarien}
Betriebsszenarien helfen die vorher beschriebene Kosten in Anwendung zu bringen und zeigen mögliche Entwicklung bis zum Jahr 2050.
Für die Betriebsszenarien wurde der Flughafen Frankfurt ausgesucht. Das ist der größte Flughafen in Deutschland und als 
funktioniert ein Hub. Bei einem größeren Flughafen werden die Betriebsdifferenzen deutlicher, da Verkehrsaufkommen viel größer ist.
Fraport meldete im Jahr 2023 insgesamt 423764 gewerbliche Flugbewegungen, das sind im Durchschnitt 1160 Flugbewegungen pro Tag. 
Es wird angenommen, dass die Hälfte davon ist die Abflüge, also müssen 580 Flugzeuge am Tag abgefertigt werden.

Die Gesamtbewegungen teilen sich nach Entfernungen folgend aus:
\begin{itemize}
    \item Kurzstrecken (bis 2500 km) sind bei 72,8 \%;
    \item Mittelstrecken (bis 6000 km) sind 9,3 \%;
    \item Langstrecken die restlichen 17,9 \%. 
    \end{itemize}
Diese Verteilung wird auch für die Betriebsszenarien benutzt. Es wird angenommen, dass die Flüge von 6 bis 24 Uhr stattfinden. Dies entspricht
18 Stunden.
Die Kurzstrecken-Flüge können durch die Batterieantriebe ersetzt werden. 
Es ist nennenswert, dass die Nachfrage des Kurzstrecken-Bedarfs dadurch nicht bedeckt werden kann, jedoch für diese Arbeit wird angenommen,
dass die gleiche Anzahl an Flugzeugen am Flughafen abgewickelt werden.



/subsection{Betriebsszenario I}

In dem ersten Betriebsszenario wird es angenommen, dass die Langstrecken durch die Batterieantriebe komplett ersetzt werden.
Die Mittelstrecken werden komplett durch Wasserstoff-Antrieben ausgetauscht.

/subsection{Betriebsszenario II}