
\section{Betriebsszenarien}
Betriebsszenarien helfen die vorher beschriebene Kosten in Anwendung zu bringen und zeigen eine mögliche Entwicklung im Jahr 2050.
Für die Betriebsszenarien wurde der Flughafen Frankfurt gewählt. 
Dies ist der größte Verkehrsflughafen Deutschlands und fungiert als bedeutendes Luftverkehrsdrehkreuz. 
Bei einem größeren Flughafen werden die Betriebsdifferenzen deutlicher, da das Verkehrsaufkommen wesentlich höher ist.
Der Fraport meldete im Jahr 2023 insgesamt 423764 gewerbliche Flugbewegungen, das sind im Durchschnitt 1160 Flugbewegungen pro Tag. 
Es wird angenommen, dass die Hälfte davon Abflüge sind, also müssen 580 Flugzeuge am Tag abgefertigt werden.

Die Gesamtbewegungen teilen sich nach Entfernungen folgend auf:
\begin{itemize}
    \item Kurzstrecken (bis 2500 km) sind bei 72,8 \%;
    \item Mittelstrecken (bis 6000 km) sind 9,3 \%;
    \item Langstrecken die restlichen 17,9 \%. 
    \end{itemize}
Diese Verteilung wird auch für die Betriebsszenarien genutzt. 
Es wird angenommen, dass die Flüge von 6 bis 24 Uhr stattfinden. 
Dies entspricht 18 Stunden.
Kurzstrecken-Flüge können durch den Einsatz von batteriebetriebenen Flugzeugen ersetzt werden. 
Es ist nennenswert, dass die Nachfrage des Kurzstrecken-Bedarfs dadurch nicht gedeckt werden kann, 
jedoch in dieser Arbeit angenommen, dass die gleiche Anzahl an Flugzeugen am Flughafen abgewickelt werden.

/subsection{Betriebsszenario I}

In dem ersten Betriebsszenario wird angenommen, dass die Langstrecken durch die Batterieantriebe komplett ersetzt werden.
Die Mittelstrecken werden komplett durch Wasserstoff-Antrieben ersetzt.

/subsection{Betriebsszenario II}
