\chapter{Änderungen durch neue Antriebe, Annahmen und Methodik}
\label{ch:Änderungen durch neue Antriebe, Annahmen und Methodik}

Konzepte mit neuen Antrieben befinden sich im Entwicklungsprozess und bis jetzt es ist ratsam, wie zukünftige Flughäfen aussehen werden
und welche Ausstattung für die Flugzeug-Abfertigung ausgesucht wird. Wasserstoff-Flugzeuge werden ab Jahr 2035 in 
auf den Markt eintreten, wobei die BA-Flugzeuge schon in den nächsten Jahren erwartet werden.
Der Wechsel zu nachhaltigen Antrieben kann es zu deutlichen Änderungen in der Infrastruktur und Abläufen am Vorfeld führen, die
in diesem Kapitel beschrieben werden.
Außerdem auf der Grundlage der unterschiedlichen Quellen und vernünftigen Behauptungen wird eine Reihe der Annahmen getroffen.
%Dieser Teil der Arbeit beschäftigt sich mit der vorhandenen Infrastruktur-Optionen. %und im Fall des Wasserstoffs Lieferketten.
%
Die Größe des Flughafens beeinflusst die Infrastrukturkosten. Größere Flughäfen können mehr Flugzeuge als Regionalflughäfen 
abfertigen, was dazu führt, dass mehr Abfertigungsplätze umgerüstet/versorgt werden müssen und mehr Arbeitskräfte geschult werden müssen. 
In diesem Teil wird näher auf die Änderungen in der Infrastruktur durch die Einführung 
von neuen Antrieben eingegangen.
%Wenn die Infrastruktur auf den Regionalflughäfen gemacht wird, 
%macht es nicht so ein Ausmaß wie auf größeren Flughäfen, wo viele Abfertigungsplätze umgerüstet werden müssen.

\section{Änderungen an der Abfertigung und dazugehörige Kosten von alternativen Antrieben}
\label{s:Änderungen an der Abfertigung und dazugehörige Kosten von alternativen Antrieben}

\subsection{SAF}
SAF ist zwar nicht die beste langfristige Lösung wegen vorhandenen Emissionen, aber wegen benötigter Entwicklung der anderen nachhaltigen
Antrieben stellt eine gute Option dar. In der nahen Zukunft werden vor allem die großen Flugzeuge mit BA-Antrieb nicht entwickelt, 
deswegen können SAF für die Langstreckenflüge benutzt werden \cite{dalmia2022powering}.

SAF benötigt keine Infrastrukturänderung und darf in bestehenden Systemen und Flugzeugen benutzt werden \cite{dalmia2022powering}.
Dadurch, dass SAF zu herkömmlichen Treibstoffen beigemischt wird, ist zurzeit einen zusätzlichen Treibstofftank 
für das gemischten Kraftstoff gebraucht. Bis jetzt Transport von SAF mit einer Pipeline nicht zugelassen (Quelle?).
Bei der intensiven Recherche wurde keine Information gefunden, die besagt, dass es verboten wird reine SAF nicht als
Drop-In zu benutzen. Aus diesem Grund gilt für die Arbeit, dass die Lieferung von SAF mit bestehenden Pipelines 
und genauso wie Kerosin-Betankung zugelassen wird.

\subsection{Batterie-Antrieb}
Batteriegetriebene Flugzeuge brauchen größere Veränderung am Flughafen als bei der Nutzung von SAF.
Bis zum Jahr 2050 sind die BAs auf die kleineren Flugzeuge und damit auf Kurz- und Regionalverkehr beschränkt (Quelle). 

Wartungsverfahren unterscheidet sich stark von konventionellen Flugzeugen. 
Es ist zu erwarten, dass die Wartungskosten niedriger sein können durch weniger bewegende Teile in dem Antrieb. 
Jedoch bei einem Fehler muss den ganzen Motor ausgetauscht werden \cite{dalmia2022powering}, 
was die hohen Kosten mit sich bringt.
Die elektrischen Motoren werden keine Kontrolle Gasturbinentriebwerks ("Hot-Section Inspection") benötigen, 
jedoch die Teile wie Lager ausgetauscht werden müssen und 
es werden mögliche Schäden durch Gelangen den Fremdpartikeln in den Motor verursacht \cite{reimers2018introduction}, was ungeplante
Wartungen benötigt.
Die Änderung der Lebensdauer von Flugzeugen wird auch nicht erwartet \cite{reimers2018introduction}.

In der Literatur werden zwei Lademöglichkeiten diskutiert, die Batterien zu wechseln (Swap-Methode), d.h. aus dem Flugzeug herauszunehmen und 
an einer Ladestation zu laden, oder Ladekabel in das Flugzeug einzustecken (Plug-In).
%

Bei BA muss beachtet werden, welche Lebensdauer eine Batterie hat und wie die Batterien geladen werden. 
Je länger die Ladung dauert, desto mehr Kosten auf dem Boden verursacht werden. Nichtsdestotrotz kann schnelle Ladung 
zur Stagnation von Lebensdauer einer Batterie bringen.
Bei \textit{Swap-Methode} kann Aus- und Einbau der Batterie aus dem/in das Flugzeug lange dauern \cite{dalmia2022powering}. 
Guo et al. \cite{guo2020aviation} ist zum Schluss gekommen, dass Batteriewechsel effizienter und ökonomischer ist, 
wenn die batteriebetriebenen Flugzeuge nur ein kleiner Teil (unter 10 \%) der Flotte ist, in anderem Fall lohnt sich eine Plug-In-Ladung. 
Für die Batteriewechsel müssen auch Transport und Hebegeräte gestellt werden, um die Batterien bewegen zu können \cite{reimers2018introduction}.
Jedoch mit Batteriewechsel können die Abfertigungszeiten reduziert werden (Quelle), was an einem großen Flughafen von der Bedeutung ist. 
Batteriewechsel bietet gleichmäßigere Deckung der Nachfrage, 
da der Austausch einer Batterie viel schneller ist, als Dauer einer Plug-In Ladung \cite{guo2020aviation}.
Aus diesen Gründen werden in nächsten Teilen die Kosten für diese Option ausgewertet.

Austausch ermöglicht langsameres Laden und macht das Laden mit geringer Leistung möglich.
Anstieg in Turnaround-Zeiten aufgrund nicht zurzeit möglichen Schnellladen möglich \cite{avogadro2024demystifying}.

Die konstante Spitzenleistung von Batteriewechselsystem von 250 kW wird angenommen. Der Strom ist normalerweise von Tag und Nacht abhängig
und davon welche Nutzung herrscht wird. Angenommen ist die Batteriekapazität, wie bei der ES-19, dauert es 3,6 h bis die Batterie geladen wird.
d.h. 10 fzg jeder braucht batterie, dann diese batterie kann erst in 4 std benutzt werden.
%
Bei \textit{Plug-In} ...
%
Für die Nachhaltigkeit des Batterieantriebes sind die Energiequellen von der Bedeutung. 
Der Strom aus dem Stromnetz kann sein Ursprung aus den Kraftwerken und Verteilerzentren haben \cite{dalmia2022powering}. 
Was dazu führt, dass die fossilen Brennstoffe für die Verbrennung benutzt werden und dadurch zu Emissionen beitragen. 
Als Alternative wäre Nutzung der erneuerbaren Energiequellen, wie Windenergie oder Solarenergie. Diese Energie ist normalerweise teurer
und die Produktionsmenge ist bis jetzt nicht ausreichend, um die Luftverkehr-Nachfrage zu decken (Quelle).
Anstieg in Turnaround-Zeiten aufgrund nicht zurzeit möglichen Schnellladen möglich \cite{avogadro2024demystifying}.
%Für die Ladung den batteriegetriebene-Flugzeugen Infrastruktur:
%Infrastrukturmodell nach Guo et al. "Die EA-Aufladung wird teilweise von einer flughafenbasierten Solar-PV-Anlage geliefert.
%Jährliche Betriebskosten (OPEX)= Strombezugskosten aus dem Netz in der Sommer und Wintersaison (typische Tage) basierend auf der Nachfrage. \cite{guo2020aviation}
Modulares System: beide gleichzeitig

\subsection{Wasserstoff}
%
Die Betankungsanlagen am Flughafen müssen ausgetauscht werden oder neue Lieferketten angeschafft werden müssen, 
um Wasserstoff als Treibstoff benutzen zu können.
Diese Investitionskosten werden die Betreiber beeinträchtigen. 
Wasserstoff kann hochentzündlich sein \cite{dalmia2022powering}. %(prüfen ob entzündlicher als Kerosin)
Wird der flüssigen Wasserstoff verschüttet, wird es aufgrund seiner Leichtigkeit vertikal nach oben verdampft \cite{colpan2022fuel}. 
Die Dauer des Brandes von LH2 ist kürzer als bei Kerosin \cite{colpan2022fuel}.
% 
Die Effizienz der Produktion- und Lieferkosten ist geografisch determiniert. 
Für europäische Distanzen sind die Wasserstoff-Pipelines günstiger als Transport mit chemischen Verbindungen, 
welcher bei längeren Distanzen in Betracht kommt \cite{undertaking2022strategic}. Bei Änderung von Erdgasleitung für Wasserstoff können Kosten gespart
werden und muss keine neue Infrastruktur gebaut werden, sondern die Leitungen für Wasserstoff umgerüstet werden können \cite{undertaking2022strategic}.
Lieferung
Logistik ist ein wichtiger Teil der Produktionskette. Nach Schenke et al. \cite{schenke2024lh2} kann Lieferung den flüssigen Wasserstoff mit einem LKW bei einer großen Zahl an Flügen 
kostengünstiger als andere Lieferalternativen sein. Transport von LH2 erfordert speziell konstruierte Tanks \cite{mulder2019outlook}.
Der Transport ist durch Pipelines im gasförmigen Zustand, LKW und Zügen sowohl im gasförmigen, als auch im flüssigen Zustand möglich. 

Wasserstoff kann unterirdisch in Salzkavernen und in erschöpften Gasfelder gespeichert werden \cite{undertaking2022strategic}, 
Sie müssen sich in der unmittelbaren Nähe zum Flughafen befinden. Da es sich je nach Flughafenstandort variiert, wird dieser Speicheroption nicht weiter behandelt.
Außerdem kann es für die Lagerung ein oberirdischer Druckzylinder, wo flüssiger Wasserstoff oder in festen Materialien (wie Metallhybriden) gespeichert wird.
Diese Zylinder oder Tankern müssen gut isoliert und kryogen sein \cite{undertaking2022strategic}.
Andernfalls wird flüssiger Wasserstoff bei der Lagerung verdampft, was zum Verlust der Menge kommt, 
jedoch der größte Teil der Verdampfung findet durch Transferphase statt \cite{undertaking2022strategic}.
%Wasserstoff kann in Hochdrucktanks und Salzkavernen gelagert werden \cite{mulder2019outlook}
Der Weg zwischen Betankung und dem Tank, damit Verdampfungsverluste minimiert werden können \cite{colpan2022fuel}
Die externe Produktion ist am Anfang sinnvoll \cite{colpan2022fuel}. Der Transfer von LH2 mit vakuumsisolierte Pipeline beschränkt sich auf die kurze Distanzen
wegen Skalierung der Verluste zu Leitungslänge (proportional) \cite{colpan2022fuel}.
Wird der größe Menge benötigt, weder LKW noch Pipeline sinnvoll \cite{colpan2022fuel}

Obwohl auf Wasserstoff und dazugehörige Infrastruktur wartet eine Reihe der Zertifizierungen, die Anschaffungskosten können bereits gefunden werden.

Aufgrund aufwendigen Infrastrukturprozessen für die Produktion und Verflüssigung von Wasserstoff werden wahrscheinlich Flughäfen, 
besonders kleineren, anfangs auf die "On-Site" Produktion verzichten.

Die Verdampfung wird mit größeren Lager kleiner \cite{colpan2022fuel}.
Eine weitere mögliche Betankungsoption ist die Austausch des Flugzeugtanks als Kapseln, dabei werden die leeren Kapseln an die 
Wasserstoffproduktionstelle zurückgegeben, wo die wieder nachgefüllt werden können \cite{colpan2022fuel}. 
Diese Möglichkeit kann vor allem für die kleineren Flughäfen als Alternative sein, da man kein Wasserstoffspeicher und 
sonstige Anlagen bauen muss.
Für längeres Parken am Flughafen, werden für kalte Tanks eine sichere Verbindung mit der Wasserstoffinfrastruktur \cite{colpan2022fuel} %weiß nicht
Transportkosten sind von der geografische Lage des Flughafens abhängig, befindet sich ein Flughafen nahe einer Wasserstoff-Pipeline...
%"current FCEV system costs higher than 200 €/kW for passenger cars but need 
%to fall below 50 €/kW for mass market. " \cite{undertaking2022strategic}

Auslastung eines Flugzeugs mit Wasserstoff-Antrieb kann im Vergleich zu konventionellen Flugzeug sinken, da eventuell WA-Flugzeuge wegen der Betankungsprozesse
mehr Zeit auf dem Boden verbringen werden (Quelle).

Lieferung von gasförmigem Wasserstoff mit niedrigem Druck mit dem Straßenverkehr bis jetzt nur für ungenügende Menge möglich \cite{undertaking2022strategic}

Wartung:
Brennstoffzellen haben auch weniger bewegende Teile \cite{dalmia2022powering} und in dem Bereich zu weniger Wartungskosten kommen kann,
allerdings der benötigte Wasserstofftank braucht öfteren Wartung (Quelle).
%
"Flüssiger Wasserstoff muss jedoch bei Minusgraden gelagert werden, was eine Verbesserung der Speichertechnologien sowohl im Flugzeug selbst als auch auf Flughäfen erfordert."
Wasserstoff kann entweder als Brennstoff für Verbrennung genutzt werden oder als Wasserbrennstoffzelle für die elektrischen Flugzeuge. \cite{dalmia2022powering}

In dem Unterkapitel \ref{Wasserstoff-Antrieb} wurde angeführt, dass die Produktion des Wasserstoffs hohe Kosten, viel Platz und Energie benötigt. 
Deswegen für die Flughäfen wäre es die Alternative besser, das Wasserstoff, woanders einzukaufen und zum Flughafen 
mit LKW oder Pipelines liefern zu lassen \cite{gu2023hydrogen}. Die Änderung für Abfertingsprozesse können erheblich sein \cite{ati_hydrogen_infrastructure}. Es ist zu erwarten, dass die Wasserstoff-Flugzeuge länger als konventionelle Flugzeuge werden
und dass Kraftstoffsicherheitszone bei Anschließen und Trennung von Wasserstoff-Betankung zum Jahr 2030 auf 20 m reduziert werden \cite{hoelzen2022h2}.

Aufgrund hohen Unterschiedes zu herkömmlichen Treibstoffen und neuen Ausrüstungen müssen die Mitarbeiter neu geschult werden, 
um mögliche Gefahren zu erkennen und zu vermeiden \cite{gu2023hydrogen}.

In der Arbeit Dalmia et al. \cite{dalmia2022powering} wird die Produktion am Flughafen diskutiert. 
Dazu wird einen Elektrolyseur für Erzeugung des gasförmigen Wasserstoffs, einen Kompressor und einen Tankwagen oder
eine Pipeline mit modularem Tanksystem für Betankung der Flugzeuge benötigt. Modulares Tanksystem kann in bestehenden Flugzeugen eingesetzt werden
und wird als wie Fracht in das Flugzeug geladen. %(nochmal nachlesen)

%Für die Wasserstofferzeugung wäre die ideale Methode der Elektrolyse, die mit Strom betrieben würde, der vor Ort aus 
%erneuerbaren Ressourcen erzeugt würde, wie z. B. Solarenergie aus nicht reflektierenden Paneelen (um eine Blendung der 
%Solarmodule für Piloten zu vermeiden). Das Wasser, das für den Prozess verwendet wird, wird aus nahe gelegenen 
%Wasserquellen stammen, einschließlich Süßwasserflüssen und Seen. Diese Wasserstoffproduktion wird vor Ort an Flughäfen 
%stattfinden. Die benötigte Infrastruktur umfasst einen Elektrolyseur, einen Kompressor und einen Tankwagen. 
%Der Elektrolyseur wird gasförmigen Wasserstoff erzeugen. Der Kompressor erhöht dann den Druck des Wasserstoffs, 
%indem er sein Volumen für die Speicherung reduziert, und der Tankwagen transportiert den komprimierten Wasserstoff 
%zum Elektroflugzeug. Eine weitere Option für den Transport wäre die Installation einer landseitigen bis luftseitigen Pipeline,
% die den komprimierten gasförmigen Wasserstoff in einem modularen Tanksystem transportiert. Dieses System ermöglicht 
% die Nachrüstung von bereits im Einsatz befindlichen Flugzeugen. Die Wasserstofftanks würden ähnlich wie Fracht 
% in ein Flugzeug geladen, festgeschnallt und sicher mit dem Flugzeug verbunden. Wenn das Ziel erreicht ist, 
% wird der leere Tank durch einen neuen Tank ersetzt, der dann zurückgeschickt wird, um in der Wasserstoffproduktionsanlage 
% am Flughafen wieder aufgefüllt zu werden. Dieses System ermöglicht den Einsatz von Wasserstoff in allen Bereichen des Fluges,
% wodurch die Effizienz maximiert, das Gesamtgewicht reduziert und die Nutzlast und Reichweite verbessert werden.²³\cite{dalmia2022powering}
%
%Gerade wird Wasserstoff durch die Pipelines transportiert \cite{mulder2019outlook}. ist es so?
%Der Transport ist durch Pipelines im gasförmigen Zustand, LKW und Zügen sowohl im gasförmigen,
% als auch im flüssigen Zustand möglich. 
% Transport von LH2 erfordert speziell konstruierte Tanks \cite{mulder2019outlook}
Mulder et al. \cite{mulder2019outlook} schätzt die Investitionskosten für Wasserstoff durch Elektrolyse deutlicher günstiger als ein Kohlekraftwerk. 
Obwohl die hohen Kapitalkosten von Pipeline-Anlage zu erwarten, die Betriebskosten niedriger sein werden. Also bei größerem Umfang lohnt es Pipelines, sonst Lkw.
Annahmen:
Es wird davon ausgegangen, dass Sicherheitsradius beim Wasserstoffbetankung nicht erweitert wird.
Da sich die Flughafenposition topografisch unterscheidet, wird die Annahme getroffen, dass die Lieferung mit LKW stattfindet 
und die Lieferkosten bereits in dem Preis von flüssigem Wasserstoff inbegriffen sind.
%eigentlich same
Lieferkosten für flüssigen Wasserstoff \ce{LH2} und HEFA werden nicht explizit ausgerechnet,
da die schon in Betriebskosten von Wasserstoff-getriebene Flugzeuge eingeschlossen sind. Es wird vor allem angenommen, dass die Produktion 
des Wasserstoffes und Verflüssigung nicht am Flughafen stattfindet, sondern eingekauft und zum Flughafen transportiert.


"Eine Kryopumpe 
bringt den Wasserstoff auf den benötigten Druck von 1000 bar. " ?file:///C:/Users/henri/Downloads/765438.pdf

\textit{Annahmen für die Analyse}
Die Gesamtinvestitionen sind von der Wahl der Produktion, Speicherung, Lieferketten als auch Betankungsentscheidung abhängig.
Da derzeit nicht einsehbar ist, welcher Technologie umgesetzt wird, fokussiert sich die Arbeit auf eine bestimmte VersorgungsWeg.

Lieferkosten für flüssigen Wasserstoff \ce{LH2} werden nicht explizit ausgerechnet,
da die schon in Betriebskosten von Wasserstoff-getriebene Flugzeuge eingeschlossen sind. Es wird vor allem angenommen, dass die Produktion 
des Wasserstoffes und Verflüssigung nicht am Flughafen stattfindet, sondern eingekauft und zum Flughafen transportiert wird.
Die Abbildung \ref{supply_wasserstoff} stellt eine Variante von Produktion- und Lieferketten für flüssigen Wasserstoff dar.

\begin{figure}[h]
	\centering
	\includegraphics[width=0.9\linewidth]{Bilder/Supply_hydrogen.png}
	\caption[Lieferkette von flüssigem Wasserstoff mit externer Herstellung und interner Lagerung bzw. die Betankung]{Lieferkette von flüssigem Wasserstoff mit externer Herstellung und interne Lagerung bzw. die Betankung \cite{schenke2024lh2}}
	\label{supply_wasserstoff}
\end{figure}

Batterie-Antrieb
Es wird angenommen, dass die Batterien für BA zur Flughafen-Infrastruktur gehören, 
d.h. die Anschaffungskosten für Flughafen anfallen und danach werden von für Fluggesellschaften ausgeliehen in Form von Leasing.
